\chapter{Deductive Calculus of First Order Logic}

\section{Deductive Calculus}

Proofs are (purely) syntactic constructs that caputre \emph{derivability of facts}. Deductive calculi provide descriptions of proofs in logic. There are more than one forms for deductive calculi.

We adopt the Hilbert-style deductive calculus, which contains

\begin{itemize}
    \item A set $\Lambda$ of wffs called \textbf{logical axioms}
    \item A \emph{single} \textbf{rule of inference} for forming a new wff from a pair of wffs
\end{itemize}

We then systematically generate a set of wffs from the axioms by using the rules of inference. They are called \textbf{derivable} wffs.

Another deductive calculus is called the natural calculus, which has many rules of inference and one single axiom (\emph{“排中律”}, that a proposition is either True or False).

Yet another deductive calculus is called the sequent calculus, which contains no axiom and a set of symmetric rules.

\subsection{Introduction to Soundness and Completeness}

The goal is to prove that for any language $\mathbb{L}$, the following are equivalent

\begin{itemize}
    \item The set of derivable wffs in $\mathbb{L}$
    \item The set of valid wffs in $\mathbb{L}$
\end{itemize}

This is done by proving the soundness and completeness

\begin{theorem}[Soundness]
    \label{thm:FOSoundness}
    Every derivable of wff is valid.
\end{theorem}

\begin{theorem}[Completeness]
    \label{thm:FOCompleteness}
    Every valid wff is derivable.
\end{theorem}

\section{Logical Axioms}

\begin{definition}[Generalization]
    A \textbf{generalization} of the wff $\alpha$ is any wff obtained by putting zero or more universal quantifiers in front of $\alpha$
\end{definition}

For example, $\forall x\forall y\forall y\alpha$ is a generalization of $\alpha$.

Note that every wff is a generalization of itself.

\begin{definition}[Axioms]
    Let $\mathbb{L}$ be a first-order language. The set $\Lambda$ of logical axioms of $\mathbb{L}$ consists of all generalizations of the wffs in the following groups
    \begin{axiom}
        \label{axiom:InstanceOfTautology}
        Instances of tautologies.
    \end{axiom}
    \begin{axiom}
        \label{axiom:Substitution}
        Wffs of the form $\forall x\alpha \to \alpha_t^x$ such that the term $t$ is \textbf{substitutable} for $x$ in $\alpha$. As a special case, $\forall x\alpha \to \alpha$, where we replace $x$ with $t=x$
    \end{axiom}
    \begin{axiom}
        \label{axiom:PushUniversalIntoImplication}
        Wffs of the form $\forall x(\alpha\to\beta) \to (\forall x \alpha\to \forall x \beta)$.
    \end{axiom}
    \begin{axiom}
        \label{axiom:QuantifyBoundedVar}
        Wffs of the form $\alpha\to\forall x \alpha$ if $x$ \emph{does not occur free} in $\alpha$
    \end{axiom}
    \begin{axiom}
        \label{axiom:Equality}
        Wffs of the form $x \doteq x$
    \end{axiom}
    \begin{axiom}
        \label{axiom:EqualitySubstitution}
        Wffs of the form $x \doteq y \to (\alpha \to \alpha')$ where $\alpha$ is atomic and $\alpha'$ is obtained from $\alpha$ by replacing zero or more free occurrences of $x$ in $\alpha$ by $y$
    \end{axiom}
\end{definition}

\begin{lemma}
    A wff $\varphi$ is valid $\iff$ $\forall{x}\varphi$ is valid.
\end{lemma}

For example,

\begin{itemize}
    \item $\alpha=Py$, $Py \to \forall x Py$ is an instance of Axiom \ref{axiom:QuantifyBoundedVar}
    \item $\alpha=Py$, $\forall y(Py\to \forall xPy)$ is a \emph{generalization} of instance of Axiom \ref{axiom:QuantifyBoundedVar}
    \item $\alpha = Px$, $x \doteq y \to Px\to Py$ is an instance of Axiom \ref{axiom:EqualitySubstitution}
    \item $\alpha = Px$, $x \doteq y \to Px \to Px$ is also an instance of Axiom \ref{axiom:EqualitySubstitution} because zero or more $x$ can be substituted
    \item $\alpha = Qxx$, $x\doteq y \to Qxx \to Qxy$ is also an instance of Axiom~\ref{axiom:EqualitySubstitution}
\end{itemize}

Despite being axioms, we can prove the validity of some of these axioms. We detail the proof of \ref{axiom:InstanceOfTautology} and \ref{axiom:Substitution}. Proof of others are trivial.

\subsection{Instance of Tautologies}

In this section we show the validity of Axiom~\ref{axiom:InstanceOfTautology}.

\begin{definition}[Instance of WFFs of Sentential Logic]
    Let $\alpha_1,\dots,\alpha_n,\dots$ be an infinite sequence of wffs of the first-order language of $\mathbb{L}$, $\varphi$ be a wff of sentential logic with junst the connectives $\to$ and $\neg$, $\varphi^\ast$ be the wff of $\mathbb{L}$ obtained by replacing every occurrence of the sentence symbol $A_n$ in $\varphi$ by $\alpha_n$ for each $m$. We say that $\varphi^\ast$ is an \textbf{instance} of $\varphi$
\end{definition}

For example, let $\varphi=A_1\to A_3$, $\varphi^\ast = \alpha_1 \to \alpha_3$ is an instance of $\varphi$.

\begin{definition}[Instance of Tautologies]
    For any tautologies $\varphi$ in the sentential logic, $\varphi^\ast$ is an instance of the tautology $\varphi$
\end{definition}

\begin{lemma}
    Given a structure $\frakA$ for language $\mathbb{L}$,
    \begin{itemize}
        \item $s$ be an assignment function
        \item $\varphi$ be a wff of sentential logic
        \item $\varphi^\ast$ be the instance of $\varphi$
        \item $v$ be the truth assignment such that
        \[ v(A_i) = T \iff \sat{A}{\alpha_i}{s} \]
    \end{itemize}
    Then
    \[ \bar{v}(\varphi) = T \iff \sat{A}{\varphi^\ast}{s} \]
\end{lemma}
\begin{proof}
    Prove by induction.
    \begin{itemize}
        \item[] \textbf{Base.} $\varphi = A_i$ follows immediately from the assumption that $v(A_i) = T \iff \sat{A}{\alpha_i}{s}$
        \item[] \textbf{Inductive.} \begin{enumerate}
            \item $\varphi = \neg \beta$.
            \begin{itemize}
                \item[] $\bar{v}(\neg\beta) = T \iff \sat{A}{(\neg\beta)^\ast}{s}$
                \item[$\equiv$] $\bar{v}(\beta) = F \iff \unsat{A}{\beta^\ast}{s}$ 
                \item[$\equiv$] $\bar{\beta} = T \iff \sat{A}{\beta^\ast}{s}$
            \end{itemize}
            \item $\varphi = \gamma \to \beta$
            \begin{itemize}
                \item[] $\bar{v}(\gamma\to\beta) = T \iff \sat{A}{\gamma\to\beta}{s}$
                \item[$\equiv$] If $\bar{v}(\gamma) = T$ then $\bar{v}(\beta) = T$ $\iff$ If $\sat{A}{\gamma}{s}$ then $\sat{A}{\beta}{s}$ 
            \end{itemize}
        \end{enumerate}
    \end{itemize}
\end{proof}

\begin{corollary}
    Every instance of a tautology is valid.
\end{corollary}

\subsection{Substitutions}

\begin{definition}[Substitution for Terms]
    Let $u$ be a term, $x$ be a variable, and $t$ be a term. $u_t^x$ is the result of replacing every occurrence of $x$ in $u$ by $t$
\end{definition}

\begin{definition}[Substitution for Formulas]
    For a wff $\alpha$, a variable $x$, let $\alpha_t^x$ be the result of replacing every free occurence of $x$ in $\alpha$ by $t$,

    \begin{itemize}
        \item If $\alpha$ is atomic. $\alpha = Pu_1,\dots,u_n$, then $\alpha_t^x = Pu_{1t}^x,\dots,u_{nt}^x$
        \item If $\alpha = \neg \beta$, then $\alpha_t^x = \neg\beta_t^x$
        \item if $\alpha = \beta\to\gamma$, then $\alpha_t^x = \beta_t^x \to \gamma_t^x$
        \item If $\alpha = \forall y \beta$, then $\alpha_t^x = \alpha$ if $y=x$, and $\alpha_t^x = \forall y \beta_t^x$
    \end{itemize}
\end{definition}

Now we return to Axiom~\ref{axiom:Substitution}. Is every $\forall x \alpha \to \alpha_t^x$ valid?

Of course the answer is ``No'', or otherwise we would not have required that $t$ is \emph{substitutable} for $x$ in $\alpha$.

To see this, let $\alpha = \exists y x\neq y$, $t=y$, then

\[ \forall x \exists y x\neq y \to \exists y y\neq y \]

which is obviously not valid.

The problem here is that by this substitution we made $x$ ``local/bounded'' (\emph{capture of free variables}), and the new formula is semantically different from the original one.

\begin{definition}[Substitutability]
    Let $\alpha$ be a wff, $x$ be a variable, and $t$ be a term. $t$ is substitutable for $x$ in $\alpha$ if
    \begin{itemize}
        \item $\alpha$ is atomic
        \item $\alpha = \neg\beta$ and $t$ is substitutable for $x$ in $\beta$.
        \item $\alpha = \beta \to \gamma$ and $t$ is substitutable for $x$ in $\beta$ and $\gamma$
        \item $\alpha = \forall y \beta$ and
        \begin{itemize}
            \item either $x$ does not occur free in $\forall y \beta$
            \item or $x$ occurs free in $\forall y\beta$ (which implies $x\neg y$), and $t$ is substitutable for $x$ in $\beta$, and $y$ does not occur in $t$
        \end{itemize}
    \end{itemize}
\end{definition}

We can check this with our previous example $\alpha = \exists y x\neq y$, $t=y$. $\alpha = \neg\forall y (\neg x \neq y)$. Notice that $y$ ocurrs in $t$ ($t=y$), therefore $y$ is not substitutable for $x$ in $\alpha$.

\subsubsection{Substitution Lemma}

\begin{lemma}
    Given a first-order language $\mathbb{L}$, let $\frakA$ be a structure for $\mathbb{L}$, $s$ be an assignment for $\frakA$, $u$ and $t$ be two terms and $x$ be a variable. Then
    \[ \bar{s}(u_t^x) = \overline{s(x|\bar{s}(t))}(u) \]
\end{lemma}

This looks very intuitive. It just states that the assignment for substitution is equal to first changing the assignment function and then apply the assignment on original term.

\begin{proof}
    Prove by induction.
    \begin{itemize}
        \item[] \textbf{Base.} (1) If $u = c$, trivial. (2) If $u=x$, then LHS would be $\bar{s}(t)$, RHS would also be $\bar{s}(t)$. (3) $u=y\neq x$, then LHS and RHS will both be $\bar{s}(y)$.
        \item[] \textbf{Inductive.} If $u = ft_1\dots t_n$.
        \[ \bar{s}(u_t^x) = \bar{s}(ft_{1t}^x\dots f(t_{nt}^x)) = f(\bar{s}(t_{1t}^x\dots t_{nt}^x)) \]
        which can be shown equal to RHS by applying inductive hypothesis
    \end{itemize}
\end{proof}

\begin{lemma}[Substitution Lemma]
    \label{lem:SubstitutionLemma}
    Let $s$ be an assignment function for $\frakA$. If $t$ is substitutable for $x$ in $\alpha$, then
    \[ \sat{A}{\alpha_t^x}{s} \iff \sat{A}{\alpha}{s(x|\bar{s}(t))} \]
\end{lemma}
\begin{proof}
    Prove by induction on $\alpha$.
    \begin{itemize}
        \item[] \textbf{Base.} $\alpha = Pt_1\dots t_n$
        \begin{itemize}
            \item[] $\sat{A}{(Pt_1\dots t_n)}{s}$
            \item[$\iff$] $\sat{A}{Pt_{1t}^x\dots t_{nt}^x}{s}$ 
            \item[$\iff$] $(\bar{s}(t_{1t}^x),\dots,\bar{s}(t_{nt}^x)) \in P^\frakA$
            \item[$\iff$] $(\bar{s'}(t_1),\dots,\bar{s'}(t_n)) \in P^\frakA$
            \item[$\iff$] $\sat{A}{\alpha}{s(x|\bar{s}(t))}$
        \end{itemize}
        \item[] \textbf{Inductive.} \begin{enumerate}
            \item  $\alpha = \neg \beta$
            \begin{itemize}
                \item[] $\sat{A}{\neg \beta_t^x}{s}$
                \item[$\iff$] $\unsat{A}{\beta_t^x}{s}$
                \item[$\iff$] $\unsat{A}{\beta}{s'}$
                \item[$\iff$] $\sat{A}{\neg\beta}{s'}$
            \end{itemize}
            \item $\alpha = \beta \to \gamma$. Similar.
            \item $\alpha = \forall y \beta$.
            \begin{itemize}
                \item[] $\sat{A}{\forall y \beta_t^x}{s}$
                \item[$\iff$] For any $d \in |\frakA|$, $\sat{A}{\beta_t^x}{s(y|d)}$
                \item[] $\sat{A}{\forall y\beta}{s'}$
                \item[$\iff$] For any $d \in |\frakA|$, $\sat{A}{\beta}{s(x|\bar{s}(t))(y|d)}$
                \item[] Since $y \neq x$, we can exchange the order of assignment overwriting
                \item[] For any $d \in |\frakA|$, $\sat{A}{\beta}{s(y|d)(x|\bar{s}(t))}$. For brevity denote $s(y|d)$ by $s''$. By induction hypothesis, we have $\sat{A}{\beta_t^x}{s''} \iff \sat{A}{\beta}{s''(x|\bar{s}''(t))}$. And by Substitutability, we have $\bar{s}''(t) = \bar{s}(t)$. Then we have proved the lemma.
            \end{itemize}
        \end{enumerate} 
    \end{itemize}
\end{proof}

\subsubsection{Validity of Axiom~\ref{axiom:Substitution}}

\begin{theorem}
    If $t$ is substitutable for $x$ in $\alpha$, then $\forall \alpha \to \alpha_t^x$ is valid.
\end{theorem}
\begin{proof}
    If $\sat{A}{\forall x \alpha}{s}$, since $s(x|\bar{s}(t))$ is an instance of ``for all $x$'', it holds that $\sat{A}{\alpha}{s(x|\bar{s}(t))}$. Then by the Substitution Lemma~\ref{lem:SubstitutionLemma}, we have $\sat{A}{\alpha_t^x}{s}$.
\end{proof}

Now we have proved that every member in the set of axioms $\Lambda$ is valid.

\section{Deductions}

\subsection{Rule of Inference}

\begin{definition}[Modus Ponens]
    Given any wffs $\alpha$ and $\beta$, the rule of \textbf{modus ponens} provides the operation for deriving $\beta$ from $\alpha\to\beta$ and $\alpha$.
\end{definition}

\subsection{Deduction}

\begin{definition}[Deduction]
    Let $\Gamma$ be a set of wffs of $\mathbb{L}$. A \textbf{deduction from $\Gamma$} is a finite sequence
    \[ \alpha_0,\dots,\alpha_n \]
    of wffs such that for every $\alpha_i$, at least one of the following holds
    \begin{itemize}
        \item $\alpha\in\Gamma$
        \item $\alpha\in\Lambda$
        \item $\alpha$ is inferred by modus ponens from two wffs $\alpha_j$ and $\alpha_k$ s.t. $j,k < i$.
    \end{itemize}
\end{definition}

\begin{definition}
    $\Gamma\vdash\alpha$ ($\alpha$ is a theorem of $\Gamma$) if there is a deduction $\alpha_0,\dots,\alpha_n$ from $\Gamma$ s.t. $\alpha=\alpha_n$.

    We write $\vdash\alpha$ for $\emptyset\vdash\alpha$
\end{definition}

Notice that deduction is purely based on symtax, and the logical implications we have learned before requires semantics.

As an example, we show that $\vdash Px \to \exists yPy$. Since $\Gamma = \emptyset$, we can only derive the conclusion from the axioms and modus ponens.

\[ Px \to \exists yPy \iff Px \to \neg\forall y \neg Py \]

$\forall y \neg Py \to \neg x$ is in axiom group~\ref{axiom:Substitution}. Further, $(\forall y \neg Py \to \neg x) \to (Px \to \neg \forall y \neg Py)$ is an instance of tautology $(A_1 \to \neg A_2) \to (A_2 \to \neg A_1)$. We can then finish the proof by applying modus ponens.

The proof can be formatted into a tree. But I cannot draw it so please refer to slides. A key problem is that few human beings (if any) is able to prove things like this, as it gradually ``factorizes'' the target wffs into members in $\Gamma$ or $\Lambda$. Fortunately we will later see some more helper rules to help normal people prove theorems in first-order logic.

\subsection{Properties of Deductions}

\begin{itemize}
    \item If $\Gamma\vdash\varphi$, then there must be a finite subset $\Delta$ of $\Gamma$ s.t. $\Gamma\vdash\varphi$. This is guaranteed by definition of deduction: it is finite.
    \item If $\alpha_1,\dots,\alpha_n$ is a deduction from $\Gamma$ and $\beta_1,\dots,\beta_m$ is a deduction from $\Gamma$, then $\alpha_1,\dots,\alpha_n,\beta_1,\dots,\beta_m$ is also a deduction from $\Gamma$
    \item If $\varphi\in\Gamma$ then $\Gamma\vdash\varphi$
    \item If $\Gamma\vdash\varphi$ and $\Gamma\subseteq\Delta$ then $\Delta\vdash\varphi$
    \item If $\Gamma\vdash\alpha$ and $\Gamma\vdash\alpha\to\beta$, then $\Gamma\vdash\beta$
    \item (Cut Rule) If $\Gamma\vdash\alpha$ and $\alpha\to\beta$, then $\Gamma\vdash\beta$. This property allows proving some theorem $\beta$ by one or more lemmas $\alpha$
    \item If $\Gamma\to\alpha$ then for any $\beta$, $\Gamma\vdash\beta\to\alpha$. Notice that $\alpha\to\left( \beta\to\alpha \right)$ is an instance of tautology.
    \item If $\Gamma\vdash\alpha\wedge\beta$, then $\Gamma\vdash\alpha$ and $\Gamma\vdash\beta$. Notice that $\alpha\wedge\beta = \neg\left( \alpha\to\neg\beta \right)$, and that $\neg\left( \alpha\to\neg\beta \right) \to \alpha$ (or $\beta$) are two instances of tautologies.
    \item If $\Gamma\vdash\alpha$ and $\Gamma\vdash\beta$ then $\Gamma\vdash\alpha\wedge\beta$
    \item If $\Gamma\vdash\alpha$ or $\Gamma\vdash\beta$ then $\Gamma\vdash\alpha\vee\beta$
    \item If $\Gamma\vdash\alpha\to\beta$ then $\Gamma;\alpha\to\beta$
\end{itemize}

\subsection{Relations between Tautologies and Derivations}

\begin{definition}
    In first-order logic, a set of wffs $\Gamma$ \textbf{tautologically implies} $\varphi$ if there is a wff $\alpha$ and a set $\Sigma$ of wffs in the sentential logic such that for some $*$ of $\mathbb{L}$
    \begin{itemize}
        \item For every $\beta\in\Sigma$, $\beta^*\in\Gamma$
        \item $\varphi=\alpha^*$
        \item $\Sigma$ tautologically implies $\alpha$ in the sentential logic ($\Sigma\vDash\alpha$)
    \end{itemize}
\end{definition}

\begin{lemma}
    $\Gamma\vdash\varphi$ iff $\Gamma\cup\Lambda$ tautologically implies $\varphi$
\end{lemma}

\begin{lemma}[Rule T (Enderton)]
    \label{lem:RuleT}
    If $\Gamma\vdash\alpha_1,\dots,\Gamma\vdash\alpha_n$ and $\{\alpha_1,\dots,\alpha_n\}$ tautologically implies $\beta$ then $\Gamma\vdash\beta$
\end{lemma}

\section{Deduction Theorem and Generalization Theorem}

\subsection{The Deduction Theorem}

\begin{theorem}[Deduction Theorem]
    \label{thm:DeductionTheorem}
    If $\Gamma;\alpha\to\beta$, then $\Gamma\vdash\alpha\to\beta$
\end{theorem}
\begin{proof}
    By induction on $\Gamma;\alpha\to\beta$.
    \begin{itemize}
        \item[] \textbf{Base.} If $\beta\in\Lambda$, then
        \[ \vdash\beta \Rightarrow \vdash \alpha\to\beta \Rightarrow \Gamma\vdash\alpha\to\beta \]
        If $\beta\in\Gamma$, then
        \[ \Gamma\vdash\beta \Rightarrow \Gamma\vdash\alpha\to\beta \]
        If $\beta=\alpha$, then it is an instance of tautology so it is in $\Lambda$
        \item[] \textbf{Inductive.} If $\Gamma;\alpha\vdash\gamma\to\beta$ and $\Gamma;\alpha\vdash\gamma$. By IH, $\Gamma\vdash\alpha\to(\gamma\to\beta)$ and $\Gamma\vdash\alpha\to\gamma$. Notice that $\{ \alpha\to(\gamma\to\beta), \alpha\to\gamma \}$ tautologically imply $\alpha\to\beta$. Then we apply ``Rule T''~\ref{lem:RuleT}.
    \end{itemize}
\end{proof}

\subsubsection{Contraposition}

\emph{Contraposition} is a corollary of the deduction theorem

\begin{theorem}[Contraposition]
    $\Gamma;\varphi\vdash\neg\psi$ iff $\Gamma;\psi\vdash\neg\varphi$
\end{theorem}
\begin{proof}
    \begin{itemize}
        \item[$\Rightarrow$] If $\Gamma;\varphi\vdash\neg\psi$, then by Deduction Theorem~\ref{thm:DeductionTheorem}, $\Gamma\vdash\varphi\to\neg\psi$. Notice that $\left( \varphi\to\neg\psi \right) \to \left( \psi\to\neg\varphi \right)$ is an instance of tautological implication in sentential logic. Therefore by Rule T~\ref{lem:RuleT} $\Gamma\vdash\psi\to\neg\varphi$, and $\Gamma;\psi\to\neg\varphi$. The converse is similar
    \end{itemize}
\end{proof}

\subsection{The Generalization Theorem}

\begin{theorem}[Generalization Theorem]
    \label{thm:GeneralizationTheorem}
    If $\Gamma\vdash\varphi$ and $x$ does not occur free in any member of $\Gamma$ then $\Gamma\vdash\forall x \varphi$
\end{theorem}
\begin{proof}
    By induction on $\varphi$
    \begin{itemize}
        \item[] \textbf{Base.} If $\varphi\in\Gamma$, then $x$ does not occur free in $\varphi$. Notice that we have $\varphi\to\forall x \varphi \in \Lambda$ (axiom~\ref{axiom:QuantifyBoundedVar}), and we have $\varphi\in\Gamma$, by modus ponens we have $\forall x \varphi$.
        
        If $\varphi\in\Lambda$, $\forall x\varphi$ is a generalization of $\varphi$.
        \item[] \textbf{Inductive.} $\Gamma\vdash\gamma\to\varphi$; $\Gamma\vdash\gamma$. By inductive hypothesis, $\Gamma\to\forall x(\gamma\to\varphi)$, and $\Gamma\vdash\forall x \gamma$. Notice that $\forall x(\gamma\to\varphi) \to \forall x \gamma\to\forall x\varphi$ (axiom~\ref{axiom:PushUniversalIntoImplication}).
    \end{itemize}
\end{proof}

The Deduction Theorem and the Generalization Theorem are ``meta-theorem''s that applies to deductive calculi. In Hilbert-style Calculus, we use the six logical axioms to derive these meta-theorems. In other calculi, they may use other axioms, or directly use these meta-theorems.

\subsubsection{Generalization on Constants}

\begin{theorem}
    \label{thm:GeneralizationOnConsts}
    If $\Gamma\vdash\varphi$ and $c$ is a constant symbol that is not in any member of $\Gamma$, then there is some variable $y$ not in $\varphi$ s.t. $\Gamma\vdash\forall y \varphi_y^c$ ($c$ replaced by $y$).

    Furthermore, there is a deduction of $\forall y \varphi_y^c$ from $\Gamma$ in which $c$ does not occur.
\end{theorem}

Intuitively, if $\varphi$ can be derived from $\Gamma$, then the constant $c$ in $y$ is somewhat like a quantified variable $y$.

\begin{proof}
    Since $\Gamma\vdash\varphi$, we have a deduction
    \[ \varphi_0,\varphi_1,\dots,\varphi_n \]
    s.t. $\varphi_n=\varphi$.

    We first show by induction (on deduction $\Gamma\vdash\varphi_i$) that
    \[ \varphi_{0y}^c,\dots,\varphi_{ny}^c \]
    is a deduction from $\Gamma$.
    \begin{itemize}
        \item[] \textbf{Base.} If $\varphi_i \in \Gamma$, since $c$ does not occur in members of $\Gamma$, we have $\varphi_{iy}^c = \varphi_i$. Then $\Gamma\vdash\varphi_{iy}^c$. If $\varphi_i\in\Lambda$, it can be verified by checking all 6 axioms.
        \item[] \textbf{Inductive.} If $\varphi_j = \varphi_k\to\varphi_i$, $\varphi_k$, $k,j < i$. By IH, $\Gamma\vdash\varphi_{ky}^c \to \varphi_{iy}^c$ and $\Gamma\vdash\varphi_{ky}^c$. We are done by Modus Ponens
    \end{itemize}

    We have shown so far that $\Gamma\vdash\varphi_y^c$ and $y$ does not occur in $\varphi$ (and the finite deduction sequence $\Delta$ that derives $\varphi$). Therefore we have $\Gamma\vdash\forall y \varphi_y^c$

    It also follows from this proof that the deduction does not contain $c$
\end{proof}

\begin{corollary}[Corollary 24G]
    \label{coroll:Corollary24G}
    If $\Gamma\vdash\varphi_c^x$, and $c$ is a constant symbol that is not in $\varphi$ or any member of $\Gamma$, then $\Gamma\vdash\varphi$
\end{corollary}

\begin{corollary}[Rule El]
    If $\Gamma;\varphi_c^x\vdash\psi$ and $c$ does not occur in $\varphi,\psi,\Gamma$, then $\Gamma;\exists x \varphi\vdash\psi$.
    
    Furthermore, there is a deduction of $\psi$ from $\Gamma;\exists x \varphi$ in which $c$ does not occur.
\end{corollary}
\begin{proof}
    To be completed.
\end{proof}

\subsection{The Re-Replacement Lemma}

\begin{lemma}[Re-Replacement Lemma]
    \label{lem:ReReplacementLemma}
    If $y$ does not occur in $\varphi$, then $x$ is substitutable for $y$ in $\varphi_y^x$ and $\varphi_{yx}^{xy} = \varphi$
\end{lemma}

\subsection{Consistency}

\begin{definition}[Consistency]
    $\Gamma$ is \textbf{inconsistent} if there is some wff $\alpha$ s.t. $\Gamma\vdash\alpha$ and $\Gamma\vdash\neg\alpha$

    $\Gamma$ is \textbf{consistent} if it is not inconsistent.
\end{definition}

Some properties of consistency

\begin{itemize}
    \item If $\Gamma$ is inconsistent then for every $\beta$, $\Gamma\vdash\beta$
    \begin{itemize}
        \item[] $\Gamma\vdash\alpha$, $\Gamma\vdash\alpha$
        \item[$\Rightarrow$] $\Gamma;\beta\vdash\alpha$ $\Gamma;\neg\beta\vdash\alpha$ (Add $\beta$ to $\Gamma$ and the deduction should still hold)
        \item[$\Rightarrow$] $\Gamma\vdash\neg\beta\to\alpha$ $\Gamma\vdash\neg\beta\to\neg\alpha$ (Deduction Theorem)
        \item[$\Rightarrow$] Notice that $\neg\beta\to\alpha, \neg\beta\to\neg\alpha$ tautologically implies $\beta$. By Rule T we have $\Gamma\vdash\beta$ 
    \end{itemize}
    \item If $\Gamma\nvdash\alpha$ $\iff$ $\Gamma;\neg\alpha$ is consistent.
    \begin{itemize}
        \item Equivalently, $\Gamma\vdash\alpha$ iff $\Gamma;\neg\alpha$ is inconsistent. $\Rightarrow$ is trivial.
        \item $\Leftarrow$ If $\Gamma;\neg\alpha$ is inconsistent, then there exists some $\beta$ s.t. $\Gamma;\neg\alpha\vdash\beta$ and $\Gamma;\neg\alpha\vdash\neg\beta$. The rest is similar to the previous proof.
    \end{itemize}
    \item $\Gamma$ is consistent $\iff$ every finite subset of $\Gamma$ is consistent.
    \begin{itemize}
        \item[] Equivalent to $\Gamma$ is inconsistent $\iff$ there is some subset of $\Gamma$ which is inconsistent
        \item[$\Rightarrow$] $\Gamma\vdash\alpha$, $\Gamma\vdash\neg\alpha$. Then there exists some finite subset $\beta_1\vdash\alpha$ and $\beta_2\vdash\neg\alpha$. And the union of $\beta_1$ and $\beta_2$ is the desired subset.
        \item[$\Leftarrow$] Trivial. 
    \end{itemize}
    \item If $\Gamma$ is consistent, then for every $\alpha$, either $\Gamma;\alpha$ is consistent or $\Gamma;\neg\alpha$ is consistent.
    \begin{itemize}
        \item Prove by contradiction. Assume for some $\alpha$, $\Gamma;\alpha$ is inconsistent, and $\Gamma;\neg\alpha$ is also inconsistent. Notice that $\Gamma;\neg\alpha$ is inconsistent iff $\Gamma\vdash\alpha$, so we have $\Gamma\vdash\alpha$. We also have $\Gamma\vdash\neg\alpha$ iff $\Gamma;\neg\neg\alpha$\footnote{We are working on pure grammatical level, so if the wffs ``looks'' different, then they are different, so we cannot directly say that $\Gamma;\neg\neg\alpha$ is inconsistent iff $\Gamma;\alpha$ is inconsistent.}. Further, it can be shown that $\Gamma;\neg\neg\alpha$ is inconsistent iff $\Gamma;\alpha$ is inconsistent.
    \end{itemize}
\end{itemize}

\begin{theorem}[Reductio Ad Absurdum]
    \label{thm:ReductioAdAbsurdum}
    If $\Gamma;\alpha$ is inconsistent, then $\Gamma\vdash\neg\alpha$
\end{theorem}

\section{Backward Inference and Prove Strategies}

\subsection{Backward Inference}

With all the properties derived so far, we can formulate a general method to prove theorems.

Assume we are showing $\Gamma\vdash\varphi$

\begin{itemize}
    \item If $\varphi = \alpha\to\beta$, then it suffices to show $\Gamma;\alpha\to\beta$ (Deduction Theorem)
    \item If $\varphi = \forall x \alpha$, and $x$ does not occur free in $\Gamma$, then it suffices to show $\Gamma\vdash\alpha$ (Generalization Theorem)
    \item If $\varphi = \forall x\alpha$, and $x$ occurs free in $\Gamma$, then we pick a variable $y$ that does not occur in $\alpha$ and $\Gamma$, then we have $\forall y \alpha_y^x\vdash\forall x\alpha$ and it suffices to show $\Gamma\vdash\forall y \alpha_y^x$
    \begin{itemize}
        \item To show why we have $\forall y \alpha_y^x\vdash\forall x\alpha$, it suffices to show $\forall y \alpha_y^x\vdash\alpha$. We have $\forall y \alpha_y^x \to \left( \alpha_y^x \right)_x^y = \alpha$ (by Axiom of Substitution and Re-Replacement Lemma).
    \end{itemize}
    \item If $\varphi = \neg\left( \alpha\to\beta \right)$, then it suffices to show $\Gamma\vdash\alpha$ and $\Gamma\vdash\neg\beta$ (Rule T)
    \item If $\varphi=\neg\neg\alpha$, then it suffices to show that $\alpha$ (Rule T)
    \item If $\varphi=\neg\forall x \alpha$, then it suffices to show that $\Gamma\vdash\neg\alpha_t^x$ for some $t$ substitutable for $x$ in $\alpha$
    \begin{itemize}
        \item This is not always possible, because there are cases where $\Gamma\vdash\neg\forall x\alpha$, but $\Gamma\nvdash\neg\alpha_t^x$ for all $t$. In this case, try Contraposition or Reductio Ad Absurdum
    \end{itemize}
    \item For atomic and negations of atomic formula, try contraposition.
\end{itemize}

Note that due to that applying this method will not always lead to a proof, or may lead to something that is not provable, this method is \emph{incomplete}.

\subsubsection{Examples}

\begin{enumerate}
    \item Assume $x$ does not occur free in $\beta$, derive

    \begin{itemize}
        \item $\vdash\forall x \left( \alpha \to \beta \right) \rightarrow \left( \alpha\to\forall x \beta \right)$
        \item[$\Leftarrow$] $\forall x \left( \alpha\to\beta \right)\vdash\exists x\alpha\to\beta$ (Deduction Thm)
        \item[$\Leftarrow$] $\forall x \left( \alpha\to\beta \right), \exists x \alpha \vdash \beta$ (Deduction Thm)
        \item[$\Leftarrow$] $\forall x \left( \alpha\to\beta \right), \exists x \alpha \vdash \neg\neg\beta$ (Rule T)
        \item[$\Leftarrow$] $\forall x \left( \alpha\to\beta \right),\neg\beta \vdash \neg\neg\forall x\neg \alpha$ (Contraposition)
        \item[$\Leftarrow$] $\dots \vdash \forall x\neg \alpha$
        \item[$\Leftarrow$] $\dots\vdash\neg\alpha$ (Generalization, note that $x$ does not occur free in $\beta$)
        \item[$\Leftarrow$] $\forall x \left( \alpha\to\beta \right),\alpha\vdash\neg\neg\beta$ (Contraposition)
        \item[$\Leftarrow$] $\dots\vdash\beta$   
    \end{itemize}

    \item Show $\vdash\forall x \forall z Pxz \to \forall{y} Pyy$
    \begin{itemize}
        \item[$\Leftarrow$] $\forall x \forall z Pxz \vdash \forall y Pyy$
        \item[$\Leftarrow$] $\forall x \forall z Pxz \vdash Pyy$
        \item[$\Leftarrow$] $\forall x \forall z Pxz \to \forall z Pyz$ (Substitution)
        \item[] Suffice to show $\forall zPyz \vdash \forall z Pyz$, again can be proved by substitution.  
        \item[] However, cannot prove $\vdash\forall x\forall y Pxy \to \forall y Pyy$ like this, although this formula is semantically valid ($\vDash$)
    \end{itemize}

    \item Show $\vdash\forall x \forall y Pxy \to \forall{y} Pyy$
    \begin{itemize}
        \item[] We have shown that Show $\vdash\forall x \forall z Pxz \to \forall{y} Pyy$
        \item[] We will show that $\forall x\forall y Pxy \vdash\forall{x}\forall{z}Pxz$
        \item[$\Leftarrow$] $\forall{x}\forall{y}Pxy \vdash Pxz$ (Generalization)
        \item[] Done. (Substitution)
    \end{itemize}
\end{enumerate}

\subsection{Alphabetic Variants}

\begin{theorem}[Alphabetic Variants]
    \label{thm:AlphabeticVariants}
    Given a wff $\alpha$, a term $t$ and a variable $x$. There is a wff $\alpha'$ s.t. $\alpha'$ differs from $\alpha$ only in quantified variables; $\vdash \alpha\leftrightarrow\alpha'$ and $t$ is substitutable for $x$ in $\alpha'$.
\end{theorem}

Intuitively, we can always find a $\alpha'$ such that we can do substitution whenever there is a conflicting variable in $\alpha$. For the examples mentioned above, $\alpha=\forall{x}\forall{y}Pxy$, and $\alpha'=\forall{x}\forall{z}Pxz$

Can be proved by induction on $\alpha$, and can refer to Enderton.

\subsubsection{Example}

Show $\forall{x}\forall{y}Pxy\vdash\forall{y}Pyy$, and $y$ is \emph{not} substitutable for $x$ in $\forall{y}Pxy$.

Apply the alphabetic variants theorem to $\forall{y}Pxy$ to get $\forall{z}Pxz$.

\begin{itemize}
    \item $\vdash\forall{y}Pxy\leftrightarrow\forall{z}Pxz$
    \item $\forall{x}\forall{y}Pxy\to\forall{x}\forall{z}Pxz$ (Generalization)
    \item $\forall{x}\forall{z}Pxz$ TO BE COMPLETED.
\end{itemize}

\section{The Soundness Theorem}

\begin{lemma}
    Given a set $\Gamma$ of wffs and the wffs $\alpha$ and $\beta$, if $\Gamma\vDash\alpha\to\beta$ and $\Gamma\vDash\alpha$, then $\Gamma\vDash\beta$
\end{lemma}

\begin{theorem}[Soundness Theorem]
    \label{thm:SoundnessTheorem}
    If $\Gamma\vdash\alpha$, then $\Gamma\vDash\alpha$
\end{theorem}
\begin{proof}
    Prove by induction on the deduction. Assume $\Gamma=\{ \gamma_0,\dots,\gamma_n \}$, we show that $\Gamma\vdash\varphi_i$ for all $i$
    \begin{itemize}
        \item[] \textbf{Base.} If $\varphi_i \in \Lambda$, by validity of all axioms, it holds. If $\varphi_i \in \Gamma$, then obviously $\Gamma\vDash\varphi_i$
        \item[] \textbf{Inductive.} Suppose there exists some $\varphi_j$ and $\varphi_k$ s.t. $j,k \le i$. Let $\varphi_k = \varphi_j \to \varphi_i$, and $\Gamma\vdash\varphi_j, \Gamma\vdash\varphi_k$. By inductive hypothesis we have $\Gamma\vDash\varphi_j$ and $\Gamma\vDash\varphi_k=\varphi_j\to\varphi_i$. Therefore by lemma we have $\Gamma\vDash\varphi_i$. 
    \end{itemize}
\end{proof}

\begin{corollary}
    If $\vdash\alpha$, then $\vDash\alpha$.
\end{corollary}

\begin{corollary}
    If $\vdash\varphi\leftrightarrow\psi$, then $\varphi$ and $\psi$ are logically equivalent.
\end{corollary}
\begin{proof}
    If $\vdash\varphi\rightarrow\psi$, then $\varphi\vDash\psi$. The converse is similar.
\end{proof}

\begin{corollary}
    If $\varphi'$ is an alphabetic variant of $\varphi$, then $\varphi$ and $\varphi'$ are logically equivalent.
\end{corollary}

\subsection{Soundness and Satisfiability}

\begin{corollary}
    If $\Gamma$ is satisfiable, then $\Gamma$ is consistent.
\end{corollary}

In fact, soundness is equivalent to the above corollary.

\begin{theorem}
    The following two statements are equivalent.
    \begin{itemize}
        \item For any $\Gamma$ and $\alpha$, if $\Gamma\vdash\alpha$, then $\Gamma\vDash\alpha$
        \item For any $\Gamma$, if $\Gamma$ is satisfiable, then $\Gamma$ is consistent.
    \end{itemize}
\end{theorem}
\begin{proof}
    \begin{itemize}
        \item[$\Rightarrow$] If we have soundness, we prove consistency by contradiction. Assume $\Gamma$ is satisfiable, but $\Gamma$ is inconsistent. Then there exists some $\varphi$ s.t. $\Gamma\vdash\varphi$ and $\Gamma\vdash\neg\varphi$. By soundness we have $\Gamma\vDash\varphi$ and $\Gamma\vDash\neg\varphi$. Since $\Gamma$ is satisfiable, there is some structure $\frakA$ and assignment $s$ s.t. $\sat{A}{\varphi}{s}$ and $\sat{A}{\neg\varphi}{s}$. 寄!
        \item[$\Leftarrow$] Conversely, if we have the latter, we prove soundness by contradiction. Assume $\Delta\vdash\alpha$ but $\Delta\nvDash\alpha$. Then $\Delta;\neg\alpha$ is satisfiable. So $\Delta;\neg\alpha$ is consistent. By property of consistency we have $\Delta\nvdash\alpha$. 寄!
    \end{itemize}
\end{proof}

\section{The Completeness Theorem}

\begin{theorem}[G\"{o}del Extended Completeness Theorem]
    If $\Gamma\vDash\alpha$, then $\Gamma\vdash\alpha$.
\end{theorem}

\begin{corollary}[G\"odel Completeness Theorem]
    If $\vDash\alpha$, then $\vdash\alpha$.
\end{corollary}

\subsection{Equivalent Statement for Completeness}

\begin{theorem}
    The following statements are equivalent
    \begin{itemize}
        \item For any $\Gamma$ and $\alpha$, if $\Gamma\vDash\alpha$, then $\Gamma\vdash\alpha$
        \item For any $\Gamma$, if $\Gamma$ is consistent then $\Gamma$ is satisfiable
    \end{itemize}
\end{theorem}
\begin{proof}
    \begin{itemize}
        \item[$\Rightarrow$] Assmue completeness, and assume $\Gamma$ is consistent. Assume for contradiction that $\Gamma$ is unsatisfiable. Then $\Gamma\vDash\alpha$ and $\Gamma\vDash\neg\alpha$. By completeness $\Gamma\vdash\alpha$ and $\Gamma\vdash\neg\alpha$, which means $\Gamma$ is inconsistent.
        \item[$\Leftarrow$] Assume the latter, we prove completeness by constradiction. Assume $\Gamma\vDash\alpha$ but $\Gamma\nvdash\alpha$. Then by property of consistency $\Gamma;\neg\alpha$ is consistent and thus satisfiable. Therefore $\Gamma\vDash\neg\alpha$. 寄!
    \end{itemize}
\end{proof}

\subsection{Proof of Completeness}

\emph{“20世纪逻辑学最重要的发现之一。”}

The proof is similar to that for compactness.

\begin{enumerate}
    \item Extend $\Gamma$ to $\Delta\supseteq\Gamma$ s.t. $\Delta$ is consistent and maximal (for any $\alpha$, either $\alpha\in\Delta$ or $\neg\alpha\in\Delta$)
    \item Define a structure $\frakA$ and an assignment $s$ for $\frakA$ s.t. $\frakA$ satisfies $\Gamma$ with $s$
\end{enumerate}

But the actual proof is more complex, because we have to deal with $\doteq$. The actual proof consists of 6 steps.

\subsubsection{Expanding a Language with Constants}

Let $\Gamma$ be a consistent set of wffs in a \emph{countable} language. We expand the language with a countably infinite set of new constant symbols $c_1,\dots,c_n,\dots$

Assume $\Gamma$ is defined in $\mathbb{L}$, and let
\[ \mathbb{L}' = \mathbb{L}\cup\{ c_1,\dots,c_n,\dots \} \]

We show by contradiction that $\Gamma$ is also consistent in $\mathbb{L}'$.

\begin{proof}
    Assume $\Gamma$ is inconsistent in $\mathbb{L}'$. Then there is some wff $\alpha$ s.t. $\Gamma\vdash\alpha$ and $\Gamma\vdash\neg\alpha$. Therefore $\Gamma\vdash\alpha\wedge\neg\alpha$. $\alpha$ may contain the new introduced constant symbols, but since we have Generalization on Constants, we can show that there is some deduction
    \[ \alpha_1',\dots,\alpha_n' = \alpha'\wedge\neg\alpha' \]
    where $\alpha_i'$ is $\alpha_i$ with all constants not in $\mathbb{L}$ replaced with some variable. Therefore $\Gamma$ is inconsistent in $\mathbb{L}$.
\end{proof}

\subsubsection{Preparing for Satisfiability of Quantified WFFs}

In the new language, for any pair of wff $\varphi$ and variable $x$, we introduce a formula
\[ \neg\forall{x}\varphi \to \neg\varphi_c^x \]

where $c$ is a new constant symbol. Let $\Theta$ be the set of all these formulas. This is essentially (1) $c$ identifies a \emph{counter example} for $\varphi$, and (2) $\Gamma\cup\Theta$ is consistent.

Note that the pairs of $\varphi$ and $x$ are enumerable

\[ (\varphi_1,x_1), (\varphi_2,x_2),\dots \]

and the newly introduced constant are also enumerable

\[ c_1,c_2,\dots \]

Since each $\varphi$ contains finitely many constants, there must be some certain way of enumeration such that $c_{i+1}$ does not occur in $\varphi_1,\dots,\varphi_i$.

We now show that $\Gamma\cup\Theta$ is consistent.

\begin{proof}
    Assume for contradiction that $\Gamma\cup\Theta$ is inconsistent. Then $\Gamma\cup\Theta\vdash\alpha$ and $\Gamma\cup\Theta\vdash\neg\alpha$.

    So there is some finite subset $\Theta_k = \{\theta_1,\dots,\theta_k\} \subseteq \Theta$ s.t. $\Gamma\cup\Theta_k$ is inconsistent. This is guaranteed by the finiteness of deduction. Notice that since $\Gamma$ itself is consistent, $\Theta_k$ cannot be empty. Let $k$ be the minimal number s.t. $\Gamma\cup\Theta_k$ is inconsistent.

    By Reductio Ad Absurdum, we have
    \[ \Gamma:=\Gamma\cup\{ \theta_1,\dots,\theta_{k-1} \} \vdash \neg \theta_k \]
    where
    \[ \theta_k = \neg\forall x \varphi_k \to \neg \varphi_{kc_k}^x \]

    Therefore
    \[ \Gamma' \vdash \neg\forall{x}\varphi_k \wedge \varphi_{kc_k}^k \]
    i.e. $\Gamma' \vdash \neg\forall{x}\varphi_k$ and $\Gamma'\vdash\varphi_{kc_k}^k$

    Note that from a corollary of Generalization on constants \ref{coroll:Corollary24G}. We have $\Gamma'\vdash\varphi_{kc_k}^k\Longrightarrow \Gamma'\vdash\forall{x}\varphi_k$. This shows that $\Gamma'$ is inconsistent, which is contradictory to our assumption that $k$ (not $k-1$) is the smallest number that makes $\Gamma\cup\Theta_k$ inconsistent.
\end{proof}

\subsubsection{Generate Maximally Consistent Set}

We extend $\Gamma\cup\Theta$ to a set $\Delta$ s.t.

\begin{itemize}
    \item $\Delta$ is consistent.
    \item For each wff $\alpha$, either $\alpha\in\Delta$ or $\neg\alpha\in\Delta$
\end{itemize}

\begin{proof}
    $\Gamma\cup\Theta$ is consistent.
    \begin{itemize}
        \item There is no $\beta$ s.t. $\Gamma\cup\Theta\vdash\beta$ and $\Gamma\cup\Theta\vdash\neg\beta$.
        \item[$\Leftrightarrow$] There is no $\beta$ s.t. $\Gamma\cup\Theta\cup\Lambda$ tautologically imply $\beta$ and $\neg\beta$
        \item[$\Rightarrow$] Therefore $\Gamma\cup\Theta\cup\Lambda$ is satisfiable \emph{in the sense of sentential logic}
        \item[$\Rightarrow$] There is some truth assignment $v$ satisfying $\Gamma\cup\Theta\cup\Lambda$. Therefore we pick
        \[ \Delta = \{ \varphi | \bar{v}(\varphi) = T \} \]
    \end{itemize}
\end{proof}

We continue to show some properties of $\Delta$.

\begin{proposition}
    For any $\alpha$, either $\alpha\in\Delta$ or $\neg\alpha\in\Delta$, but not both.
\end{proposition}
\begin{proof}
    If $\bar{v}(\alpha) = T$, then $\alpha\in\Delta$. $\bar{v}(\neg\alpha)=F$. Then $\neg\alpha\notin\Delta$. Conversely, if $\bar{v}(\alpha)=F$, then $\neg\alpha\in\Delta$, and thus $\alpha\notin\Delta$.
\end{proof}

\begin{proposition}
    $\Delta$ is a \emph{theory}. That is,
    \[ \Delta\vdash\alpha \Longrightarrow \alpha\in\Delta \]
\end{proposition}
\begin{proof}
    If $\Delta\vdash\alpha$, then $\Delta\cup\Lambda$ tautologically implies $\alpha$. This is equivalent to $\Delta$ tautologically implies $\alpha$, because by definition of $\Delta$ we have $\Lambda\subseteq\Delta$. Since we know that $v$ satisfies $\Delta$ (in sentential logic), by tautological implication we know that $v$ should also satisfy $\alpha$. Thus $\alpha\in\Delta$
\end{proof}

\begin{proposition}
    $\Delta$ is consistent.
\end{proposition}
\begin{proof}
    If $\Delta$ is inconsistent, then there exists some formula $\beta$ s.t. $\Delta\vdash\beta$ and $\neg\beta$. Then $\beta\in\Delta$ and $\neg\beta\in\Delta$. 噔噔咚
\end{proof}

\subsubsection{Make a Structure for the New Language}

We make a structure $|\frakA|$ from $\Delta$ for the new language where $\doteq$ is replaced by a 2-nary symbol $E$.

sudo make install!

\begin{itemize}
    \item Let $|\frakA|$ be the set of all terms in the new language
    \item $(u,t)\in E^\frakA \iff u\doteq t \in \Delta$
    \item For any n-ary predicate symbol,
    \[ (t_1,\dots,t_n) \in P^\frakA \iff Pt_1\dots t_n \in \Delta \]
    \item For any n-ary predicate symbol,
    \[ f^\frakA (t_1,\dots,t_n) = ft_1\dots t_n \]
    \item For any constant symbol $c$, $c^\frakA=c$
\end{itemize}

Then we make an assignment function $s:V\mapsto|\frakA|$
\[ s(x)=x \]

Then $\bar{s}(t)=t$. For any wff $\varphi$, let $\varphi^*$ be $\varphi$ with $\doteq$ replaced by $E$. We have
\[ \sat{A}{\varphi^*}{s} \iff \varphi\in\Delta \]

Notice that once we prove this, we have already found a structure $\frakA$ and an assignment $s$ that satisfies $\Delta$. That is, if the language does not contain $\doteq$, we have proved completeness.

\begin{proof}
    Assume $n$ is the number of connectives in $\varphi$. Prove by induction on $n$. We do induction on $n$ to avoid $\doteq$'s in $\varphi$ that may cause troubles.
    \begin{itemize}
        \item[] \textbf{Base.} $n=0$. Then $\varphi$ is $u\doteq t$ or $Pt_1\dots t_n$. If $\varphi$ is $u\doteq t$, then $\varphi^*$ is $Eut$.
        \[ \sat{A}{Eut}{s} \iff (\bar{s}(u), \bar{s}(t)) \in E^\frakA \iff (u,t) \in E^\frakA \iff u\doteq t \in \Delta \]
        If $\varphi$ is $Pt_1\dots t_n$, the proof is similar.

        \item[] \textbf{Inductive.} If $\varphi = \neg\alpha$, then $\varphi^*=\neg\alpha^*$.
        \[ \sat{A}{\neg\alpha^*}{s} \iff \unsat{A}{\alpha^*}{s} \iff \alpha\notin\Delta\iff\neg\alpha\in\Delta \]
        The second step uses the inductive hypothesis.

        If $\varphi=\alpha\to\beta$, then $\varphi^*=\alpha^*\to\beta^*$.
        \[ \sat{A}{\alpha^*\to\beta^*}{s} \iff \unsat{A}{\alpha^*}{s}\text{ or }\sat{A}{\beta^*}{s} \iff \alpha\notin\Delta \text{ or }\beta\in\Delta \]
        \begin{itemize}
            \item We first show $\alpha\notin\Delta$ or $\beta\in\Delta$ implies $\alpha\to\beta\in\Delta$. To show this, it suffices to show that $\Delta\vdash\alpha\to\beta$, and it suffices to show that $\Delta;\alpha\vdash\beta$. Then 分类讨论两种前提 and we are done.\footnote{“事实上我也忘了这个怎么证明,我们来证一下。”}
            \item To show $\alpha\to\beta\in\Delta\Rightarrow\alpha\notin\Delta$ or $\beta\in\Delta$.
            \[ \alpha\to\beta\in\Delta\iff\Delta\vdash\alpha\to\beta\Longleftarrow\Delta;\alpha\vdash\beta \]
        \end{itemize}

        If $\varphi=\forall{x}\alpha$. $\varphi^*=\forall{x}\alpha^*$.
        \[ \sat{A}{\forall{x}\alpha^*}{s} \iff \text{For every $t\in|\frakA|$,} \sat{A}{\alpha^*}{s(x|t)} \]
        \begin{itemize}
            \item[$\iff$] $\sat{A}{\alpha^*}{s(x|\bar{s}(t))}$
            \item[$\iff$] $\sat{A}{(\alpha^*)^x_t}{s}$ (Substitution Lemma)
            \item[$\iff$] $\sat{A}{(\alpha^x_t)^*}{s}$
            \item[$\iff$] $\alpha_t^x \in \Delta$ (IH) 
            \item[$\Longrightarrow$] $\alpha_c^x\in\Delta$ 
        \end{itemize}
        Notice that $\neg\forall{x}\alpha\to\neg\alpha_c^x \in \Delta$. Since we have $\alpha_c^x\in\Delta$, the condition $\neg\forall{x}\alpha$ cannot hold, so $\forall{x}\alpha\in\Delta$

        So far we have proved $\Rightarrow$. Now consider the other side $\forall{x}\alpha\in\Delta\Rightarrow\sat{A}{\forall{x}\alpha^*}{s}$. This is equivalent to showing
        \[ \unsat{A}{\forall{x}\alpha^*}{s} \Longrightarrow \forall{x}\alpha\notin\Delta \]
        Assume there is some $t$ s.t. $\unsat{A}{\alpha^*}{s(x|t)}$, equivalent to $\unsat{A}{(\alpha_t^x)^*}{s}$.
        
        Let $\beta$ be alphabetic equivalent to $\alpha$ s.t. $t$ is substitutable for $x$ in $\beta$. We have $\alpha\vDash\Dashv\beta$ by alphabetical equivalence.

        Therefore previous equation is equivalent to
        \[ \unsat{A}{(\beta_t^x)^*}{s} \iff \beta_t^x \notin \Delta \]

        If $\forall{x}\alpha\in\Delta$, then $\Delta\vdash\forall{x}\alpha$, and thus $\Delta\vdash\forall{x}\beta$. We have axiom $\forall{x}\beta\to\beta_t^x$ since $t$ is substitutable for $x$ in $\beta$. Thus $\Delta\vdash\beta_t^x$ and $\beta_t^x\in\Delta$. Contradiction. So $\forall{x}\alpha\notin\Delta$, we are done.
    \end{itemize}
\end{proof}

Until now, we are done if the language does not contain $\doteq$. But to actually complete the proof for all cases, we need an extra step.

\subsubsection{Deal with Equality}

Not enough time. Pigeoned. Refer to slides or reference books.
