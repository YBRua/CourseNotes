\chapter{The Informal Notions of Algorithms}

\emph{“形式语言与自动机课程速成。”}

\section{Algorithms}

\begin{definition}[Algorithms (Informal)]
    \label{def:Algorithm}
    An algorithm is a \textbf{finite ordered list} of instructions.
\end{definition}

Possible outcomes of running an algorithm
\begin{itemize}
    \item The algorithm does not halt
    \item The algorithm halts
    \begin{itemize}
        \item In an erroneous state (fails)
        \item Gives valid outputs
    \end{itemize}
\end{itemize}
Cases other than the algorithm giving valid outputs are collectively identified as ``no output''.

\begin{definition}
    An algorithm for \emph{determining membership} in a set $A \subseteq \mathbb{N}$ has an input, and two possible outputs ``yes'' and ``no''. If the algorithm is run on input $n$, it will halt in finite steps with output ``yes'' if $n \in A$ and ``no'' if $n \notin A$.
\end{definition}

\begin{definition}[Effectively Decidable Sets]
    \label{def:EffectivelyDecidableSet}
    Let $A$ be a subset of $\mathbb{N}$. $A$ is \textbf{effectively decidable} if there is an algorithm for determining membership of $A$.
\end{definition}
\begin{theorem}
    If $A$ and $B$ are effectively decidable subsets of $\mathbb{N}$, then $\mathbb{N}\backslash A$, $A \cap B$ and $A \cup B$ are all effectively decidable.
\end{theorem}

Algorithms can have different kinds of outputs and inputs.

\begin{definition}[Diophantine Equations]
    Consider polynomials with integer coefficients (and any number of variables), a \textbf{diophantine equation} is an equation of the form $p=0$, where $p$ is sunch a polynomial. (e.g., $3x^2 + 5xy - 2z^4 +3 = 0$)
\end{definition}

\textbf{Hilbert's 10th Problem.} Is there an algorithm for determining whether or not diophantine equations have integer solutions?
