\documentclass[oneside]{book}
\usepackage[UTF8]{ctex}
\usepackage{amsmath}
\usepackage{mathtools}
\usepackage{listings} % lstlist插入代码
\usepackage{booktabs}
\usepackage{ulem}
\usepackage{enumerate}
\usepackage{amsfonts}
\usepackage{amssymb}
\usepackage{amsthm}
\usepackage{proof}
\usepackage{setspace} % spacing环境设置行间距
\usepackage[ruled, vlined]{algorithm2e} % 算法与伪代码 
\usepackage{bm} % 数学公式中的加粗
\usepackage{pifont} % 打圈的数字。172-211。\ding
\usepackage{graphicx}
\usepackage{float}
\usepackage[dvipsnames]{xcolor}
%\usepackage{indentfirst}
\usepackage{ulem} %\sout{}打删除线
\normalem % 使用默认normalem
\usepackage{lmodern}
\usepackage{subcaption}
\usepackage[colorlinks, linkcolor=blue]{hyperref}
\usepackage{cleveref}
\usepackage[a4paper]{geometry}
\usepackage{titlesec}
\usepackage{graphicx}
\usepackage{stmaryrd}

\theoremstyle{definition}
\newtheorem{definition}{Definition}[section]
\newtheorem{theorem}{Theorem}[section]
\newtheorem*{optTheorem}{Theorem}
\newtheorem{proposition}{Proposition}[section]
\newtheorem{lemma}{Lemma}[section]
\newtheorem{corollary}{Corollary}[section]
\newtheorem{axiom}{Axiom}[section]
\theoremstyle{remark}
\newtheorem*{remark}{Remark}
\newtheorem*{sketchproof}{Sketch of Proof}
\renewcommand{\proofname}{Proof}

\newcommand{\range}{\textrm{rng}}
\newcommand{\domain}{\textrm{dom}}

\newcommand{\Dashv}{\rotatebox[origin=c]{180}{\ensuremath\vDash}}

\newcommand{\questeq}{\stackrel{?}{=}}

\newcommand{\semanticalImply}[2]{#1\vDash{}#2}
\newcommand{\tautology}[1]{\vDash{}#1}

\newcommand{\sat}[3]{\vDash_{\mathfrak{#1}}#2[#3]}
\newcommand{\sentSat}[2]{\vDash_{\mathfrak{#1}}#2}
\newcommand{\unsat}[3]{\nvDash_{\mathfrak{#1}}#2[#3]}
\newcommand{\sentunsat}[2]{\nvDash_{\mathfrak{#1}}#2}
\newcommand{\assignSat}[3]{\vDash_{\mathfrak{#1}}#2\llbracket #3 \rrbracket}

\newcommand{\naturalSet}{\mathbb{N}}
\newcommand{\realSet}{\mathbb{R}}
\newcommand{\naturalStruct}{\mathfrak{N}}
\newcommand{\realStruct}{\mathfrak{R}}

% \newcommand{\iff}{\Leftrightarrow}

\newcommand{\frakA}{\mathfrak{A}}
\newcommand{\frakB}{\mathfrak{B}}


\title{Mathematical Logic}
\author{\textsc{YBiuR}}
\date{A long long time ago in a far far away SJTU}


\begin{document}
\setlength{\parskip}{1em}
\setlength{\parindent}{0em}

\frontmatter
\maketitle
\chapter*{Preface}
\emph{“道可道,非常道。”}

\mainmatter
\tableofcontents

\include{SetTheory.tex}
\chapter{The Informal Notions of Algorithms}

\emph{“形式语言与自动机课程速成。”}

\section{Algorithms}

\begin{definition}[Algorithms (Informal)]
    \label{def:Algorithm}
    An algorithm is a \textbf{finite ordered list} of instructions.
\end{definition}

Possible outcomes of running an algorithm
\begin{itemize}
    \item The algorithm does not halt
    \item The algorithm halts
    \begin{itemize}
        \item In an erroneous state (fails)
        \item Gives valid outputs
    \end{itemize}
\end{itemize}
Cases other than the algorithm giving valid outputs are collectively identified as ``no output''.

\begin{definition}
    An algorithm for \emph{determining membership} in a set $A \subseteq \mathbb{N}$ has an input, and two possible outputs ``yes'' and ``no''. If the algorithm is run on input $n$, it will halt in finite steps with output ``yes'' if $n \in A$ and ``no'' if $n \notin A$.
\end{definition}

\begin{definition}[Effectively Decidable Sets]
    \label{def:EffectivelyDecidableSet}
    Let $A$ be a subset of $\mathbb{N}$. $A$ is \textbf{effectively decidable} if there is an algorithm for determining membership of $A$.
\end{definition}
\begin{theorem}
    If $A$ and $B$ are effectively decidable subsets of $\mathbb{N}$, then $\mathbb{N}\backslash A$, $A \cap B$ and $A \cup B$ are all effectively decidable.
\end{theorem}

Algorithms can have different kinds of outputs and inputs.

\begin{definition}[Diophantine Equations]
    Consider polynomials with integer coefficients (and any number of variables), a \textbf{diophantine equation} is an equation of the form $p=0$, where $p$ is sunch a polynomial. (e.g., $3x^2 + 5xy - 2z^4 +3 = 0$)
\end{definition}

\textbf{Hilbert's 10th Problem.} Is there an algorithm for determining whether or not diophantine equations have integer solutions?

\chapter{Sentential Logic}

\section{Grammar}
\label{sec:Gramma}

\subsection{Symbols}
\label{sub:Symbols}
\begin{itemize}
    \item Logical Symbols
    \begin{itemize}
        \item Sentential Connectives
        \begin{itemize}
            \item $\neg$
            \item $\wedge$
            \item $\vee$
            \item $\rightarrow$
            \item $\leftrightarrow$
        \end{itemize}
        \item Parentheses
    \end{itemize}
    \item Non-logical Symbols: An enumerable set of elements
\end{itemize}

\subsection{Expressions}
\label{sub:Expressions}

\begin{definition}[Expression]
    \label{def:Expression}
    An expression is a finite sequence of symbols.
\end{definition}
\begin{remark}
    The set of all expressions is enumerable.
\end{remark}

We often use Greek alphabets $\alpha,\beta,\dots$ to represent expressions.

\subsection{Well-Formed Formulas}
\label{sub:WellFormedFormulas}

\begin{definition}[Well-Formed Formula]
    \label{def:WFF}
    A \textbf{well-formed formula} (or formula or wff) is an expression built up from sentence symbols by applying some finite times of \emph{formula building operations}
\end{definition}

\begin{definition}[Formula Building Operations]~{}
    \begin{itemize}
        \item $\xi_{\neg}(\alpha) = (\neg\alpha)$
        \item $\xi_{\wedge}(\alpha,\beta) = (\alpha \wedge \beta)$
        \item $\xi_{\vee}(\alpha,\beta) = (\alpha\vee\beta)$
        \item $\xi_{\rightarrow}(\alpha,\beta) = (\alpha\rightarrow\beta)$
        \item $\xi_{\leftrightarrow}(\alpha,\beta) = (\alpha\leftrightarrow\beta)$
    \end{itemize}
\end{definition}
\begin{remark}
    Do NOT omit the parentheses.
\end{remark}

\begin{definition}[Well-Formed Sequences of Expressions]
    \label{def:WellFormedSeqOfExpr}
    A \textbf{well-formed sequence of expressions} is a finite sequence $\alpha_1,\alpha_2,\dots,\alpha_n$ of expressions such that each $\alpha_i$ is either
    \begin{itemize}
        \item A sentence symbol
        \item $(\neg\alpha_j)$ for some $j < i$
        \item $(\alpha_j \wedge \beta_k)$ for some $j,k<i$
        \item $(\alpha_j \vee \beta_k)$ for some $j,k<i$
        \item $(\alpha_j \leftarrow \beta_k)$ for some $j,k<i$
        \item $(\alpha_j \leftrightarrow \beta_k)$ for some $j,k<i$
    \end{itemize}
\end{definition}

\begin{proposition}
    An expression $\alpha$ is a well-formed formula iff there is a well-formed sequence $(\alpha_1,\dots,\alpha_n)$ s.t. $\alpha = \alpha_n$
\end{proposition}

\section{The Induction Principle}
\label{sec:Induction}

Well-formed formulas are a form of inductive definitions with
\begin{itemize}
    \item Basic building blocks
    \item Closing operations
\end{itemize}

\begin{theorem}[The Induction Principle]
    \label{thm:InductionPrinciple}
    Let $S$ be a set of wffs ($S \subseteq W$), if
    \begin{enumerate}
        \item Every sentence symbol is in $S$
        \item For each wff $\alpha$ and $\beta$, if $\alpha$ and $\beta$ are in $S$ then each of the following are in $S$
        \begin{itemize}
            \item $(\neg \alpha)$
            \item $(\alpha \wedge \beta)$
            \item $(\alpha \vee \beta)$
            \item $(\alpha \rightarrow \beta)$
            \item $(\alpha \leftrightarrow \beta)$
        \end{itemize}
    \end{enumerate}
    Then $S$ is the set of \emph{all wffs} ($S = W$).
\end{theorem}

\chapter{First Order Logic}

\emph{"All men are mortal. Socrates is a man. Socrates is mortal."}

\section{Syntax of First-Order Logic}

We start with the symbols of a \textbf{first-order language} $\mathbb{L}$

There are two types of symbols

\begin{itemize}
    \item \textbf{Logical symbols}
    \item Non-logical symbols, a.k.a. \textbf{parameters}
\end{itemize}

\subsection{Symbols}

\subsubsection{Logical Symbols}

In a first-order language $\mathbb{L}$, we have the following symbols

\begin{enumerate}
    \item \textbf{Parentheses}. Two symbols `(' and `)'.
    \item \textbf{Logical connective symbols}. $\to$ and $\neg$
    \item \textbf{Variables}. An enumerable list of symbols $v_1,\dots,v_n,\dots$
    \item \textbf{Identity or Equalily Symbol} $=$ or $\doteq$. It may or may not be present in a particular first-order language
\end{enumerate}

Notice that we do not need $\vee$, $\wedge$. $\leftrightarrow$ because $\{\to, \neg\}$ is complete.

\subsubsection{Parameters}

\begin{enumerate}
    \item \textbf{Universal quantifier}. $\forall$
    \item For each $n>0$, there is a set (possibly empty) of objects called n-ary (or n-place) \textbf{predicate symbols}
    \item For each $n>0$, there is a set (possibly empty) of objects called n-ary (or n-place) \textbf{function symbols}
    \item A set of (possibly empty) of objects \textbf{constant symbols}
\end{enumerate}

\subsubsection{Further Requirements}

\begin{itemize}
    \item $\doteq$ is a 2-ary predicate symbols
    \item There is at least one predicate symbol
    \item The symbols are distinct, and no symbol is equal to a finite sequence of other symbols
\end{itemize}

\subsubsection{Example: Set Theory as First-Order Logic}

The Set Theory can be described by the following language

\begin{itemize}
    \item Equality
    \item Predicate symbols: 2-place $\dot{\in}$
    \item Constant symbols: empty set $\dot{\emptyset}$
    \item Function symbols: None
\end{itemize}

Note that the symbols are (currently) just interpreted as symbols and they do not have semantic meanings.

\begin{remark}
    We do not put restrictions or requirements on number of predicate, function or constant symbols.
\end{remark}

\subsection{Expressions}

An \textbf{expression} in a language $\mathbb{L}$ is a finite sequence of symbols.

\subsubsection{Terms}

\begin{definition}[Term Building Operation]
    \label{def:TermBuildingOperation}
    Given any n-ary function symbol $f$, the term-building operation $\mathcal{F}_f$ is defined by
    \[ \mathcal{F}_f (\sigma_1,\dots,\sigma_n) = f \sigma_1\dots\sigma_n \]
    We call $\sigma_i$ the arguments to $f$
\end{definition}

\begin{definition}[Term]
    \label{def:Term}
    A \textbf{term} is an expression built up from constant symbols and variables by applying some finite times (zero or more times) of term-building operations.
\end{definition}

For example, let $f$ and $g$ be 2-ary and 3-ary function symbols, then $gfc_1c_2v_3c_1$ is a term.

\begin{definition}[Term Sequence]
    \label{def:TermSequence}
    A \textbf{term sequence} is a finite sequence $t_1,\dots,t_n$ of expressions s.t. each $t_i$ is
    \begin{itemize}
        \item either a variable, a constant
        \item or is in the form of $f\sigma_1\dots\sigma_k$ where $f$ is a $f$-ary function and each $\sigma_1,\dots,\sigma_k$ occurs earlier in the sequence
    \end{itemize}
\end{definition}

\begin{proposition}
    An expression $t$ is a term iff there is a term sequence $t_1,\dots,t_n$ such that $t=t_n$
\end{proposition}

\subsubsection{Atomic Formulas}

\begin{definition}[Atomic Formula]
    \label{def:AtomicFormula}
    An expression is an \textbf{atomic formula} if it is of the form $P t_1\dots t_n$ where $t_1,\dots,t_n$ are terms and $P$ is a n-ary predicate symbol.
\end{definition}

\subsection{Well-Formed Formulas}

\begin{definition}[Formula-Building Operations]
    \label{def:FormulaBuildingOperation}
    \begin{itemize}~{}
        \item $\xi_\neg(\alpha) = (\neg \alpha)$
        \item $\xi_\to(\alpha, \beta) = (\alpha\to\beta)$
        \item $\mathcal{Q}_i(\gamma) = \forall v_i\gamma$
    \end{itemize}
\end{definition}

\begin{definition}[Well-Formed Formula]
    A \textbf{well-formed formula} (wff) is an expression built up from atomic formulas by applying some finite times of term-building operations.
\end{definition}

\begin{definition}[Well-Formed Sequence]
    A \textbf{well-formed sequence} is a finite sequence $\alpha_1,\dots,\alpha_n$ of expressions such that each $\alpha_i$ is
    \begin{itemize}
        \item either an atomic formula
        \item or is of the form of $(\neg \beta)$ or $(\beta\to\gamma)$ where $\beta$ and $\gamma$ occur earlier in the list
        \item or is of the form $\forall v_i\beta$ where $\beta$ occurs earlier in the list
    \end{itemize}
\end{definition}

\begin{proposition}
    The expression $\alpha$ is a wff if there is a well-formed sequence $\alpha_1,\dots,\alpha_k$ such that $\alpha = \alpha_k$
\end{proposition}

\subsection{Abbreviations}

\begin{itemize}
    \item $(\alpha\vee\beta)$ abbreviates $((\neg\alpha)\to\beta)$
    \item $(\alpha\wedge\beta)$ abbreviates $(\neg(\alpha \to (\neg\beta)))$
    \item $(\alpha\leftrightarrow\beta)$ abbreviates $(\alpha\to\beta)\wedge(\beta\to\alpha)$
    \item $\exists x\alpha$ abbreviates $(\neg\forall x(\neg\alpha))$
    \item $u\doteq t$ abbreviates $\doteq ut$
    \item $u \dot{\neq} t$ abbreviates $\dot{\neq} ut$
    \item Outer-most parentheses can be omitted
    \item $\neg$, $\forall$, $\exists$ apply to as little as possible
    \item $\wedge$, $\vee$ apply to as little as possible, subject to previous operators
    \item Grouping for repeated connectives is to the right
\end{itemize}

\subsection{Free Occurrence of Variables}

\begin{definition}[Free Occurrence]
    The variable $x$ \textbf{occurs free} in an atomic wff $\varphi$ iff it occurs in $\varphi$.

    $x$ \textbf{occurs free} in $\neg\alpha$ iff $x$ occurs free in $\alpha$.

    $x$ \textbf{occurs free} in $\alpha\to\beta$ iff $x$ occurs free in $\alpha$ or in $\beta$.

    $x$ \textbf{occurs free} in $\forall y \alpha$ iff $x$ occurs free in $\alpha$ and $x \neg y$.
\end{definition}

\begin{definition}[Sentence]
    $\varphi$ is a \textbf{sentence} iff no variable occurs free in $\varphi$.
\end{definition}
\begin{remark}
    Sentences are usually represented by $\sigma$ or $\tau$.
\end{remark}

We provide some examples

\begin{itemize}
    \item $\dot{0} \dot{<} \dot{1}$ does not have any free occurrence. It is a sentence,
    \item $\forall x(x\dot{<}y)$. $y$ occurs free but $x$ does not.
    \item $\forall x(\neg x \dot{<} \dot{0})$. No free occurrence.
    \item $\forall x\forall y (x \dot{<} y \to \exists z x\dot{<}z\wedge z\dot{<}y)$. No free occurrence.
\end{itemize}

\section{Semantics of First-Order Logic}

\subsection{Structures}

\begin{definition}
    Given a first order language $\mathbb{L}$, a \textbf{structure} $\mathfrak{A}$ for $\mathbb{L}$ consists of
    \begin{itemize}
        \item A non-empty set called the \textbf{universe} or \textbf{domain} of the structure, written as $|\mathfrak{A}|$
        \item For each n-ary predicate symbol $P$ of $\mathcal{L}$, other than $\doteq$, an n-ary relation $\mathbb{P}^{\mathfrak{A}}$ on $|\mathfrak{A}|$
        \item $\doteq^{\mathfrak{A}}$ is the identity relation on $|\mathfrak{A}|$. $\doteq^{\mathfrak{A}} = \{(a,b)|a,b\in|\mathfrak{A}, a =b|\}$
        \item For each n-ary function symbol $f$ of $\mathbb{L}$, an n-ary ooperation on the universe, i.e. an n-ary function $f^{\mathfrak{A}}: |\mathfrak{A}|\times\cdots\times|\mathfrak{A}|\mapsto|\mathfrak{A}|$
        \item For each constant symbol $c$ of $\mathbb{L}$, $c^{\mathfrak{A}}\in|\mathfrak{A}|$
    \end{itemize}
\end{definition}

\subsection{Assignments}

Let $\mathfrak{A}$ be a structure for language $\mathbb{L}$. Let $V$ be the set of variables, and $T$ be the set of terms.

\begin{definition}[Assignment Functions]
    An \textbf{assignment} for $\mathfrak{A}$ is a function $s:V\mapsto|\mathfrak{A}|$.
\end{definition}

\begin{definition}[Assignment to Terms]
    An assignment $s$ is extended to a function $\bar{s}:T\mapsto|\mathfrak{A}|$.
    \begin{itemize}
        \item $\bar{s}(v) = s(v)$ if $v$ is a variable
        \item $\bar{s}(c) = c^{\mathfrak{A}}$ if $c$ is a constant
        \item $\bar{s}(ft_1\dots t_n) = f^{\mathfrak{A}}(\bar{s}(t_1),\dots,\bar{s}(t_n))$ if $f$ is a n-ary function symbol and $t_1,\dots,t_n$ are terms
    \end{itemize}
\end{definition}

\subsubsection{Changing the Assignment Function}

Let $s$ be an assignment function, $x$ be a variable and $a\in\mathfrak{A}$. Then $s(x|a)$ is the new assignment, where for each variable $y$

\[s(x|a)(y) = \begin{cases}
    s(y) &\quad \text{if $(y\neq x)$}\\
    a &\quad \text{if $(y=x)$}
\end{cases}\]

This operation actually ``overrides'' the assignment of $s$ to $x$ and makes the assignment to $x$ equal to $a$.

\section{Satisfaction}

Given a first-order language $\mathbb{L}$, let $\mathfrak{A}$ be a structure for $\mathbb{L}$, let $s$ be an assignment for $\mathbb{L}$ and let $\varphi$ be a wff in $\mathbb{L}$. We denote $\mathfrak{A}$ to satisfy $\varphi$ with $s$ by $\vDash_{\mathfrak{A}}\varphi[s]$

Informally, it means ``\emph{The translation of $\varphi$ determined by $\mathfrak{A}$, where a variable $x$ is translated as $s(x)$, is true}''

\subsection{Satisfaction for Atomic Formulas}

\begin{definition}
    Given a language $\mathbb{L}$ and a structure $\mathfrak{A}$, let $s$ be an assignment, let $P$ be a n-ary predicate,
    \[ \vDash_{\mathfrak{A}} Pt_1\dots t_n[s] \Leftrightarrow (\bar{s}(t_1),\dots,\bar{s}(t_n)) \in P^{\mathfrak{A}} \]
    \[ \vDash_{\mathfrak{A}} \doteq t_1t_2[s] \Leftrightarrow \bar{s}(t_1) = \bar{s}(t_2) \]
\end{definition}

\subsection{Satisfaction for WFF}

\begin{definition}
    Suppose $\vDash_{\mathfrak{A}}\alpha[s]$ and $\vDash_{\mathfrak{A}}\beta[s]$ have been defined, then
    \begin{itemize}
        \item $\vDash_{\mathfrak{A}} \neg\alpha[s]$ iff not $\vDash_{\mathfrak{A}}\alpha[s]$
        \item $\vDash_{\mathfrak{A}}\alpha\to\beta[s]$ iff $\vDash_{\mathfrak{A}}\alpha[s]\Longrightarrow\vDash_{\mathfrak{A}}\beta[s]$
        \item $\vDash_{\mathfrak{A}}\forall x \alpha[s]$ iff $\forall a\in|\mathfrak{A}|$, $\vDash_{\mathfrak{A}}\alpha[s(x|a)]$
    \end{itemize}
\end{definition}

If $\vDash_{\mathfrak{A}}\varphi[s]$, we say \emph{$\mathfrak{A}$ satisfies $\varphi$ with $s$}, or \emph{$s$ satisfies $\varphi$ in the structure $\mathfrak{A}$}

\subsubsection{Satisfaction Depends Only on Variables that Occur Free}

\begin{lemma}
    \label{lem:LemmaForFreeOccurrenceThm}
    Let $\mathfrak{A}$ be a structure for $\mathbb{L}$, $s_1, s_2$ be two assignment for $\mathfrak{A}$ and $\varphi$ be a term of $\mathbb{L}$.

    If $s_1(x)=s_2(x)$ for every $x$ that occurs in $t$, then
    \[ \bar{s}_1(t) = \bar{s}_2(t) \]
\end{lemma}
\begin{proof}
    Proof by induction on $t$.
    \begin{itemize}
        \item[Base] If $t=c$ is a constant. It is straightforward that $\bar{s}_1(t) = \bar{s}_2(t) = c$. If $t=x$ is a variable, then by assumption we know that $\bar{s}_1(x) = s_1(x) = s_2(x) = \bar{s}_2(x)$. So we are done.
        \item[Induction] Consider a term $t=ft_1\dots t_n$. By inductive hypothesis we know that $\forall i$, $\bar{s}_1(t_i) = \bar{s}_2(t_i)$. $\bar{s}_1(t) = f^{\mathfrak{A}}(\bar{s}_1(t), \dots, \bar{s}_1(t) = f^{\mathfrak{A}}(\bar{s}_2(t), \dots, \bar{s}_2(t)) = \bar{s}_2(t)$.
    \end{itemize}
\end{proof}

\begin{theorem}
    Let $\mathfrak{A}$ be a structure for $\mathbb{L}$, $s_1, s_2$ be two assignment for $\mathfrak{A}$ and $\varphi$ be a wff of $\mathbb{L}$.

    If $s_1(x)=s_2(x)$ for every $x$ that occurs free in $\varphi$, then
    \[ \vDash_{\mathfrak{A}}\varphi[s_1] \Leftrightarrow \vDash_{\mathfrak{A}}\varphi[s_2] \]
\end{theorem}

\begin{proof}
    Prove by induction on $\varphi$.
    \begin{itemize}
        \item[Base] If $\varphi$ is an atomic formula $Pt_1\dots t_n$.
        \[ \vDash_{\mathfrak{A}} \varphi [s_1] \Leftrightarrow P^{\mathfrak{A}}(\bar{s}_1(t_1),\dots,\bar{s}_1(t_n)) \]
        \[ \vDash_{\mathfrak{A}} \varphi [s_2] \Leftrightarrow P^{\mathfrak{A}}(\bar{s}_2(t_1),\dots,\bar{s}_2(t_n)) \]

        We need to prove that the two RHSes are equivalent. By Lemma~\ref{lem:LemmaForFreeOccurrenceThm} we know that all the terms are equal under $s_1$ and $s_2$, and therefore $\vDash_{\mathfrak{A}} \varphi [s_1] = \vDash_{\mathfrak{A}} \varphi [s_2]$.

        \item[Induction] Consider $\varphi=\neg\alpha$.
        \[ \vDash_{\mathfrak{A}}(\neg\alpha)[s_1] \Leftrightarrow \nvDash_{\mathfrak{A}}\alpha[s_1] \Leftrightarrow \nvDash_{\mathfrak{A}}\alpha[s_2] \Leftrightarrow \vDash_{\mathfrak{A}}(\neg\alpha)[s_2] \]

        The case $\varphi = \alpha\to\beta$ is similar.

        Consider the case $\forall x \alpha$. We want to prove
        \[ \sat{A}{\forall x\alpha}{s_1} \Leftrightarrow \sat{A}{\forall x\alpha}{s_2} \]
        which is equivalent to
        \[ \forall a\in|\mathfrak{A}|\sat{A}{\alpha}{s_1(x|a)} \Leftrightarrow \forall a \in |\frakA| \sat{A}{\alpha}{s_2(x|a)} \]

        We only need to prove that
        \[ \forall y \text{occurring free in $\alpha$}, s_1(x|a)(y) = s_2(x|a)(y) \]

        If $y\neq x$, then $y$ is still occurring free in $\alpha$, and by inductive hypothesis they should equal. If $y=x$, then both sides are $a$. So we are done.
    \end{itemize}
\end{proof}
\begin{remark}
    This theorem is somewhat similar to the theorem in sentential logic, which states that we only need to consider sentence symbols. Similarly, in first-order logic, we only need to consider variables that occur free.
\end{remark}

\begin{definition}
    Let $\varphi$ be a wff s.t. all variables occurring free in $\varphi$ are included amoing $v_1,\dots,v_k$. Given $a_1,\dots,a_k\in\frakA$.

    \[ \assignSat{A}{\varphi}{a_1,\dots,a_k} \]

    means that $\sat{A}{\varphi}{s}$ for some $s:V\mapsto|\frakA|$ s.t. $s(v_i) = a_i$
\end{definition}

\begin{corollary}
    If $\sigma$ is a sentence then
    \begin{itemize}
        \item either $\sat{A}{\sigma}{s}$ for every assignment $s$. We say $\sigma$ is true in $\frakA$.
        \item or $\unsat{A}{\sigma}{s}$ for every assignment $s$. We say $\sigma$ is false in $\frakA$
    \end{itemize}
\end{corollary}

Therefore a sentence does not depend on $s$, and we can simply write $\sentSat{A}{\sigma}$ or $\sentunsat{A}{\sigma}$.

\subsection{Elementary Equivalence}

\begin{definition}[Elementary Equivalence]
    \label{def:ElementaryEquivalence}
    Let $\mathfrak{A}$ and $\mathfrak{B}$ be structures for the same language $\mathbb{L}$. $\mathfrak{A}$ and $\mathfrak{B}$ are \textbf{elementarily equivalent} ($\mathfrak{A} \equiv \mathfrak{B}$) if for every \emph{sentence} of $\mathbb{L}$
    \[ \sentSat{A}{\sigma} \iff \sentSat{B}{\sigma} \]
\end{definition}

\begin{remark}
    Elementary equivalence only take into consideration sentences.
\end{remark}

\begin{proposition}
    $\mathfrak{Q}$ and $\mathfrak{R}$ are elementary equivalent. But this is beyond the scope of the course.
\end{proposition}

\section{Models}

\subsection{Models}

\begin{definition}[Model]
    $\mathfrak{A}$ is a \textbf{model} of the sentence $\sigma$ if $\sentSat{A}{\sigma}$, i.e. if $\sigma$ is true in $\mathfrak{A}$. $\mathfrak{A}$ is a \textbf{model} of a set $\Sigma$ of sentences if $\mathfrak{A}$ is a model for every sentence in $\Sigma$. i.e. every sentence in $\Sigma$ is true in $\mathfrak{A}$.
\end{definition}

For example, consider a first-order language $\mathbb{L}$, with 2-ary predicate symbols $\dot{P}$ and $\doteq$. Given a structure $\mathfrak{A}$ of $\mathbb{L}$,

\begin{itemize}
    \item $\mathfrak{A}$ is a model of $\forall x \forall y x \doteq y$
    \begin{itemize}
        \item $\Leftrightarrow \sat{A}{x\doteq y}{s(x|a)(y|b)}$ for every $a,b \in |\mathfrak{A}|$
        \item $\Leftrightarrow$ $a = b$ for every $a,b \in|\mathfrak{A}|$
        \item $\Leftrightarrow$ $|\mathfrak{A}|$ contains only one element.
        \item Note that $|\frakA|$ cannot be empty because the universe of a structure must be non-empty.
    \end{itemize}
    \item $\mathfrak{A}$ is a model of $\forall x \forall y \dot{P}xy$
    \begin{itemize}
        \item iff $P^\mathfrak{A}(a,b)$ for all $a,b\in|\mathfrak{A}|$
        \item iff $\dot{P}^{\frakA} = |\frakA| \times |\frakA|$
    \end{itemize}
    \item $\mathfrak{A}$ is a model of $\forall x \forall y \neg\dot{P}xy$
    \begin{itemize}
        \item iff $P^{\frakA}(a,b)$ does not hold for all $a,b\in|\frakA|$
        \item iff $\dot{P}^{\frakA} = \emptyset$
    \end{itemize}
    \item $\frakA$ is a model of $\forall x \exists y \dot{P}xy$
    \begin{itemize}
        \item iff forall $a\in|\frakA|$, there is a $b\in|\frakA|$ s.t. $\dot{P}^{\frakA}(a,b)$
        \item iff the domain of $\dot{P}^{\frakA}$ is $|\frakA|$
        \item Conversely, if we want the range of $\dot{P}^{\frakA}$ is $|\frakA|$, we can write $\forall y \exists x \dot{P}xy$
    \end{itemize}
\end{itemize}

\subsection{Linearly Ordered Structures}

\begin{definition}[Trichotomy]
    Let $R$ be a binary relation, $R$ satisfies \textbf{trichotomy} if exactly one of the following is true
    \[ (a,b) \in R \quad (b,a) \in R \quad a = b \]
\end{definition}

\begin{definition}[Linear Ordering]
    A binary relation $R$ is a \textbf{linear ordering} on $A$ if $R$ is transitive and satisfies trichotomy on $A$.
\end{definition}

\begin{definition}
    Let $\mathbb{L}$ be the language with a binary relation symbol $\dot{R}$ and $\doteq$ (and no other symbols). Let $\frakA = (A,R)$, i.e. $A=|\frakA|$ and $R = \dot{R}^{\frakA}$.
    \begin{itemize}
        \item $\frakA$ is transitive if $R$ is transitive
        \item $\frakA$ is a linearly ordered structure if $R$ is a linear ordering on $\frakA$
    \end{itemize}
\end{definition}

For a set of structures with some certain properties, the set can be defined by a sentence.

Let $\frakA = (A,R)$,
\begin{itemize}
    \item $\frakA$ is transitive iff $\sentSat{A}{\sigma}$, where $\sigma=\forall x \forall y \forall z \dot{R}xy \to \dot{R}yz \to \dot{R}xz$. Therefore $\sigma$ defines the set of all transitive structures
    \item $\frakA$ is linearly ordered iff $\sentSat{A}{\sigma}$ where
    \begin{itemize}
        \item $\sigma_1 = \forall x \forall y \forall z \dot{R}xy \to \dot{R}yz \to \dot{R}xz$
        \item $\sigma_2 = \forall x \forall y (\dot{R}xy \vee x=y \vee \dot{R}yx)$
        \item $\sigma_3 = \forall x \forall y (\dot{R}xy \to \neg\dot{R}yx)$
        \item $\sigma = \sigma_1 \wedge \sigma_2 \wedge \sigma_3$
    \end{itemize}
    Therefore $\sigma$ defines the set of all linearly ordered structures
    \item $\domain{R} = A$ iff $\sentSat{A}{\sigma}$ where $\sigma = \forall x \exists y \dot{R}xy$
    \item $\range{R} = A$ iff $\sentSat{A}{\sigma}$ where $\sigma = \forall y \exists x \dot{R}xy$
    \item $R$ is a (total) function iff $\sentSat{A}{\sigma}$ where
    \begin{itemize}
        \item $\sigma_4 =  \forall x \exists y \dot{R}xy$
        \item $\sigma_5 = \forall x \forall y \forall z \dot{R}xy \to \dot{R}xz \to x \doteq z$
        \item $\sigma = \sigma_4 \wedge \sigma_5$
    \end{itemize}
\end{itemize}

\subsection{Elementary Class}

\begin{definition}[Elementary Class]
    A set of strucutures $\mathcal{K}$ is an \textbf{elementary class} if there exists a sentence $\sigma$ s.t.
    \[ \mathcal{K} = \{ \frakA | \frakA \text{ is a model of $\sigma$} \} \]
    i.e.
    \[ \mathcal{K} = \{ \frakA | \sentSat{A}{\sigma} \} \]
\end{definition}

For example, the set of all graphs is an elementary class.

Let $\mathbb{L}$ be the language with a binary predicate symbol $\dot{E}$ and $\doteq$, and no other symbols. A structure $\mathfrak{G} = (G, E)$ for $\mathbb{L}$ is a graph if
\begin{itemize}
    \item $E$ is symmetric (undirected graph)
    \item For every $a \in G$, $(a, a) \notin E$ (no self-loops)
\end{itemize}

To show that the set of graphs defined above is an elementary class, we need to show that there is a sentence $\sigma$ s.t.
\[ \text{$G$ is a graph} \Leftrightarrow \sentSat{G}{\sigma} \]

Therefore $\sigma$ should be able to represent the symmetric and non-reflexible properties.

\[\sigma = (\forall x \forall y (\dot{E}xy \to \dot{E}yx)) \wedge (\forall x(\neg \dot{E}xx))\]

\subsubsection{In the Wider Sense}

\begin{definition}[Elementary Class in the Wider Sense]
    A set of structures $\mathcal{K}$ is an \textbf{elementary class in the wider sense} ($EC_\Delta$) if there is a set $\Sigma$ of sentences s.t.
    \[ \mathcal{K} = \{ \frakA | \frakA \text{ is a model of } \Sigma \} \]
    i.e.
    \[ \mathcal{K} = \{ \frakA | \sentSat{A}{\sigma} \text{ for every $\sigma \in \Sigma$} \} \]
\end{definition}

For example, we can use a set $\Sigma$ to define the set of strucutures whose univerise is infinite.

\[ \Sigma = \{ \lambda_2, \lambda_3, \dots, \lambda_n, \dots \} \]

where $\lambda_i$ denotes ``At least $i$ elements exists in $|\frakA|$'', for example $\lambda_2 = \exists x \exists y x \neq y$.

However, it is hard to decide whether there is a single sentence $\sigma$ s.t. $\frakA$ is a model of $\sigma$ iff $|\frakA|$ is infinite.

\section{Logical Implications and Satisfiability}

\subsection{Logical Implications}

\begin{definition}
    Let $\Gamma$ be a set of wffs and $\varphi$ be a wff. $\Gamma$ \textbf{logically implies} $\varphi$, written as
    \[ \Gamma \vDash \varphi \]
    if for every structure $\frakA$ and every assignment $s$, if $\frakA$ satisfies $\Gamma$ with $s$, then $\frakA$ satisfies $\varphi$ with $s$
\end{definition}

\begin{theorem}
    For a set of sentences $\Sigma$, and a sentence $\sigma$, $\Sigma\vDash\sigma$ iff for every model $\frakA$ of $\Sigma$, $\frakA$ is a model of $\sigma$.
\end{theorem}

As before, we denote $\{\alpha\}\vDash\beta$ by $\alpha\vDash\beta$

\subsection{Logical Equivalence}

\begin{definition}[Logical Equivalence]
    $\alpha$ and $\beta$ are logically equivalent, written as $\alpha\vDash\Dashv\beta$ if $\alpha\vDash\beta$ and $\beta\vDash\alpha$
\end{definition}

\subsection{Valid Formulas}

\begin{definition}[Valid WFFs]
    Let $\varphi$ be a wff in the language $\mathbb{L}$. $\varphi$ is \textbf{valid} if $\semanticalImply{\emptyset}{\varphi}$, written as $\tautology{\varphi}$.
\end{definition}

For example,

\begin{itemize}
    \item $x \doteq x$ is valid
    \item $\exists x \doteq x$ is valid
    \item $\forall x \exists y x\dot{\neq}y$ is not valid
    \item $\dot{P}x \vee \neg \dot{P}x$ is valid
    \item $\exists x(\dot{P}x \to \forall x \dot{P}x)$ is valid
\end{itemize}

We detail the proof of the last example

\begin{itemize}
    \item $\sat{A}{\exists x (Px\to\forall x Px)}{s}$
    \item[$\Leftrightarrow$] There is some $a \in |\frakA|$ s.t. $\sat{A}{Px \to \forall x Px}{s(x|a)}$
    \item[$\Leftrightarrow$] $\sat{A}{\dot{P}(x)}{s(x|a)} \Rightarrow \sat{A}{\forall x Px}{s(x|a)}$
    \item[$\Leftrightarrow$] $\sat{A}{\dot{P}(x)}{s(x|a)} \Rightarrow \sat{A}{\forall x Px}{s(x|a)}$
    \item[$\Leftrightarrow$] There is some $a$ s.t. if $a \in \dot{P}^{\frakA}$, then for every $b \in |\frakA|$, $b\in\dot{P}^{\frakA}$
    \item If there is some $a$ s.t. $a \notin \dot{P}^{\frakA}$, then the consequence holds trivially
    \item If there is no $a$ s.t. $a \notin \dot{P}^{\frakA}$, this means that for every $a \in |\frakA|$, $a\in\dot{P}^{\frakA}$. This is exactly the consequence so we are done
\end{itemize}

\subsection{Satisfiability}

\begin{definition}[Satisfiability]
    \begin{itemize}
        \item The wff $\varphi$ is \textbf{satisfiable} if there is some structure $\frakA$ and some assignment $s$ s.t. $\sat{A}{\varphi}{s}$.
        \item The set of wffs $\Gamma$ is \textbf{satisfiable} if there is some structure $\frakA$ and some assignment $s$ s.t. $\sat{A}{\varphi}{s}$ for every $\varphi$ in $\Gamma$.
    \end{itemize}
\end{definition}

\begin{theorem}
    $\varphi$ is not satisfiable iff $\neg\varphi$ is valid
\end{theorem}

\section{Definability}

\begin{definition}[Relations Defined by WFFs]
    Let $\frakA$ be a structure, and $\varphi$ be a wff, and $n$ be such that the variables occurring free in $\varphi$ are included among $v_1,\dots,v_n$

    The n-ary relation \textbf{defined by $\varphi$ in $\frakA$} is
    \[ \{ (a_1,\dots,a_n) | \assignSat{A}{\varphi}{a_1,\dots,a_n}\} \]
\end{definition}

Let $\mathfrak{N} = (\mathbb{N},\le, +, 1)$, the 2-ary relation $\{(a,b)|a < b\}$ is defined by
\[ v_1 \dot{+}\dot{1}\doteq v_2 \]

To show this, we will show that $(a,b) \in R \iff \assignSat{N}{\varphi}{a,b}$.

If $(a,b) \in R$, $\assignSat{N}{\varphi}{a,b} \iff a+1 \le b$

Conversely, if $\assignSat{N}{\varphi}{a,b}$, it's equivalent to $a+1 \le b$, which mathematically implies that $a < b$

\begin{definition}[Definability]
    The relation $R$ is \textbf{definable in the structure $\frakA$} if there is some wff $\varphi$ that defines it in $\frakA$.
\end{definition}

We show some examples for definability of functions. Let $\mathfrak{N} = (\mathbb{N}, <, +, x, 0, 1)$.

\begin{itemize}
    \item $v_1 \dot{+} v_2 \doteq v_3$ defines $\{ (a,b,c) | a + b = c\}$ which is the same as function $f$, where $f(a,b) = a + b$.
\end{itemize}

\subsection{Definable Relations}

\begin{definition}[Relations Defined by WFFs]
    Let
    \begin{itemize}
        \item $\frakA$ be a structure
        \item $\varphi$ be a wff and $n$ be such that the variables occurring free in $\varphi$ are included among $v_1,\dots,v_n$
    \end{itemize}

    The $n$-ary relation defined by $\varphi$ in $\frakA$ is
    \[ \{ (a_1,\dots,a_n) | \assignSat{A}{\varphi}{a_1,\dots,a_n} \} \]
\end{definition}

For example,

\begin{itemize}
    \item Let $\mathfrak{R} = (\mathbb{R},<,+,\times,0,1)$. The $1$-ary relation $\{ a \in \mathbb{R} | 0 \le a \}$ is defined by
    \[ \exists v_2, v_1 \doteq v_1 \times v_2 \]
    \item Let $\mathfrak{R} = (\mathbb{R},<,+,\times,0,1)$。 The $2$-ary relation $\{ (a,b) | a < b \}$ is defined by
    \[ \exists v_3 (v_1\dot{+}(\dot{1}\dot{+}v_3)\doteq v_2) \]
\end{itemize}

\begin{definition}[Definable Relations]
    The relation $R$ is \textbf{definable in structure} $\frakA$ if there is some wff that defines it in $\frakA$

    Let $f$ be a n-ary function $f$ whose domain is a subset $|\frakA| \times \dot \times |\frakA|$ and whose range is a subset of $|\frakA|$, $f$ is definable in $\frakA$ if the $(n+1)$-ary relation
    \[ \{ (a_1,\dots,a_n, b) | f(a_1,\dots,a_n) = b \} \]
    is definable in $\frakA$.
\end{definition}

Consider $\mathfrak{N} = (\mathbb{N},<,+,\times,0,1)$.
\begin{itemize}
    \item $v_1 + v_2 = v_3$ defines $\{ (a,b,c) | a + b = c \}$, which is the same as function $f(a,b) = a + b$
    \item $v_1 + v_3 = v_2$ defines $\{ (a,b,c) | a+c = b \}$, which is the same as function $f(a,b)$
    \[ f(a,b) = \begin{cases}
        b - a &\quad a \le b\\
        Undefined &\quad o.w.
    \end{cases} \]
\end{itemize}

\begin{lemma}
    Given a structure $\frakA$, the set of definable relations is \emph{enumerable}
\end{lemma}
\begin{lemma}
    Not every subset of $\mathbb{N}$ is definable.
\end{lemma}

The proof of the two lemmas are similar. Note that the set of wffs is enumerable, and every wff may define only one relation. And the set of all subsets of $\mathbb{N}$ is uncountable.

We now move from $\mathbb{N}$ to $\mathbb{R}$ and consider a more generall case. Consider whether the following subsets of $\mathbb{R}$ are definable in $\mathfrak{R} = (\mathbb{R}, <)$

\begin{itemize}
    \item $\emptyset$. Yes.
    \item $\mathbb{N}$. Yes.
    \item Anything else?
\end{itemize}

\section{Homomorphisms}

\begin{definition}[Homomorphism]
    Let $\mathfrak{A}$ and $\mathfrak{B}$ be structures for $\mathbb{L}$. A \textbf{homomorphism} from $\frakA$ to $\mathfrak{B}$ is a function $h:|\frakA| \mapsto |\mathfrak{B}|$ s.t.
    \begin{itemize}
        \item For every n-ary predicate symbol $R$, other than $\doteq$, and $a_1,\dots,a_n \in |\frakA|$,
        \[ (a_1,\dots,a_n) \in R^{\frakA} \iff (h(a_1),\dots,h(a_n))\in R^{\mathfrak{B}} \]
        \item For every n-ary function symbol $f$, and $a_1,\dots,a_n \in |\frakA|$,
        \[ h(f^\frakA(a_1,\dots,a_n)) = f^\mathfrak{B}(h(a_1),\dots,h(a_n)) \]
        \item For every constant symbol $c$
        \[ h(c^\frakA) = c^\mathfrak{B} \]
    \end{itemize}
\end{definition}

\begin{definition}[Onto of Homomorphism]
    $h$ is a homomorphism of $\frakA$ \textbf{onto} $\mathfrak{B}$ if $h$ is a homomorphism from $\mathfrak{A}$ to $\mathfrak{B}$ and $h$ maps $\frakA$ onto $\mathfrak{B}$.
\end{definition}

\begin{definition}[Isomorphism]
    A homomorphism $h$ from $\mathfrak{A}$ to $\mathfrak{B}$ is an \textbf{isomorphism} if $h$ is one-to-one
\end{definition}

\begin{definition}[Isomorphic]
    The structures $\frakA$ and $\mathfrak{B}$ are \textbf{isomorphic}, denoted by $\frakA \cong \mathfrak{B}$ if there is some \emph{isomorphism} of $\frakA$ \emph{onto} $\mathfrak{B}$. (One-to-one correspondence)
\end{definition}

\begin{definition}[Automorphism]
    An \textbf{automorphism} of $\frakA$ is an isomorphism of $\frakA$ onto $\frakA$
\end{definition}

\subsubsection{Examples of Homomorphism}

For example, let $\frakA = (\naturalSet, <^\naturalSet, +^\naturalSet)$, $\mathfrak{B} = (\mathbb{E}, <^\mathbb{E}, +^\mathbb{E})$, where $\mathbb{E}$ is the set of even non-negative integers. Then we claim that $h(n)=2n$ is an isomorphism of $\mathfrak{A}$ onto $\mathfrak{B}$

To do this, we show that 1) $h$ is a homomorphism; 2) $h$ is one-to-one; 3) $h$ is onto

\begin{itemize}
    \item[$\dot{<}$] $(a,b) \in <^\mathbb{N} \iff (h(a), h(b)) = (2a,2b) \in <^\mathbb{E}$
    \item[$\dot{+}$] $h(a +^\mathbb{N} b) = (h(a) +^\mathbb{E} h(b)) = (2a +^\mathbb{E} 2b)$
\end{itemize}

However, let $\mathfrak{C} = (\mathbb{O}, <^\mathbb{O}, +^\mathbb{O})$, where $\mathbb{O}$ is the set of all odd non-negative integers, then there is no isomorphism of $\frakA$ onto $\mathfrak{C}$. Infact $\mathfrak{C}$ is not even a valid strucutre because $+^\mathbb{O}$ is not closed.

\subsubsection{Automorphism of $\mathfrak{R}=(\realSet, <)$}

Consider which of the following $h$ are automorphisms of $\mathfrak{R}$. Note that to show this, we need to show that 1) $h$ is a homomorphism; 2) $h$ is one-to-one; 3) $h$ is onto; and 4) $h$ maps $\realSet$ to $\realSet$.

\begin{itemize}
    \item The identity function. Obviously yes.
    \item $h(a) = a + 3$. Yes.
    \item $h(a) = a - 4$. Yes.
    \item $h(a) = 2a$. Yes.
    \item $h(a) = -a$. Yes.
    \item $h(a) = ka + l$. Yes if $k>0$.
    \item $h(a) = a^3$. Yes.
    \item $h(a) = a^2$. No.
\end{itemize}

\subsubsection{Automorphism of $\mathfrak{N} = (\naturalSet, <)$}

Obviously the identity function is an automorphism. We consider other cases.

If we map $0$ to any $n>0$, i.e. $h(0) = n > 0$. Since $h$ is onto, there exists some $m > 0$ s.t. $h(m) = 0$. And here comes a problem
\[ m > 0 \iff h(m) = 0 > h(0) = n > 0 \]
Therefore $0$ can only be mapped to $0$, i.e. $h(0) = 0$

Similarly, $h(1)$ can only be mapped to $1$, and for each $n$, $h(n) = n$. Therefore the identity function $h(n) = n$ is the \emph{only} automorphism of $\naturalStruct$.

\subsection{Substructures}

We now consider a special kind of isomorphism

\begin{definition}[Substructures]
    Let $\frakA = (A,\dots)$ and $\frakB = (B,\dots)$ be structures for $\mathbb{L}$. $\frakA$ is a \textbf{substructure} of $\frakB$, denoted by $\frakA \subseteq \frakB$ if
    \begin{itemize}
        \item $A \subseteq B$
        \item For every $k$-ary predicate symbol,
        \[ P^\frakA = P^\frakB \cap A^k \]
        Note that this is to guarantee that $(a_1,\dots,a_k)\in P^\frakB \Longrightarrow (a_1,\dots,a_k)\in P^\frakA$
        \item For every $k$-ary function $f$ and every $k$-tuple of $A$
        \[ f^\frakA(a_1,\dots,a_k) = f^\frakB(a_1,\dots,a_k) \]
        \item For every constant $c$
        \[ c^\frakA = c^\frakB \]
    \end{itemize}
\end{definition}
\begin{remark}
    Substructures are defined under identity map $h(x)=x$. The identity map is an isomorphism of $\frakA$ into $\frakB$ iff
    \begin{enumerate}
        \item For each predicate $P$, $P^\frakA$ is the restriction of $P^\frakB$ to $A$
        \item For each function $f$, $f^\frakA$ is the restriction of $f^\frakB$ to $A$
        \item $c^\frakA = c^\frakB$.
    \end{enumerate}
    If these conditions are met, then $\frakA$ is called a \textbf{substructure} of $\frakB$.
\end{remark}

\subsection{Homomorphism Theorem}

\begin{lemma}
    \label{lem:HomomorphismLemma}
    Let $\frakA$ and $\frakB$ be structures for the language $\mathbb{L}$. Let $h$ be a homomorphism from $\frakA$ to $\frakB$, and $s:V\to|\frakA|$ be an assignment for $\frakA$. Then for every term $t$ of $\mathbb{L}$
    \[ h(\bar{s}(t)) = \overline{h\circ s}(t) \]
\end{lemma}
\begin{proof}
    Prove by induction.
    \begin{itemize}
        \item[] \textbf{Base Case.} If $t=c$, then
        \[ h(\bar{s}(c)) = h(c^\frakA) = c^\frakB = \overline{h\circ s(c)} \]
        If $t=v$, then
        \[ h(\bar{s}(x)) = h(s(x)) = h \circ s(x) \]
        \item[] \textbf{Inductive Case.} If $t = ft_1,\dots,t_n$,
        \[ h(\bar{s}(t)) = h(f^\frakA(\bar{s}(t_1),\dots,\bar{s}(t_n))) = f^\frakB(\bar{s}(t_1),\dots,\bar{s}(t_n)) \]
        \[ \overline{h\circ s} = f^\frakB(\overline{h\circ s}(t_1),\dots,\overline{h\circ s}(t_n)) \]
        From inductive hypothesis we know $\overline{h\circ s}(t_k) = h(\bar{s}(t_k))$, and we are done.
    \end{itemize}
\end{proof}

\begin{theorem}[Homomorphism Theorem]
    \label{thm:HomomorphismTheorem}
    Let $h$ be a homomorphism form $\frakA$ to $\frakB$ and $s$ be an assignment function for $\frakA$. The statement
    \[ \sat{A}{\varphi}{s} \iff \sat{B}{\varphi}{h \circ s} \]
    \begin{itemize}
        \item is true for every quantifier-free wff $\varphi$ not containing $\doteq$
        \item is true for every quantifier-free wff $\varphi$ if $h$ is one-to-one
        \item is true for every wff $\varphi$ wff $\varphi$ not containing $\doteq$ if $h$ is onto
        \item is true for every wff $\varphi$ if $h$ is an isomorphism of $\frakA$ onto $\frakB$ ($\frakA \cong \frakB$)
    \end{itemize}
\end{theorem}
\begin{proof}
    Prove by induction on $\varphi$.
    \begin{itemize}
        \item[] \textbf{Base Case.} Let $\varphi = Pt_1\dots t_n$.
         Since $h$ is a homomorphism, by definition we have
         \[ (\bar{s}(t_1),\dots,\bar{s}(t_n))\in P^\frakA \iff (h(\bar{s}(t_1)),\dots, h(\bar{s}(t_n))) \in P^\frakB \]
         Further, by Lemma~\ref{lem:HomomorphismLemma},
         \[ (\overline{h\circ s}(t_1),\dots,\overline{h\circ s}(t_2)) = (h(\bar{s}(t_1)),\dots, h(\bar{s}(t_n))) \in P^\frakB \]
         So we are done.
         \item[] \textbf{Inductive Case.}\begin{itemize}
             \item If $\varphi = \neg \alpha$, by induction hypothesis
             \[ \sat{A}{\alpha}{s} \iff \sat{B}{\alpha}{h \circ s} \]
             We can negate both sides
             \[ \unsat{A}{\alpha}{s} \iff \unsat{B}{\alpha}{h \circ s} \]
             Therefore,
             \[ \sat{A}{\neg\alpha}{s} \iff \sat{B}{\neg\alpha}{h\circ s} \]
             \item If $\varphi = \alpha \to \beta$.
             \[ \sat{A}{\alpha\to\beta}{s} \iff \sat{B}{\alpha\to\beta}{h\circ s} \]
             is equivalent to
             \[ \text{If} \sat{A}{\alpha}{s} \text{then} \sat{A}{\beta}{s} \iff \text{If} \sat{B}{\alpha}{h \circ s} \text{then} \sat{B}{\beta}{s} \]
             By induction hypothesis, we are done.
         \end{itemize}
    \end{itemize}
    Until now we have proved (a). We now further consider $\doteq$ and $\forall$

    We start from $\doteq$, which is a base case,
    If $\varphi = t_1\doteq t_2$
        \[ \sat{A}{t_1\doteq t_2}{s} \iff \sat{B}{t_1\doteq t_2}{h\circ s} \]
        \[ \bar{s}(t_1) = \bar{s}(t_2) \iff \overline{h\circ s}(t_1) = \overline{h\circ s}(t_2) \iff h(\bar{s}(t_1)) = h(\bar{s}(t_2)) \]
        $\Leftrightarrow$ always holds. However, $\Leftarrow$ requires an additional condition that $h$ is one-to-one.

    Finally, consider $\varphi=\forall x\alpha$ in the inductive case.
    \[ \sat{A}{\forall x \alpha}{s} \iff \sat{B}{\forall x \alpha}{h\circ s} \]
    \[ \text{For any $a\in|\frakA|$,} \sat{A}{\alpha}{s(x|a)} \iff \text{For any $b\in|\frakB|$,}\sat{B}{\alpha}{(h\circ s)(x|b)} \]
    Assume LHS, we prove RHS. Note that since $a$ and $b$ are arbitrary, we cannot relate them without additional conditions. Therefore to prove this we would require $h$ to be \emph{onto}. Then there is some $a'\in|\frakA|$ s.t. $h(a')=b$.
    \[ \sat{B}{\alpha}{(h\circ s)(x|h(a'))} \iff \sat{B}{\alpha}{h\circ(s(x|a'))} \]
    By LHS, we have $\sat{A}{\alpha}{s(x|a')}$, and by hypothesis, we have $\sat{B}{\alpha}{h\circ (s(x|a'))}$, so we are done.

    Assume RHS, we prove LHS. Let $b=h(a)$,
    \[ \sat{B}{\alpha}{(h\circ s)(x|h(a))} \iff \sat{B}{\alpha}{h\circ(s(x|a))} \]
    By hypothesis, we have $\sat{A}{\alpha}{s(x|a)}$. Done.
\end{proof}

\begin{corollary}
    \label{cor:HomomorphismToElementaryEquiv}
    If $\frakA \cong \frakB$, then $\frakA \equiv \frakB$. Recall that $\equiv$ means Elementary Equivalence (Def~\ref{def:ElementaryEquivalence}).
\end{corollary}

Corollary~\ref{cor:HomomorphismToElementaryEquiv} follows immediately from Homomorphism Theorem~\ref{thm:HomomorphismTheorem} because sentences do not care about assignments.

But the converse is not true. Take $\mathfrak{R} = (\realSet, <)$ and $\mathfrak{Q}=(\mathbb{Q}, <)$ as a counter example. We have claimed that they are elementary equivalent (though the proof is beyond the scope).

\begin{corollary}[Automorphism Theorem]
    \label{cor:AutomorphismTheorem}
    Let $h$ be an automorphism of $\frakA$. Let $R$ be an n-ary relation on $|\frakA|$ that is definable in $\frakA$. For every n-tuple $(a_1,\dots,a_n)$ of elements of $\frakA$,
    \[ (a_1,\dots,a_n) \in R \iff (h(a_1),\dots,h(a_n)) \in R \]
\end{corollary}
\begin{proof}
    Since $R$ is definable in $\frakA$,
    \begin{itemize}
        \item[] $(a_1,\dots,a_n) \in R$
        \item[$\iff$] $\assignSat{A}{\varphi}{a_1,\dots,a_n}$ 
        \item[$\iff$] $\assignSat{A}{\varphi}{h(a_1),\dots,h(a_n)}$ 
        \item[$\iff$] $(h(a_1),\dots,h(a_n))\in R$ 
    \end{itemize}
\end{proof}

Corollary~\ref{cor:AutomorphismTheorem} is often used for proving some relations are \emph{not definable}. For example, consider $\mathfrak{R} = (\realSet, <)$, its subset $\mathbb{N}$ is not definable in $\realStruct$.

Assume $\naturalSet$ is definable in $\realStruct$. Let $h(a)=a^3$. Obviously $h$ is an automorphism of $\realStruct$.

\begin{itemize}
    \item[] $a \in \mathbb{N}$
    \item[$\iff$] $\assignSat{R}{\varphi}{a}$
    \item[$\iff$] $\assignSat{R}{\varphi}{h(a)}$
    \item[$\iff$] $h(a)\in\naturalSet$
    \item[$\iff$] $a^3 \in \naturalSet$.  
\end{itemize}

This leads to a contradiction. If $a^3 = 2$, then no $a \in \naturalSet$.

\chapter{Deductive Calculus of First Order Logic}

\section{Deductive Calculus}

Proofs are (purely) syntactic constructs that caputre \emph{derivability of facts}. Deductive calculi provide descriptions of proofs in logic. There are more than one forms for deductive calculi.

We adopt the Hilbert-style deductive calculus, which contains

\begin{itemize}
    \item A set $\Lambda$ of wffs called \textbf{logical axioms}
    \item A \emph{single} \textbf{rule of inference} for forming a new wff from a pair of wffs
\end{itemize}

We then systematically generate a set of wffs from the axioms by using the rules of inference. They are called \textbf{derivable} wffs.

Another deductive calculus is called the natural calculus, which has many rules of inference and one single axiom (\emph{“排中律”}, that a proposition is either True or False).

Yet another deductive calculus is called the sequent calculus, which contains no axiom and a set of symmetric rules.

\subsection{Introduction to Soundness and Completeness}

The goal is to prove that for any language $\mathbb{L}$, the following are equivalent

\begin{itemize}
    \item The set of derivable wffs in $\mathbb{L}$
    \item The set of valid wffs in $\mathbb{L}$
\end{itemize}

This is done by proving the soundness and completeness

\begin{theorem}[Soundness]
    \label{thm:FOSoundness}
    Every derivable of wff is valid.
\end{theorem}

\begin{theorem}[Completeness]
    \label{thm:FOCompleteness}
    Every valid wff is derivable.
\end{theorem}

\section{Logical Axioms}

\begin{definition}[Generalization]
    A \textbf{generalization} of the wff $\alpha$ is any wff obtained by putting zero or more universal quantifiers in front of $\alpha$
\end{definition}

For example, $\forall x\forall y\forall y\alpha$ is a generalization of $\alpha$.

Note that every wff is a generalization of itself.

\begin{definition}[Axioms]
    Let $\mathbb{L}$ be a first-order language. The set $\Lambda$ of logical axioms of $\mathbb{L}$ consists of all generalizations of the wffs in the following groups
    \begin{axiom}
        \label{axiom:InstanceOfTautology}
        Instances of tautologies.
    \end{axiom}
    \begin{axiom}
        \label{axiom:Substitution}
        Wffs of the form $\forall x\alpha \to \alpha_t^x$ such that the term $t$ is \textbf{substitutable} for $x$ in $\alpha$. As a special case, $\forall x\alpha \to \alpha$, where we replace $x$ with $t=x$
    \end{axiom}
    \begin{axiom}
        \label{axiom:PushUniversalIntoImplication}
        Wffs of the form $\forall x(\alpha\to\beta) \to (\forall x \alpha\to \forall x \beta)$.
    \end{axiom}
    \begin{axiom}
        \label{axiom:QuantifyBoundedVar}
        Wffs of the form $\alpha\to\forall x \alpha$ if $x$ \emph{does not occur free} in $\alpha$
    \end{axiom}
    \begin{axiom}
        \label{axiom:Equality}
        Wffs of the form $x \doteq x$
    \end{axiom}
    \begin{axiom}
        \label{axiom:EqualitySubstitution}
        Wffs of the form $x \doteq y \to (\alpha \to \alpha')$ where $\alpha$ is atomic and $\alpha'$ is obtained from $\alpha$ by replacing zero or more free occurrences of $x$ in $\alpha$ by $y$
    \end{axiom}
\end{definition}

\begin{lemma}
    A wff $\varphi$ is valid $\iff$ $\forall{x}\varphi$ is valid.
\end{lemma}

For example,

\begin{itemize}
    \item $\alpha=Py$, $Py \to \forall x Py$ is an instance of Axiom \ref{axiom:QuantifyBoundedVar}
    \item $\alpha=Py$, $\forall y(Py\to \forall xPy)$ is a \emph{generalization} of instance of Axiom \ref{axiom:QuantifyBoundedVar}
    \item $\alpha = Px$, $x \doteq y \to Px\to Py$ is an instance of Axiom \ref{axiom:EqualitySubstitution}
    \item $\alpha = Px$, $x \doteq y \to Px \to Px$ is also an instance of Axiom \ref{axiom:EqualitySubstitution} because zero or more $x$ can be substituted
    \item $\alpha = Qxx$, $x\doteq y \to Qxx \to Qxy$ is also an instance of Axiom~\ref{axiom:EqualitySubstitution}
\end{itemize}

Despite being axioms, we can prove the validity of some of these axioms. We detail the proof of \ref{axiom:InstanceOfTautology} and \ref{axiom:Substitution}. Proof of others are trivial.

\subsection{Instance of Tautologies}

In this section we show the validity of Axiom~\ref{axiom:InstanceOfTautology}.

\begin{definition}[Instance of WFFs of Sentential Logic]
    Let $\alpha_1,\dots,\alpha_n,\dots$ be an infinite sequence of wffs of the first-order language of $\mathbb{L}$, $\varphi$ be a wff of sentential logic with junst the connectives $\to$ and $\neg$, $\varphi^\ast$ be the wff of $\mathbb{L}$ obtained by replacing every occurrence of the sentence symbol $A_n$ in $\varphi$ by $\alpha_n$ for each $m$. We say that $\varphi^\ast$ is an \textbf{instance} of $\varphi$
\end{definition}

For example, let $\varphi=A_1\to A_3$, $\varphi^\ast = \alpha_1 \to \alpha_3$ is an instance of $\varphi$.

\begin{definition}[Instance of Tautologies]
    For any tautologies $\varphi$ in the sentential logic, $\varphi^\ast$ is an instance of the tautology $\varphi$
\end{definition}

\begin{lemma}
    Given a structure $\frakA$ for language $\mathbb{L}$,
    \begin{itemize}
        \item $s$ be an assignment function
        \item $\varphi$ be a wff of sentential logic
        \item $\varphi^\ast$ be the instance of $\varphi$
        \item $v$ be the truth assignment such that
        \[ v(A_i) = T \iff \sat{A}{\alpha_i}{s} \]
    \end{itemize}
    Then
    \[ \bar{v}(\varphi) = T \iff \sat{A}{\varphi^\ast}{s} \]
\end{lemma}
\begin{proof}
    Prove by induction.
    \begin{itemize}
        \item[] \textbf{Base.} $\varphi = A_i$ follows immediately from the assumption that $v(A_i) = T \iff \sat{A}{\alpha_i}{s}$
        \item[] \textbf{Inductive.} \begin{enumerate}
            \item $\varphi = \neg \beta$.
            \begin{itemize}
                \item[] $\bar{v}(\neg\beta) = T \iff \sat{A}{(\neg\beta)^\ast}{s}$
                \item[$\equiv$] $\bar{v}(\beta) = F \iff \unsat{A}{\beta^\ast}{s}$ 
                \item[$\equiv$] $\bar{\beta} = T \iff \sat{A}{\beta^\ast}{s}$
            \end{itemize}
            \item $\varphi = \gamma \to \beta$
            \begin{itemize}
                \item[] $\bar{v}(\gamma\to\beta) = T \iff \sat{A}{\gamma\to\beta}{s}$
                \item[$\equiv$] If $\bar{v}(\gamma) = T$ then $\bar{v}(\beta) = T$ $\iff$ If $\sat{A}{\gamma}{s}$ then $\sat{A}{\beta}{s}$ 
            \end{itemize}
        \end{enumerate}
    \end{itemize}
\end{proof}

\begin{corollary}
    Every instance of a tautology is valid.
\end{corollary}

\subsection{Substitutions}

\begin{definition}[Substitution for Terms]
    Let $u$ be a term, $x$ be a variable, and $t$ be a term. $u_t^x$ is the result of replacing every occurrence of $x$ in $u$ by $t$
\end{definition}

\begin{definition}[Substitution for Formulas]
    For a wff $\alpha$, a variable $x$, let $\alpha_t^x$ be the result of replacing every free occurence of $x$ in $\alpha$ by $t$,

    \begin{itemize}
        \item If $\alpha$ is atomic. $\alpha = Pu_1,\dots,u_n$, then $\alpha_t^x = Pu_{1t}^x,\dots,u_{nt}^x$
        \item If $\alpha = \neg \beta$, then $\alpha_t^x = \neg\beta_t^x$
        \item if $\alpha = \beta\to\gamma$, then $\alpha_t^x = \beta_t^x \to \gamma_t^x$
        \item If $\alpha = \forall y \beta$, then $\alpha_t^x = \alpha$ if $y=x$, and $\alpha_t^x = \forall y \beta_t^x$
    \end{itemize}
\end{definition}

Now we return to Axiom~\ref{axiom:Substitution}. Is every $\forall x \alpha \to \alpha_t^x$ valid?

Of course the answer is ``No'', or otherwise we would not have required that $t$ is \emph{substitutable} for $x$ in $\alpha$.

To see this, let $\alpha = \exists y x\neq y$, $t=y$, then

\[ \forall x \exists y x\neq y \to \exists y y\neq y \]

which is obviously not valid.

The problem here is that by this substitution we made $x$ ``local/bounded'' (\emph{capture of free variables}), and the new formula is semantically different from the original one.

\begin{definition}[Substitutability]
    Let $\alpha$ be a wff, $x$ be a variable, and $t$ be a term. $t$ is substitutable for $x$ in $\alpha$ if
    \begin{itemize}
        \item $\alpha$ is atomic
        \item $\alpha = \neg\beta$ and $t$ is substitutable for $x$ in $\beta$.
        \item $\alpha = \beta \to \gamma$ and $t$ is substitutable for $x$ in $\beta$ and $\gamma$
        \item $\alpha = \forall y \beta$ and
        \begin{itemize}
            \item either $x$ does not occur free in $\forall y \beta$
            \item or $x$ occurs free in $\forall y\beta$ (which implies $x\neg y$), and $t$ is substitutable for $x$ in $\beta$, and $y$ does not occur in $t$
        \end{itemize}
    \end{itemize}
\end{definition}

We can check this with our previous example $\alpha = \exists y x\neq y$, $t=y$. $\alpha = \neg\forall y (\neg x \neq y)$. Notice that $y$ ocurrs in $t$ ($t=y$), therefore $y$ is not substitutable for $x$ in $\alpha$.

\subsubsection{Substitution Lemma}

\begin{lemma}
    Given a first-order language $\mathbb{L}$, let $\frakA$ be a structure for $\mathbb{L}$, $s$ be an assignment for $\frakA$, $u$ and $t$ be two terms and $x$ be a variable. Then
    \[ \bar{s}(u_t^x) = \overline{s(x|\bar{s}(t))}(u) \]
\end{lemma}

This looks very intuitive. It just states that the assignment for substitution is equal to first changing the assignment function and then apply the assignment on original term.

\begin{proof}
    Prove by induction.
    \begin{itemize}
        \item[] \textbf{Base.} (1) If $u = c$, trivial. (2) If $u=x$, then LHS would be $\bar{s}(t)$, RHS would also be $\bar{s}(t)$. (3) $u=y\neq x$, then LHS and RHS will both be $\bar{s}(y)$.
        \item[] \textbf{Inductive.} If $u = ft_1\dots t_n$.
        \[ \bar{s}(u_t^x) = \bar{s}(ft_{1t}^x\dots f(t_{nt}^x)) = f(\bar{s}(t_{1t}^x\dots t_{nt}^x)) \]
        which can be shown equal to RHS by applying inductive hypothesis
    \end{itemize}
\end{proof}

\begin{lemma}[Substitution Lemma]
    \label{lem:SubstitutionLemma}
    Let $s$ be an assignment function for $\frakA$. If $t$ is substitutable for $x$ in $\alpha$, then
    \[ \sat{A}{\alpha_t^x}{s} \iff \sat{A}{\alpha}{s(x|\bar{s}(t))} \]
\end{lemma}
\begin{proof}
    Prove by induction on $\alpha$.
    \begin{itemize}
        \item[] \textbf{Base.} $\alpha = Pt_1\dots t_n$
        \begin{itemize}
            \item[] $\sat{A}{(Pt_1\dots t_n)}{s}$
            \item[$\iff$] $\sat{A}{Pt_{1t}^x\dots t_{nt}^x}{s}$ 
            \item[$\iff$] $(\bar{s}(t_{1t}^x),\dots,\bar{s}(t_{nt}^x)) \in P^\frakA$
            \item[$\iff$] $(\bar{s'}(t_1),\dots,\bar{s'}(t_n)) \in P^\frakA$
            \item[$\iff$] $\sat{A}{\alpha}{s(x|\bar{s}(t))}$
        \end{itemize}
        \item[] \textbf{Inductive.} \begin{enumerate}
            \item  $\alpha = \neg \beta$
            \begin{itemize}
                \item[] $\sat{A}{\neg \beta_t^x}{s}$
                \item[$\iff$] $\unsat{A}{\beta_t^x}{s}$
                \item[$\iff$] $\unsat{A}{\beta}{s'}$
                \item[$\iff$] $\sat{A}{\neg\beta}{s'}$
            \end{itemize}
            \item $\alpha = \beta \to \gamma$. Similar.
            \item $\alpha = \forall y \beta$.
            \begin{itemize}
                \item[] $\sat{A}{\forall y \beta_t^x}{s}$
                \item[$\iff$] For any $d \in |\frakA|$, $\sat{A}{\beta_t^x}{s(y|d)}$
                \item[] $\sat{A}{\forall y\beta}{s'}$
                \item[$\iff$] For any $d \in |\frakA|$, $\sat{A}{\beta}{s(x|\bar{s}(t))(y|d)}$
                \item[] Since $y \neq x$, we can exchange the order of assignment overwriting
                \item[] For any $d \in |\frakA|$, $\sat{A}{\beta}{s(y|d)(x|\bar{s}(t))}$. For brevity denote $s(y|d)$ by $s''$. By induction hypothesis, we have $\sat{A}{\beta_t^x}{s''} \iff \sat{A}{\beta}{s''(x|\bar{s}''(t))}$. And by Substitutability, we have $\bar{s}''(t) = \bar{s}(t)$. Then we have proved the lemma.
            \end{itemize}
        \end{enumerate} 
    \end{itemize}
\end{proof}

\subsubsection{Validity of Axiom~\ref{axiom:Substitution}}

\begin{theorem}
    If $t$ is substitutable for $x$ in $\alpha$, then $\forall \alpha \to \alpha_t^x$ is valid.
\end{theorem}
\begin{proof}
    If $\sat{A}{\forall x \alpha}{s}$, since $s(x|\bar{s}(t))$ is an instance of ``for all $x$'', it holds that $\sat{A}{\alpha}{s(x|\bar{s}(t))}$. Then by the Substitution Lemma~\ref{lem:SubstitutionLemma}, we have $\sat{A}{\alpha_t^x}{s}$.
\end{proof}

Now we have proved that every member in the set of axioms $\Lambda$ is valid.

\section{Deductions}

\subsection{Rule of Inference}

\begin{definition}[Modus Ponens]
    Given any wffs $\alpha$ and $\beta$, the rule of \textbf{modus ponens} provides the operation for deriving $\beta$ from $\alpha\to\beta$ and $\alpha$.
\end{definition}

\subsection{Deduction}

\begin{definition}[Deduction]
    Let $\Gamma$ be a set of wffs of $\mathbb{L}$. A \textbf{deduction from $\Gamma$} is a finite sequence
    \[ \alpha_0,\dots,\alpha_n \]
    of wffs such that for every $\alpha_i$, at least one of the following holds
    \begin{itemize}
        \item $\alpha\in\Gamma$
        \item $\alpha\in\Lambda$
        \item $\alpha$ is inferred by modus ponens from two wffs $\alpha_j$ and $\alpha_k$ s.t. $j,k < i$.
    \end{itemize}
\end{definition}

\begin{definition}
    $\Gamma\vdash\alpha$ ($\alpha$ is a theorem of $\Gamma$) if there is a deduction $\alpha_0,\dots,\alpha_n$ from $\Gamma$ s.t. $\alpha=\alpha_n$.

    We write $\vdash\alpha$ for $\emptyset\vdash\alpha$
\end{definition}

Notice that deduction is purely based on symtax, and the logical implications we have learned before requires semantics.

As an example, we show that $\vdash Px \to \exists yPy$. Since $\Gamma = \emptyset$, we can only derive the conclusion from the axioms and modus ponens.

\[ Px \to \exists yPy \iff Px \to \neg\forall y \neg Py \]

$\forall y \neg Py \to \neg x$ is in axiom group~\ref{axiom:Substitution}. Further, $(\forall y \neg Py \to \neg x) \to (Px \to \neg \forall y \neg Py)$ is an instance of tautology $(A_1 \to \neg A_2) \to (A_2 \to \neg A_1)$. We can then finish the proof by applying modus ponens.

The proof can be formatted into a tree. But I cannot draw it so please refer to slides. A key problem is that few human beings (if any) is able to prove things like this, as it gradually ``factorizes'' the target wffs into members in $\Gamma$ or $\Lambda$. Fortunately we will later see some more helper rules to help normal people prove theorems in first-order logic.

\subsection{Properties of Deductions}

\begin{itemize}
    \item If $\Gamma\vdash\varphi$, then there must be a finite subset $\Delta$ of $\Gamma$ s.t. $\Gamma\vdash\varphi$. This is guaranteed by definition of deduction: it is finite.
    \item If $\alpha_1,\dots,\alpha_n$ is a deduction from $\Gamma$ and $\beta_1,\dots,\beta_m$ is a deduction from $\Gamma$, then $\alpha_1,\dots,\alpha_n,\beta_1,\dots,\beta_m$ is also a deduction from $\Gamma$
    \item If $\varphi\in\Gamma$ then $\Gamma\vdash\varphi$
    \item If $\Gamma\vdash\varphi$ and $\Gamma\subseteq\Delta$ then $\Delta\vdash\varphi$
    \item If $\Gamma\vdash\alpha$ and $\Gamma\vdash\alpha\to\beta$, then $\Gamma\vdash\beta$
    \item (Cut Rule) If $\Gamma\vdash\alpha$ and $\alpha\to\beta$, then $\Gamma\vdash\beta$. This property allows proving some theorem $\beta$ by one or more lemmas $\alpha$
    \item If $\Gamma\to\alpha$ then for any $\beta$, $\Gamma\vdash\beta\to\alpha$. Notice that $\alpha\to\left( \beta\to\alpha \right)$ is an instance of tautology.
    \item If $\Gamma\vdash\alpha\wedge\beta$, then $\Gamma\vdash\alpha$ and $\Gamma\vdash\beta$. Notice that $\alpha\wedge\beta = \neg\left( \alpha\to\neg\beta \right)$, and that $\neg\left( \alpha\to\neg\beta \right) \to \alpha$ (or $\beta$) are two instances of tautologies.
    \item If $\Gamma\vdash\alpha$ and $\Gamma\vdash\beta$ then $\Gamma\vdash\alpha\wedge\beta$
    \item If $\Gamma\vdash\alpha$ or $\Gamma\vdash\beta$ then $\Gamma\vdash\alpha\vee\beta$
    \item If $\Gamma\vdash\alpha\to\beta$ then $\Gamma;\alpha\to\beta$
\end{itemize}

\subsection{Relations between Tautologies and Derivations}

\begin{definition}
    In first-order logic, a set of wffs $\Gamma$ \textbf{tautologically implies} $\varphi$ if there is a wff $\alpha$ and a set $\Sigma$ of wffs in the sentential logic such that for some $*$ of $\mathbb{L}$
    \begin{itemize}
        \item For every $\beta\in\Sigma$, $\beta^*\in\Gamma$
        \item $\varphi=\alpha^*$
        \item $\Sigma$ tautologically implies $\alpha$ in the sentential logic ($\Sigma\vDash\alpha$)
    \end{itemize}
\end{definition}

\begin{lemma}
    $\Gamma\vdash\varphi$ iff $\Gamma\cup\Lambda$ tautologically implies $\varphi$
\end{lemma}

\begin{lemma}[Rule T (Enderton)]
    \label{lem:RuleT}
    If $\Gamma\vdash\alpha_1,\dots,\Gamma\vdash\alpha_n$ and $\{\alpha_1,\dots,\alpha_n\}$ tautologically implies $\beta$ then $\Gamma\vdash\beta$
\end{lemma}

\section{Deduction Theorem and Generalization Theorem}

\subsection{The Deduction Theorem}

\begin{theorem}[Deduction Theorem]
    \label{thm:DeductionTheorem}
    If $\Gamma;\alpha\to\beta$, then $\Gamma\vdash\alpha\to\beta$
\end{theorem}
\begin{proof}
    By induction on $\Gamma;\alpha\to\beta$.
    \begin{itemize}
        \item[] \textbf{Base.} If $\beta\in\Lambda$, then
        \[ \vdash\beta \Rightarrow \vdash \alpha\to\beta \Rightarrow \Gamma\vdash\alpha\to\beta \]
        If $\beta\in\Gamma$, then
        \[ \Gamma\vdash\beta \Rightarrow \Gamma\vdash\alpha\to\beta \]
        If $\beta=\alpha$, then it is an instance of tautology so it is in $\Lambda$
        \item[] \textbf{Inductive.} If $\Gamma;\alpha\vdash\gamma\to\beta$ and $\Gamma;\alpha\vdash\gamma$. By IH, $\Gamma\vdash\alpha\to(\gamma\to\beta)$ and $\Gamma\vdash\alpha\to\gamma$. Notice that $\{ \alpha\to(\gamma\to\beta), \alpha\to\gamma \}$ tautologically imply $\alpha\to\beta$. Then we apply ``Rule T''~\ref{lem:RuleT}.
    \end{itemize}
\end{proof}

\subsubsection{Contraposition}

\emph{Contraposition} is a corollary of the deduction theorem

\begin{theorem}[Contraposition]
    $\Gamma;\varphi\vdash\neg\psi$ iff $\Gamma;\psi\vdash\neg\varphi$
\end{theorem}
\begin{proof}
    \begin{itemize}
        \item[$\Rightarrow$] If $\Gamma;\varphi\vdash\neg\psi$, then by Deduction Theorem~\ref{thm:DeductionTheorem}, $\Gamma\vdash\varphi\to\neg\psi$. Notice that $\left( \varphi\to\neg\psi \right) \to \left( \psi\to\neg\varphi \right)$ is an instance of tautological implication in sentential logic. Therefore by Rule T~\ref{lem:RuleT} $\Gamma\vdash\psi\to\neg\varphi$, and $\Gamma;\psi\to\neg\varphi$. The converse is similar
    \end{itemize}
\end{proof}

\subsection{The Generalization Theorem}

\begin{theorem}[Generalization Theorem]
    \label{thm:GeneralizationTheorem}
    If $\Gamma\vdash\varphi$ and $x$ does not occur free in any member of $\Gamma$ then $\Gamma\vdash\forall x \varphi$
\end{theorem}
\begin{proof}
    By induction on $\varphi$
    \begin{itemize}
        \item[] \textbf{Base.} If $\varphi\in\Gamma$, then $x$ does not occur free in $\varphi$. Notice that we have $\varphi\to\forall x \varphi \in \Lambda$ (axiom~\ref{axiom:QuantifyBoundedVar}), and we have $\varphi\in\Gamma$, by modus ponens we have $\forall x \varphi$.
        
        If $\varphi\in\Lambda$, $\forall x\varphi$ is a generalization of $\varphi$.
        \item[] \textbf{Inductive.} $\Gamma\vdash\gamma\to\varphi$; $\Gamma\vdash\gamma$. By inductive hypothesis, $\Gamma\to\forall x(\gamma\to\varphi)$, and $\Gamma\vdash\forall x \gamma$. Notice that $\forall x(\gamma\to\varphi) \to \forall x \gamma\to\forall x\varphi$ (axiom~\ref{axiom:PushUniversalIntoImplication}).
    \end{itemize}
\end{proof}

The Deduction Theorem and the Generalization Theorem are ``meta-theorem''s that applies to deductive calculi. In Hilbert-style Calculus, we use the six logical axioms to derive these meta-theorems. In other calculi, they may use other axioms, or directly use these meta-theorems.

\subsubsection{Generalization on Constants}

\begin{theorem}
    \label{thm:GeneralizationOnConsts}
    If $\Gamma\vdash\varphi$ and $c$ is a constant symbol that is not in any member of $\Gamma$, then there is some variable $y$ not in $\varphi$ s.t. $\Gamma\vdash\forall y \varphi_y^c$ ($c$ replaced by $y$).

    Furthermore, there is a deduction of $\forall y \varphi_y^c$ from $\Gamma$ in which $c$ does not occur.
\end{theorem}

Intuitively, if $\varphi$ can be derived from $\Gamma$, then the constant $c$ in $y$ is somewhat like a quantified variable $y$.

\begin{proof}
    Since $\Gamma\vdash\varphi$, we have a deduction
    \[ \varphi_0,\varphi_1,\dots,\varphi_n \]
    s.t. $\varphi_n=\varphi$.

    We first show by induction (on deduction $\Gamma\vdash\varphi_i$) that
    \[ \varphi_{0y}^c,\dots,\varphi_{ny}^c \]
    is a deduction from $\Gamma$.
    \begin{itemize}
        \item[] \textbf{Base.} If $\varphi_i \in \Gamma$, since $c$ does not occur in members of $\Gamma$, we have $\varphi_{iy}^c = \varphi_i$. Then $\Gamma\vdash\varphi_{iy}^c$. If $\varphi_i\in\Lambda$, it can be verified by checking all 6 axioms.
        \item[] \textbf{Inductive.} If $\varphi_j = \varphi_k\to\varphi_i$, $\varphi_k$, $k,j < i$. By IH, $\Gamma\vdash\varphi_{ky}^c \to \varphi_{iy}^c$ and $\Gamma\vdash\varphi_{ky}^c$. We are done by Modus Ponens
    \end{itemize}

    We have shown so far that $\Gamma\vdash\varphi_y^c$ and $y$ does not occur in $\varphi$ (and the finite deduction sequence $\Delta$ that derives $\varphi$). Therefore we have $\Gamma\vdash\forall y \varphi_y^c$

    It also follows from this proof that the deduction does not contain $c$
\end{proof}

\begin{corollary}[Corollary 24G]
    \label{coroll:Corollary24G}
    If $\Gamma\vdash\varphi_c^x$, and $c$ is a constant symbol that is not in $\varphi$ or any member of $\Gamma$, then $\Gamma\vdash\varphi$
\end{corollary}

\begin{corollary}[Rule El]
    If $\Gamma;\varphi_c^x\vdash\psi$ and $c$ does not occur in $\varphi,\psi,\Gamma$, then $\Gamma;\exists x \varphi\vdash\psi$.
    
    Furthermore, there is a deduction of $\psi$ from $\Gamma;\exists x \varphi$ in which $c$ does not occur.
\end{corollary}
\begin{proof}
    To be completed.
\end{proof}

\subsection{The Re-Replacement Lemma}

\begin{lemma}[Re-Replacement Lemma]
    \label{lem:ReReplacementLemma}
    If $y$ does not occur in $\varphi$, then $x$ is substitutable for $y$ in $\varphi_y^x$ and $\varphi_{yx}^{xy} = \varphi$
\end{lemma}

\subsection{Consistency}

\begin{definition}[Consistency]
    $\Gamma$ is \textbf{inconsistent} if there is some wff $\alpha$ s.t. $\Gamma\vdash\alpha$ and $\Gamma\vdash\neg\alpha$

    $\Gamma$ is \textbf{consistent} if it is not inconsistent.
\end{definition}

Some properties of consistency

\begin{itemize}
    \item If $\Gamma$ is inconsistent then for every $\beta$, $\Gamma\vdash\beta$
    \begin{itemize}
        \item[] $\Gamma\vdash\alpha$, $\Gamma\vdash\alpha$
        \item[$\Rightarrow$] $\Gamma;\beta\vdash\alpha$ $\Gamma;\neg\beta\vdash\alpha$ (Add $\beta$ to $\Gamma$ and the deduction should still hold)
        \item[$\Rightarrow$] $\Gamma\vdash\neg\beta\to\alpha$ $\Gamma\vdash\neg\beta\to\neg\alpha$ (Deduction Theorem)
        \item[$\Rightarrow$] Notice that $\neg\beta\to\alpha, \neg\beta\to\neg\alpha$ tautologically implies $\beta$. By Rule T we have $\Gamma\vdash\beta$ 
    \end{itemize}
    \item If $\Gamma\nvdash\alpha$ $\iff$ $\Gamma;\neg\alpha$ is consistent.
    \begin{itemize}
        \item Equivalently, $\Gamma\vdash\alpha$ iff $\Gamma;\neg\alpha$ is inconsistent. $\Rightarrow$ is trivial.
        \item $\Leftarrow$ If $\Gamma;\neg\alpha$ is inconsistent, then there exists some $\beta$ s.t. $\Gamma;\neg\alpha\vdash\beta$ and $\Gamma;\neg\alpha\vdash\neg\beta$. The rest is similar to the previous proof.
    \end{itemize}
    \item $\Gamma$ is consistent $\iff$ every finite subset of $\Gamma$ is consistent.
    \begin{itemize}
        \item[] Equivalent to $\Gamma$ is inconsistent $\iff$ there is some subset of $\Gamma$ which is inconsistent
        \item[$\Rightarrow$] $\Gamma\vdash\alpha$, $\Gamma\vdash\neg\alpha$. Then there exists some finite subset $\beta_1\vdash\alpha$ and $\beta_2\vdash\neg\alpha$. And the union of $\beta_1$ and $\beta_2$ is the desired subset.
        \item[$\Leftarrow$] Trivial. 
    \end{itemize}
    \item If $\Gamma$ is consistent, then for every $\alpha$, either $\Gamma;\alpha$ is consistent or $\Gamma;\neg\alpha$ is consistent.
    \begin{itemize}
        \item Prove by contradiction. Assume for some $\alpha$, $\Gamma;\alpha$ is inconsistent, and $\Gamma;\neg\alpha$ is also inconsistent. Notice that $\Gamma;\neg\alpha$ is inconsistent iff $\Gamma\vdash\alpha$, so we have $\Gamma\vdash\alpha$. We also have $\Gamma\vdash\neg\alpha$ iff $\Gamma;\neg\neg\alpha$\footnote{We are working on pure grammatical level, so if the wffs ``looks'' different, then they are different, so we cannot directly say that $\Gamma;\neg\neg\alpha$ is inconsistent iff $\Gamma;\alpha$ is inconsistent.}. Further, it can be shown that $\Gamma;\neg\neg\alpha$ is inconsistent iff $\Gamma;\alpha$ is inconsistent.
    \end{itemize}
\end{itemize}

\begin{theorem}[Reductio Ad Absurdum]
    \label{thm:ReductioAdAbsurdum}
    If $\Gamma;\alpha$ is inconsistent, then $\Gamma\vdash\neg\alpha$
\end{theorem}

\section{Backward Inference and Prove Strategies}

\subsection{Backward Inference}

With all the properties derived so far, we can formulate a general method to prove theorems.

Assume we are showing $\Gamma\vdash\varphi$

\begin{itemize}
    \item If $\varphi = \alpha\to\beta$, then it suffices to show $\Gamma;\alpha\to\beta$ (Deduction Theorem)
    \item If $\varphi = \forall x \alpha$, and $x$ does not occur free in $\Gamma$, then it suffices to show $\Gamma\vdash\alpha$ (Generalization Theorem)
    \item If $\varphi = \forall x\alpha$, and $x$ occurs free in $\Gamma$, then we pick a variable $y$ that does not occur in $\alpha$ and $\Gamma$, then we have $\forall y \alpha_y^x\vdash\forall x\alpha$ and it suffices to show $\Gamma\vdash\forall y \alpha_y^x$
    \begin{itemize}
        \item To show why we have $\forall y \alpha_y^x\vdash\forall x\alpha$, it suffices to show $\forall y \alpha_y^x\vdash\alpha$. We have $\forall y \alpha_y^x \to \left( \alpha_y^x \right)_x^y = \alpha$ (by Axiom of Substitution and Re-Replacement Lemma).
    \end{itemize}
    \item If $\varphi = \neg\left( \alpha\to\beta \right)$, then it suffices to show $\Gamma\vdash\alpha$ and $\Gamma\vdash\neg\beta$ (Rule T)
    \item If $\varphi=\neg\neg\alpha$, then it suffices to show that $\alpha$ (Rule T)
    \item If $\varphi=\neg\forall x \alpha$, then it suffices to show that $\Gamma\vdash\neg\alpha_t^x$ for some $t$ substitutable for $x$ in $\alpha$
    \begin{itemize}
        \item This is not always possible, because there are cases where $\Gamma\vdash\neg\forall x\alpha$, but $\Gamma\nvdash\neg\alpha_t^x$ for all $t$. In this case, try Contraposition or Reductio Ad Absurdum
    \end{itemize}
    \item For atomic and negations of atomic formula, try contraposition.
\end{itemize}

Note that due to that applying this method will not always lead to a proof, or may lead to something that is not provable, this method is \emph{incomplete}.

\subsubsection{Examples}

\begin{enumerate}
    \item Assume $x$ does not occur free in $\beta$, derive

    \begin{itemize}
        \item $\vdash\forall x \left( \alpha \to \beta \right) \rightarrow \left( \alpha\to\forall x \beta \right)$
        \item[$\Leftarrow$] $\forall x \left( \alpha\to\beta \right)\vdash\exists x\alpha\to\beta$ (Deduction Thm)
        \item[$\Leftarrow$] $\forall x \left( \alpha\to\beta \right), \exists x \alpha \vdash \beta$ (Deduction Thm)
        \item[$\Leftarrow$] $\forall x \left( \alpha\to\beta \right), \exists x \alpha \vdash \neg\neg\beta$ (Rule T)
        \item[$\Leftarrow$] $\forall x \left( \alpha\to\beta \right),\neg\beta \vdash \neg\neg\forall x\neg \alpha$ (Contraposition)
        \item[$\Leftarrow$] $\dots \vdash \forall x\neg \alpha$
        \item[$\Leftarrow$] $\dots\vdash\neg\alpha$ (Generalization, note that $x$ does not occur free in $\beta$)
        \item[$\Leftarrow$] $\forall x \left( \alpha\to\beta \right),\alpha\vdash\neg\neg\beta$ (Contraposition)
        \item[$\Leftarrow$] $\dots\vdash\beta$   
    \end{itemize}

    \item Show $\vdash\forall x \forall z Pxz \to \forall{y} Pyy$
    \begin{itemize}
        \item[$\Leftarrow$] $\forall x \forall z Pxz \vdash \forall y Pyy$
        \item[$\Leftarrow$] $\forall x \forall z Pxz \vdash Pyy$
        \item[$\Leftarrow$] $\forall x \forall z Pxz \to \forall z Pyz$ (Substitution)
        \item[] Suffice to show $\forall zPyz \vdash \forall z Pyz$, again can be proved by substitution.  
        \item[] However, cannot prove $\vdash\forall x\forall y Pxy \to \forall y Pyy$ like this, although this formula is semantically valid ($\vDash$)
    \end{itemize}

    \item Show $\vdash\forall x \forall y Pxy \to \forall{y} Pyy$
    \begin{itemize}
        \item[] We have shown that Show $\vdash\forall x \forall z Pxz \to \forall{y} Pyy$
        \item[] We will show that $\forall x\forall y Pxy \vdash\forall{x}\forall{z}Pxz$
        \item[$\Leftarrow$] $\forall{x}\forall{y}Pxy \vdash Pxz$ (Generalization)
        \item[] Done. (Substitution)
    \end{itemize}
\end{enumerate}

\subsection{Alphabetic Variants}

\begin{theorem}[Alphabetic Variants]
    \label{thm:AlphabeticVariants}
    Given a wff $\alpha$, a term $t$ and a variable $x$. There is a wff $\alpha'$ s.t. $\alpha'$ differs from $\alpha$ only in quantified variables; $\vdash \alpha\leftrightarrow\alpha'$ and $t$ is substitutable for $x$ in $\alpha'$.
\end{theorem}

Intuitively, we can always find a $\alpha'$ such that we can do substitution whenever there is a conflicting variable in $\alpha$. For the examples mentioned above, $\alpha=\forall{x}\forall{y}Pxy$, and $\alpha'=\forall{x}\forall{z}Pxz$

Can be proved by induction on $\alpha$, and can refer to Enderton.

\subsubsection{Example}

Show $\forall{x}\forall{y}Pxy\vdash\forall{y}Pyy$, and $y$ is \emph{not} substitutable for $x$ in $\forall{y}Pxy$.

Apply the alphabetic variants theorem to $\forall{y}Pxy$ to get $\forall{z}Pxz$.

\begin{itemize}
    \item $\vdash\forall{y}Pxy\leftrightarrow\forall{z}Pxz$
    \item $\forall{x}\forall{y}Pxy\to\forall{x}\forall{z}Pxz$ (Generalization)
    \item $\forall{x}\forall{z}Pxz$ TO BE COMPLETED.
\end{itemize}

\section{The Soundness Theorem}

\begin{lemma}
    Given a set $\Gamma$ of wffs and the wffs $\alpha$ and $\beta$, if $\Gamma\vDash\alpha\to\beta$ and $\Gamma\vDash\alpha$, then $\Gamma\vDash\beta$
\end{lemma}

\begin{theorem}[Soundness Theorem]
    \label{thm:SoundnessTheorem}
    If $\Gamma\vdash\alpha$, then $\Gamma\vDash\alpha$
\end{theorem}
\begin{proof}
    Prove by induction on the deduction. Assume $\Gamma=\{ \gamma_0,\dots,\gamma_n \}$, we show that $\Gamma\vdash\varphi_i$ for all $i$
    \begin{itemize}
        \item[] \textbf{Base.} If $\varphi_i \in \Lambda$, by validity of all axioms, it holds. If $\varphi_i \in \Gamma$, then obviously $\Gamma\vDash\varphi_i$
        \item[] \textbf{Inductive.} Suppose there exists some $\varphi_j$ and $\varphi_k$ s.t. $j,k \le i$. Let $\varphi_k = \varphi_j \to \varphi_i$, and $\Gamma\vdash\varphi_j, \Gamma\vdash\varphi_k$. By inductive hypothesis we have $\Gamma\vDash\varphi_j$ and $\Gamma\vDash\varphi_k=\varphi_j\to\varphi_i$. Therefore by lemma we have $\Gamma\vDash\varphi_i$. 
    \end{itemize}
\end{proof}

\begin{corollary}
    If $\vdash\alpha$, then $\vDash\alpha$.
\end{corollary}

\begin{corollary}
    If $\vdash\varphi\leftrightarrow\psi$, then $\varphi$ and $\psi$ are logically equivalent.
\end{corollary}
\begin{proof}
    If $\vdash\varphi\rightarrow\psi$, then $\varphi\vDash\psi$. The converse is similar.
\end{proof}

\begin{corollary}
    If $\varphi'$ is an alphabetic variant of $\varphi$, then $\varphi$ and $\varphi'$ are logically equivalent.
\end{corollary}

\subsection{Soundness and Satisfiability}

\begin{corollary}
    If $\Gamma$ is satisfiable, then $\Gamma$ is consistent.
\end{corollary}

In fact, soundness is equivalent to the above corollary.

\begin{theorem}
    The following two statements are equivalent.
    \begin{itemize}
        \item For any $\Gamma$ and $\alpha$, if $\Gamma\vdash\alpha$, then $\Gamma\vDash\alpha$
        \item For any $\Gamma$, if $\Gamma$ is satisfiable, then $\Gamma$ is consistent.
    \end{itemize}
\end{theorem}
\begin{proof}
    \begin{itemize}
        \item[$\Rightarrow$] If we have soundness, we prove consistency by contradiction. Assume $\Gamma$ is satisfiable, but $\Gamma$ is inconsistent. Then there exists some $\varphi$ s.t. $\Gamma\vdash\varphi$ and $\Gamma\vdash\neg\varphi$. By soundness we have $\Gamma\vDash\varphi$ and $\Gamma\vDash\neg\varphi$. Since $\Gamma$ is satisfiable, there is some structure $\frakA$ and assignment $s$ s.t. $\sat{A}{\varphi}{s}$ and $\sat{A}{\neg\varphi}{s}$. 寄!
        \item[$\Leftarrow$] Conversely, if we have the latter, we prove soundness by contradiction. Assume $\Delta\vdash\alpha$ but $\Delta\nvDash\alpha$. Then $\Delta;\neg\alpha$ is satisfiable. So $\Delta;\neg\alpha$ is consistent. By property of consistency we have $\Delta\nvdash\alpha$. 寄!
    \end{itemize}
\end{proof}

\section{The Completeness Theorem}

\begin{theorem}[G\"{o}del Extended Completeness Theorem]
    If $\Gamma\vDash\alpha$, then $\Gamma\vdash\alpha$.
\end{theorem}

\begin{corollary}[G\"odel Completeness Theorem]
    If $\vDash\alpha$, then $\vdash\alpha$.
\end{corollary}

\subsection{Equivalent Statement for Completeness}

\begin{theorem}
    The following statements are equivalent
    \begin{itemize}
        \item For any $\Gamma$ and $\alpha$, if $\Gamma\vDash\alpha$, then $\Gamma\vdash\alpha$
        \item For any $\Gamma$, if $\Gamma$ is consistent then $\Gamma$ is satisfiable
    \end{itemize}
\end{theorem}
\begin{proof}
    \begin{itemize}
        \item[$\Rightarrow$] Assmue completeness, and assume $\Gamma$ is consistent. Assume for contradiction that $\Gamma$ is unsatisfiable. Then $\Gamma\vDash\alpha$ and $\Gamma\vDash\neg\alpha$. By completeness $\Gamma\vdash\alpha$ and $\Gamma\vdash\neg\alpha$, which means $\Gamma$ is inconsistent.
        \item[$\Leftarrow$] Assume the latter, we prove completeness by constradiction. Assume $\Gamma\vDash\alpha$ but $\Gamma\nvdash\alpha$. Then by property of consistency $\Gamma;\neg\alpha$ is consistent and thus satisfiable. Therefore $\Gamma\vDash\neg\alpha$. 寄!
    \end{itemize}
\end{proof}

\subsection{Proof of Completeness}

\emph{“20世纪逻辑学最重要的发现之一。”}

The proof is similar to that for compactness.

\begin{enumerate}
    \item Extend $\Gamma$ to $\Delta\supseteq\Gamma$ s.t. $\Delta$ is consistent and maximal (for any $\alpha$, either $\alpha\in\Delta$ or $\neg\alpha\in\Delta$)
    \item Define a structure $\frakA$ and an assignment $s$ for $\frakA$ s.t. $\frakA$ satisfies $\Gamma$ with $s$
\end{enumerate}

But the actual proof is more complex, because we have to deal with $\doteq$. The actual proof consists of 6 steps.

\subsubsection{Expanding a Language with Constants}

Let $\Gamma$ be a consistent set of wffs in a \emph{countable} language. We expand the language with a countably infinite set of new constant symbols $c_1,\dots,c_n,\dots$

Assume $\Gamma$ is defined in $\mathbb{L}$, and let
\[ \mathbb{L}' = \mathbb{L}\cup\{ c_1,\dots,c_n,\dots \} \]

We show by contradiction that $\Gamma$ is also consistent in $\mathbb{L}'$.

\begin{proof}
    Assume $\Gamma$ is inconsistent in $\mathbb{L}'$. Then there is some wff $\alpha$ s.t. $\Gamma\vdash\alpha$ and $\Gamma\vdash\neg\alpha$. Therefore $\Gamma\vdash\alpha\wedge\neg\alpha$. $\alpha$ may contain the new introduced constant symbols, but since we have Generalization on Constants, we can show that there is some deduction
    \[ \alpha_1',\dots,\alpha_n' = \alpha'\wedge\neg\alpha' \]
    where $\alpha_i'$ is $\alpha_i$ with all constants not in $\mathbb{L}$ replaced with some variable. Therefore $\Gamma$ is inconsistent in $\mathbb{L}$.
\end{proof}

\subsubsection{Preparing for Satisfiability of Quantified WFFs}

In the new language, for any pair of wff $\varphi$ and variable $x$, we introduce a formula
\[ \neg\forall{x}\varphi \to \neg\varphi_c^x \]

where $c$ is a new constant symbol. Let $\Theta$ be the set of all these formulas. This is essentially (1) $c$ identifies a \emph{counter example} for $\varphi$, and (2) $\Gamma\cup\Theta$ is consistent.

Note that the pairs of $\varphi$ and $x$ are enumerable

\[ (\varphi_1,x_1), (\varphi_2,x_2),\dots \]

and the newly introduced constant are also enumerable

\[ c_1,c_2,\dots \]

Since each $\varphi$ contains finitely many constants, there must be some certain way of enumeration such that $c_{i+1}$ does not occur in $\varphi_1,\dots,\varphi_i$.

We now show that $\Gamma\cup\Theta$ is consistent.

\begin{proof}
    Assume for contradiction that $\Gamma\cup\Theta$ is inconsistent. Then $\Gamma\cup\Theta\vdash\alpha$ and $\Gamma\cup\Theta\vdash\neg\alpha$.

    So there is some finite subset $\Theta_k = \{\theta_1,\dots,\theta_k\} \subseteq \Theta$ s.t. $\Gamma\cup\Theta_k$ is inconsistent. This is guaranteed by the finiteness of deduction. Notice that since $\Gamma$ itself is consistent, $\Theta_k$ cannot be empty. Let $k$ be the minimal number s.t. $\Gamma\cup\Theta_k$ is inconsistent.

    By Reductio Ad Absurdum, we have
    \[ \Gamma:=\Gamma\cup\{ \theta_1,\dots,\theta_{k-1} \} \vdash \neg \theta_k \]
    where
    \[ \theta_k = \neg\forall x \varphi_k \to \neg \varphi_{kc_k}^x \]

    Therefore
    \[ \Gamma' \vdash \neg\forall{x}\varphi_k \wedge \varphi_{kc_k}^k \]
    i.e. $\Gamma' \vdash \neg\forall{x}\varphi_k$ and $\Gamma'\vdash\varphi_{kc_k}^k$

    Note that from a corollary of Generalization on constants \ref{coroll:Corollary24G}. We have $\Gamma'\vdash\varphi_{kc_k}^k\Longrightarrow \Gamma'\vdash\forall{x}\varphi_k$. This shows that $\Gamma'$ is inconsistent, which is contradictory to our assumption that $k$ (not $k-1$) is the smallest number that makes $\Gamma\cup\Theta_k$ inconsistent.
\end{proof}

\subsubsection{Generate Maximally Consistent Set}

We extend $\Gamma\cup\Theta$ to a set $\Delta$ s.t.

\begin{itemize}
    \item $\Delta$ is consistent.
    \item For each wff $\alpha$, either $\alpha\in\Delta$ or $\neg\alpha\in\Delta$
\end{itemize}

\begin{proof}
    $\Gamma\cup\Theta$ is consistent.
    \begin{itemize}
        \item There is no $\beta$ s.t. $\Gamma\cup\Theta\vdash\beta$ and $\Gamma\cup\Theta\vdash\neg\beta$.
        \item[$\Leftrightarrow$] There is no $\beta$ s.t. $\Gamma\cup\Theta\cup\Lambda$ tautologically imply $\beta$ and $\neg\beta$
        \item[$\Rightarrow$] Therefore $\Gamma\cup\Theta\cup\Lambda$ is satisfiable \emph{in the sense of sentential logic}
        \item[$\Rightarrow$] There is some truth assignment $v$ satisfying $\Gamma\cup\Theta\cup\Lambda$. Therefore we pick
        \[ \Delta = \{ \varphi | \bar{v}(\varphi) = T \} \]
    \end{itemize}
\end{proof}

We continue to show some properties of $\Delta$.

\begin{proposition}
    For any $\alpha$, either $\alpha\in\Delta$ or $\neg\alpha\in\Delta$, but not both.
\end{proposition}
\begin{proof}
    If $\bar{v}(\alpha) = T$, then $\alpha\in\Delta$. $\bar{v}(\neg\alpha)=F$. Then $\neg\alpha\notin\Delta$. Conversely, if $\bar{v}(\alpha)=F$, then $\neg\alpha\in\Delta$, and thus $\alpha\notin\Delta$.
\end{proof}

\begin{proposition}
    $\Delta$ is a \emph{theory}. That is,
    \[ \Delta\vdash\alpha \Longrightarrow \alpha\in\Delta \]
\end{proposition}
\begin{proof}
    If $\Delta\vdash\alpha$, then $\Delta\cup\Lambda$ tautologically implies $\alpha$. This is equivalent to $\Delta$ tautologically implies $\alpha$, because by definition of $\Delta$ we have $\Lambda\subseteq\Delta$. Since we know that $v$ satisfies $\Delta$ (in sentential logic), by tautological implication we know that $v$ should also satisfy $\alpha$. Thus $\alpha\in\Delta$
\end{proof}

\begin{proposition}
    $\Delta$ is consistent.
\end{proposition}
\begin{proof}
    If $\Delta$ is inconsistent, then there exists some formula $\beta$ s.t. $\Delta\vdash\beta$ and $\neg\beta$. Then $\beta\in\Delta$ and $\neg\beta\in\Delta$. 噔噔咚
\end{proof}

\subsubsection{Make a Structure for the New Language}

We make a structure $|\frakA|$ from $\Delta$ for the new language where $\doteq$ is replaced by a 2-nary symbol $E$.

sudo make install!

\begin{itemize}
    \item Let $|\frakA|$ be the set of all terms in the new language
    \item $(u,t)\in E^\frakA \iff u\doteq t \in \Delta$
    \item For any n-ary predicate symbol,
    \[ (t_1,\dots,t_n) \in P^\frakA \iff Pt_1\dots t_n \in \Delta \]
    \item For any n-ary predicate symbol,
    \[ f^\frakA (t_1,\dots,t_n) = ft_1\dots t_n \]
    \item For any constant symbol $c$, $c^\frakA=c$
\end{itemize}

Then we make an assignment function $s:V\mapsto|\frakA|$
\[ s(x)=x \]

Then $\bar{s}(t)=t$. For any wff $\varphi$, let $\varphi^*$ be $\varphi$ with $\doteq$ replaced by $E$. We have
\[ \sat{A}{\varphi^*}{s} \iff \varphi\in\Delta \]

Notice that once we prove this, we have already found a structure $\frakA$ and an assignment $s$ that satisfies $\Delta$. That is, if the language does not contain $\doteq$, we have proved completeness.

\begin{proof}
    Assume $n$ is the number of connectives in $\varphi$. Prove by induction on $n$. We do induction on $n$ to avoid $\doteq$'s in $\varphi$ that may cause troubles.
    \begin{itemize}
        \item[] \textbf{Base.} $n=0$. Then $\varphi$ is $u\doteq t$ or $Pt_1\dots t_n$. If $\varphi$ is $u\doteq t$, then $\varphi^*$ is $Eut$.
        \[ \sat{A}{Eut}{s} \iff (\bar{s}(u), \bar{s}(t)) \in E^\frakA \iff (u,t) \in E^\frakA \iff u\doteq t \in \Delta \]
        If $\varphi$ is $Pt_1\dots t_n$, the proof is similar.

        \item[] \textbf{Inductive.} If $\varphi = \neg\alpha$, then $\varphi^*=\neg\alpha^*$.
        \[ \sat{A}{\neg\alpha^*}{s} \iff \unsat{A}{\alpha^*}{s} \iff \alpha\notin\Delta\iff\neg\alpha\in\Delta \]
        The second step uses the inductive hypothesis.

        If $\varphi=\alpha\to\beta$, then $\varphi^*=\alpha^*\to\beta^*$.
        \[ \sat{A}{\alpha^*\to\beta^*}{s} \iff \unsat{A}{\alpha^*}{s}\text{ or }\sat{A}{\beta^*}{s} \iff \alpha\notin\Delta \text{ or }\beta\in\Delta \]
        \begin{itemize}
            \item We first show $\alpha\notin\Delta$ or $\beta\in\Delta$ implies $\alpha\to\beta\in\Delta$. To show this, it suffices to show that $\Delta\vdash\alpha\to\beta$, and it suffices to show that $\Delta;\alpha\vdash\beta$. Then 分类讨论两种前提 and we are done.\footnote{“事实上我也忘了这个怎么证明,我们来证一下。”}
            \item To show $\alpha\to\beta\in\Delta\Rightarrow\alpha\notin\Delta$ or $\beta\in\Delta$.
            \[ \alpha\to\beta\in\Delta\iff\Delta\vdash\alpha\to\beta\Longleftarrow\Delta;\alpha\vdash\beta \]
        \end{itemize}

        If $\varphi=\forall{x}\alpha$. $\varphi^*=\forall{x}\alpha^*$.
        \[ \sat{A}{\forall{x}\alpha^*}{s} \iff \text{For every $t\in|\frakA|$,} \sat{A}{\alpha^*}{s(x|t)} \]
        \begin{itemize}
            \item[$\iff$] $\sat{A}{\alpha^*}{s(x|\bar{s}(t))}$
            \item[$\iff$] $\sat{A}{(\alpha^*)^x_t}{s}$ (Substitution Lemma)
            \item[$\iff$] $\sat{A}{(\alpha^x_t)^*}{s}$
            \item[$\iff$] $\alpha_t^x \in \Delta$ (IH) 
            \item[$\Longrightarrow$] $\alpha_c^x\in\Delta$ 
        \end{itemize}
        Notice that $\neg\forall{x}\alpha\to\neg\alpha_c^x \in \Delta$. Since we have $\alpha_c^x\in\Delta$, the condition $\neg\forall{x}\alpha$ cannot hold, so $\forall{x}\alpha\in\Delta$

        So far we have proved $\Rightarrow$. Now consider the other side $\forall{x}\alpha\in\Delta\Rightarrow\sat{A}{\forall{x}\alpha^*}{s}$. This is equivalent to showing
        \[ \unsat{A}{\forall{x}\alpha^*}{s} \Longrightarrow \forall{x}\alpha\notin\Delta \]
        Assume there is some $t$ s.t. $\unsat{A}{\alpha^*}{s(x|t)}$, equivalent to $\unsat{A}{(\alpha_t^x)^*}{s}$.
        
        Let $\beta$ be alphabetic equivalent to $\alpha$ s.t. $t$ is substitutable for $x$ in $\beta$. We have $\alpha\vDash\Dashv\beta$ by alphabetical equivalence.

        Therefore previous equation is equivalent to
        \[ \unsat{A}{(\beta_t^x)^*}{s} \iff \beta_t^x \notin \Delta \]

        If $\forall{x}\alpha\in\Delta$, then $\Delta\vdash\forall{x}\alpha$, and thus $\Delta\vdash\forall{x}\beta$. We have axiom $\forall{x}\beta\to\beta_t^x$ since $t$ is substitutable for $x$ in $\beta$. Thus $\Delta\vdash\beta_t^x$ and $\beta_t^x\in\Delta$. Contradiction. So $\forall{x}\alpha\notin\Delta$, we are done.
    \end{itemize}
\end{proof}

Until now, we are done if the language does not contain $\doteq$. But to actually complete the proof for all cases, we need an extra step.

\subsubsection{Deal with Equality}

Not enough time. Pigeoned. Refer to slides or reference books.

\chapter{Wrapping Up}

\section{Programs and Proofs}

\begin{itemize}
    \item Law of Excluded Middle $\forall{Q} Q\vee\neg Q$
    \item $Programs \equiv Proofs$
\end{itemize}

Not all proofs have a corresponding program.

Let $isProgram(P)$ and $halts(P)$ be two predicates.
\[ \varphi \triangleq \forall{P} isProgram(P) \to halts(P) \vee \neg halts(P) \]
$\varphi$ is valid. i.e. $\vDash\varphi$. By completeness of first order logic, $\vdash\varphi$. So there must exist a proof (deduction) $\Phi=\{\varphi_1,\dots,\varphi_n\}$ s.t. $\varphi_n = \varphi$. If every proof had a corresponding program, then this program would be able to solve the Halting problem.

Not all programs have a corresponding proof.

\begin{lstlisting}[language=c]
    while(1) {}
\end{lstlisting}

This program has a dead loop, which implies ``false''. If this program had a proof, then we would be able to prove ``false''.


\end{document}