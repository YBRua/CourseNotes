\documentclass[oneside]{book}
\usepackage{xeCJK}
\usepackage{amsmath}
\usepackage{mathtools}
\usepackage{listings} % lstlist插入代码
\usepackage{booktabs}
\usepackage{ulem}
\usepackage{enumerate}
\usepackage{amsfonts}
\usepackage{amssymb}
\usepackage{amsthm}
\usepackage{proof}
\usepackage{setspace} % spacing环境设置行间距
\usepackage[ruled, vlined]{algorithm2e} % 算法与伪代码 
\usepackage{bm} % 数学公式中的加粗
\usepackage{pifont} % 打圈的数字。172-211。\ding
\usepackage{graphicx}
\usepackage{float}
\usepackage[dvipsnames]{xcolor}
%\usepackage{indentfirst}
\usepackage{ulem} %\sout{}打删除线
\normalem % 使用默认normalem
\usepackage{lmodern}
\usepackage{subcaption}
\usepackage[colorlinks, linkcolor=blue]{hyperref}
\usepackage{cleveref}
\usepackage[a4paper]{geometry}
\usepackage{titlesec}

\theoremstyle{definition}
\newtheorem{definition}{Definition}[section]
\newtheorem{theorem}{Theorem}[section]
\newtheorem*{optTheorem}{Theorem}
\newtheorem{proposition}{Proposition}[section]
\newtheorem{lemma}{Lemma}[section]
\newtheorem{corollary}{Corollary}[section]
\theoremstyle{remark}
\newtheorem*{remark}{Remark}
\newtheorem*{sketchproof}{Sketch of Proof}

\newcommand{\range}{\textrm{rng}}
\newcommand{\domain}{\textrm{dom}}

\newcommand{\questeq}{\stackrel{?}{=}}


\title{Mathematical Logic}
\author{\textsc{YBiuR}}
\date{A long long time ago in a far far away SJTU}


\begin{document}
\setlength{\parskip}{1em}
\setlength{\parindent}{0em}

\frontmatter
\maketitle
\chapter*{Preface}
\emph{“道可道,非常道。”}

\mainmatter
\tableofcontents
\begin{spacing}{1.2}

\chapter{Set Theory}
\emph{Salute to Discrete Mathematics.}

\section{Sets}
\label{sec:Sets}
(Informally,) A set is a collection of elements.

\section{Relations and Functions}
\label{sec:RelationsAndFunctions}

\subsection{Relations}
\label{sub:Relation}
Given $n$ sets $A_1,\dots,A_n$, a relation $R$ over them is a subset of $A_1 \times \cdots \times A_n$.

\begin{definition}[Binary Relation]
    \label{def:BinaryRelation}
    A \textbf{binary relation} $R$ is a relation over $A \times B$ given some $A$ and $B$.
    \begin{itemize}
        \item The \textbf{domain} of $R$, denoted by $\textrm{dom}(R)$, is $\{ x | \exists y: \langle x,y \rangle \in R \}$
        \item The \textbf{range} of $R$, denoted by $\textrm{rng}(R)$, is $\{ y | \exists x: \langle x,y \rangle \in R \}$
    \end{itemize}
\end{definition}

A binary relation is
\begin{itemize}
    \item \textbf{Reflexive} iff $\langle x, x \rangle \in R$ for each $x \in A$.
    \item \textbf{Symmetric} iff $\langle x,y \rangle \in R \longrightarrow \langle y,x \rangle \in R$.
    \item \textbf{Transitive} iff $\langle x,y \rangle \in R \land \langle y,z \rangle \in R \longrightarrow \langle x,z \rangle \in R$.
\end{itemize}

A relation is an \textbf{equivalence relation} if it is reflexive, symmetric and transitive.

\subsection{Functions}
\label{sub:Function}
\begin{definition}[Functions]
    \label{def:Function}
    A \textbf{function} $f: A \rightarrow B$ is a binary relation over $A \times B$ satisfying:
    \begin{itemize}
        \item Its domain is $A$.
        \item For each $x \in A$, there is a unique $y \in B$ such that $\langle x, y \rangle \in f$.
    \end{itemize}
\end{definition}

\begin{definition}[One-to-One (injective)]
    \label{def:Injective}
    A function $f: A \rightarrow B$ is \textbf{one-to-one (injective)} if for each $x, y \in A$,
    \[ f(x) = f(y) \Longrightarrow x = y. \]
\end{definition}

\begin{definition}[Onto (surjective)]
    \label{def:Surjective}
    A function is \textbf{onto (surjective)} if for each $y \in B$, there is some $x \in A$ such that $f(x) = y$.
\end{definition}

\begin{definition}[One-to-One Correspondence (bijective)]
    \label{def:Bijective}
    A function is a \textbf{one-to-one correspondence (bijective)} between $A$ and $B$ if $f$ is both one-to-one and onto.
\end{definition}

\subsection{Finite Sets}
\label{sub:FiniteSets}
\begin{definition}[Finite sets]
    \label{def:FiniteSets}
    The set $X$ is \textbf{finite} if there is a natural number $n$ and a one-to-one correspondence between $X$ and $\{0, 1, \dots, n\}$。

    The set $X$ is \textbf{infinite} if it is not finite.
\end{definition}

We use the notion of one-to-one correspondence between infinite sets to talk about the ``sizes'' of these infinite sets.

\begin{definition}[Enumerable Sets]
    \label{def:EnumerableSets}
    The set $X$ is \textbf{enumerable} if there is a one-to-one correspondence between $X$ and $\mathbb{N}$.
\end{definition}

\begin{definition}[Listing of Sets]
    \label{def:ListingOfSets}
    Let $A$ be a set. The sequence $a_0, a_1, \dots, a_n, \dots$ is a \textbf{listing} of $A$ if
    \begin{enumerate}
        \item $a_i \in A$ for each $a_i$.
        \item Every member of $A$ is equal to $a_n$ for some $n \in \mathbb{N}$.
    \end{enumerate}
\end{definition}

\begin{theorem}
    \label{thm:ListingAndEnumerableSets}
    The set $A$ is enumerable iff there is some listing without repetitions of $A$.
\end{theorem}
\begin{proof}
    $\Rightarrow$.

    $\exists f: A \mapsto \mathbb{N}$, and $f$ is a one-to-one correspondence. Therefore there exists an inverse function $f^{-1}: \mathbb{N} \mapsto A$.

    Therefore we can construct a listing
    $$ f^{-1}(0), f^{-1}(1), f^{-1}(2), \cdots $$

    $\Leftarrow$.

    Let $g(i) = a_i$. $g$ is one-to-one because the listing has no identical elements. $g$ is onto because every member in $A$ is equal to some $a_n$ in the listing.

    $f = g^{-1}$.
\end{proof}

\begin{definition}[Countable Sets]
    \label{def:CountableSets}
    A set is \textbf{countable} if it is finite or enumerable. A set is \textbf{uncountable} if it is not countable.
\end{definition}

\begin{theorem}
    \label{thm:CountableSets}
    A set $X$ is countable iff there is a one-to-one mapping $f: X \mapsto \mathbb{N}$.

    A set $X$ is countably infinite iff it is enumerable.
\end{theorem}
\begin{sketchproof}
    $\Rightarrow$: Straightforward. Follows immediately from the definition of finite \ref{sub:FiniteSets} and enumerable \ref{def:EnumerableSets} sets.
    
    $\Leftarrow$: Let $B = \textrm{rng}(f)$. Notice that $f$ is a one-to-one correspondence between $X$ and $B$.
    \begin{itemize}
        \item If $B$ is finite, then there exists some function $h: B \mapsto \{1,2,\dots,n\}$ s.t. $h$ is a one-to-one correspondence. Let $g = h \circ f$, then $g$ is a one-to-one correspondence from $X$ to $\{1,2,\dots,n\}$. Therefore by definition \ref{def:FiniteSets}, $X$ is finite and is thus countable.
        
        \item If $B$ is infinite, since $B \subseteq \mathbb{N}$, we can sort elements of $B$ in ascending order:
        \[ b_0 < b_1 < b_2 < \cdots < b_n < \cdots \]
        
        We have constructed a listing of $B$, and therefore $B$ is enumerable. Therefore by definition \ref{def:EnumerableSets}, there exists a one-to-one correspondence $h:B \mapsto \mathbb{N}$. Construct $g = h \circ f$, then $g$ must also be a one-to-one correspondence between $X$ and $\mathbb{N}$. Therefore by definition \ref{def:EnumerableSets}, the set $X$ is enumerable and thus countable.
    \end{itemize}
\end{sketchproof}

\begin{theorem}
    \label{thm:ListingAndCountableSets}
    The set $A$ is countable and nonempty iff there is some listing with possible repetitions of $A$.
\end{theorem}
\begin{sketchproof}
    $\Rightarrow$. If $A$ is enumerable, we are done by Theorem \ref{thm:ListingAndEnumerableSets}. If $A$ is finite, we can construct a listing $a_0, a_1, \cdots, a_n, a_n, \cdots$.

    $\Leftarrow$. Let $a_0, a_1, \cdots, a_n, \cdots$ be the listing. Define $f:A \mapsto \mathbb{N}$ by $f(a_i) = \min_j\left\{ j: a_j=a_i \right\}$. $f$ is one-to-one and by Theorem \ref{thm:CountableSets} we can conclude that $A$ is countable.
\end{sketchproof}

\begin{proposition}
    \label{prop:SubsetOfEnumerableSetsIsCountable}
    If $A$ is enumerable and $B \subseteq A$, then $B$ is countable.
\end{proposition}

\begin{theorem}
    \label{thm:LotsOfCountableSets}
    ~{}
    \begin{itemize}
        \item If $A$ and $B$ are countable, then $A \cup B$, $A \cap B$, and $A \times B$ are all countable.
        \item If each of $A_0, \dots, A_n, \dots$ is countable, then the union of these sets is also countable.
        \item If $A$ is countable and nonempty, then the set of all finite sequences of members of $A$ is countable.
    \end{itemize}
\end{theorem}

\subsection{Constructing Infinite Sets}
\label{sub:ConstructingInfiniteSets}

\begin{definition}[Characteristic Functions]
    \label{def:CharacteristicFunctions}
    Let $X$ be a set and let $A \subseteq X$. FOr any $a \in A$, let
    $$ C_A(a) = \begin{cases}
        1, &\quad a \in A\\
        0, &\quad a \notin A
    \end{cases} $$
    $C_A(a)$ is called the characteristic function of $A$.
\end{definition}

\begin{definition}[Power Sets]    \label{def:PowerSets}
    Let $A$ be a set, the power set of $A$ is
    \[ \mathcal{P}(A) = \{ X|X \subseteq A \} \]
\end{definition}

\begin{theorem}[Cantor's Theorem]
    \label{thm:CantorsTheorem}
    $\mathcal{P}(\mathbb{N})$ is uncountable.
\end{theorem}
\begin{proof}
    Proof by contradiction. Suppose $\mathcal{P}(\mathbb{N})$ is countable. Obviously $\mathcal{P}(\mathbb{N})$ is not finite, so it must be enumerable.

    By Theorem \ref{thm:ListingAndEnumerableSets}, we can construct a listing with no repetitions.

    We can list all subsets of $A$ using characteristic functions.
    \[\begin{bmatrix}
        \emptyset & \begin{bmatrix} 0 & 0 & \cdots & 0 \end{bmatrix}\\
        \{0\} & \begin{bmatrix} 1 & 0 & \cdots & 0 \end{bmatrix}\\
        \{1\} & \begin{bmatrix} 0 & 1 & \cdots & 0 \end{bmatrix}\\
        \{0, 1\} & \begin{bmatrix} 1 & 1 & \cdots & 0 \end{bmatrix}\\
        \cdots & \cdots\\
    \end{bmatrix}\]

    We can then select all the bits along the diagonal and flip these bits to form a new bit sequence. This sequence can also be interpreted as a listing of some subset of $\mathcal{P}(\mathbb{N})$. However, this listing cannot exist in the listed listings, because it has at least one bit that is different from any existing listings. This listing fails to enumerate all subsets of $\mathcal{P}(\mathbb{N})$, and therefore the set $\mathcal{P}(\mathbb{N})$ is uncountable.
\end{proof}

\begin{corollary}
    $\mathbb{R}$ is uncountable.
\end{corollary}

\subsection{Domination of Sets}

\begin{definition}[Domination of Sets]
    \label{def:SetDomination}
    ~{}
    \begin{itemize}
        \item $A \preceq B$ if there is a one-to-one function $f: A \mapsto B$.
        \item $ A \prec B$ if $A \preceq B$ but $B \npreceq A$.
        \item $A \equiv B$ if $A \preceq B$ and $B \preceq A$.
    \end{itemize}
\end{definition}

\begin{theorem}[Cantor-Schr\"oder-Bernstein]
    \label{thm:CSB}
    $A \equiv B$ iff there is a one-to-one correspondence between $A$ and $B$.
\end{theorem}
\begin{proof}
    Refer to supplementary material on Canvas.
\end{proof}

\begin{theorem}[Cantor's Theorem]
    \label{thm:Cantor}
    For every set $A$,
    \[ A \prec \mathcal{P}(A) \]
\end{theorem}
\begin{proof}
    We need to prove 1) $A \preceq \mathcal{P}(A)$ and 2) $\mathcal{P}(A) \npreceq A$. 1) is easy because $\forall a \in A$, $\{a\} \subseteq A$ and therefore $\{a\} \in \mathcal{P}(A)$.

    To prove 2), assume $\mathcal{P}(A) \preceq A$, by theorem \ref{thm:CSB} we have $\mathcal{P}(A) \equiv A$, and therefore there should exist a one-to-one correspondence $f: A \mapsto \mathcal{P}(A)$.

    Let
    \[ B = \{x \in A|x \notin f(x)\} \]

    Then $B \subseteq A$, and $B \in \mathcal{P}(A)$. Since $f$ is onto, there must exist some $b \in A$ s.t. $f(b) = B$.

    \begin{enumerate}
        \item If $b \in f(b) = B$. Then $b \in f(b) \Rightarrow b \notin f(b)$. Boom.
        \item If $b \notin f(b)$. Then $b \in B = f(b)$. Boom.
    \end{enumerate}

    Therefore $\mathcal{P}(A) \npreceq A$.
\end{proof}
\begin{corollary}
    For every $A$ there exists a $B$ s.t. $A \prec B$.
\end{corollary}
\begin{corollary}
    $\mathcal{P}(\mathbb{N})$ is uncountable.
\end{corollary}
\begin{proof}
    $\mathcal{P}(\mathbb{N}) \npreceq \mathbb{N}$.
\end{proof}

\subsection{Sets of Functions}

\begin{definition}[Set of Functions]
    \label{def:SetOfFunction}
    $~^{A} B$ is the set of all functions that map $A$ into $B$.
\end{definition}
\begin{remark}
    The set $~^A\{0,1\}$ is the set of all characteristic functions (\ref{def:CharacteristicFunctions}).
\end{remark}

\begin{theorem}
    There is a one-to-one correspondence between $~^{\mathbb{N}}\{0,1\}$ and $\mathcal{P}(\mathbb{N})$.
\end{theorem}
\begin{remark}
    A generalized version: there is a one-to-one correspondence between $~^{\mathbb{A}}\{0,1\}$ and $\mathcal{P}(\mathbb{A})$
\end{remark}
\begin{corollary}
    \label{cor:RealIsUncountable}
    $\mathbb{R}$ is uncountable.
\end{corollary}
\begin{theorem}
    \label{thm:NNisUncountable}
    $~^\mathbb{N}\mathbb{N}$ is uncountable.
\end{theorem}
\begin{proof}
    Assume $~^\mathbb{N}\mathbb{N}$ is countable. Since $~^\mathbb{N}\{0,1\} \subseteq ~^\mathbb{N}\mathbb{N}$. $~^\mathbb{N}\{0,1\}$ must be countable. Boom.
\end{proof}

\subsection{Generalized Continuum Hypothesis}

\begin{center}
    \textbf{Is there a set $A$ s.t. $\mathbb{N} \prec A \prec \mathbb{R} \equiv \mathcal{P}(\mathbb{N})$?}
\end{center}

\chapter{The Informal Notions of Algorithms}

\emph{“形式语言与自动机课程速成。”}

\section{Algorithms}

\begin{definition}[Algorithms (Informal)]
    \label{def:Algorithm}
    An algorithm is a \textbf{finite ordered list} of instructions.
\end{definition}

Possible outcomes of running an algorithm
\begin{itemize}
    \item The algorithm does not halt
    \item The algorithm halts
    \begin{itemize}
        \item In an erroneous state (fails)
        \item Gives valid outputs
    \end{itemize}
\end{itemize}
Cases other than the algorithm giving valid outputs are collectively identified as ``no output''.

\subsection{Algorithms for Determining Membership}
\label{sub:AlgoForDeterminingMembership}

\begin{definition}[Algorithms for Determining Membership]
    \label{def:AlgoForDeterminingMembership}
    An algorithm for \emph{determining membership} in a set $A \subseteq \mathbb{N}$ has an input, and two possible outputs ``yes'' and ``no''. If the algorithm is run on input $n$, it will halt in finite steps with output ``yes'' if $n \in A$ and ``no'' if $n \notin A$.
\end{definition}

\begin{definition}[Effectively Decidable Sets]
    \label{def:EffectivelyDecidableSet}
    Let $A$ be a subset of $\mathbb{N}$. $A$ is \textbf{effectively decidable} if there is an algorithm for determining membership of $A$.
\end{definition}

\begin{itemize}
    \item $\mathbb{N}$ is effectively decidable. (Always ``yes'')
    \item $\emptyset$ is effectively decidable. (Always ``no'')
\end{itemize}

\begin{theorem}
    If $A$ and $B$ are effectively decidable subsets of $\mathbb{N}$, then $\mathbb{N}\backslash A$, $A \cap B$ and $A \cup B$ are all effectively decidable.
\end{theorem}

Algorithms can have different kinds of outputs and inputs.

\begin{definition}[Diophantine Equations]
    Consider polynomials with integer coefficients (and any number of variables), a \textbf{diophantine equation} is an equation of the form $p=0$, where $p$ is sunch a polynomial. (e.g., $3x^2 + 5xy - 2z^4 +3 = 0$)
\end{definition}

\paragraph{Hilbert's 10th Problem.} Is there an algorithm for determining whether or not diophantine equations have integer solutions?

\subsection{Algorithm for Listing Members of Sets}
\label{sub:AlgoForListingMembersOfSets}

\begin{definition}[Algorithm for Listing Members of Sets]
    \label{def:AlgoForListingMembersOfSets}
    Let $A \subseteq \mathbb{N}$. An algorithm for \textbf{listing the members of $A$} prints (or enumerates) a list of numbers $a_0,a_1,\dots$ s.t.
    \begin{itemize}
        \item $a_n \in A$ for all $n \in \mathbb{N}$
        \item If $a \in A$, then $a = a_n$ for some $n$
    \end{itemize}
\end{definition}
\begin{remark} ~{}
    \begin{itemize}
        \item No input.
        \item Repetitions in the listing are permitted.
        \item If $A$ is finite, the algorithm may terminate or run forever (by iterating endlessly over the finite set).
        \item If $A$ is infinite, the algorithm should run forever.
    \end{itemize}
\end{remark}

\begin{definition}[Effectively Enumerable Sets]
    \label{def:EffectivelyEnumerableSet}
    A set $A \subseteq \mathbb{N}$ is \textbf{effectively enumerable} if there is an algorithm for listing its members.
\end{definition}

\subsection{Effectively Computable Functions}
\label{sub:EffectivelyComputableFunction}

\begin{definition}[Partial Functions]
    \label{def:PartialFunction}
    A function $f:\mathbb{N}\mapsto\mathbb{N}$ is a \textbf{partial number-theoretic function} if its domain is a subset of $\mathbb{N}$.
\end{definition}
\begin{definition}[Total Functions]
    \label{def:TotalFunction}
    A partial function whose domain is all of $\mathbb{N}$ is a total function.
\end{definition}

\begin{definition}[Notation for Definability]
    For a partial function $f$
    \begin{itemize}
        \item $n \in \textrm{dom}(f)$, then $f(n)\downarrow$
        \item $n \notin \textrm{dom}(f)$, then $f(n)\uparrow$
    \end{itemize}
\end{definition}

\begin{definition}[Effectively Computable Functions]
    \label{def:EffectivelyComputableFunction}
    The partial function $f$ is \textbf{effectively computable} if there is an algorithm $\mathcal{A}$ s.t. on input $n$
    \begin{itemize}
        \item Prints output $f(n)$ if $f(n)\downarrow$
        \item No output if $f(n)\uparrow$
    \end{itemize}
    $\mathcal{A}$ is called an algorithm for computing $f$.
\end{definition}

\begin{remark}
    The definition of effectively decidable and enumerable sets, and effectively computable functions can be extended to $\mathbb{N}^k$.
\end{remark}

\subsection{Relations among Effectively Decidable Sets, Effectively Enumerable Sets and Effectively Computable Functions}

\begin{proposition}
    Let $A \subseteq \mathbb{N}$. If $A$ is effectively decidable then $A$ is effectively enumerable.
\end{proposition}

\begin{proposition}
    Let $A \subseteq \mathbb{N}$. $A$ is effectively decidable iff the characteristic function (\ref{def:CharacteristicFunctions}) $C_A$ is effectively computable.
\end{proposition}

\begin{proposition}
    Let $f:\mathbb{N}\mapsto\mathbb{N}$. If $f$ is an effectively computable function, then $\textrm{rng}(f)$ is an effectively enumerable set.
\end{proposition}

\begin{proposition}
    Combining the above two propositions, we have: Let $A$ be a non-empty subsect of $\mathbb{N}$, $A$ is effectively enumerable iff there is an effectively computable function whose range is $A$.
\end{proposition}

\begin{proposition}
    If $A\neq\emptyset$ is an effectively enumerable subset of $\mathbb{N}$, then there is an effectively computable function $f$ with $\range(f)=A$.
\end{proposition}

\section{Enumerability of Algorithms}
\label{sec:EnumerabilityOfAlgos}

\begin{theorem}
    There are only \emph{enumerably many} algorithms.
\end{theorem}
\begin{corollary}
    There are only enumerably many partial effectively computable functoins. Since there are uncountably many total number-theoretic functions, there must exist total number-theoretic functions that are not effectively computable.
\end{corollary}
\begin{corollary}
    There are only enumerably many effectively enumerable subsets of $\mathbb{N}$. Since there are uncountably many subsets of $\mathbb{N}$, there must exist subsets of $\mathbb{N}$ that are not effectively enumerable.
\end{corollary}

\subsubsection{Listing of Effectively Computable Total Function}

Suppose $f_0,\dots$ is a listing of all effectively computable total function (this listing is unknown, we only know a listing of all computable partial functions). Let $g$ be a function such that

\[ g(m,n) = f_m(n) \]

$g$ is not effectively computable.

\begin{proof}
    By diagonal argument. Let $h(n) = g(n,n) + 1$. If $g$ is effectively computable, then so is $h$. Then $h$ must be in the listing $\{f_n\}$. Suppose $f_k = h$. Then
    \[ f_k(k) = h(k) = g(k,k)+1 = f_k(k) + 1 \]
    噔噔咚。
\end{proof}

\section{The Halting Problem}

Is there an algorithm for deciding whether any algorithm halts on given input or not?

\begin{theorem}
    \label{thm:HaltingProblem}
    Let $\mathcal{A}_0,\dots,\mathcal{A}_n,\dots$ be a listing of algorithm with space for only one input. There does not exist an algorithm $\mathcal{H}$ with space for two number inputs s.t. on input $(m,n)$
    \begin{itemize}
        \item Halts with output ``yes'' if $\mathcal{A}_m$ halts on input $n$.
        \item Halts with output ``no'' if $\mathcal{A}_m$ does not halt on input $n$
    \end{itemize}
\end{theorem}
\begin{proof}
    Construct $\mathcal{C}(n)$ by:
    \begin{itemize}
        \item Run $\mathcal{H}(n,n)$
        \item If ``yes'' then loop forever
        \item If ``no'' then print ``yes'' and return
    \end{itemize}

    Then $\mathcal{C}$ must be in the listing $\{\mathcal{A}_n\}$. Suppose $\mathcal{A}_k = \mathcal{C}$.
    \begin{itemize}
        \item $\mathcal{C}(k)$ halts $\Leftrightarrow$ $\mathcal{H}(k,k)$ prints ``no'' $\Leftrightarrow$ $\mathcal{A}_k(k) = \mathcal{C}(k)$ does not halt.
    \end{itemize}
    噔噔咚。
\end{proof}
\chapter{Sentential Logic}

\section{Grammar}
\label{sec:Gramma}

\subsection{Symbols}
\label{sub:Symbols}
\begin{itemize}
    \item Logical Symbols
    \begin{itemize}
        \item Sentential Connectives
        \begin{itemize}
            \item $\neg$
            \item $\wedge$
            \item $\vee$
            \item $\rightarrow$
            \item $\leftrightarrow$
        \end{itemize}
        \item Parentheses
    \end{itemize}
    \item Non-logical Symbols: An enumerable set of elements
\end{itemize}

\subsection{Expressions}
\label{sub:Expressions}

\begin{definition}[Expression]
    \label{def:Expression}
    An expression is a finite sequence of symbols.
\end{definition}
\begin{remark}
    The set of all expressions is enumerable.
\end{remark}

We often use Greek alphabets $\alpha,\beta,\dots$ to represent expressions.

\subsection{Well-Formed Formulas}
\label{sub:WellFormedFormulas}

\begin{definition}[Well-Formed Formula]
    \label{def:WFF}
    A \textbf{well-formed formula} (or formula or wff) is an expression built up from sentence symbols by applying some finite times of \emph{formula building operations}
\end{definition}

\begin{definition}[Formula Building Operations]~{}
    \begin{itemize}
        \item $\xi_{\neg}(\alpha) = (\neg\alpha)$
        \item $\xi_{\wedge}(\alpha,\beta) = (\alpha \wedge \beta)$
        \item $\xi_{\vee}(\alpha,\beta) = (\alpha\vee\beta)$
        \item $\xi_{\rightarrow}(\alpha,\beta) = (\alpha\rightarrow\beta)$
        \item $\xi_{\leftrightarrow}(\alpha,\beta) = (\alpha\leftrightarrow\beta)$
    \end{itemize}
\end{definition}
\begin{remark}
    Do NOT omit the parentheses.
\end{remark}

\begin{definition}[Well-Formed Sequences of Expressions]
    \label{def:WellFormedSeqOfExpr}
    A \textbf{well-formed sequence of expressions} is a finite sequence $\alpha_1,\alpha_2,\dots,\alpha_n$ of expressions such that each $\alpha_i$ is either
    \begin{itemize}
        \item A sentence symbol
        \item $(\neg\alpha_j)$ for some $j < i$
        \item $(\alpha_j \wedge \beta_k)$ for some $j,k<i$
        \item $(\alpha_j \vee \beta_k)$ for some $j,k<i$
        \item $(\alpha_j \leftarrow \beta_k)$ for some $j,k<i$
        \item $(\alpha_j \leftrightarrow \beta_k)$ for some $j,k<i$
    \end{itemize}
\end{definition}

\begin{proposition}
    An expression $\alpha$ is a well-formed formula iff there is a well-formed sequence $(\alpha_1,\dots,\alpha_n)$ s.t. $\alpha = \alpha_n$
\end{proposition}

\subsection{The Induction Principle}
\label{sub:Induction}

Well-formed formulas are a form of inductive definitions with
\begin{itemize}
    \item Basic building blocks
    \item Closing operations
\end{itemize}

\begin{theorem}[The Induction Principle]
    \label{thm:InductionPrinciple}
    Let $S$ be a set of wffs ($S \subseteq W$), if
    \begin{enumerate}
        \item Every sentence symbol is in $S$
        \item For each wff $\alpha$ and $\beta$, if $\alpha$ and $\beta$ are in $S$ then each of the following are in $S$
        \begin{itemize}
            \item $(\neg \alpha)$
            \item $(\alpha \wedge \beta)$
            \item $(\alpha \vee \beta)$
            \item $(\alpha \rightarrow \beta)$
            \item $(\alpha \leftrightarrow \beta)$
        \end{itemize}
    \end{enumerate}
    Then $S$ is the set of \emph{all wffs} ($S = W$).
\end{theorem}

We can see an example of Indunction.

\begin{proposition}
    Every wff has the same number of left parentheses as right parenthesis
\end{proposition}
\begin{proof}
    Let $S\triangleq \{\alpha|\alpha\text{has equal number of left and right parentheses}\}$.
    \begin{itemize}
        \item[Base] $\alpha = A$. Straightforward. Sentense symbols do not have parenthesis
        \item[Step]
        \begin{enumerate}
            \item Let $\beta \in S$, $\alpha = (\neg \beta) \in S$.
            \item Let $\alpha_1, \alpha_2 \in S$, $\alpha = (\alpha_1 \wedge \alpha_2) \in S$.
            \item $\cdots$
        \end{enumerate} 
    \end{itemize}
\end{proof}

\subsection{Parsing Formulas}
\label{sub:ParsingFormulas}

The induction principle actually gives an algorithm for parsing formulas.

On input expression $\alpha$

\begin{enumerate}
    \item If is leaf node, we are done. Return.
    \item The first symbol must be `('.
    \item If the second symbol is `$\neg$', then expect an non-empty expression $\beta$ and parse $\beta$.
    \item If the second symbol is not `$\neg$', then expect a non-empty expression $\beta_1$, an operator and another expression $\beta_2$.
\end{enumerate}

\subsection{Abbreviations}
\label{sub:Abbreviations}

\begin{itemize}
    \item The outermost parentheses can be omitted.
    \item $\neg$ appplies to as little as possible, with the highest precedence.
    \item $\wedge$ and $\vee$ apply to as little as possible, subject to $\neg$.
    \item $\rightarrow$ and $\leftrightarrow$ apply to as little as possible, subject to other operators.
    \item When handling operators with the same precedence, grouping is always to the right. $A \rightarrow B \rightarrow C= (A \rightarrow (B\rightarrow C))$.
\end{itemize}

\section{Semantics}
\label{sec:Semantics}

\subsection{Truth Assignments}

Consider a math domain $\{T,F\}$ of truth values

\begin{itemize}
    \item T is called truth
    \item F is called falsity
\end{itemize}

\begin{definition}[Truth Assignment]
    A truth assignment for a set $\mathcal{S}$ of sentence symbols is a function
    \[ v:\mathcal{S}\mapsto\{T,F\} \]
\end{definition}

\begin{definition}[Extended Truth Assignment]
    Let $\bar{\mathcal{S}}$ be the set of wffs that can be built up from $\mathcal{S}$ by formula-building operations. Let $v$ be a truth assignment for $\mathcal{S}$. An \textbf{extension} $\bar{v}$ of $v$
    \[ \bar{v}:\bar{\mathcal{S}}\mapsto \{T,F\} \]
    assigns truth values to every wff in $\mathcal{S}$ s.t.
    \begin{itemize}
        \item $\bar{v}(\alpha) = v(\alpha)$ if $\alpha \in \mathcal{S}$
        \item $\bar{v}(\neg(\alpha))$ is T if $\bar{v}(\alpha)$ is F and F otherwise.
        \item $\bar{v}((\alpha\wedge\beta))$ is T if $\bar{v}(\alpha)$ is T and $\bar{v}(\beta)$ is T and F otherwise.
        \item $\bar{v}((\alpha\vee\beta))$ is T if $\bar{v}(\alpha)$ is T or $\bar{v}(\beta)$ is T and F otherwise.
        \item $\bar{v}((\alpha\to\beta))$ is F if $\bar{v}(\alpha)$ is T and $\bar{v}(\beta)$ is F and T otherwise.\footnote{Emphasizes the promise of a condition implying a consequence. If the condition is falsy then no guarantee for the consequence. \emph{“骗你是小狗”}}
        \item $\bar{v}((\alpha\leftrightarrow\beta))$ is T if $\bar{v}(\alpha) = \bar{v}(\beta)$ and is F otherwise.
    \end{itemize}
\end{definition}

\begin{theorem}[Determinacy of Truth Assignments]
    \label{thm:DeterminacyofTruthAssignments}
    For every $v_1$ and $v_2$ and wff $\alpha$, if
    \[ v_1(A) = v_2(A) \]
    for every sentence symbol that occurs in $\alpha$, then
    \[ \bar{v}_1(\alpha) = \bar{v}_2(\alpha) \]
\end{theorem}

\begin{remark}
    To determine the value of $\bar{v}(\alpha)$, we only need to know the value of $v$ on the sentence symbols that occur in $\alpha$. This leads to the method of \textbf{truth tables}.
\end{remark}

\subsection{Satisfiability}
\label{sub:Satisfiability}

We first introduce some new notations. We use captial Greek letters, $\Delta$, $\Sigma$, etc. to represent sets of wffs. And we use $\Sigma;\alpha$ to represent $\Sigma \cup \{\alpha\}$.

\begin{definition}~{}
    \begin{itemize}
        \item $v$ satisfies $\alpha$ if $\bar{v}(\alpha) = T$
        \item $v$ satisfies $\Sigma$ if $\bar{v}(\alpha) = T$ for every $\alpha \in \Sigma$.
    \end{itemize}
\end{definition}

\begin{definition}[Satisfiability]~{}
    \label{def:Satisfiability}
    \begin{itemize}
        \item $\alpha$ is satisfiable if there exists some $v$ that satisfies $\alpha$
        \item $\Sigma$ is satisfiable if there exists some $v$ that satisfies $\Sigma$
    \end{itemize}
\end{definition}

\begin{remark}
    Every $v$ satisfies $\emptyset$. Because $v$ satisfies $\emptyset$ iff
    \[ \forall \alpha, \alpha\in\emptyset \Longrightarrow \bar{v}(\alpha) = T \]
    The assumption itself is false, and therefore the consequence is always true.
\end{remark}

\subsection{Semantic Implications}
\label{sub:SemanticImplications}

\begin{definition}
    A set of wffs $\Sigma$ semantically implies $\alpha$ when every truth assignment satisfying $\Sigma$ also satisfies $\alpha$.
    \begin{itemize}
        \item $\Sigma \vDash \alpha$ denotes that $\Sigma$ implies $\alpha$
        \item $\alpha \vDash \beta$ denotes that $\{\alpha\} \vDash \beta$
    \end{itemize}
    If $\Sigma \vDash \alpha$, we call $\alpha$ a semantic consiquence of $\Sigma$.
\end{definition}

Semantic implication is also referred to as tautological implication.

\begin{remark}
    Note that $\{\alpha,\neg\alpha\} \vDash \beta$ also holds, because the assumption does not hold, so the consequence trivially holds.
\end{remark}

\subsection{Tautologies}
\label{sub:Tautologies}

\begin{definition}[Tautologies]
    \label{def:Tautology}
    $\alpha$ is a tautology if $\emptyset \vDash \alpha$, denoted by $\vDash \alpha$.
\end{definition}
\begin{remark}
    ~{}
    \begin{itemize}
        \item $\alpha$ is a tautology iff $\forall v$, $\bar{v}(\alpha)=T$.
        \item $\alpha$ is a tautology iff $\neg \alpha$ is \emph{not satisfiable}
        \item $\alpha$ is satisfiable iff $\neg \alpha$ is not a tautology.
    \end{itemize}
\end{remark}

\subsection{Semantic Equivalence}
\label{sub:SemanticEquivalence}

\begin{definition}[Semantic Equivalence]
    Two wffs $\alpha$ and $\beta$ are semantically equivalent if both $\alpha\vDash\beta$ and $\beta\vDash\alpha$ hold. We use $\alpha\vDash\Dashv\beta$
\end{definition}

\begin{proposition}
    The following are equivalent
    \begin{itemize}
        \item $\alpha$ and $\beta$ are semantically equivalent
        \item For every $v$, $\bar{v}(\alpha)=\bar{v}(\beta)$
        \item $\alpha$ and $\beta$ have the same truth table
    \end{itemize}
\end{proposition}

We can use semantic equivalence to derive truthfulness of wffs. If $\alpha\vDash\models\beta$, we can freely exchange one for the other in deriving the truth of some formula $\sigma$ where $\alpha$ and/or $\beta$ occur.

Remember not to mix syntax and semantics

\begin{itemize}
    \item $\alpha=T$ is incorrect. Use $\bar{v}(\alpha)=T$.
    \item $v(\Sigma)=T$ is incorrect. Use $v$ satisfies $\Sigma$.
\end{itemize}

\subsection{Properties of Satisfaction and Implication}

\begin{itemize}
    \item If $\alpha$ is a tautology, then $\Sigma\vDash\alpha$ for every $\Sigma$
    \item If $\alpha\in\Sigma$ then $\Sigma\vDash\alpha$
    \item If $\Sigma\vDash\alpha$ and $\Sigma\vDash\alpha\to\beta$ then $\Sigma\vDash\beta$
    \item If $\Sigma\vDash\alpha$ and $\alpha\vDash\beta$ then $\Sigma\vDash\beta$.
    \item If $\Sigma\vDash\alpha$ then for all $\beta$, $\Sigma\vDash\beta\to\alpha$
    \item If $\Sigma\vDash\alpha$ and $\Sigma\vDash\beta$ then $\Sigma\vDash \alpha\wedge\beta$
    \item If $\Sigma\vDash\alpha$ or $\Sigma\vDash\beta$ then $\Sigma\vDash \alpha\vee\beta$
    \item $\Sigma\nvDash\alpha$ iff $\Sigma\cup\{\neg\alpha\}$ is satisfiable
    \item $\Sigma \vDash \alpha$ iff $\Sigma \cup \{\neg \alpha\}$ is not satisfiable
    \item $\Sigma \vDash \alpha \to \beta$ iff $\Sigma;\alpha \vDash \beta$
    \item If $\Sigma$ is not satisfiable, then for every $\alpha$, $\Sigma\vDash\alpha$
    \item If $\Sigma\vDash\alpha$ and $\Sigma\subseteq\Delta$ then $\Delta\vDash\alpha$
    \item If $\Sigma$ is satisfiable then every subset of $\Sigma$ is satisfiable
    \item If every subset of $\Sigma$ is satisfiable then $\Sigma$ is satisfiable.
    \item If every finite subset of $\Sigma$ is satisfiable then $\Sigma$ is satisfiable
    \item If $\Sigma\vDash\alpha$ then there is a finite subset $\Delta$ of $\Sigma$ such that $\Delta\vDash\alpha$
\end{itemize}

\section{Normal Forms}
\label{sec:NormalForms}

\subsection{Disjunctive Normal Forms}
\label{sub:DisjunctiveNormalForms}

\begin{definition}[Disjunctive Normal Form]
    \label{def:DisjunctiveNormalForm}
    The wff $\alpha$ is in \textbf{disjunctive normal form} if $\alpha=\gamma_1\vee\gamma_2\vee\cdots\vee\gamma_k$ where each $\gamma_i$ is a conjunction
    \[ \gamma_i = \beta_{i1}\wedge\beta_{i2}\wedge\cdots\wedge\beta_{in_i} \]
    where each $\beta_{ij}$ is either a sentence symbol or the negation of a sentence symbol
\end{definition}

\subsection{Conjunctive Normal Forms}
\label{sub:ConjunctiveNormalForms}

\begin{definition}[Conjunctive Normal Form]
    \label{def:ConjunctiveNormalForm}
    The wff $\alpha$ is in \textbf{conjunctive normal form} if $\alpha=\gamma_1\wedge\gamma_2\wedge\cdots\wedge\gamma_k$ where each $\gamma_i$ is a disjunction
    \[ \gamma_i = \beta_{i1}\vee\beta_{i2}\vee\cdots\vee\beta_{in_i} \]
    where each $\beta_{ij}$ is either a sentence symbol or the negation of a sentence symbol
\end{definition}

\subsection{Completeness of Normal Forms}

\begin{theorem}[Completeness of DNFs]
    \label{thm:CompletenessOfDNF}
    Every wff is semantically equivalent to a wff in disjunctive normal form
\end{theorem}
\begin{proof}
    \begin{enumerate}
        \item Construct the disjunctive normal form truth tables
        \item Select all assignments $v$ s.t. $\bar{v}(\alpha)=T$
        \item Construct $\gamma_i$ where $\beta_{ij} = A_j$ if $A_j$ is assigned $T$ and $\beta_{ij} = \neg A_j$ otherwise
    \end{enumerate}
\end{proof}

\begin{theorem}[Completeness of CNFs]
    \label{thm:CompletenessOfCNF}
    Every wff is semantically equivalent to a wff in conjunctive normal form
\end{theorem}
\begin{proof}
    Given $\alpha=\gamma_1\vee\cdots\vee\gamma_n$ in DNF. If every $\gamma_i$ is a sentence symbol or the negation of a sentence symbol, we are done. Otherwise, there is some $\gamma_i = \beta_{i1}\wedge\beta_{i2}$. Then
    \[ \alpha \equiv (\beta_{i1} \wedge \beta_{i2}) \vee \alpha' \equiv (\beta_{i1}\vee\alpha') \wedge (\beta_{i2}\vee\alpha') \]
    where $\alpha'$ is the disjunction of $\{\gamma_k|k \neq i\}$

    And we recursively repeat the steps
\end{proof}

\section{Finite Satisfiability}
\label{sec:FiniteSatisfiability}

\begin{definition}[Finite Satisfiability]
    \label{def:FiniteSatisfiability}
    $\Sigma$ is \textbf{finitely satisfiable} if every finite subset of $\Sigma$ is satisfiable.
\end{definition}
\begin{remark}
    Suppose $\Delta$ is finitely satisfiable, and for every $\alpha$, $\alpha\in\Delta$ or $\neg\alpha\in\Delta$

    Then $\alpha\in\Delta$ iff $\neg\alpha\notin\Delta$
\end{remark}


\subsection{Compactness Theorem}

\begin{theorem}[Compactness Theorem]
    \label{thm:CompactnessTheorem}
    If $\Sigma$ is finitely satisfiable, then $\Sigma$ is satisfiable
\end{theorem}
\begin{sketchproof}
    We break down the proof into the following steps
    \begin{enumerate}
        \item From $\Sigma$, construct its superset $\Delta$ s.t.
        \begin{enumerate}
            \item $\Delta$ is finitely satisfiable
            \item For every wff $\alpha$, $\alpha\in\Delta$ or $\neg\alpha\in\Delta$
        \end{enumerate}
        \item Show that $\Delta$ is satisfiable, so that $\Sigma$ is also satisfiable
    \end{enumerate}
\end{sketchproof}

\begin{lemma}
    \label{lem:CompactnessLemma1}
    If $\Delta$ is finitely satisfiable, then for every wff $\alpha$,
    \begin{itemize}
        \item either $\Delta\cup\{\alpha\}$ is finitely satisfiable
        \item or $\Delta\cup\{\neg\alpha\}$ is finitely satisfiable
    \end{itemize}
\end{lemma}
\begin{proof}
    Prove by contradiction. Suppose neither $\Delta\cup\{\alpha\}$ nor $\Delta\cup\{\neg\alpha\}$ is finitely satisfiable. Then there are some finite subsets $\Delta_1 \subseteq \Delta\cup\{\alpha\}$ and $\Delta_2\subseteq\Delta\cup\{\neg\alpha\}$ which are not satisfiable. Notice that $\alpha$ must be in $\Delta_1$ and $\Delta_2$, or otherwise $\Delta_1$ and $\Delta_2$ would be satisfiable by finite satisfiability of $\Delta$. Therefore $\Delta_1 = \Delta'_1\cup\{\alpha\}$ and $\Delta_2'\cup\{\neg\alpha\}$.

    Then we construct $\Delta' = \Delta_1\cup\Delta_2 = \Delta_1'\cup\Delta_2'\cup\{\alpha,\neg\alpha\}$. Then there exists an assignment $v$ such that $v$ satisfies $\Delta_1'\cup\Delta_2'$, and that $v$ satisfies either $\alpha$ or $\neg\alpha$. Suppose $\bar{v}(\alpha)=T$, then $\Delta_1$ is satisfiable, which causes contradiction. Conversely, if $\bar{v}(\alpha) = F$ then $\bar{v}(\neg\alpha)=T$, then $\Delta_2$ is satisfiable, which is also a contradiction.
\end{proof}
\begin{remark}
    This lemma implies that we can expand $\Sigma$ for one step, by including $\alpha$ or $\neg\alpha$
\end{remark}


\begin{lemma}
    \label{lem:CompactnessLemma2}
    If $\Sigma$ is finitely satisfiable, then there is a $\Delta \supseteq \Sigma$ such that
    \begin{itemize}
        \item $\Delta$ is finitely satisfiable
        \item For each wff $\alpha$, $\alpha\in\Delta$ or $\neg\alpha\in\Delta$
    \end{itemize}
\end{lemma}
\begin{proof}
    The set of all wffs is enumerable, so we can write all wffs in a sequence
    \[ \alpha_1, \alpha_2,\dots,\alpha_n,\dots \]
    Then we can scan through the sequence and check if $\alpha_i$ can be added to $\Sigma$. Starting from $\Delta_0 = \Sigma$,
    \[\Delta_{i+1}\begin{cases}
        \Delta_i \cup \{\alpha_i\} &\quad\text{if it is finitely satisfiable}\\
        \Delta_i \cup \{\neg \alpha_i\} &\quad \text{otherwise}
    \end{cases}\]

    It can be proved by induction and Lemma~\ref{lem:CompactnessLemma1} that $\forall i,\Delta_i$ is finitely satisfiable.

    And $\Delta$ can be constructed by the union of all $\Delta_i$'s

    \[ \Delta = \bigcup_{i\in\mathbb{N}} \Delta_i \]

    $\Delta$ is finitely satisifiable because for each $\Delta'\subseteq\Delta$, $\Delta'\subseteq\Delta_i$ for some $\Delta_i$. Since $\Delta_i$ is finitely sat, $\Delta'$ is also satisfiable, and therefore $\Delta$ is finitely sat.

    For each wff $\alpha$, either $\alpha\in\Delta$ or $\neg\alpha\in\Delta$. Because $\alpha$ must exist as $\alpha_i$ in the sequence of all wffs, then either $\alpha_i\in\Delta_{i+1}$ or $\neg\alpha_i\in\Delta_{i+1}$.
\end{proof}

\begin{lemma}
    \label{lem:CompactnessLemma3}
    Let $\Delta$ be a set of wffs such that
    \begin{itemize}
        \item $\Delta$ is finitely satisifiable
        \item For every wff $\alpha$, $\alpha\in\Delta$ or $\neg\alpha\in\Delta$
    \end{itemize}
    Then $\Delta$ is satisfiable
\end{lemma}
\begin{proof}
    Consider the sentence symbols. All sentence symbols (or their negations) must be in $\Delta$ because they are also wffs. Therefore if there is an assignment $v$ that satisfies $\Delta$, its values is already determined by the sentence symbols in $\Delta$.

    \[v(A)=\begin{cases}
        T &\quad A\in\Delta\\
        F &\quad \neg A\in\Delta
    \end{cases}\]

    Then proving Lemma~\ref{lem:CompactnessLemma3} is equivalent to proving
    \[ \forall \alpha, \alpha\in\Delta\Leftrightarrow\bar{v}(\alpha) = T \]
    This can be proved by induction
    \begin{itemize}
        \item[base] Consider $\alpha=A$. Obviously it holds.
        \item[induction] \begin{enumerate}[(a)]
            \item $\alpha=\neg\beta$. The hypothesis is $\beta\in\Delta \Leftrightarrow \bar{v}(\beta)=T$. If $\neg\beta\in\Delta$, then $\beta$ cannot be in $\Delta$ due to the finite satisfiability of $\Delta$, and therefore $\bar{v}(\beta) = F$, and therefore $\bar{v}(\neg\beta) = T$. Conversely, If $\bar{v}(\neg\beta) = T$, then $\bar{v}(\beta) =F$, and therefore $\beta\notin\Delta$ and thus $\neg\beta$ must be in $\Delta$.
        \end{enumerate}
    \end{itemize}
\end{proof}

\begin{corollary}
    \label{coroll:CorollaryOfTheCompactnessTheorem}
    If $\Sigma\vDash\tau$, then there is a finite subset $\Delta$ of $\Sigma$ such that $\Delta\vDash\tau$
\end{corollary}
\begin{proof}
    Assume for every subset $\Delta$ of $\Sigma$, $\Delta \nvDash \tau$. Then $\Delta;\tau$ is satisfiable. So $\Sigma;\tau$ is finitely satisfiable, and by compactness theorem we have $\Sigma;\tau$ is satisfiable, which implies $\Sigma\nvDash\tau$, and this leads to a contradiction.
\end{proof}

\subsection{Decidability Results for Semantic Implications}

\begin{theorem}
    Given any finite set $\Sigma$ of wffs and any wff $\alpha$, there is an algorithm for deciding whether or not $\Sigma\vDash\alpha$.
\end{theorem}
\begin{enumerate}
    \item Collect all sentence symbols in $\Sigma$ and $\alpha$
    \item Use truth table
\end{enumerate}

\begin{corollary}
    Given a finite set of wffs $\Sigma$, the set of its semantic consequences is effectively decidable. In particular, the set of tautologies is effectively decidable.
\end{corollary}

\subsection{Enumerability Results for Semantic Implications}

\begin{theorem}
    If $\Sigma$ is an effectively enumerable set of wffs, then the set of semantic consequences of $\Sigma$ is effectively enumerable.
\end{theorem}
\begin{proof}
    Let $\beta_1,\dots,\beta_n,\dots$ be an effective enumeration of $\Sigma$. Let $\Delta_n=\beta_1,\dots,\beta_n$. Let $\alpha_1,\dots,\alpha_m,\dots$ be an effective enumeration of all wffs. We construct a table $T$ where $T_{ij} = \Delta_i \vDash \alpha_j$. Due to Corollary~\ref{coroll:CorollaryOfTheCompactnessTheorem}, if some $T_{ij}$ holds, then $\alpha_j$ is a semantic consequence of $\Sigma$.
\end{proof}

\end{spacing}
\end{document}