\documentclass[oneside]{book}
\usepackage[UTF8]{ctex}
\usepackage{amsmath}
\usepackage{mathtools}
\usepackage{listings} % lstlist插入代码
\usepackage{booktabs}
\usepackage{ulem}
\usepackage{enumerate}
\usepackage{amsfonts}
\usepackage{amssymb}
\usepackage{amsthm}
\usepackage{proof}
\usepackage{setspace} % spacing环境设置行间距
\usepackage[ruled, vlined]{algorithm2e} % 算法与伪代码 
\usepackage{bm} % 数学公式中的加粗
\usepackage{pifont} % 打圈的数字。172-211。\ding
\usepackage{graphicx}
\usepackage{float}
\usepackage[dvipsnames]{xcolor}
%\usepackage{indentfirst}
\usepackage{ulem} %\sout{}打删除线
\normalem % 使用默认normalem
\usepackage{lmodern}
\usepackage{subcaption}
\usepackage[colorlinks, linkcolor=blue]{hyperref}
\usepackage{cleveref}
\usepackage[a4paper]{geometry}
\usepackage{titlesec}
\usepackage{graphicx}
\usepackage{stmaryrd}

\theoremstyle{definition}
\newtheorem{definition}{Definition}[section]
\newtheorem{theorem}{Theorem}[section]
\newtheorem*{optTheorem}{Theorem}
\newtheorem{proposition}{Proposition}[section]
\newtheorem{lemma}{Lemma}[section]
\newtheorem{corollary}{Corollary}[section]
\theoremstyle{remark}
\newtheorem*{remark}{Remark}
\newtheorem*{sketchproof}{Sketch of Proof}
\renewcommand{\proofname}{Proof}

\newcommand{\range}{\textrm{rng}}
\newcommand{\domain}{\textrm{dom}}

\newcommand{\Dashv}{\rotatebox[origin=c]{180}{\ensuremath\vDash}}

\newcommand{\questeq}{\stackrel{?}{=}}

\newcommand{\semanticalImply}[2]{#1\vDash{}#2}
\newcommand{\tautology}[1]{\vDash{}#1}

\newcommand{\sat}[3]{\vDash_{\mathfrak{#1}}#2[#3]}
\newcommand{\sentSat}[2]{\vDash_{\mathfrak{#1}}#2}
\newcommand{\unsat}[3]{\nvDash_{\mathfrak{#1}}#2[#3]}
\newcommand{\sentunsat}[2]{\nvDash_{\mathfrak{#1}}#2}
\newcommand{\assignSat}[3]{\vDash_{\mathfrak{#1}}#2\llbracket #3 \rrbracket}

\newcommand{\naturalSet}{\mathbb{N}}
\newcommand{\realSet}{\mathbb{R}}
\newcommand{\naturalStruct}{\mathfrak{N}}
\newcommand{\realStruct}{\mathfrak{R}}

% \newcommand{\iff}{\Leftrightarrow}

\newcommand{\frakA}{\mathfrak{A}}
\newcommand{\frakB}{\mathfrak{B}}


\title{Mathematical Logic}
\author{\textsc{YBiuR}}
\date{A long long time ago in a far far away SJTU}


\begin{document}
\setlength{\parskip}{1em}
\setlength{\parindent}{0em}

\frontmatter
\maketitle
\chapter*{Preface}
\emph{“道可道,非常道。”}

\mainmatter
\tableofcontents

\chapter{Set Theory}
\emph{Salute to Discrete Mathematics.}

\section{Sets}
\label{sec:Sets}
(Informally,) A set is a collection of elements.

\section{Relations and Functions}
\label{sec:RelationsAndFunctions}

\subsection{Relations}
\label{sub:Relation}
Given $n$ sets $A_1,\dots,A_n$, a relation $R$ over them is a subset of $A_1 \times \cdots \times A_n$.

\begin{definition}[Binary Relation]
    \label{def:BinaryRelation}
    A \textbf{binary relation} $R$ is a relation over $A \times B$ given some $A$ and $B$.
    \begin{itemize}
        \item The \textbf{domain} of $R$, denoted by $\textrm{dom}(R)$, is $\{ x | \exists y: \langle x,y \rangle \in R \}$
        \item The \textbf{range} of $R$, denoted by $\textrm{rng}(R)$, is $\{ y | \exists x: \langle x,y \rangle \in R \}$
    \end{itemize}
\end{definition}

A binary relation is
\begin{itemize}
    \item \textbf{Reflexive} iff $\langle x, x \rangle \in R$ for each $x \in A$.
    \item \textbf{Symmetric} iff $\langle x,y \rangle \in R \longrightarrow \langle y,x \rangle \in R$.
    \item \textbf{Transitive} iff $\langle x,y \rangle \in R \land \langle y,z \rangle \in R \longrightarrow \langle x,z \rangle \in R$.
\end{itemize}

A relation is an \textbf{equivalence relation} if it is reflexive, symmetric and transitive.

\subsection{Functions}
\label{sub:Function}
\begin{definition}[Functions]
    \label{def:Function}
    A \textbf{function} $f: A \rightarrow B$ is a binary relation over $A \times B$ satisfying:
    \begin{itemize}
        \item Its domain is $A$.
        \item For each $x \in A$, there is a unique $y \in B$ such that $\langle x, y \rangle \in f$.
    \end{itemize}
\end{definition}

\begin{definition}[One-to-One (injective)]
    \label{def:Injective}
    A function $f: A \rightarrow B$ is \textbf{one-to-one (injective)} if for each $x, y \in A$,
    \[ f(x) = f(y) \Longrightarrow x = y. \]
\end{definition}

\begin{definition}[Onto (surjective)]
    \label{def:Surjective}
    A function is \textbf{onto (surjective)} if for each $y \in B$, there is some $x \in A$ such that $f(x) = y$.
\end{definition}

\begin{definition}[One-to-One Correspondence (bijective)]
    \label{def:Bijective}
    A function is a \textbf{one-to-one correspondence (bijective)} between $A$ and $B$ if $f$ is both one-to-one and onto.
\end{definition}

\subsection{Finite Sets}
\label{sub:FiniteSets}
\begin{definition}[Finite sets]
    \label{def:FiniteSets}
    The set $X$ is \textbf{finite} if there is a natural number $n$ and a one-to-one correspondence between $X$ and $\{0, 1, \dots, n\}$。

    The set $X$ is \textbf{infinite} if it is not finite.
\end{definition}

We use the notion of one-to-one correspondence between infinite sets to talk about the ``sizes'' of these infinite sets.

\begin{definition}[Enumerable Sets]
    \label{def:EnumerableSets}
    The set $X$ is \textbf{enumerable} if there is a one-to-one correspondence between $X$ and $\mathbb{N}$.
\end{definition}

\begin{definition}[Listing of Sets]
    \label{def:ListingOfSets}
    Let $A$ be a set. The sequence $a_0, a_1, \dots, a_n, \dots$ is a \textbf{listing} of $A$ if
    \begin{enumerate}
        \item $a_i \in A$ for each $a_i$.
        \item Every member of $A$ is equal to $a_n$ for some $n \in \mathbb{N}$.
    \end{enumerate}
\end{definition}

\begin{theorem}
    \label{thm:ListingAndEnumerableSets}
    The set $A$ is enumerable iff there is some listing without repetitions of $A$.
\end{theorem}
\begin{proof}
    $\Rightarrow$.

    $\exists f: A \mapsto \mathbb{N}$, and $f$ is a one-to-one correspondence. Therefore there exists an inverse function $f^{-1}: \mathbb{N} \mapsto A$.

    Therefore we can construct a listing
    $$ f^{-1}(0), f^{-1}(1), f^{-1}(2), \cdots $$

    $\Leftarrow$.

    Let $g(i) = a_i$. $g$ is one-to-one because the listing has no identical elements. $g$ is onto because every member in $A$ is equal to some $a_n$ in the listing.

    $f = g^{-1}$.
\end{proof}

\begin{definition}[Countable Sets]
    \label{def:CountableSets}
    A set is \textbf{countable} if it is finite or enumerable. A set is \textbf{uncountable} if it is not countable.
\end{definition}

\begin{theorem}
    \label{thm:CountableSets}
    A set $X$ is countable iff there is a one-to-one mapping $f: X \mapsto \mathbb{N}$.

    A set $X$ is countably infinite iff it is enumerable.
\end{theorem}
\begin{sketchproof}
    $\Rightarrow$: Straightforward. Follows immediately from the definition of finite \ref{sub:FiniteSets} and enumerable \ref{def:EnumerableSets} sets.
    
    $\Leftarrow$: Let $B = \textrm{rng}(f)$. Notice that $f$ is a one-to-one correspondence between $X$ and $B$.
    \begin{itemize}
        \item If $B$ is finite, then there exists some function $h: B \mapsto \{1,2,\dots,n\}$ s.t. $h$ is a one-to-one correspondence. Let $g = h \circ f$, then $g$ is a one-to-one correspondence from $X$ to $\{1,2,\dots,n\}$. Therefore by definition \ref{def:FiniteSets}, $X$ is finite and is thus countable.
        
        \item If $B$ is infinite, since $B \subseteq \mathbb{N}$, we can sort elements of $B$ in ascending order:
        \[ b_0 < b_1 < b_2 < \cdots < b_n < \cdots \]
        
        We have constructed a listing of $B$, and therefore $B$ is enumerable. Therefore by definition \ref{def:EnumerableSets}, there exists a one-to-one correspondence $h:B \mapsto \mathbb{N}$. Construct $g = h \circ f$, then $g$ must also be a one-to-one correspondence between $X$ and $\mathbb{N}$. Therefore by definition \ref{def:EnumerableSets}, the set $X$ is enumerable and thus countable.
    \end{itemize}
\end{sketchproof}

\begin{theorem}
    \label{thm:ListingAndCountableSets}
    The set $A$ is countable and nonempty iff there is some listing with possible repetitions of $A$.
\end{theorem}
\begin{sketchproof}
    $\Rightarrow$. If $A$ is enumerable, we are done by Theorem \ref{thm:ListingAndEnumerableSets}. If $A$ is finite, we can construct a listing $a_0, a_1, \cdots, a_n, a_n, \cdots$.

    $\Leftarrow$. Let $a_0, a_1, \cdots, a_n, \cdots$ be the listing. Define $f:A \mapsto \mathbb{N}$ by $f(a_i) = \min_j\left\{ j: a_j=a_i \right\}$. $f$ is one-to-one and by Theorem \ref{thm:CountableSets} we can conclude that $A$ is countable.
\end{sketchproof}

\begin{proposition}
    \label{prop:SubsetOfEnumerableSetsIsCountable}
    If $A$ is enumerable and $B \subseteq A$, then $B$ is countable.
\end{proposition}

\begin{theorem}
    \label{thm:LotsOfCountableSets}
    ~{}
    \begin{itemize}
        \item If $A$ and $B$ are countable, then $A \cup B$, $A \cap B$, and $A \times B$ are all countable.
        \item If each of $A_0, \dots, A_n, \dots$ is countable, then the union of these sets is also countable.
        \item If $A$ is countable and nonempty, then the set of all finite sequences of members of $A$ is countable.
    \end{itemize}
\end{theorem}

\subsection{Constructing Infinite Sets}
\label{sub:ConstructingInfiniteSets}

\begin{definition}[Characteristic Functions]
    \label{def:CharacteristicFunctions}
    Let $X$ be a set and let $A \subseteq X$. FOr any $a \in A$, let
    $$ C_A(a) = \begin{cases}
        1, &\quad a \in A\\
        0, &\quad a \notin A
    \end{cases} $$
    $C_A(a)$ is called the characteristic function of $A$.
\end{definition}

\begin{definition}[Power Sets]    \label{def:PowerSets}
    Let $A$ be a set, the power set of $A$ is
    \[ \mathcal{P}(A) = \{ X|X \subseteq A \} \]
\end{definition}

\begin{theorem}[Cantor's Theorem]
    \label{thm:CantorsTheorem}
    $\mathcal{P}(\mathbb{N})$ is uncountable.
\end{theorem}
\begin{proof}
    Proof by contradiction. Suppose $\mathcal{P}(\mathbb{N})$ is countable. Obviously $\mathcal{P}(\mathbb{N})$ is not finite, so it must be enumerable.

    By Theorem \ref{thm:ListingAndEnumerableSets}, we can construct a listing with no repetitions.

    We can list all subsets of $A$ using characteristic functions.
    \[\begin{bmatrix}
        \emptyset & \begin{bmatrix} 0 & 0 & \cdots & 0 \end{bmatrix}\\
        \{0\} & \begin{bmatrix} 1 & 0 & \cdots & 0 \end{bmatrix}\\
        \{1\} & \begin{bmatrix} 0 & 1 & \cdots & 0 \end{bmatrix}\\
        \{0, 1\} & \begin{bmatrix} 1 & 1 & \cdots & 0 \end{bmatrix}\\
        \cdots & \cdots\\
    \end{bmatrix}\]

    We can then select all the bits along the diagonal and flip these bits to form a new bit sequence. This sequence can also be interpreted as a listing of some subset of $\mathcal{P}(\mathbb{N})$. However, this listing cannot exist in the listed listings, because it has at least one bit that is different from any existing listings. This listing fails to enumerate all subsets of $\mathcal{P}(\mathbb{N})$, and therefore the set $\mathcal{P}(\mathbb{N})$ is uncountable.
\end{proof}

\begin{corollary}
    $\mathbb{R}$ is uncountable.
\end{corollary}

\subsection{Domination of Sets}

\begin{definition}[Domination of Sets]
    \label{def:SetDomination}
    ~{}
    \begin{itemize}
        \item $A \preceq B$ if there is a one-to-one function $f: A \mapsto B$.
        \item $ A \prec B$ if $A \preceq B$ but $B \npreceq A$.
        \item $A \equiv B$ if $A \preceq B$ and $B \preceq A$.
    \end{itemize}
\end{definition}

\begin{theorem}[Cantor-Schr\"oder-Bernstein]
    \label{thm:CSB}
    $A \equiv B$ iff there is a one-to-one correspondence between $A$ and $B$.
\end{theorem}
\begin{proof}
    Refer to supplementary material on Canvas.
\end{proof}

\begin{theorem}[Cantor's Theorem]
    \label{thm:Cantor}
    For every set $A$,
    \[ A \prec \mathcal{P}(A) \]
\end{theorem}
\begin{proof}
    We need to prove 1) $A \preceq \mathcal{P}(A)$ and 2) $\mathcal{P}(A) \npreceq A$. 1) is easy because $\forall a \in A$, $\{a\} \subseteq A$ and therefore $\{a\} \in \mathcal{P}(A)$.

    To prove 2), assume $\mathcal{P}(A) \preceq A$, by theorem \ref{thm:CSB} we have $\mathcal{P}(A) \equiv A$, and therefore there should exist a one-to-one correspondence $f: A \mapsto \mathcal{P}(A)$.

    Let
    \[ B = \{x \in A|x \notin f(x)\} \]

    Then $B \subseteq A$, and $B \in \mathcal{P}(A)$. Since $f$ is onto, there must exist some $b \in A$ s.t. $f(b) = B$.

    \begin{enumerate}
        \item If $b \in f(b) = B$. Then $b \in f(b) \Rightarrow b \notin f(b)$. Boom.
        \item If $b \notin f(b)$. Then $b \in B = f(b)$. Boom.
    \end{enumerate}

    Therefore $\mathcal{P}(A) \npreceq A$.
\end{proof}
\begin{corollary}
    For every $A$ there exists a $B$ s.t. $A \prec B$.
\end{corollary}
\begin{corollary}
    $\mathcal{P}(\mathbb{N})$ is uncountable.
\end{corollary}
\begin{proof}
    $\mathcal{P}(\mathbb{N}) \npreceq \mathbb{N}$.
\end{proof}

\subsection{Sets of Functions}

\begin{definition}[Set of Functions]
    \label{def:SetOfFunction}
    $~^{A} B$ is the set of all functions that map $A$ into $B$.
\end{definition}
\begin{remark}
    The set $~^A\{0,1\}$ is the set of all characteristic functions (\ref{def:CharacteristicFunctions}).
\end{remark}

\begin{theorem}
    There is a one-to-one correspondence between $~^{\mathbb{N}}\{0,1\}$ and $\mathcal{P}(\mathbb{N})$.
\end{theorem}
\begin{remark}
    A generalized version: there is a one-to-one correspondence between $~^{\mathbb{A}}\{0,1\}$ and $\mathcal{P}(\mathbb{A})$
\end{remark}
\begin{corollary}
    \label{cor:RealIsUncountable}
    $\mathbb{R}$ is uncountable.
\end{corollary}
\begin{theorem}
    \label{thm:NNisUncountable}
    $~^\mathbb{N}\mathbb{N}$ is uncountable.
\end{theorem}
\begin{proof}
    Assume $~^\mathbb{N}\mathbb{N}$ is countable. Since $~^\mathbb{N}\{0,1\} \subseteq ~^\mathbb{N}\mathbb{N}$. $~^\mathbb{N}\{0,1\}$ must be countable. Boom.
\end{proof}

\subsection{Generalized Continuum Hypothesis}

\begin{center}
    \textbf{Is there a set $A$ s.t. $\mathbb{N} \prec A \prec \mathbb{R} \equiv \mathcal{P}(\mathbb{N})$?}
\end{center}

\chapter{The Informal Notions of Algorithms}

\emph{“形式语言与自动机课程速成。”}

\section{Algorithms}

\begin{definition}[Algorithms (Informal)]
    \label{def:Algorithm}
    An algorithm is a \textbf{finite ordered list} of instructions.
\end{definition}

Possible outcomes of running an algorithm
\begin{itemize}
    \item The algorithm does not halt
    \item The algorithm halts
    \begin{itemize}
        \item In an erroneous state (fails)
        \item Gives valid outputs
    \end{itemize}
\end{itemize}
Cases other than the algorithm giving valid outputs are collectively identified as ``no output''.

\subsection{Algorithms for Determining Membership}
\label{sub:AlgoForDeterminingMembership}

\begin{definition}[Algorithms for Determining Membership]
    \label{def:AlgoForDeterminingMembership}
    An algorithm for \emph{determining membership} in a set $A \subseteq \mathbb{N}$ has an input, and two possible outputs ``yes'' and ``no''. If the algorithm is run on input $n$, it will halt in finite steps with output ``yes'' if $n \in A$ and ``no'' if $n \notin A$.
\end{definition}

\begin{definition}[Effectively Decidable Sets]
    \label{def:EffectivelyDecidableSet}
    Let $A$ be a subset of $\mathbb{N}$. $A$ is \textbf{effectively decidable} if there is an algorithm for determining membership of $A$.
\end{definition}

\begin{itemize}
    \item $\mathbb{N}$ is effectively decidable. (Always ``yes'')
    \item $\emptyset$ is effectively decidable. (Always ``no'')
\end{itemize}

\begin{theorem}
    If $A$ and $B$ are effectively decidable subsets of $\mathbb{N}$, then $\mathbb{N}\backslash A$, $A \cap B$ and $A \cup B$ are all effectively decidable.
\end{theorem}

Algorithms can have different kinds of outputs and inputs.

\begin{definition}[Diophantine Equations]
    Consider polynomials with integer coefficients (and any number of variables), a \textbf{diophantine equation} is an equation of the form $p=0$, where $p$ is sunch a polynomial. (e.g., $3x^2 + 5xy - 2z^4 +3 = 0$)
\end{definition}

\paragraph{Hilbert's 10th Problem.} Is there an algorithm for determining whether or not diophantine equations have integer solutions?

\subsection{Algorithm for Listing Members of Sets}
\label{sub:AlgoForListingMembersOfSets}

\begin{definition}[Algorithm for Listing Members of Sets]
    \label{def:AlgoForListingMembersOfSets}
    Let $A \subseteq \mathbb{N}$. An algorithm for \textbf{listing the members of $A$} prints (or enumerates) a list of numbers $a_0,a_1,\dots$ s.t.
    \begin{itemize}
        \item $a_n \in A$ for all $n \in \mathbb{N}$
        \item If $a \in A$, then $a = a_n$ for some $n$
    \end{itemize}
\end{definition}
\begin{remark} ~{}
    \begin{itemize}
        \item No input.
        \item Repetitions in the listing are permitted.
        \item If $A$ is finite, the algorithm may terminate or run forever (by iterating endlessly over the finite set).
        \item If $A$ is infinite, the algorithm should run forever.
    \end{itemize}
\end{remark}

\begin{definition}[Effectively Enumerable Sets]
    \label{def:EffectivelyEnumerableSet}
    A set $A \subseteq \mathbb{N}$ is \textbf{effectively enumerable} if there is an algorithm for listing its members.
\end{definition}

\subsection{Effectively Computable Functions}
\label{sub:EffectivelyComputableFunction}

\begin{definition}[Partial Functions]
    \label{def:PartialFunction}
    A function $f:\mathbb{N}\mapsto\mathbb{N}$ is a \textbf{partial number-theoretic function} if its domain is a subset of $\mathbb{N}$.
\end{definition}
\begin{definition}[Total Functions]
    \label{def:TotalFunction}
    A partial function whose domain is all of $\mathbb{N}$ is a total function.
\end{definition}

\begin{definition}[Notation for Definability]
    For a partial function $f$
    \begin{itemize}
        \item $n \in \textrm{dom}(f)$, then $f(n)\downarrow$
        \item $n \notin \textrm{dom}(f)$, then $f(n)\uparrow$
    \end{itemize}
\end{definition}

\begin{definition}[Effectively Computable Functions]
    \label{def:EffectivelyComputableFunction}
    The partial function $f$ is \textbf{effectively computable} if there is an algorithm $\mathcal{A}$ s.t. on input $n$
    \begin{itemize}
        \item Prints output $f(n)$ if $f(n)\downarrow$
        \item No output if $f(n)\uparrow$
    \end{itemize}
    $\mathcal{A}$ is called an algorithm for computing $f$.
\end{definition}

\begin{remark}
    The definition of effectively decidable and enumerable sets, and effectively computable functions can be extended to $\mathbb{N}^k$.
\end{remark}

\subsection{Relations among Effectively Decidable Sets, Effectively Enumerable Sets and Effectively Computable Functions}

\begin{proposition}
    Let $A \subseteq \mathbb{N}$. If $A$ is effectively decidable then $A$ is effectively enumerable.
\end{proposition}

\begin{proposition}
    Let $A \subseteq \mathbb{N}$. $A$ is effectively decidable iff the characteristic function (\ref{def:CharacteristicFunctions}) $C_A$ is effectively computable.
\end{proposition}

\begin{proposition}
    Let $f:\mathbb{N}\mapsto\mathbb{N}$. If $f$ is an effectively computable function, then $\textrm{rng}(f)$ is an effectively enumerable set.
\end{proposition}

\begin{proposition}
    Combining the above two propositions, we have: Let $A$ be a non-empty subsect of $\mathbb{N}$, $A$ is effectively enumerable iff there is an effectively computable function whose range is $A$.
\end{proposition}

\begin{proposition}
    If $A\neq\emptyset$ is an effectively enumerable subset of $\mathbb{N}$, then there is an effectively computable function $f$ with $\range(f)=A$.
\end{proposition}

\section{Enumerability of Algorithms}
\label{sec:EnumerabilityOfAlgos}

\begin{theorem}
    There are only \emph{enumerably many} algorithms.
\end{theorem}
\begin{corollary}
    There are only enumerably many partial effectively computable functoins. Since there are uncountably many total number-theoretic functions, there must exist total number-theoretic functions that are not effectively computable.
\end{corollary}
\begin{corollary}
    There are only enumerably many effectively enumerable subsets of $\mathbb{N}$. Since there are uncountably many subsets of $\mathbb{N}$, there must exist subsets of $\mathbb{N}$ that are not effectively enumerable.
\end{corollary}

\subsubsection{Listing of Effectively Computable Total Function}

Suppose $f_0,\dots$ is a listing of all effectively computable total function (this listing is unknown, we only know a listing of all computable partial functions). Let $g$ be a function such that

\[ g(m,n) = f_m(n) \]

$g$ is not effectively computable.

\begin{proof}
    By diagonal argument. Let $h(n) = g(n,n) + 1$. If $g$ is effectively computable, then so is $h$. Then $h$ must be in the listing $\{f_n\}$. Suppose $f_k = h$. Then
    \[ f_k(k) = h(k) = g(k,k)+1 = f_k(k) + 1 \]
    噔噔咚。
\end{proof}

\section{The Halting Problem}

Is there an algorithm for deciding whether any algorithm halts on given input or not?

\begin{theorem}
    \label{thm:HaltingProblem}
    Let $\mathcal{A}_0,\dots,\mathcal{A}_n,\dots$ be a listing of algorithm with space for only one input. There does not exist an algorithm $\mathcal{H}$ with space for two number inputs s.t. on input $(m,n)$
    \begin{itemize}
        \item Halts with output ``yes'' if $\mathcal{A}_m$ halts on input $n$.
        \item Halts with output ``no'' if $\mathcal{A}_m$ does not halt on input $n$
    \end{itemize}
\end{theorem}
\begin{proof}
    Construct $\mathcal{C}(n)$ by:
    \begin{itemize}
        \item Run $\mathcal{H}(n,n)$
        \item If ``yes'' then loop forever
        \item If ``no'' then print ``yes'' and return
    \end{itemize}

    Then $\mathcal{C}$ must be in the listing $\{\mathcal{A}_n\}$. Suppose $\mathcal{A}_k = \mathcal{C}$.
    \begin{itemize}
        \item $\mathcal{C}(k)$ halts $\Leftrightarrow$ $\mathcal{H}(k,k)$ prints ``no'' $\Leftrightarrow$ $\mathcal{A}_k(k) = \mathcal{C}(k)$ does not halt.
    \end{itemize}
    噔噔咚。
\end{proof}
\chapter{Sentential Logic}

\section{Grammar}
\label{sec:Gramma}

\subsection{Symbols}
\label{sub:Symbols}
\begin{itemize}
    \item Logical Symbols
    \begin{itemize}
        \item Sentential Connectives
        \begin{itemize}
            \item $\neg$
            \item $\wedge$
            \item $\vee$
            \item $\rightarrow$
            \item $\leftrightarrow$
        \end{itemize}
        \item Parentheses
    \end{itemize}
    \item Non-logical Symbols: An enumerable set of elements
\end{itemize}

\subsection{Expressions}
\label{sub:Expressions}

\begin{definition}[Expression]
    \label{def:Expression}
    An expression is a finite sequence of symbols.
\end{definition}
\begin{remark}
    The set of all expressions is enumerable.
\end{remark}

We often use Greek alphabets $\alpha,\beta,\dots$ to represent expressions.

\subsection{Well-Formed Formulas}
\label{sub:WellFormedFormulas}

\begin{definition}[Well-Formed Formula]
    \label{def:WFF}
    A \textbf{well-formed formula} (or formula or wff) is an expression built up from sentence symbols by applying some finite times of \emph{formula building operations}
\end{definition}

\begin{definition}[Formula Building Operations]~{}
    \begin{itemize}
        \item $\xi_{\neg}(\alpha) = (\neg\alpha)$
        \item $\xi_{\wedge}(\alpha,\beta) = (\alpha \wedge \beta)$
        \item $\xi_{\vee}(\alpha,\beta) = (\alpha\vee\beta)$
        \item $\xi_{\rightarrow}(\alpha,\beta) = (\alpha\rightarrow\beta)$
        \item $\xi_{\leftrightarrow}(\alpha,\beta) = (\alpha\leftrightarrow\beta)$
    \end{itemize}
\end{definition}
\begin{remark}
    Do NOT omit the parentheses.
\end{remark}

\begin{definition}[Well-Formed Sequences of Expressions]
    \label{def:WellFormedSeqOfExpr}
    A \textbf{well-formed sequence of expressions} is a finite sequence $\alpha_1,\alpha_2,\dots,\alpha_n$ of expressions such that each $\alpha_i$ is either
    \begin{itemize}
        \item A sentence symbol
        \item $(\neg\alpha_j)$ for some $j < i$
        \item $(\alpha_j \wedge \beta_k)$ for some $j,k<i$
        \item $(\alpha_j \vee \beta_k)$ for some $j,k<i$
        \item $(\alpha_j \leftarrow \beta_k)$ for some $j,k<i$
        \item $(\alpha_j \leftrightarrow \beta_k)$ for some $j,k<i$
    \end{itemize}
\end{definition}

\begin{proposition}
    An expression $\alpha$ is a well-formed formula iff there is a well-formed sequence $(\alpha_1,\dots,\alpha_n)$ s.t. $\alpha = \alpha_n$
\end{proposition}

\subsection{The Induction Principle}
\label{sub:Induction}

Well-formed formulas are a form of inductive definitions with
\begin{itemize}
    \item Basic building blocks
    \item Closing operations
\end{itemize}

\begin{theorem}[The Induction Principle]
    \label{thm:InductionPrinciple}
    Let $S$ be a set of wffs ($S \subseteq W$), if
    \begin{enumerate}
        \item Every sentence symbol is in $S$
        \item For each wff $\alpha$ and $\beta$, if $\alpha$ and $\beta$ are in $S$ then each of the following are in $S$
        \begin{itemize}
            \item $(\neg \alpha)$
            \item $(\alpha \wedge \beta)$
            \item $(\alpha \vee \beta)$
            \item $(\alpha \rightarrow \beta)$
            \item $(\alpha \leftrightarrow \beta)$
        \end{itemize}
    \end{enumerate}
    Then $S$ is the set of \emph{all wffs} ($S = W$).
\end{theorem}

We can see an example of Indunction.

\begin{proposition}
    Every wff has the same number of left parentheses as right parenthesis
\end{proposition}
\begin{proof}
    Let $S\triangleq \{\alpha|\alpha\text{has equal number of left and right parentheses}\}$.
    \begin{itemize}
        \item[Base] $\alpha = A$. Straightforward. Sentense symbols do not have parenthesis
        \item[Step]
        \begin{enumerate}
            \item Let $\beta \in S$, $\alpha = (\neg \beta) \in S$.
            \item Let $\alpha_1, \alpha_2 \in S$, $\alpha = (\alpha_1 \wedge \alpha_2) \in S$.
            \item $\cdots$
        \end{enumerate} 
    \end{itemize}
\end{proof}

\subsection{Parsing Formulas}
\label{sub:ParsingFormulas}

The induction principle actually gives an algorithm for parsing formulas.

On input expression $\alpha$

\begin{enumerate}
    \item If is leaf node, we are done. Return.
    \item The first symbol must be `('.
    \item If the second symbol is `$\neg$', then expect an non-empty expression $\beta$ and parse $\beta$.
    \item If the second symbol is not `$\neg$', then expect a non-empty expression $\beta_1$, an operator and another expression $\beta_2$.
\end{enumerate}

\subsection{Abbreviations}
\label{sub:Abbreviations}

\begin{itemize}
    \item The outermost parentheses can be omitted.
    \item $\neg$ appplies to as little as possible, with the highest precedence.
    \item $\wedge$ and $\vee$ apply to as little as possible, subject to $\neg$.
    \item $\rightarrow$ and $\leftrightarrow$ apply to as little as possible, subject to other operators.
    \item When handling operators with the same precedence, grouping is always to the right. $A \rightarrow B \rightarrow C= (A \rightarrow (B\rightarrow C))$.
\end{itemize}

\section{Semantics}
\label{sec:Semantics}

\subsection{Truth Assignments}

Consider a math domain $\{T,F\}$ of truth values

\begin{itemize}
    \item T is called truth
    \item F is called falsity
\end{itemize}

\begin{definition}[Truth Assignment]
    A truth assignment for a set $\mathcal{S}$ of sentence symbols is a function
    \[ v:\mathcal{S}\mapsto\{T,F\} \]
\end{definition}

\begin{definition}[Extended Truth Assignment]
    Let $\bar{\mathcal{S}}$ be the set of wffs that can be built up from $\mathcal{S}$ by formula-building operations. Let $v$ be a truth assignment for $\mathcal{S}$. An \textbf{extension} $\bar{v}$ of $v$
    \[ \bar{v}:\bar{\mathcal{S}}\mapsto \{T,F\} \]
    assigns truth values to every wff in $\mathcal{S}$ s.t.
    \begin{itemize}
        \item $\bar{v}(\alpha) = v(\alpha)$ if $\alpha \in \mathcal{S}$
        \item $\bar{v}(\neg(\alpha))$ is T if $\bar{v}(\alpha)$ is F and F otherwise.
        \item $\bar{v}((\alpha\wedge\beta))$ is T if $\bar{v}(\alpha)$ is T and $\bar{v}(\beta)$ is T and F otherwise.
        \item $\bar{v}((\alpha\vee\beta))$ is T if $\bar{v}(\alpha)$ is T or $\bar{v}(\beta)$ is T and F otherwise.
        \item $\bar{v}((\alpha\to\beta))$ is F if $\bar{v}(\alpha)$ is T and $\bar{v}(\beta)$ is F and T otherwise.\footnote{Emphasizes the promise of a condition implying a consequence. If the condition is falsy then no guarantee for the consequence. \emph{“骗你是小狗”}}
        \item $\bar{v}((\alpha\leftrightarrow\beta))$ is T if $\bar{v}(\alpha) = \bar{v}(\beta)$ and is F otherwise.
    \end{itemize}
\end{definition}

\begin{theorem}[Determinacy of Truth Assignments]
    \label{thm:DeterminacyofTruthAssignments}
    For every $v_1$ and $v_2$ and wff $\alpha$, if
    \[ v_1(A) = v_2(A) \]
    for every sentence symbol that occurs in $\alpha$, then
    \[ \bar{v}_1(\alpha) = \bar{v}_2(\alpha) \]
\end{theorem}

\begin{remark}
    To determine the value of $\bar{v}(\alpha)$, we only need to know the value of $v$ on the sentence symbols that occur in $\alpha$. This leads to the method of \textbf{truth tables}.
\end{remark}

\subsection{Satisfiability}
\label{sub:Satisfiability}

We first introduce some new notations. We use captial Greek letters, $\Delta$, $\Sigma$, etc. to represent sets of wffs. And we use $\Sigma;\alpha$ to represent $\Sigma \cup \{\alpha\}$.

\begin{definition}~{}
    \begin{itemize}
        \item $v$ satisfies $\alpha$ if $\bar{v}(\alpha) = T$
        \item $v$ satisfies $\Sigma$ if $\bar{v}(\alpha) = T$ for every $\alpha \in \Sigma$.
    \end{itemize}
\end{definition}

\begin{definition}[Satisfiability]~{}
    \label{def:Satisfiability}
    \begin{itemize}
        \item $\alpha$ is satisfiable if there exists some $v$ that satisfies $\alpha$
        \item $\Sigma$ is satisfiable if there exists some $v$ that satisfies $\Sigma$
    \end{itemize}
\end{definition}

\begin{remark}
    Every $v$ satisfies $\emptyset$. Because $v$ satisfies $\emptyset$ iff
    \[ \forall \alpha, \alpha\in\emptyset \Longrightarrow \bar{v}(\alpha) = T \]
    The assumption itself is false, and therefore the consequence is always true.
\end{remark}

\subsection{Semantic Implications}
\label{sub:SemanticImplications}

\begin{definition}
    A set of wffs $\Sigma$ semantically implies $\alpha$ when every truth assignment satisfying $\Sigma$ also satisfies $\alpha$.
    \begin{itemize}
        \item $\Sigma \vDash \alpha$ denotes that $\Sigma$ implies $\alpha$
        \item $\alpha \vDash \beta$ denotes that $\{\alpha\} \vDash \beta$
    \end{itemize}
    If $\Sigma \vDash \alpha$, we call $\alpha$ a semantic consiquence of $\Sigma$.
\end{definition}

Semantic implication is also referred to as tautological implication.

\begin{remark}
    Note that $\{\alpha,\neg\alpha\} \vDash \beta$ also holds, because the assumption does not hold, so the consequence trivially holds.
\end{remark}

\subsection{Tautologies}
\label{sub:Tautologies}

\begin{definition}[Tautologies]
    \label{def:Tautology}
    $\alpha$ is a tautology if $\emptyset \vDash \alpha$, denoted by $\vDash \alpha$.
\end{definition}
\begin{remark}
    ~{}
    \begin{itemize}
        \item $\alpha$ is a tautology iff $\forall v$, $\bar{v}(\alpha)=T$.
        \item $\alpha$ is a tautology iff $\neg \alpha$ is \emph{not satisfiable}
        \item $\alpha$ is satisfiable iff $\neg \alpha$ is not a tautology.
    \end{itemize}
\end{remark}

\subsection{Semantic Equivalence}
\label{sub:SemanticEquivalence}

\begin{definition}[Semantic Equivalence]
    Two wffs $\alpha$ and $\beta$ are semantically equivalent if both $\alpha\vDash\beta$ and $\beta\vDash\alpha$ hold. We use $\alpha\vDash\Dashv\beta$
\end{definition}

\begin{proposition}
    The following are equivalent
    \begin{itemize}
        \item $\alpha$ and $\beta$ are semantically equivalent
        \item For every $v$, $\bar{v}(\alpha)=\bar{v}(\beta)$
        \item $\alpha$ and $\beta$ have the same truth table
    \end{itemize}
\end{proposition}

We can use semantic equivalence to derive truthfulness of wffs. If $\alpha\vDash\models\beta$, we can freely exchange one for the other in deriving the truth of some formula $\sigma$ where $\alpha$ and/or $\beta$ occur.

Remember not to mix syntax and semantics

\begin{itemize}
    \item $\alpha=T$ is incorrect. Use $\bar{v}(\alpha)=T$.
    \item $v(\Sigma)=T$ is incorrect. Use $v$ satisfies $\Sigma$.
\end{itemize}

\subsection{Properties of Satisfaction and Implication}

\begin{itemize}
    \item If $\alpha$ is a tautology, then $\Sigma\vDash\alpha$ for every $\Sigma$
    \item If $\alpha\in\Sigma$ then $\Sigma\vDash\alpha$
    \item If $\Sigma\vDash\alpha$ and $\Sigma\vDash\alpha\to\beta$ then $\Sigma\vDash\beta$
    \item If $\Sigma\vDash\alpha$ and $\alpha\vDash\beta$ then $\Sigma\vDash\beta$.
    \item If $\Sigma\vDash\alpha$ then for all $\beta$, $\Sigma\vDash\beta\to\alpha$
    \item If $\Sigma\vDash\alpha$ and $\Sigma\vDash\beta$ then $\Sigma\vDash \alpha\wedge\beta$
    \item If $\Sigma\vDash\alpha$ or $\Sigma\vDash\beta$ then $\Sigma\vDash \alpha\vee\beta$
    \item $\Sigma\nvDash\alpha$ iff $\Sigma\cup\{\neg\alpha\}$ is satisfiable
    \item $\Sigma \vDash \alpha$ iff $\Sigma \cup \{\neg \alpha\}$ is not satisfiable
    \item $\Sigma \vDash \alpha \to \beta$ iff $\Sigma;\alpha \vDash \beta$
    \item If $\Sigma$ is not satisfiable, then for every $\alpha$, $\Sigma\vDash\alpha$
    \item If $\Sigma\vDash\alpha$ and $\Sigma\subseteq\Delta$ then $\Delta\vDash\alpha$
    \item If $\Sigma$ is satisfiable then every subset of $\Sigma$ is satisfiable
    \item If every subset of $\Sigma$ is satisfiable then $\Sigma$ is satisfiable.
    \item If every finite subset of $\Sigma$ is satisfiable then $\Sigma$ is satisfiable
    \item If $\Sigma\vDash\alpha$ then there is a finite subset $\Delta$ of $\Sigma$ such that $\Delta\vDash\alpha$
\end{itemize}

\section{Normal Forms}
\label{sec:NormalForms}

\subsection{Disjunctive Normal Forms}
\label{sub:DisjunctiveNormalForms}

\begin{definition}[Disjunctive Normal Form]
    \label{def:DisjunctiveNormalForm}
    The wff $\alpha$ is in \textbf{disjunctive normal form} if $\alpha=\gamma_1\vee\gamma_2\vee\cdots\vee\gamma_k$ where each $\gamma_i$ is a conjunction
    \[ \gamma_i = \beta_{i1}\wedge\beta_{i2}\wedge\cdots\wedge\beta_{in_i} \]
    where each $\beta_{ij}$ is either a sentence symbol or the negation of a sentence symbol
\end{definition}

\subsection{Conjunctive Normal Forms}
\label{sub:ConjunctiveNormalForms}

\begin{definition}[Conjunctive Normal Form]
    \label{def:ConjunctiveNormalForm}
    The wff $\alpha$ is in \textbf{conjunctive normal form} if $\alpha=\gamma_1\wedge\gamma_2\wedge\cdots\wedge\gamma_k$ where each $\gamma_i$ is a disjunction
    \[ \gamma_i = \beta_{i1}\vee\beta_{i2}\vee\cdots\vee\beta_{in_i} \]
    where each $\beta_{ij}$ is either a sentence symbol or the negation of a sentence symbol
\end{definition}

\subsection{Completeness of Normal Forms}

\begin{theorem}[Completeness of DNFs]
    \label{thm:CompletenessOfDNF}
    Every wff is semantically equivalent to a wff in disjunctive normal form
\end{theorem}
\begin{proof}
    \begin{enumerate}
        \item Construct the disjunctive normal form truth tables
        \item Select all assignments $v$ s.t. $\bar{v}(\alpha)=T$
        \item Construct $\gamma_i$ where $\beta_{ij} = A_j$ if $A_j$ is assigned $T$ and $\beta_{ij} = \neg A_j$ otherwise
    \end{enumerate}
\end{proof}

\begin{theorem}[Completeness of CNFs]
    \label{thm:CompletenessOfCNF}
    Every wff is semantically equivalent to a wff in conjunctive normal form
\end{theorem}
\begin{proof}
    Given $\alpha=\gamma_1\vee\cdots\vee\gamma_n$ in DNF. If every $\gamma_i$ is a sentence symbol or the negation of a sentence symbol, we are done. Otherwise, there is some $\gamma_i = \beta_{i1}\wedge\beta_{i2}$. Then
    \[ \alpha \equiv (\beta_{i1} \wedge \beta_{i2}) \vee \alpha' \equiv (\beta_{i1}\vee\alpha') \wedge (\beta_{i2}\vee\alpha') \]
    where $\alpha'$ is the disjunction of $\{\gamma_k|k \neq i\}$

    And we recursively repeat the steps
\end{proof}

\section{Finite Satisfiability}
\label{sec:FiniteSatisfiability}

\begin{definition}[Finite Satisfiability]
    \label{def:FiniteSatisfiability}
    $\Sigma$ is \textbf{finitely satisfiable} if every finite subset of $\Sigma$ is satisfiable.
\end{definition}
\begin{remark}
    Suppose $\Delta$ is finitely satisfiable, and for every $\alpha$, $\alpha\in\Delta$ or $\neg\alpha\in\Delta$

    Then $\alpha\in\Delta$ iff $\neg\alpha\notin\Delta$
\end{remark}


\subsection{Compactness Theorem}

\begin{theorem}[Compactness Theorem]
    \label{thm:CompactnessTheorem}
    If $\Sigma$ is finitely satisfiable, then $\Sigma$ is satisfiable
\end{theorem}
\begin{sketchproof}
    We break down the proof into the following steps
    \begin{enumerate}
        \item From $\Sigma$, construct its superset $\Delta$ s.t.
        \begin{enumerate}
            \item $\Delta$ is finitely satisfiable
            \item For every wff $\alpha$, $\alpha\in\Delta$ or $\neg\alpha\in\Delta$
        \end{enumerate}
        \item Show that $\Delta$ is satisfiable, so that $\Sigma$ is also satisfiable
    \end{enumerate}
\end{sketchproof}

\begin{lemma}
    \label{lem:CompactnessLemma1}
    If $\Delta$ is finitely satisfiable, then for every wff $\alpha$,
    \begin{itemize}
        \item either $\Delta\cup\{\alpha\}$ is finitely satisfiable
        \item or $\Delta\cup\{\neg\alpha\}$ is finitely satisfiable
    \end{itemize}
\end{lemma}
\begin{proof}
    Prove by contradiction. Suppose neither $\Delta\cup\{\alpha\}$ nor $\Delta\cup\{\neg\alpha\}$ is finitely satisfiable. Then there are some finite subsets $\Delta_1 \subseteq \Delta\cup\{\alpha\}$ and $\Delta_2\subseteq\Delta\cup\{\neg\alpha\}$ which are not satisfiable. Notice that $\alpha$ must be in $\Delta_1$ and $\Delta_2$, or otherwise $\Delta_1$ and $\Delta_2$ would be satisfiable by finite satisfiability of $\Delta$. Therefore $\Delta_1 = \Delta'_1\cup\{\alpha\}$ and $\Delta_2'\cup\{\neg\alpha\}$.

    Then we construct $\Delta' = \Delta_1\cup\Delta_2 = \Delta_1'\cup\Delta_2'\cup\{\alpha,\neg\alpha\}$. Then there exists an assignment $v$ such that $v$ satisfies $\Delta_1'\cup\Delta_2'$, and that $v$ satisfies either $\alpha$ or $\neg\alpha$. Suppose $\bar{v}(\alpha)=T$, then $\Delta_1$ is satisfiable, which causes contradiction. Conversely, if $\bar{v}(\alpha) = F$ then $\bar{v}(\neg\alpha)=T$, then $\Delta_2$ is satisfiable, which is also a contradiction.
\end{proof}
\begin{remark}
    This lemma implies that we can expand $\Sigma$ for one step, by including $\alpha$ or $\neg\alpha$
\end{remark}


\begin{lemma}
    \label{lem:CompactnessLemma2}
    If $\Sigma$ is finitely satisfiable, then there is a $\Delta \supseteq \Sigma$ such that
    \begin{itemize}
        \item $\Delta$ is finitely satisfiable
        \item For each wff $\alpha$, $\alpha\in\Delta$ or $\neg\alpha\in\Delta$
    \end{itemize}
\end{lemma}
\begin{proof}
    The set of all wffs is enumerable, so we can write all wffs in a sequence
    \[ \alpha_1, \alpha_2,\dots,\alpha_n,\dots \]
    Then we can scan through the sequence and check if $\alpha_i$ can be added to $\Sigma$. Starting from $\Delta_0 = \Sigma$,
    \[\Delta_{i+1}\begin{cases}
        \Delta_i \cup \{\alpha_i\} &\quad\text{if it is finitely satisfiable}\\
        \Delta_i \cup \{\neg \alpha_i\} &\quad \text{otherwise}
    \end{cases}\]

    It can be proved by induction and Lemma~\ref{lem:CompactnessLemma1} that $\forall i,\Delta_i$ is finitely satisfiable.

    And $\Delta$ can be constructed by the union of all $\Delta_i$'s

    \[ \Delta = \bigcup_{i\in\mathbb{N}} \Delta_i \]

    $\Delta$ is finitely satisifiable because for each $\Delta'\subseteq\Delta$, $\Delta'\subseteq\Delta_i$ for some $\Delta_i$. Since $\Delta_i$ is finitely sat, $\Delta'$ is also satisfiable, and therefore $\Delta$ is finitely sat.

    For each wff $\alpha$, either $\alpha\in\Delta$ or $\neg\alpha\in\Delta$. Because $\alpha$ must exist as $\alpha_i$ in the sequence of all wffs, then either $\alpha_i\in\Delta_{i+1}$ or $\neg\alpha_i\in\Delta_{i+1}$.
\end{proof}

\begin{lemma}
    \label{lem:CompactnessLemma3}
    Let $\Delta$ be a set of wffs such that
    \begin{itemize}
        \item $\Delta$ is finitely satisifiable
        \item For every wff $\alpha$, $\alpha\in\Delta$ or $\neg\alpha\in\Delta$
    \end{itemize}
    Then $\Delta$ is satisfiable
\end{lemma}
\begin{proof}
    Consider the sentence symbols. All sentence symbols (or their negations) must be in $\Delta$ because they are also wffs. Therefore if there is an assignment $v$ that satisfies $\Delta$, its values is already determined by the sentence symbols in $\Delta$.

    \[v(A)=\begin{cases}
        T &\quad A\in\Delta\\
        F &\quad \neg A\in\Delta
    \end{cases}\]

    Then proving Lemma~\ref{lem:CompactnessLemma3} is equivalent to proving
    \[ \forall \alpha, \alpha\in\Delta\Leftrightarrow\bar{v}(\alpha) = T \]
    This can be proved by induction
    \begin{itemize}
        \item[base] Consider $\alpha=A$. Obviously it holds.
        \item[induction] \begin{enumerate}[(a)]
            \item $\alpha=\neg\beta$. The hypothesis is $\beta\in\Delta \Leftrightarrow \bar{v}(\beta)=T$. If $\neg\beta\in\Delta$, then $\beta$ cannot be in $\Delta$ due to the finite satisfiability of $\Delta$, and therefore $\bar{v}(\beta) = F$, and therefore $\bar{v}(\neg\beta) = T$. Conversely, If $\bar{v}(\neg\beta) = T$, then $\bar{v}(\beta) =F$, and therefore $\beta\notin\Delta$ and thus $\neg\beta$ must be in $\Delta$.
        \end{enumerate}
    \end{itemize}
\end{proof}

\begin{corollary}
    \label{coroll:CorollaryOfTheCompactnessTheorem}
    If $\Sigma\vDash\tau$, then there is a finite subset $\Delta$ of $\Sigma$ such that $\Delta\vDash\tau$
\end{corollary}
\begin{proof}
    Assume for every subset $\Delta$ of $\Sigma$, $\Delta \nvDash \tau$. Then $\Delta;\tau$ is satisfiable. So $\Sigma;\tau$ is finitely satisfiable, and by compactness theorem we have $\Sigma;\tau$ is satisfiable, which implies $\Sigma\nvDash\tau$, and this leads to a contradiction.
\end{proof}

\subsection{Decidability Results for Semantic Implications}

\begin{theorem}
    Given any finite set $\Sigma$ of wffs and any wff $\alpha$, there is an algorithm for deciding whether or not $\Sigma\vDash\alpha$.
\end{theorem}
\begin{enumerate}
    \item Collect all sentence symbols in $\Sigma$ and $\alpha$
    \item Use truth table
\end{enumerate}

\begin{corollary}
    Given a finite set of wffs $\Sigma$, the set of its semantic consequences is effectively decidable. In particular, the set of tautologies is effectively decidable.
\end{corollary}

\subsection{Enumerability Results for Semantic Implications}

\begin{theorem}
    If $\Sigma$ is an effectively enumerable set of wffs, then the set of semantic consequences of $\Sigma$ is effectively enumerable.
\end{theorem}
\begin{proof}
    Let $\beta_1,\dots,\beta_n,\dots$ be an effective enumeration of $\Sigma$. Let $\Delta_n=\beta_1,\dots,\beta_n$. Let $\alpha_1,\dots,\alpha_m,\dots$ be an effective enumeration of all wffs. We construct a table $T$ where $T_{ij} = \Delta_i \vDash \alpha_j$. Due to Corollary~\ref{coroll:CorollaryOfTheCompactnessTheorem}, if some $T_{ij}$ holds, then $\alpha_j$ is a semantic consequence of $\Sigma$.
\end{proof}
\chapter{First Order Logic}

\emph{"All men are mortal. Socrates is a man. Socrates is mortal."}

\section{Syntax of First-Order Logic}

We start with the symbols of a \textbf{first-order language} $\mathbb{L}$

There are two types of symbols

\begin{itemize}
    \item \textbf{Logical symbols}
    \item Non-logical symbols, a.k.a. \textbf{parameters}
\end{itemize}

\subsection{Symbols}

\subsubsection{Logical Symbols}

In a first-order language $\mathbb{L}$, we have the following symbols

\begin{enumerate}
    \item \textbf{Parentheses}. Two symbols `(' and `)'.
    \item \textbf{Logical connective symbols}. $\to$ and $\neg$
    \item \textbf{Variables}. An enumerable list of symbols $v_1,\dots,v_n,\dots$
    \item \textbf{Identity or Equalily Symbol} $=$ or $\doteq$. It may or may not be present in a particular first-order language
\end{enumerate}

Notice that we do not need $\vee$, $\wedge$. $\leftrightarrow$ because $\{\to, \neg\}$ is complete.

\subsubsection{Parameters}

\begin{enumerate}
    \item \textbf{Universal quantifier}. $\forall$
    \item For each $n>0$, there is a set (possibly empty) of objects called n-ary (or n-place) \textbf{predicate symbols}
    \item For each $n>0$, there is a set (possibly empty) of objects called n-ary (or n-place) \textbf{function symbols}
    \item A set of (possibly empty) of objects \textbf{constant symbols}
\end{enumerate}

\subsubsection{Further Requirements}

\begin{itemize}
    \item $\doteq$ is a 2-ary predicate symbols
    \item There is at least one predicate symbol
    \item The symbols are distinct, and no symbol is equal to a finite sequence of other symbols
\end{itemize}

\subsubsection{Example: Set Theory as First-Order Logic}

The Set Theory can be described by the following language

\begin{itemize}
    \item Equality
    \item Predicate symbols: 2-place $\dot{\in}$
    \item Constant symbols: empty set $\dot{\emptyset}$
    \item Function symbols: None
\end{itemize}

Note that the symbols are (currently) just interpreted as symbols and they do not have semantic meanings.

\begin{remark}
    We do not put restrictions or requirements on number of predicate, function or constant symbols.
\end{remark}

\subsection{Expressions}

An \textbf{expression} in a language $\mathbb{L}$ is a finite sequence of symbols.

\subsubsection{Terms}

\begin{definition}[Term Building Operation]
    \label{def:TermBuildingOperation}
    Given any n-ary function symbol $f$, the term-building operation $\mathcal{F}_f$ is defined by
    \[ \mathcal{F}_f (\sigma_1,\dots,\sigma_n) = f \sigma_1\dots\sigma_n \]
    We call $\sigma_i$ the arguments to $f$
\end{definition}

\begin{definition}[Term]
    \label{def:Term}
    A \textbf{term} is an expression built up from constant symbols and variables by applying some finite times (zero or more times) of term-building operations.
\end{definition}

For example, let $f$ and $g$ be 2-ary and 3-ary function symbols, then $gfc_1c_2v_3c_1$ is a term.

\begin{definition}[Term Sequence]
    \label{def:TermSequence}
    A \textbf{term sequence} is a finite sequence $t_1,\dots,t_n$ of expressions s.t. each $t_i$ is
    \begin{itemize}
        \item either a variable, a constant
        \item or is in the form of $f\sigma_1\dots\sigma_k$ where $f$ is a $f$-ary function and each $\sigma_1,\dots,\sigma_k$ occurs earlier in the sequence
    \end{itemize}
\end{definition}

\begin{proposition}
    An expression $t$ is a term iff there is a term sequence $t_1,\dots,t_n$ such that $t=t_n$
\end{proposition}

\subsubsection{Atomic Formulas}

\begin{definition}[Atomic Formula]
    \label{def:AtomicFormula}
    An expression is an \textbf{atomic formula} if it is of the form $P t_1\dots t_n$ where $t_1,\dots,t_n$ are terms and $P$ is a n-ary predicate symbol.
\end{definition}

\subsection{Well-Formed Formulas}

\begin{definition}[Formula-Building Operations]~{}
    \label{def:FormulaBuildingOperation}
    \begin{itemize}
        \item $\xi_\neg(\alpha) = (\neg \alpha)$
        \item $\xi_\to(\alpha, \beta) = (\alpha\to\beta)$
        \item $\mathcal{Q}_i(\gamma) = \forall v_i\gamma$
    \end{itemize}
\end{definition}

\begin{definition}[Well-Formed Formula]
    A \textbf{well-formed formula} (wff) is an expression built up from atomic formulas by applying some finite times of term-building operations.
\end{definition}

\begin{definition}[Well-Formed Sequence]
    A \textbf{well-formed sequence} is a finite sequence $\alpha_1,\dots,\alpha_n$ of expressions such that each $\alpha_i$ is
    \begin{itemize}
        \item either an atomic formula
        \item or is of the form of $(\neg \beta)$ or $(\beta\to\gamma)$ where $\beta$ and $\gamma$ occur earlier in the list
        \item or is of the form $\forall v_i\beta$ where $\beta$ occurs earlier in the list
    \end{itemize}
\end{definition}

\begin{proposition}
    The expression $\alpha$ is a wff if there is a well-formed sequence $\alpha_1,\dots,\alpha_k$ such that $\alpha = \alpha_k$
\end{proposition}

\subsection{Abbreviations}

\begin{itemize}
    \item $(\alpha\vee\beta)$ abbreviates $((\neg\alpha)\to\beta)$
    \item $(\alpha\wedge\beta)$ abbreviates $(\neg(\alpha \to (\neg\beta)))$
    \item $(\alpha\leftrightarrow\beta)$ abbreviates $(\alpha\to\beta)\wedge(\beta\to\alpha)$
    \item $\exists x\alpha$ abbreviates $(\neg\forall x(\neg\alpha))$
    \item $u\doteq t$ abbreviates $\doteq ut$
    \item $u \dot{\neq} t$ abbreviates $\dot{\neq} ut$
    \item Outer-most parentheses can be omitted
    \item $\neg$, $\forall$, $\exists$ apply to as little as possible
    \item $\wedge$, $\vee$ apply to as little as possible, subject to previous operators
    \item Grouping for repeated connectives is to the right
\end{itemize}

\subsection{Free Occurrence of Variables}

\begin{definition}[Free Occurrence]
    The variable $x$ \textbf{occurs free} in an atomic wff $\varphi$ iff it occurs in $\varphi$.

    $x$ \textbf{occurs free} in $\neg\alpha$ iff $x$ occurs free in $\alpha$.

    $x$ \textbf{occurs free} in $\alpha\to\beta$ iff $x$ occurs free in $\alpha$ or in $\beta$.

    $x$ \textbf{occurs free} in $\forall y \alpha$ iff $x$ occurs free in $\alpha$ and $x \neq y$.
\end{definition}

\begin{definition}[Sentence]
    $\varphi$ is a \textbf{sentence} iff no variable occurs free in $\varphi$.
\end{definition}
\begin{remark}
    Sentences are usually represented by $\sigma$ or $\tau$.
\end{remark}

We provide some examples

\begin{itemize}
    \item $\dot{0} \dot{<} \dot{1}$ does not have any free occurrence. It is a sentence,
    \item $\forall x(x\dot{<}y)$. $y$ occurs free but $x$ does not.
    \item $\forall x(\neg x \dot{<} \dot{0})$. No free occurrence.
    \item $\forall x\forall y (x \dot{<} y \to \exists z x\dot{<}z\wedge z\dot{<}y)$. No free occurrence.
\end{itemize}

\section{Semantics of First-Order Logic}

\subsection{Structures}

\begin{definition}
    Given a first order language $\mathbb{L}$, a \textbf{structure} $\mathfrak{A}$ for $\mathbb{L}$ consists of
    \begin{itemize}
        \item A non-empty set called the \textbf{universe} or \textbf{domain} of the structure, written as $|\mathfrak{A}|$
        \item For each n-ary predicate symbol $P$ of $\mathcal{L}$, other than $\doteq$, an n-ary relation $\mathbb{P}^{\mathfrak{A}}$ on $|\mathfrak{A}|$
        \item $\doteq^{\mathfrak{A}}$ is the identity relation on $|\mathfrak{A}|$. $\doteq^{\mathfrak{A}} = \{(a,b)|a,b\in|\mathfrak{A}, a =b|\}$
        \item For each n-ary function symbol $f$ of $\mathbb{L}$, an n-ary ooperation on the universe, i.e. an n-ary function $f^{\mathfrak{A}}: |\mathfrak{A}|\times\cdots\times|\mathfrak{A}|\mapsto|\mathfrak{A}|$
        \item For each constant symbol $c$ of $\mathbb{L}$, $c^{\mathfrak{A}}\in|\mathfrak{A}|$
    \end{itemize}
\end{definition}

\subsection{Assignments}

Let $\mathfrak{A}$ be a structure for language $\mathbb{L}$. Let $V$ be the set of variables, and $T$ be the set of terms.

\begin{definition}[Assignment Functions]
    An \textbf{assignment} for $\mathfrak{A}$ is a function $s:V\mapsto|\mathfrak{A}|$.
\end{definition}

\begin{definition}[Assignment to Terms]
    An assignment $s$ is extended to a function $\bar{s}:T\mapsto|\mathfrak{A}|$.
    \begin{itemize}
        \item $\bar{s}(v) = s(v)$ if $v$ is a variable
        \item $\bar{s}(c) = c^{\mathfrak{A}}$ if $c$ is a constant
        \item $\bar{s}(ft_1\dots t_n) = f^{\mathfrak{A}}(\bar{s}(t_1),\dots,\bar{s}(t_n))$ if $f$ is a n-ary function symbol and $t_1,\dots,t_n$ are terms
    \end{itemize}
\end{definition}

\subsubsection{Changing the Assignment Function}

Let $s$ be an assignment function, $x$ be a variable and $a\in\mathfrak{A}$. Then $s(x|a)$ is the new assignment, where for each variable $y$

\[s(x|a)(y) = \begin{cases}
    s(y) &\quad \text{if $(y\neq x)$}\\
    a &\quad \text{if $(y=x)$}
\end{cases}\]

This operation actually ``overrides'' the assignment of $s$ to $x$ and makes the assignment to $x$ equal to $a$.

\section{Satisfaction}

Given a first-order language $\mathbb{L}$, let $\mathfrak{A}$ be a structure for $\mathbb{L}$, let $s$ be an assignment for $\mathbb{L}$ and let $\varphi$ be a wff in $\mathbb{L}$. We denote $\mathfrak{A}$ to satisfy $\varphi$ with $s$ by $\vDash_{\mathfrak{A}}\varphi[s]$

Informally, it means ``\emph{The translation of $\varphi$ determined by $\mathfrak{A}$, where a variable $x$ is translated as $s(x)$, is true}''

\subsection{Satisfaction for Atomic Formulas}

\begin{definition}
    Given a language $\mathbb{L}$ and a structure $\mathfrak{A}$, let $s$ be an assignment, let $P$ be a n-ary predicate,
    \[ \vDash_{\mathfrak{A}} Pt_1\dots t_n[s] \Leftrightarrow (\bar{s}(t_1),\dots,\bar{s}(t_n)) \in P^{\mathfrak{A}} \]
    \[ \vDash_{\mathfrak{A}} \doteq t_1t_2[s] \Leftrightarrow \bar{s}(t_1) = \bar{s}(t_2) \]
\end{definition}

\subsection{Satisfaction for WFF}

\begin{definition}
    Suppose $\vDash_{\mathfrak{A}}\alpha[s]$ and $\vDash_{\mathfrak{A}}\beta[s]$ have been defined, then
    \begin{itemize}
        \item $\vDash_{\mathfrak{A}} \neg\alpha[s]$ iff not $\vDash_{\mathfrak{A}}\alpha[s]$
        \item $\vDash_{\mathfrak{A}}\alpha\to\beta[s]$ iff $\vDash_{\mathfrak{A}}\alpha[s]\Longrightarrow\vDash_{\mathfrak{A}}\beta[s]$
        \item $\vDash_{\mathfrak{A}}\forall x \alpha[s]$ iff $\forall a\in|\mathfrak{A}|$, $\vDash_{\mathfrak{A}}\alpha[s(x|a)]$
    \end{itemize}
\end{definition}

If $\vDash_{\mathfrak{A}}\varphi[s]$, we say \emph{$\mathfrak{A}$ satisfies $\varphi$ with $s$}, or \emph{$s$ satisfies $\varphi$ in the structure $\mathfrak{A}$}

\subsubsection{Satisfaction Depends Only on Variables that Occur Free}

\begin{lemma}
    \label{lem:LemmaForFreeOccurrenceThm}
    Let $\mathfrak{A}$ be a structure for $\mathbb{L}$, $s_1, s_2$ be two assignment for $\mathfrak{A}$ and $t$ be a term of $\mathbb{L}$.

    If $s_1(x)=s_2(x)$ for every $x$ that occurs in $t$, then
    \[ \bar{s}_1(t) = \bar{s}_2(t) \]
\end{lemma}
\begin{proof}
    Proof by induction on $t$.
    \begin{itemize}
        \item[Base] If $t=c$ is a constant. It is straightforward that $\bar{s}_1(t) = \bar{s}_2(t) = c$. If $t=x$ is a variable, then by assumption we know that $\bar{s}_1(x) = s_1(x) = s_2(x) = \bar{s}_2(x)$. So we are done.
        \item[Induction] Consider a term $t=ft_1\dots t_n$. By inductive hypothesis we know that $\forall i$, $\bar{s}_1(t_i) = \bar{s}_2(t_i)$. $\bar{s}_1(t) = f^{\mathfrak{A}}(\bar{s}_1(t), \dots, \bar{s}_1(t) = f^{\mathfrak{A}}(\bar{s}_2(t), \dots, \bar{s}_2(t)) = \bar{s}_2(t)$.
    \end{itemize}
\end{proof}

\begin{theorem}
    Let $\mathfrak{A}$ be a structure for $\mathbb{L}$, $s_1, s_2$ be two assignment for $\mathfrak{A}$ and $\varphi$ be a wff of $\mathbb{L}$.

    If $s_1(x)=s_2(x)$ for every $x$ that occurs free in $\varphi$, then
    \[ \vDash_{\mathfrak{A}}\varphi[s_1] \iff \vDash_{\mathfrak{A}}\varphi[s_2] \]
\end{theorem}

\begin{proof}
    Prove by induction on $\varphi$.
    \begin{itemize}
        \item[Base] If $\varphi$ is an atomic formula $Pt_1\dots t_n$.
        \[ \vDash_{\mathfrak{A}} \varphi [s_1] \Leftrightarrow P^{\mathfrak{A}}(\bar{s}_1(t_1),\dots,\bar{s}_1(t_n)) \]
        \[ \vDash_{\mathfrak{A}} \varphi [s_2] \Leftrightarrow P^{\mathfrak{A}}(\bar{s}_2(t_1),\dots,\bar{s}_2(t_n)) \]

        We need to prove that the two RHSes are equivalent. By Lemma~\ref{lem:LemmaForFreeOccurrenceThm} we know that all the terms are equal under $s_1$ and $s_2$, and therefore $\vDash_{\mathfrak{A}} \varphi [s_1] = \vDash_{\mathfrak{A}} \varphi [s_2]$.

        \item[Induction] Consider $\varphi=\neg\alpha$.
        \[ \vDash_{\mathfrak{A}}(\neg\alpha)[s_1] \Leftrightarrow \nvDash_{\mathfrak{A}}\alpha[s_1] \Leftrightarrow \nvDash_{\mathfrak{A}}\alpha[s_2] \Leftrightarrow \vDash_{\mathfrak{A}}(\neg\alpha)[s_2] \]

        The case $\varphi = \alpha\to\beta$ is similar.

        Consider the case $\forall x \alpha$. We want to prove
        \[ \sat{A}{\forall x\alpha}{s_1} \Leftrightarrow \sat{A}{\forall x\alpha}{s_2} \]
        which is equivalent to
        \[ \forall a\in|\mathfrak{A}|\sat{A}{\alpha}{s_1(x|a)} \Leftrightarrow \forall a \in |\frakA| \sat{A}{\alpha}{s_2(x|a)} \]

        We only need to prove that
        \[ \forall y \text{occurring free in $\alpha$}, s_1(x|a)(y) = s_2(x|a)(y) \]

        If $y\neq x$, then $y$ is still occurring free in $\alpha$, and by inductive hypothesis they should equal. If $y=x$, then both sides are $a$. So we are done.
    \end{itemize}
\end{proof}
\begin{remark}
    This theorem is somewhat similar to the theorem in sentential logic, which states that we only need to consider sentence symbols. Similarly, in first-order logic, we only need to consider variables that occur free.
\end{remark}

\begin{definition}
    Let $\varphi$ be a wff s.t. all variables occurring free in $\varphi$ are included amoing $v_1,\dots,v_k$. Given $a_1,\dots,a_k\in\frakA$.

    \[ \assignSat{A}{\varphi}{a_1,\dots,a_k} \]

    means that $\sat{A}{\varphi}{s}$ for some $s:V\mapsto|\frakA|$ s.t. $s(v_i) = a_i$
\end{definition}

\begin{corollary}
    If $\sigma$ is a sentence then
    \begin{itemize}
        \item either $\sat{A}{\sigma}{s}$ for every assignment $s$. We say $\sigma$ is true in $\frakA$.
        \item or $\unsat{A}{\sigma}{s}$ for every assignment $s$. We say $\sigma$ is false in $\frakA$
    \end{itemize}
\end{corollary}

Therefore a sentence does not depend on $s$, and we can simply write $\sentSat{A}{\sigma}$ or $\sentunsat{A}{\sigma}$.

\subsection{Elementary Equivalence}

\begin{definition}[Elementary Equivalence]
    \label{def:ElementaryEquivalence}
    Let $\mathfrak{A}$ and $\mathfrak{B}$ be structures for the same language $\mathbb{L}$. $\mathfrak{A}$ and $\mathfrak{B}$ are \textbf{elementarily equivalent} ($\mathfrak{A} \equiv \mathfrak{B}$) if for every \emph{sentence} of $\mathbb{L}$
    \[ \sentSat{A}{\sigma} \iff \sentSat{B}{\sigma} \]
\end{definition}

\begin{remark}
    Elementary equivalence only take into consideration sentences.
\end{remark}

\begin{proposition}
    $\mathfrak{Q}$ and $\mathfrak{R}$ are elementary equivalent. But this is beyond the scope of the course.
\end{proposition}

\section{Models}

\subsection{Models}

\begin{definition}[Model]
    $\mathfrak{A}$ is a \textbf{model} of the sentence $\sigma$ if $\sentSat{A}{\sigma}$, i.e. if $\sigma$ is true in $\mathfrak{A}$. $\mathfrak{A}$ is a \textbf{model} of a set $\Sigma$ of sentences if $\mathfrak{A}$ is a model for every sentence in $\Sigma$. i.e. every sentence in $\Sigma$ is true in $\mathfrak{A}$.
\end{definition}

For example, consider a first-order language $\mathbb{L}$, with 2-ary predicate symbols $\dot{P}$ and $\doteq$. Given a structure $\mathfrak{A}$ of $\mathbb{L}$,

\begin{itemize}
    \item $\mathfrak{A}$ is a model of $\forall x \forall y x \doteq y$
    \begin{itemize}
        \item $\Leftrightarrow \sat{A}{x\doteq y}{s(x|a)(y|b)}$ for every $a,b \in |\mathfrak{A}|$
        \item $\Leftrightarrow$ $a = b$ for every $a,b \in|\mathfrak{A}|$
        \item $\Leftrightarrow$ $|\mathfrak{A}|$ contains only one element.
        \item Note that $|\frakA|$ cannot be empty because the universe of a structure must be non-empty.
    \end{itemize}
    \item $\mathfrak{A}$ is a model of $\forall x \forall y \dot{P}xy$
    \begin{itemize}
        \item iff $P^\mathfrak{A}(a,b)$ for all $a,b\in|\mathfrak{A}|$
        \item iff $\dot{P}^{\frakA} = |\frakA| \times |\frakA|$
    \end{itemize}
    \item $\mathfrak{A}$ is a model of $\forall x \forall y \neg\dot{P}xy$
    \begin{itemize}
        \item iff $P^{\frakA}(a,b)$ does not hold for all $a,b\in|\frakA|$
        \item iff $\dot{P}^{\frakA} = \emptyset$
    \end{itemize}
    \item $\frakA$ is a model of $\forall x \exists y \dot{P}xy$
    \begin{itemize}
        \item iff forall $a\in|\frakA|$, there is a $b\in|\frakA|$ s.t. $\dot{P}^{\frakA}(a,b)$
        \item iff the domain of $\dot{P}^{\frakA}$ is $|\frakA|$
        \item Conversely, if we want the range of $\dot{P}^{\frakA}$ is $|\frakA|$, we can write $\forall y \exists x \dot{P}xy$
    \end{itemize}
\end{itemize}

\subsection{Linearly Ordered Structures}

\begin{definition}[Trichotomy]
    Let $R$ be a binary relation, $R$ satisfies \textbf{trichotomy} if exactly one of the following is true
    \[ (a,b) \in R \quad (b,a) \in R \quad a = b \]
\end{definition}

\begin{definition}[Linear Ordering]
    A binary relation $R$ is a \textbf{linear ordering} on $A$ if $R$ is transitive and satisfies trichotomy on $A$.
\end{definition}

\begin{definition}
    Let $\mathbb{L}$ be the language with a binary relation symbol $\dot{R}$ and $\doteq$ (and no other symbols). Let $\frakA = (A,R)$, i.e. $A=|\frakA|$ and $R = \dot{R}^{\frakA}$.
    \begin{itemize}
        \item $\frakA$ is transitive if $R$ is transitive
        \item $\frakA$ is a linearly ordered structure if $R$ is a linear ordering on $\frakA$
    \end{itemize}
\end{definition}

For a set of structures with some certain properties, the set can be defined by a sentence.

Let $\frakA = (A,R)$,
\begin{itemize}
    \item $\frakA$ is transitive iff $\sentSat{A}{\sigma}$, where $\sigma=\forall x \forall y \forall z \dot{R}xy \to \dot{R}yz \to \dot{R}xz$. Therefore $\sigma$ defines the set of all transitive structures
    \item $\frakA$ is linearly ordered iff $\sentSat{A}{\sigma}$ where
    \begin{itemize}
        \item $\sigma_1 = \forall x \forall y \forall z \dot{R}xy \to \dot{R}yz \to \dot{R}xz$
        \item $\sigma_2 = \forall x \forall y (\dot{R}xy \vee x=y \vee \dot{R}yx)$
        \item $\sigma_3 = \forall x \forall y (\dot{R}xy \to \neg\dot{R}yx)$
        \item $\sigma = \sigma_1 \wedge \sigma_2 \wedge \sigma_3$
    \end{itemize}
    Therefore $\sigma$ defines the set of all linearly ordered structures
    \item $\domain{R} = A$ iff $\sentSat{A}{\sigma}$ where $\sigma = \forall x \exists y \dot{R}xy$
    \item $\range{R} = A$ iff $\sentSat{A}{\sigma}$ where $\sigma = \forall y \exists x \dot{R}xy$
    \item $R$ is a (total) function iff $\sentSat{A}{\sigma}$ where
    \begin{itemize}
        \item $\sigma_4 =  \forall x \exists y \dot{R}xy$
        \item $\sigma_5 = \forall x \forall y \forall z \dot{R}xy \to \dot{R}xz \to x \doteq z$
        \item $\sigma = \sigma_4 \wedge \sigma_5$
    \end{itemize}
\end{itemize}

\subsection{Elementary Class}

\begin{definition}[Elementary Class]
    A set of strucutures $\mathcal{K}$ is an \textbf{elementary class} if there exists a sentence $\sigma$ s.t.
    \[ \mathcal{K} = \{ \frakA | \frakA \text{ is a model of $\sigma$} \} \]
    i.e.
    \[ \mathcal{K} = \{ \frakA | \sentSat{A}{\sigma} \} \]
\end{definition}

For example, the set of all graphs is an elementary class.

Let $\mathbb{L}$ be the language with a binary predicate symbol $\dot{E}$ and $\doteq$, and no other symbols. A structure $\mathfrak{G} = (G, E)$ for $\mathbb{L}$ is a graph if
\begin{itemize}
    \item $E$ is symmetric (undirected graph)
    \item For every $a \in G$, $(a, a) \notin E$ (no self-loops)
\end{itemize}

To show that the set of graphs defined above is an elementary class, we need to show that there is a sentence $\sigma$ s.t.
\[ \text{$G$ is a graph} \Leftrightarrow \sentSat{G}{\sigma} \]

Therefore $\sigma$ should be able to represent the symmetric and non-reflexible properties.

\[\sigma = (\forall x \forall y (\dot{E}xy \to \dot{E}yx)) \wedge (\forall x(\neg \dot{E}xx))\]

\subsubsection{In the Wider Sense}

\begin{definition}[Elementary Class in the Wider Sense]
    A set of structures $\mathcal{K}$ is an \textbf{elementary class in the wider sense} ($EC_\Delta$) if there is a set $\Sigma$ of sentences s.t.
    \[ \mathcal{K} = \{ \frakA | \frakA \text{ is a model of } \Sigma \} \]
    i.e.
    \[ \mathcal{K} = \{ \frakA | \sentSat{A}{\sigma} \text{ for every $\sigma \in \Sigma$} \} \]
\end{definition}

For example, we can use a set $\Sigma$ to define the set of strucutures whose univerise is infinite.

\[ \Sigma = \{ \lambda_2, \lambda_3, \dots, \lambda_n, \dots \} \]

where $\lambda_i$ denotes ``At least $i$ elements exists in $|\frakA|$'', for example $\lambda_2 = \exists x \exists y x \neq y$.

However, it is hard to decide whether there is a single sentence $\sigma$ s.t. $\frakA$ is a model of $\sigma$ iff $|\frakA|$ is infinite.

\section{Logical Implications and Satisfiability}

\subsection{Logical Implications}

\begin{definition}
    Let $\Gamma$ be a set of wffs and $\varphi$ be a wff. $\Gamma$ \textbf{logically implies} $\varphi$, written as
    \[ \Gamma \vDash \varphi \]
    if for every structure $\frakA$ and every assignment $s$, if $\frakA$ satisfies $\Gamma$ with $s$, then $\frakA$ satisfies $\varphi$ with $s$
\end{definition}

\begin{theorem}
    For a set of sentences $\Sigma$, and a sentence $\sigma$, $\Sigma\vDash\sigma$ iff for every model $\frakA$ of $\Sigma$, $\frakA$ is a model of $\sigma$.
\end{theorem}

As before, we denote $\{\alpha\}\vDash\beta$ by $\alpha\vDash\beta$

\subsection{Logical Equivalence}

\begin{definition}[Logical Equivalence]
    $\alpha$ and $\beta$ are logically equivalent, written as $\alpha\vDash\Dashv\beta$ if $\alpha\vDash\beta$ and $\beta\vDash\alpha$
\end{definition}

\subsection{Valid Formulas}

\begin{definition}[Valid WFFs]
    Let $\varphi$ be a wff in the language $\mathbb{L}$. $\varphi$ is \textbf{valid} if $\semanticalImply{\emptyset}{\varphi}$, written as $\tautology{\varphi}$.
\end{definition}

For example,

\begin{itemize}
    \item $x \doteq x$ is valid
    \item $\exists x \doteq x$ is valid
    \item $\forall x \exists y x\dot{\neq}y$ is not valid
    \item $\dot{P}x \vee \neg \dot{P}x$ is valid
    \item $\exists x(\dot{P}x \to \forall x \dot{P}x)$ is valid
\end{itemize}

We detail the proof of the last example

\begin{itemize}
    \item $\sat{A}{\exists x (Px\to\forall x Px)}{s}$
    \item[$\Leftrightarrow$] There is some $a \in |\frakA|$ s.t. $\sat{A}{Px \to \forall x Px}{s(x|a)}$
    \item[$\Leftrightarrow$] $\sat{A}{\dot{P}(x)}{s(x|a)} \Rightarrow \sat{A}{\forall x Px}{s(x|a)}$
    \item[$\Leftrightarrow$] $\sat{A}{\dot{P}(x)}{s(x|a)} \Rightarrow \sat{A}{\forall x Px}{s(x|a)}$
    \item[$\Leftrightarrow$] There is some $a$ s.t. if $a \in \dot{P}^{\frakA}$, then for every $b \in |\frakA|$, $b\in\dot{P}^{\frakA}$
    \item If there is some $a$ s.t. $a \notin \dot{P}^{\frakA}$, then the consequence holds trivially
    \item If there is no $a$ s.t. $a \notin \dot{P}^{\frakA}$, this means that for every $a \in |\frakA|$, $a\in\dot{P}^{\frakA}$. This is exactly the consequence so we are done
\end{itemize}

\subsection{Satisfiability}

\begin{definition}[Satisfiability]
    \begin{itemize}
        \item The wff $\varphi$ is \textbf{satisfiable} if there is some structure $\frakA$ and some assignment $s$ s.t. $\sat{A}{\varphi}{s}$.
        \item The set of wffs $\Gamma$ is \textbf{satisfiable} if there is some structure $\frakA$ and some assignment $s$ s.t. $\sat{A}{\varphi}{s}$ for every $\varphi$ in $\Gamma$.
    \end{itemize}
\end{definition}

\begin{theorem}
    $\varphi$ is not satisfiable iff $\neg\varphi$ is valid
\end{theorem}

\section{Definability}

\begin{definition}[Relations Defined by WFFs]
    Let $\frakA$ be a structure, and $\varphi$ be a wff, and $n$ be such that the variables occurring free in $\varphi$ are included among $v_1,\dots,v_n$

    The n-ary relation \textbf{defined by $\varphi$ in $\frakA$} is
    \[ \{ (a_1,\dots,a_n) | \assignSat{A}{\varphi}{a_1,\dots,a_n}\} \]
\end{definition}

Let $\mathfrak{N} = (\mathbb{N},\le, +, 1)$, the 2-ary relation $\{(a,b)|a < b\}$ is defined by
\[ v_1 \dot{+}\dot{1}\doteq v_2 \]

To show this, we will show that $(a,b) \in R \iff \assignSat{N}{\varphi}{a,b}$.

If $(a,b) \in R$, $\assignSat{N}{\varphi}{a,b} \iff a+1 \le b$

Conversely, if $\assignSat{N}{\varphi}{a,b}$, it's equivalent to $a+1 \le b$, which mathematically implies that $a < b$

\begin{definition}[Definability]
    The relation $R$ is \textbf{definable in the structure $\frakA$} if there is some wff $\varphi$ that defines it in $\frakA$.
\end{definition}

We show some examples for definability of functions. Let $\mathfrak{N} = (\mathbb{N}, <, +, x, 0, 1)$.

\begin{itemize}
    \item $v_1 \dot{+} v_2 \doteq v_3$ defines $\{ (a,b,c) | a + b = c\}$ which is the same as function $f$, where $f(a,b) = a + b$.
\end{itemize}

\subsection{Definable Relations}

\begin{definition}[Relations Defined by WFFs]
    Let
    \begin{itemize}
        \item $\frakA$ be a structure
        \item $\varphi$ be a wff and $n$ be such that the variables occurring free in $\varphi$ are included among $v_1,\dots,v_n$
    \end{itemize}

    The $n$-ary relation defined by $\varphi$ in $\frakA$ is
    \[ \{ (a_1,\dots,a_n) | \assignSat{A}{\varphi}{a_1,\dots,a_n} \} \]
\end{definition}

For example,

\begin{itemize}
    \item Let $\mathfrak{R} = (\mathbb{R},<,+,\times,0,1)$. The $1$-ary relation $\{ a \in \mathbb{R} | 0 \le a \}$ is defined by
    \[ \exists v_2, v_1 \doteq v_1 \times v_2 \]
    \item Let $\mathfrak{R} = (\mathbb{R},<,+,\times,0,1)$。 The $2$-ary relation $\{ (a,b) | a < b \}$ is defined by
    \[ \exists v_3 (v_1\dot{+}(\dot{1}\dot{+}v_3)\doteq v_2) \]
\end{itemize}

\begin{definition}[Definable Relations]
    The relation $R$ is \textbf{definable in structure} $\frakA$ if there is some wff that defines it in $\frakA$

    Let $f$ be a n-ary function $f$ whose domain is a subset $|\frakA| \times \dots \times |\frakA|$ and whose range is a subset of $|\frakA|$, $f$ is definable in $\frakA$ if the $(n+1)$-ary relation
    \[ \{ (a_1,\dots,a_n, b) | f(a_1,\dots,a_n) = b \} \]
    is definable in $\frakA$.
\end{definition}

Consider $\mathfrak{N} = (\mathbb{N},<,+,\times,0,1)$.
\begin{itemize}
    \item $v_1 + v_2 = v_3$ defines $\{ (a,b,c) | a + b = c \}$, which is the same as function $f(a,b) = a + b$
    \item $v_1 + v_3 = v_2$ defines $\{ (a,b,c) | a+c = b \}$, which is the same as function $f(a,b)$
    \[ f(a,b) = \begin{cases}
        b - a &\quad a \le b\\
        Undefined &\quad o.w.
    \end{cases} \]
\end{itemize}

\begin{lemma}
    Given a structure $\frakA$, the set of definable relations is \emph{enumerable}
\end{lemma}
\begin{lemma}
    Not every subset of $\mathbb{N}$ is definable.
\end{lemma}

The proof of the two lemmas are similar. Note that the set of wffs is enumerable, and every wff may define only one relation. And the set of all subsets of $\mathbb{N}$ is uncountable.

We now move from $\mathbb{N}$ to $\mathbb{R}$ and consider a more generall case. Consider whether the following subsets of $\mathbb{R}$ are definable in $\mathfrak{R} = (\mathbb{R}, <)$

\begin{itemize}
    \item $\emptyset$. Yes.
    \item $\mathbb{N}$. Yes.
    \item Anything else?
\end{itemize}

\section{Homomorphisms}

\begin{definition}[Homomorphism]
    Let $\mathfrak{A}$ and $\mathfrak{B}$ be structures for $\mathbb{L}$. A \textbf{homomorphism} from $\frakA$ to $\mathfrak{B}$ is a function $h:|\frakA| \mapsto |\mathfrak{B}|$ s.t.
    \begin{itemize}
        \item For every n-ary predicate symbol $R$, other than $\doteq$, and $a_1,\dots,a_n \in |\frakA|$,
        \[ (a_1,\dots,a_n) \in R^{\frakA} \iff (h(a_1),\dots,h(a_n))\in R^{\mathfrak{B}} \]
        \item For every n-ary function symbol $f$, and $a_1,\dots,a_n \in |\frakA|$,
        \[ h(f^\frakA(a_1,\dots,a_n)) = f^\mathfrak{B}(h(a_1),\dots,h(a_n)) \]
        \item For every constant symbol $c$
        \[ h(c^\frakA) = c^\mathfrak{B} \]
    \end{itemize}
\end{definition}

\begin{definition}[Onto of Homomorphism]
    $h$ is a homomorphism of $\frakA$ \textbf{onto} $\mathfrak{B}$ if $h$ is a homomorphism from $\mathfrak{A}$ to $\mathfrak{B}$ and $h$ maps $\frakA$ onto $\mathfrak{B}$.
\end{definition}

\begin{definition}[Isomorphism]
    A homomorphism $h$ from $\mathfrak{A}$ to $\mathfrak{B}$ is an \textbf{isomorphism} if $h$ is one-to-one
\end{definition}

\begin{definition}[Isomorphic]
    The structures $\frakA$ and $\mathfrak{B}$ are \textbf{isomorphic}, denoted by $\frakA \cong \mathfrak{B}$ if there is some \emph{isomorphism} of $\frakA$ \emph{onto} $\mathfrak{B}$. (One-to-one correspondence)
\end{definition}

\begin{definition}[Automorphism]
    An \textbf{automorphism} of $\frakA$ is an isomorphism of $\frakA$ onto $\frakA$
\end{definition}

\subsubsection{Examples of Homomorphism}

For example, let $\frakA = (\naturalSet, <^\naturalSet, +^\naturalSet)$, $\mathfrak{B} = (\mathbb{E}, <^\mathbb{E}, +^\mathbb{E})$, where $\mathbb{E}$ is the set of even non-negative integers. Then we claim that $h(n)=2n$ is an isomorphism of $\mathfrak{A}$ onto $\mathfrak{B}$

To do this, we show that 1) $h$ is a homomorphism; 2) $h$ is one-to-one; 3) $h$ is onto

\begin{itemize}
    \item[$\dot{<}$] $(a,b) \in <^\mathbb{N} \iff (h(a), h(b)) = (2a,2b) \in <^\mathbb{E}$
    \item[$\dot{+}$] $h(a +^\mathbb{N} b) = (h(a) +^\mathbb{E} h(b)) = (2a +^\mathbb{E} 2b)$
\end{itemize}

However, let $\mathfrak{C} = (\mathbb{O}, <^\mathbb{O}, +^\mathbb{O})$, where $\mathbb{O}$ is the set of all odd non-negative integers, then there is no isomorphism of $\frakA$ onto $\mathfrak{C}$. Infact $\mathfrak{C}$ is not even a valid strucutre because $+^\mathbb{O}$ is not closed.

\subsubsection{Automorphism of $\mathfrak{R}=(\realSet, <)$}

Consider which of the following $h$ are automorphisms of $\mathfrak{R}$. Note that to show this, we need to show that 1) $h$ is a homomorphism; 2) $h$ is one-to-one; 3) $h$ is onto; and 4) $h$ maps $\realSet$ to $\realSet$.

\begin{itemize}
    \item The identity function. Obviously yes.
    \item $h(a) = a + 3$. Yes.
    \item $h(a) = a - 4$. Yes.
    \item $h(a) = 2a$. Yes.
    \item $h(a) = -a$. Yes.
    \item $h(a) = ka + l$. Yes if $k>0$.
    \item $h(a) = a^3$. Yes.
    \item $h(a) = a^2$. No.
\end{itemize}

\subsubsection{Automorphism of $\mathfrak{N} = (\naturalSet, <)$}

Obviously the identity function is an automorphism. We consider other cases.

If we map $0$ to any $n>0$, i.e. $h(0) = n > 0$. Since $h$ is onto, there exists some $m > 0$ s.t. $h(m) = 0$. And here comes a problem
\[ m > 0 \iff h(m) = 0 > h(0) = n > 0 \]
Therefore $0$ can only be mapped to $0$, i.e. $h(0) = 0$

Similarly, $h(1)$ can only be mapped to $1$, and for each $n$, $h(n) = n$. Therefore the identity function $h(n) = n$ is the \emph{only} automorphism of $\naturalStruct$.

\subsection{Substructures}

We now consider a special kind of isomorphism

\begin{definition}[Substructures]
    Let $\frakA = (A,\dots)$ and $\frakB = (B,\dots)$ be structures for $\mathbb{L}$. $\frakA$ is a \textbf{substructure} of $\frakB$, denoted by $\frakA \subseteq \frakB$ if
    \begin{itemize}
        \item $A \subseteq B$
        \item For every $k$-ary predicate symbol,
        \[ P^\frakA = P^\frakB \cap A^k \]
        Note that this is to guarantee that $(a_1,\dots,a_k)\in P^\frakB \Longrightarrow (a_1,\dots,a_k)\in P^\frakA$
        \item For every $k$-ary function $f$ and every $k$-tuple of $A$
        \[ f^\frakA(a_1,\dots,a_k) = f^\frakB(a_1,\dots,a_k) \]
        \item For every constant $c$
        \[ c^\frakA = c^\frakB \]
    \end{itemize}
\end{definition}
\begin{remark}
    Substructures are defined under identity map $h(x)=x$. The identity map is an isomorphism of $\frakA$ into $\frakB$ iff
    \begin{enumerate}
        \item For each predicate $P$, $P^\frakA$ is the restriction of $P^\frakB$ to $A$
        \item For each function $f$, $f^\frakA$ is the restriction of $f^\frakB$ to $A$
        \item $c^\frakA = c^\frakB$.
    \end{enumerate}
    If these conditions are met, then $\frakA$ is called a \textbf{substructure} of $\frakB$.
\end{remark}

\subsection{Homomorphism Theorem}

\begin{lemma}
    \label{lem:HomomorphismLemma}
    Let $\frakA$ and $\frakB$ be structures for the language $\mathbb{L}$. Let $h$ be a homomorphism from $\frakA$ to $\frakB$, and $s:V\to|\frakA|$ be an assignment for $\frakA$. Then for every term $t$ of $\mathbb{L}$
    \[ h(\bar{s}(t)) = \overline{h\circ s}(t) \]
\end{lemma}
\begin{proof}
    Prove by induction.
    \begin{itemize}
        \item[] \textbf{Base Case.} If $t=c$, then
        \[ h(\bar{s}(c)) = h(c^\frakA) = c^\frakB = \overline{h\circ s(c)} \]
        If $t=v$, then
        \[ h(\bar{s}(x)) = h(s(x)) = h \circ s(x) \]
        \item[] \textbf{Inductive Case.} If $t = ft_1,\dots,t_n$,
        \[ h(\bar{s}(t)) = h(f^\frakA(\bar{s}(t_1),\dots,\bar{s}(t_n))) = f^\frakB(\bar{s}(t_1),\dots,\bar{s}(t_n)) \]
        \[ \overline{h\circ s} = f^\frakB(\overline{h\circ s}(t_1),\dots,\overline{h\circ s}(t_n)) \]
        From inductive hypothesis we know $\overline{h\circ s}(t_k) = h(\bar{s}(t_k))$, and we are done.
    \end{itemize}
\end{proof}

\begin{theorem}[Homomorphism Theorem]
    \label{thm:HomomorphismTheorem}
    Let $h$ be a homomorphism form $\frakA$ to $\frakB$ and $s$ be an assignment function for $\frakA$. The statement
    \[ \sat{A}{\varphi}{s} \iff \sat{B}{\varphi}{h \circ s} \]
    \begin{itemize}
        \item is true for every quantifier-free wff $\varphi$ not containing $\doteq$
        \item is true for every quantifier-free wff $\varphi$ if $h$ is one-to-one
        \item is true for every wff $\varphi$ wff $\varphi$ not containing $\doteq$ if $h$ is onto
        \item is true for every wff $\varphi$ if $h$ is an isomorphism of $\frakA$ onto $\frakB$ ($\frakA \cong \frakB$)
    \end{itemize}
\end{theorem}
\begin{proof}
    Prove by induction on $\varphi$.
    \begin{itemize}
        \item[] \textbf{Base Case.} Let $\varphi = Pt_1\dots t_n$.
         Since $h$ is a homomorphism, by definition we have
         \[ (\bar{s}(t_1),\dots,\bar{s}(t_n))\in P^\frakA \iff (h(\bar{s}(t_1)),\dots, h(\bar{s}(t_n))) \in P^\frakB \]
         Further, by Lemma~\ref{lem:HomomorphismLemma},
         \[ (\overline{h\circ s}(t_1),\dots,\overline{h\circ s}(t_2)) = (h(\bar{s}(t_1)),\dots, h(\bar{s}(t_n))) \in P^\frakB \]
         So we are done.
         \item[] \textbf{Inductive Case.}\begin{itemize}
             \item If $\varphi = \neg \alpha$, by induction hypothesis
             \[ \sat{A}{\alpha}{s} \iff \sat{B}{\alpha}{h \circ s} \]
             We can negate both sides
             \[ \unsat{A}{\alpha}{s} \iff \unsat{B}{\alpha}{h \circ s} \]
             Therefore,
             \[ \sat{A}{\neg\alpha}{s} \iff \sat{B}{\neg\alpha}{h\circ s} \]
             \item If $\varphi = \alpha \to \beta$.
             \[ \sat{A}{\alpha\to\beta}{s} \iff \sat{B}{\alpha\to\beta}{h\circ s} \]
             is equivalent to
             \[ \text{If} \sat{A}{\alpha}{s} \text{then} \sat{A}{\beta}{s} \iff \text{If} \sat{B}{\alpha}{h \circ s} \text{then} \sat{B}{\beta}{s} \]
             By induction hypothesis, we are done.
         \end{itemize}
    \end{itemize}
    Until now we have proved (a). We now further consider $\doteq$ and $\forall$

    We start from $\doteq$, which is a base case,
    If $\varphi = t_1\doteq t_2$
        \[ \sat{A}{t_1\doteq t_2}{s} \iff \sat{B}{t_1\doteq t_2}{h\circ s} \]
        \[ \bar{s}(t_1) = \bar{s}(t_2) \iff \overline{h\circ s}(t_1) = \overline{h\circ s}(t_2) \iff h(\bar{s}(t_1)) = h(\bar{s}(t_2)) \]
        $\Leftrightarrow$ always holds. However, $\Leftarrow$ requires an additional condition that $h$ is one-to-one.

    Finally, consider $\varphi=\forall x\alpha$ in the inductive case.
    \[ \sat{A}{\forall x \alpha}{s} \iff \sat{B}{\forall x \alpha}{h\circ s} \]
    \[ \text{For any $a\in|\frakA|$,} \sat{A}{\alpha}{s(x|a)} \iff \text{For any $b\in|\frakB|$,}\sat{B}{\alpha}{(h\circ s)(x|b)} \]
    Assume LHS, we prove RHS. Note that since $a$ and $b$ are arbitrary, we cannot relate them without additional conditions. Therefore to prove this we would require $h$ to be \emph{onto}. Then there is some $a'\in|\frakA|$ s.t. $h(a')=b$.
    \[ \sat{B}{\alpha}{(h\circ s)(x|h(a'))} \iff \sat{B}{\alpha}{h\circ(s(x|a'))} \]
    By LHS, we have $\sat{A}{\alpha}{s(x|a')}$, and by hypothesis, we have $\sat{B}{\alpha}{h\circ (s(x|a'))}$, so we are done.

    Assume RHS, we prove LHS. Let $b=h(a)$,
    \[ \sat{B}{\alpha}{(h\circ s)(x|h(a))} \iff \sat{B}{\alpha}{h\circ(s(x|a))} \]
    By hypothesis, we have $\sat{A}{\alpha}{s(x|a)}$. Done.
\end{proof}

\begin{corollary}
    \label{cor:HomomorphismToElementaryEquiv}
    If $\frakA \cong \frakB$, then $\frakA \equiv \frakB$. Recall that $\equiv$ means Elementary Equivalence (Def~\ref{def:ElementaryEquivalence}).
\end{corollary}

Corollary~\ref{cor:HomomorphismToElementaryEquiv} follows immediately from Homomorphism Theorem~\ref{thm:HomomorphismTheorem} because sentences do not care about assignments.

But the converse is not true. Take $\mathfrak{R} = (\realSet, <)$ and $\mathfrak{Q}=(\mathbb{Q}, <)$ as a counter example. We have claimed that they are elementary equivalent (though the proof is beyond the scope).

\begin{corollary}[Automorphism Theorem]
    \label{cor:AutomorphismTheorem}
    Let $h$ be an automorphism of $\frakA$. Let $R$ be an n-ary relation on $|\frakA|$ that is definable in $\frakA$. For every n-tuple $(a_1,\dots,a_n)$ of elements of $\frakA$,
    \[ (a_1,\dots,a_n) \in R \iff (h(a_1),\dots,h(a_n)) \in R \]
\end{corollary}
\begin{proof}
    Since $R$ is definable in $\frakA$,
    \begin{itemize}
        \item[] $(a_1,\dots,a_n) \in R$
        \item[$\iff$] $\assignSat{A}{\varphi}{a_1,\dots,a_n}$ 
        \item[$\iff$] $\assignSat{A}{\varphi}{h(a_1),\dots,h(a_n)}$ 
        \item[$\iff$] $(h(a_1),\dots,h(a_n))\in R$ 
    \end{itemize}
\end{proof}

Corollary~\ref{cor:AutomorphismTheorem} is often used for proving some relations are \emph{not definable}. For example, consider $\mathfrak{R} = (\realSet, <)$, its subset $\mathbb{N}$ is not definable in $\realStruct$.

Assume $\naturalSet$ is definable in $\realStruct$. Let $h(a)=a^3$. Obviously $h$ is an automorphism of $\realStruct$.

\begin{itemize}
    \item[] $a \in \mathbb{N}$
    \item[$\iff$] $\assignSat{R}{\varphi}{a}$
    \item[$\iff$] $\assignSat{R}{\varphi}{h(a)}$
    \item[$\iff$] $h(a)\in\naturalSet$
    \item[$\iff$] $a^3 \in \naturalSet$.  
\end{itemize}

This leads to a contradiction. If $a^3 = 2$, then no $a \in \naturalSet$.


\end{document}