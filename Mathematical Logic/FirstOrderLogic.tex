\chapter{First Order Logic}

\emph{"All men are mortal. Socrates is a man. Socrates is mortal."}

\section{Syntax of First-Order Logic}

We start with the symbols of a \textbf{first-order language} $\mathbb{L}$

There are two types of symbols

\begin{itemize}
    \item \textbf{Logical symbols}
    \item Non-logical symbols, a.k.a. \textbf{parameters}
\end{itemize}

\subsection{Symbols}

\subsubsection{Logical Symbols}

In a first-order language $\mathbb{L}$, we have the following symbols

\begin{enumerate}
    \item \textbf{Parentheses}. Two symbols `(' and `)'.
    \item \textbf{Logical connective symbols}. $\to$ and $\neg$
    \item \textbf{Variables}. An enumerable list of symbols $v_1,\dots,v_n,\dots$
    \item \textbf{Identity or Equalily Symbol} $=$ or $\doteq$. It may or may not be present in a particular first-order language
\end{enumerate}

Notice that we do not need $\vee$, $\wedge$. $\leftrightarrow$ because $\{\to, \neg\}$ is complete.

\subsubsection{Parameters}

\begin{enumerate}
    \item \textbf{Universal quantifier}. $\forall$
    \item For each $n>0$, there is a set (possibly empty) of objects called n-ary (or n-place) \textbf{predicate symbols}
    \item For each $n>0$, there is a set (possibly empty) of objects called n-ary (or n-place) \textbf{function symbols}
    \item A set of (possibly empty) of objects \textbf{constant symbols}
\end{enumerate}

\subsubsection{Further Requirements}

\begin{itemize}
    \item $\doteq$ is a 2-ary predicate symbols
    \item There is at least one predicate symbol
    \item The symbols are distinct, and no symbol is equal to a finite sequence of other symbols
\end{itemize}

\subsubsection{Example: Set Theory as First-Order Logic}

The Set Theory can be described by the following language

\begin{itemize}
    \item Equality
    \item Predicate symbols: 2-place $\dot{\in}$
    \item Constant symbols: empty set $\dot{\emptyset}$
    \item Function symbols: None
\end{itemize}

Note that the symbols are (currently) just interpreted as symbols and they do not have semantic meanings.

\begin{remark}
    We do not put restrictions or requirements on number of predicate, function or constant symbols.
\end{remark}

\subsection{Expressions}

An \textbf{expression} in a language $\mathbb{L}$ is a finite sequence of symbols.

\subsubsection{Terms}

\begin{definition}[Term Building Operation]
    \label{def:TermBuildingOperation}
    Given any n-ary function symbol $f$, the term-building operation $\mathcal{F}_f$ is defined by
    \[ \mathcal{F}_f (\sigma_1,\dots,\sigma_n) = f \sigma_1\dots\sigma_n \]
    We call $\sigma_i$ the arguments to $f$
\end{definition}

\begin{definition}[Term]
    \label{def:Term}
    A \textbf{term} is an expression
    
\end{definition}

For example, let $f$ and $g$ be 2-ary and 3-ary function symbols, then $gfc_1c_2v_3c_1$ is a term.

\begin{definition}[Term Sequence]
    \label{def:TermSequence}
    A \textbf{term sequence} is a finite sequence $t_1,\dots,t_n$ of expressions s.t. each $t_i$ is
    \begin{itemize}
        \item either a variable, a constant
        \item or
    \end{itemize}
\end{definition}

\subsubsection{Atomic Formulas}

\begin{definition}[Atomic Formula]
    \label{def:AtomicFormula}
    An expression is an \textbf{atomic formula} if it is of the form $P t_1\dots t_n$ where $t_1,\dots,t_n$ are terms and $P$ is a n-ary predicate symbol.
\end{definition}

\subsection{Well-Formed Formulas}

\begin{definition}[Formula-Building Operations]
    \label{def:FormulaBuildingOperation}
    \begin{itemize}
        \item $\xi_\neg(\alpha) = (\neg \alpha)$
        \item $\xi_\to(\alpha, \beta) = (\alpha\to\beta)$
    \end{itemize}
    
\end{definition}
