\chapter{First Order Logic}

\emph{"All men are mortal. Socrates is a man. Socrates is mortal."}

\section{Syntax of First-Order Logic}

We start with the symbols of a \textbf{first-order language} $\mathbb{L}$

There are two types of symbols

\begin{itemize}
    \item \textbf{Logical symbols}
    \item Non-logical symbols, a.k.a. \textbf{parameters}
\end{itemize}

\subsection{Symbols}

\subsubsection{Logical Symbols}

In a first-order language $\mathbb{L}$, we have the following symbols

\begin{enumerate}
    \item \textbf{Parentheses}. Two symbols `(' and `)'.
    \item \textbf{Logical connective symbols}. $\to$ and $\neg$
    \item \textbf{Variables}. An enumerable list of symbols $v_1,\dots,v_n,\dots$
    \item \textbf{Identity or Equalily Symbol} $=$ or $\doteq$. It may or may not be present in a particular first-order language
\end{enumerate}

Notice that we do not need $\vee$, $\wedge$. $\leftrightarrow$ because $\{\to, \neg\}$ is complete.

\subsubsection{Parameters}

\begin{enumerate}
    \item \textbf{Universal quantifier}. $\forall$
    \item For each $n>0$, there is a set (possibly empty) of objects called n-ary (or n-place) \textbf{predicate symbols}
    \item For each $n>0$, there is a set (possibly empty) of objects called n-ary (or n-place) \textbf{function symbols}
    \item A set of (possibly empty) of objects \textbf{constant symbols}
\end{enumerate}

\subsubsection{Further Requirements}

\begin{itemize}
    \item $\doteq$ is a 2-ary predicate symbols
    \item There is at least one predicate symbol
    \item The symbols are distinct, and no symbol is equal to a finite sequence of other symbols
\end{itemize}

\subsubsection{Example: Set Theory as First-Order Logic}

The Set Theory can be described by the following language

\begin{itemize}
    \item Equality
    \item Predicate symbols: 2-place $\dot{\in}$
    \item Constant symbols: empty set $\dot{\emptyset}$
    \item Function symbols: None
\end{itemize}

Note that the symbols are (currently) just interpreted as symbols and they do not have semantic meanings.

\begin{remark}
    We do not put restrictions or requirements on number of predicate, function or constant symbols.
\end{remark}

\subsection{Expressions}

An \textbf{expression} in a language $\mathbb{L}$ is a finite sequence of symbols.

\subsubsection{Terms}

\begin{definition}[Term Building Operation]
    \label{def:TermBuildingOperation}
    Given any n-ary function symbol $f$, the term-building operation $\mathcal{F}_f$ is defined by
    \[ \mathcal{F}_f (\sigma_1,\dots,\sigma_n) = f \sigma_1\dots\sigma_n \]
    We call $\sigma_i$ the arguments to $f$
\end{definition}

\begin{definition}[Term]
    \label{def:Term}
    A \textbf{term} is an expression built up from constant symbols and variables by applying some finite times (zero or more times) of term-building operations.
\end{definition}

For example, let $f$ and $g$ be 2-ary and 3-ary function symbols, then $gfc_1c_2v_3c_1$ is a term.

\begin{definition}[Term Sequence]
    \label{def:TermSequence}
    A \textbf{term sequence} is a finite sequence $t_1,\dots,t_n$ of expressions s.t. each $t_i$ is
    \begin{itemize}
        \item either a variable, a constant
        \item or is in the form of $f\sigma_1\dots\sigma_k$ where $f$ is a $f$-ary function and each $\sigma_1,\dots,\sigma_k$ occurs earlier in the sequence
    \end{itemize}
\end{definition}

\begin{proposition}
    An expression $t$ is a term iff there is a term sequence $t_1,\dots,t_n$ such that $t=t_n$
\end{proposition}

\subsubsection{Atomic Formulas}

\begin{definition}[Atomic Formula]
    \label{def:AtomicFormula}
    An expression is an \textbf{atomic formula} if it is of the form $P t_1\dots t_n$ where $t_1,\dots,t_n$ are terms and $P$ is a n-ary predicate symbol.
\end{definition}

\subsection{Well-Formed Formulas}

\begin{definition}[Formula-Building Operations]
    \label{def:FormulaBuildingOperation}
    \begin{itemize}~{}
        \item $\xi_\neg(\alpha) = (\neg \alpha)$
        \item $\xi_\to(\alpha, \beta) = (\alpha\to\beta)$
        \item $\mathcal{Q}_i(\gamma) = \forall v_i\gamma$
    \end{itemize}
\end{definition}

\begin{definition}[Well-Formed Formula]
    A \textbf{well-formed formula} (wff) is an expression built up from atomic formulas by applying some finite times of term-building operations.
\end{definition}

\begin{definition}[Well-Formed Sequence]
    A \textbf{well-formed sequence} is a finite sequence $\alpha_1,\dots,\alpha_n$ of expressions such that each $\alpha_i$ is
    \begin{itemize}
        \item either an atomic formula
        \item or is of the form of $(\neg \beta)$ or $(\beta\to\gamma)$ where $\beta$ and $\gamma$ occur earlier in the list
        \item or is of the form $\forall v_i\beta$ where $\beta$ occurs earlier in the list
    \end{itemize}
\end{definition}

\begin{proposition}
    The expression $\alpha$ is a wff if there is a well-formed sequence $\alpha_1,\dots,\alpha_k$ such that $\alpha = \alpha_k$
\end{proposition}

\subsection{Abbreviations}

\begin{itemize}
    \item $(\alpha\vee\beta)$ abbreviates $((\neg\alpha)\to\beta)$
    \item $(\alpha\wedge\beta)$ abbreviates $(\neg(\alpha \to (\neg\beta)))$
    \item $(\alpha\leftrightarrow\beta)$ abbreviates $(\alpha\to\beta)\wedge(\beta\to\alpha)$
    \item $\exists x\alpha$ abbreviates $(\neg\forall x(\neg\alpha))$
    \item $u\doteq t$ abbreviates $\doteq ut$
    \item $u \dot{\neq} t$ abbreviates $\dot{\neq} ut$
    \item Outer-most parentheses can be omitted
    \item $\neg$, $\forall$, $\exists$ apply to as little as possible
    \item $\wedge$, $\vee$ apply to as little as possible, subject to previous operators
    \item Grouping for repeated connectives is to the right
\end{itemize}

\subsection{Free Occurrence of Variables}

\begin{definition}[Free Occurrence]
    The variable $x$ \textbf{occurs free} in an atomic wff $\varphi$ iff it occurs in $\varphi$.

    $x$ \textbf{occurs free} in $\neg\alpha$ iff $x$ occurs free in $\alpha$.

    $x$ \textbf{occurs free} in $\alpha\to\beta$ iff $x$ occurs free in $\alpha$ or in $\beta$.

    $x$ \textbf{occurs free} in $\forall y \alpha$ iff $x$ occurs free in $\alpha$ and $x \neg y$.
\end{definition}

\begin{definition}[Sentence]
    $\varphi$ is a \textbf{sentence} iff no variable occurs free in $\varphi$.
\end{definition}
\begin{remark}
    Sentences are usually represented by $\sigma$ or $\tau$.
\end{remark}

We provide some examples

\begin{itemize}
    \item $\dot{0} \dot{<} \dot{1}$ does not have any free occurrence. It is a sentence,
    \item $\forall x(x\dot{<}y)$. $y$ occurs free but $x$ does not.
    \item $\forall x(\neg x \dot{<} \dot{0})$. No free occurrence.
    \item $\forall x\forall y (x \dot{<} y \to \exists z x\dot{<}z\wedge z\dot{<}y)$. No free occurrence.
\end{itemize}

\section{Semantics of First-Order Logic}

\subsection{Structures}

\begin{definition}
    Given a first order language $\mathbb{L}$, a \textbf{structure} $\mathfrak{A}$ for $\mathbb{L}$ consists of
    \begin{itemize}
        \item A non-empty set called the \textbf{universe} or \textbf{domain} of the structure, written as $|\mathfrak{A}|$
        \item For each n-ary predicate symbol $P$ of $\mathcal{L}$, other than $\doteq$, an n-ary relation $\mathbb{P}^{\mathfrak{A}}$ on $|\mathfrak{A}|$
        \item $\doteq^{\mathfrak{A}}$ is the identity relation on $|\mathfrak{A}|$. $\doteq^{\mathfrak{A}} = \{(a,b)|a,b\in|\mathfrak{A}, a =b|\}$
        \item For each n-ary function symbol $f$ of $\mathbb{L}$, an n-ary ooperation on the universe, i.e. an n-ary function $f^{\mathfrak{A}}: |\mathfrak{A}|\times\cdots\times|\mathfrak{A}|\mapsto|\mathfrak{A}|$
        \item For each constant symbol $c$ of $\mathbb{L}$, $c^{\mathfrak{A}}\in|\mathfrak{A}|$
    \end{itemize}
\end{definition}

\subsection{Assignments}

Let $\mathfrak{A}$ be a structure for language $\mathbb{L}$. Let $V$ be the set of variables, and $T$ be the set of terms.

\begin{definition}[Assignment Functions]
    An \textbf{assignment} for $\mathfrak{A}$ is a function $s:V\mapsto|\mathfrak{A}|$.
\end{definition}

\begin{definition}[Assignment to Terms]
    An assignment $s$ is extended to a function $\bar{s}:T\mapsto|\mathfrak{A}|$.
    \begin{itemize}
        \item $\bar{s}(v) = s(v)$ if $v$ is a variable
        \item $\bar{s}(c) = c^{\mathfrak{A}}$ if $c$ is a constant
        \item $\bar{s}(ft_1\dots t_n) = f^{\mathfrak{A}}(\bar{s}(t_1),\dots,\bar{s}(t_n))$ if $f$ is a n-ary function symbol and $t_1,\dots,t_n$ are terms
    \end{itemize}
\end{definition}

\subsubsection{Changing the Assignment Function}

Let $s$ be an assignment function, $x$ be a variable and $a\in\mathfrak{A}$. Then $s(x|a)$ is the new assignment, where for each variable $y$

\[s(x|a)(y) = \begin{cases}
    s(y) &\quad \text{if $(y\neq x)$}\\
    a &\quad \text{if $(y=x)$}
\end{cases}\]

This operation actually ``overrides'' the assignment of $s$ to $x$ and makes the assignment to $x$ equal to $a$.

\section{Satisfaction}

Given a first-order language $\mathbb{L}$, let $\mathfrak{A}$ be a structure for $\mathbb{L}$, let $s$ be an assignment for $\mathbb{L}$ and let $\varphi$ be a wff in $\mathbb{L}$. We denote $\mathfrak{A}$ to satisfy $\varphi$ with $s$ by $\vDash_{\mathfrak{A}}\varphi[s]$

Informally, it means ``\emph{The translation of $\varphi$ determined by $\mathfrak{A}$, where a variable $x$ is translated as $s(x)$, is true}''

\subsection{Satisfaction for Atomic Formulas}

\begin{definition}
    Given a language $\mathbb{L}$ and a structure $\mathfrak{A}$, let $s$ be an assignment, let $P$ be a n-ary predicate,
    \[ \vDash_{\mathfrak{A}} Pt_1\dots t_n[s] \Leftrightarrow (\bar{s}(t_1),\dots,\bar{s}(t_n)) \in P^{\mathfrak{A}} \]
    \[ \vDash_{\mathfrak{A}} \doteq t_1t_2[s] \Leftrightarrow \bar{s}(t_1) = \bar{s}(t_2) \]
\end{definition}

\subsection{Satisfaction for WFF}

\begin{definition}
    Suppose $\vDash_{\mathfrak{A}}\alpha[s]$ and $\vDash_{\mathfrak{A}}\beta[s]$ have been defined, then
    \begin{itemize}
        \item $\vDash_{\mathfrak{A}} \neg\alpha[s]$ iff not $\vDash_{\mathfrak{A}}\alpha[s]$
        \item $\vDash_{\mathfrak{A}}\alpha\to\beta[s]$ iff $\vDash_{\mathfrak{A}}\alpha[s]\Longrightarrow\vDash_{\mathfrak{A}}\beta[s]$
        \item $\vDash_{\mathfrak{A}}\forall x \alpha[s]$ iff $\forall a\in|\mathfrak{A}|$, $\vDash_{\mathfrak{A}}\alpha[s(x|a)]$
    \end{itemize}
\end{definition}

If $\vDash_{\mathfrak{A}}\varphi[s]$, we say \emph{$\mathfrak{A}$ satisfies $\varphi$ with $s$}, or \emph{$s$ satisfies $\varphi$ in the structure $\mathfrak{A}$}

\subsubsection{Satisfaction Depends Only on Variables that Occur Free}

\begin{lemma}
    \label{lem:LemmaForFreeOccurrenceThm}
    Let $\mathfrak{A}$ be a structure for $\mathbb{L}$, $s_1, s_2$ be two assignment for $\mathfrak{A}$ and $\varphi$ be a term of $\mathbb{L}$.

    If $s_1(x)=s_2(x)$ for every $x$ that occurs in $t$, then
    \[ \bar{s}_1(t) = \bar{s}_2(t) \]
\end{lemma}
\begin{proof}
    Proof by induction on $t$.
    \begin{itemize}
        \item[Base] If $t=c$ is a constant. It is straightforward that $\bar{s}_1(t) = \bar{s}_2(t) = c$. If $t=x$ is a variable, then by assumption we know that $\bar{s}_1(x) = s_1(x) = s_2(x) = \bar{s}_2(x)$. So we are done.
        \item[Induction] Consider a term $t=ft_1\dots t_n$. By inductive hypothesis we know that $\forall i$, $\bar{s}_1(t_i) = \bar{s}_2(t_i)$. $\bar{s}_1(t) = f^{\mathfrak{A}}(\bar{s}_1(t), \dots, \bar{s}_1(t) = f^{\mathfrak{A}}(\bar{s}_2(t), \dots, \bar{s}_2(t)) = \bar{s}_2(t)$.
    \end{itemize}
\end{proof}

\begin{theorem}
    Let $\mathfrak{A}$ be a structure for $\mathbb{L}$, $s_1, s_2$ be two assignment for $\mathfrak{A}$ and $\varphi$ be a wff of $\mathbb{L}$.

    If $s_1(x)=s_2(x)$ for every $x$ that occurs free in $\varphi$, then
    \[ \vDash_{\mathfrak{A}}\varphi[s_1] \Leftrightarrow \vDash_{\mathfrak{A}}\varphi[s_2] \]
\end{theorem}

\begin{proof}
    Prove by induction on $\varphi$.
    \begin{itemize}
        \item[Base] If $\varphi$ is an atomic formula $Pt_1\dots t_n$.
        \[ \vDash_{\mathfrak{A}} \varphi [s_1] \Leftrightarrow P^{\mathfrak{A}}(\bar{s}_1(t_1),\dots,\bar{s}_1(t_n)) \]
        \[ \vDash_{\mathfrak{A}} \varphi [s_2] \Leftrightarrow P^{\mathfrak{A}}(\bar{s}_2(t_1),\dots,\bar{s}_2(t_n)) \]

        We need to prove that the two RHSes are equivalent. By Lemma~\ref{lem:LemmaForFreeOccurrenceThm} we know that all the terms are equal under $s_1$ and $s_2$, and therefore $\vDash_{\mathfrak{A}} \varphi [s_1] = \vDash_{\mathfrak{A}} \varphi [s_2]$.

        \item[Induction] Consider $\varphi=\neg\alpha$.
        \[ \vDash_{\mathfrak{A}}(\neg\alpha)[s_1] \Leftrightarrow \nvDash_{\mathfrak{A}}\alpha[s_1] \Leftrightarrow \nvDash_{\mathfrak{A}}\alpha[s_2] \Leftrightarrow \vDash_{\mathfrak{A}}(\neg\alpha)[s_2] \]

        The case $\varphi = \alpha\to\beta$ is similar.

        Consider the case $\forall x \alpha$. We want to prove
        \[ \sat{A}{\forall x\alpha}{s_1} \Leftrightarrow \sat{A}{\forall x\alpha}{s_2} \]
        which is equivalent to
        \[ \forall a\in|\mathfrak{A}|\sat{A}{\alpha}{s_1(x|a)} \Leftrightarrow \forall a \in |\frakA| \sat{A}{\alpha}{s_2(x|a)} \]

        We only need to prove that
        \[ \forall y \text{occurring free in $\alpha$}, s_1(x|a)(y) = s_2(x|a)(y) \]

        If $y\neq x$, then $y$ is still occurring free in $\alpha$, and by inductive hypothesis they should equal. If $y=x$, then both sides are $a$. So we are done.
    \end{itemize}
\end{proof}
\begin{remark}
    This theorem is somewhat similar to the theorem in sentential logic, which states that we only need to consider sentence symbols. Similarly, in first-order logic, we only need to consider variables that occur free.
\end{remark}

\begin{definition}
    Let $\varphi$ be a wff s.t. all variables occurring free in $\varphi$ are included amoing $v_1,\dots,v_k$. Given $a_1,\dots,a_k\in\frakA$.

    \[ \assignSat{A}{\varphi}{a_1,\dots,a_k} \]

    means that $\sat{A}{\varphi}{s}$ for some $s:V\mapsto|\frakA|$ s.t. $s(v_i) = a_i$
\end{definition}

\begin{corollary}
    If $\sigma$ is a sentence then
    \begin{itemize}
        \item either $\sat{A}{\sigma}{s}$ for every assignment $s$. We say $\sigma$ is true in $\frakA$.
        \item or $\unsat{A}{\sigma}{s}$ for every assignment $s$. We say $\sigma$ is false in $\frakA$
    \end{itemize}
\end{corollary}

Therefore a sentence does not depend on $s$, and we can simply write $\sentSat{A}{\sigma}$ or $\sentunsat{A}{\sigma}$.

\subsection{Elementary Equivalence}

\begin{definition}[Elementary Equivalence]
    \label{def:ElementaryEquivalence}
    Let $\mathfrak{A}$ and $\mathfrak{B}$ be structures for the same language $\mathbb{L}$. $\mathfrak{A}$ and $\mathfrak{B}$ are \textbf{elementarily equivalent} ($\mathfrak{A} \equiv \mathfrak{B}$) if for every \emph{sentence} of $\mathbb{L}$
    \[ \sentSat{A}{\sigma} \iff \sentSat{B}{\sigma} \]
\end{definition}

\begin{remark}
    Elementary equivalence only take into consideration sentences.
\end{remark}

\begin{proposition}
    $\mathfrak{Q}$ and $\mathfrak{R}$ are elementary equivalent. But this is beyond the scope of the course.
\end{proposition}

\section{Models}

\subsection{Models}

\begin{definition}[Model]
    $\mathfrak{A}$ is a \textbf{model} of the sentence $\sigma$ if $\sentSat{A}{\sigma}$, i.e. if $\sigma$ is true in $\mathfrak{A}$. $\mathfrak{A}$ is a \textbf{model} of a set $\Sigma$ of sentences if $\mathfrak{A}$ is a model for every sentence in $\Sigma$. i.e. every sentence in $\Sigma$ is true in $\mathfrak{A}$.
\end{definition}

For example, consider a first-order language $\mathbb{L}$, with 2-ary predicate symbols $\dot{P}$ and $\doteq$. Given a structure $\mathfrak{A}$ of $\mathbb{L}$,

\begin{itemize}
    \item $\mathfrak{A}$ is a model of $\forall x \forall y x \doteq y$
    \begin{itemize}
        \item $\Leftrightarrow \sat{A}{x\doteq y}{s(x|a)(y|b)}$ for every $a,b \in |\mathfrak{A}|$
        \item $\Leftrightarrow$ $a = b$ for every $a,b \in|\mathfrak{A}|$
        \item $\Leftrightarrow$ $|\mathfrak{A}|$ contains only one element.
        \item Note that $|\frakA|$ cannot be empty because the universe of a structure must be non-empty.
    \end{itemize}
    \item $\mathfrak{A}$ is a model of $\forall x \forall y \dot{P}xy$
    \begin{itemize}
        \item iff $P^\mathfrak{A}(a,b)$ for all $a,b\in|\mathfrak{A}|$
        \item iff $\dot{P}^{\frakA} = |\frakA| \times |\frakA|$
    \end{itemize}
    \item $\mathfrak{A}$ is a model of $\forall x \forall y \neg\dot{P}xy$
    \begin{itemize}
        \item iff $P^{\frakA}(a,b)$ does not hold for all $a,b\in|\frakA|$
        \item iff $\dot{P}^{\frakA} = \emptyset$
    \end{itemize}
    \item $\frakA$ is a model of $\forall x \exists y \dot{P}xy$
    \begin{itemize}
        \item iff forall $a\in|\frakA|$, there is a $b\in|\frakA|$ s.t. $\dot{P}^{\frakA}(a,b)$
        \item iff the domain of $\dot{P}^{\frakA}$ is $|\frakA|$
        \item Conversely, if we want the range of $\dot{P}^{\frakA}$ is $|\frakA|$, we can write $\forall y \exists x \dot{P}xy$
    \end{itemize}
\end{itemize}

\subsection{Linearly Ordered Structures}

\begin{definition}[Trichotomy]
    Let $R$ be a binary relation, $R$ satisfies \textbf{trichotomy} if exactly one of the following is true
    \[ (a,b) \in R \quad (b,a) \in R \quad a = b \]
\end{definition}

\begin{definition}[Linear Ordering]
    A binary relation $R$ is a \textbf{linear ordering} on $A$ if $R$ is transitive and satisfies trichotomy on $A$.
\end{definition}

\begin{definition}
    Let $\mathbb{L}$ be the language with a binary relation symbol $\dot{R}$ and $\doteq$ (and no other symbols). Let $\frakA = (A,R)$, i.e. $A=|\frakA|$ and $R = \dot{R}^{\frakA}$.
    \begin{itemize}
        \item $\frakA$ is transitive if $R$ is transitive
        \item $\frakA$ is a linearly ordered structure if $R$ is a linear ordering on $\frakA$
    \end{itemize}
\end{definition}

For a set of structures with some certain properties, the set can be defined by a sentence.

Let $\frakA = (A,R)$,
\begin{itemize}
    \item $\frakA$ is transitive iff $\sentSat{A}{\sigma}$, where $\sigma=\forall x \forall y \forall z \dot{R}xy \to \dot{R}yz \to \dot{R}xz$. Therefore $\sigma$ defines the set of all transitive structures
    \item $\frakA$ is linearly ordered iff $\sentSat{A}{\sigma}$ where
    \begin{itemize}
        \item $\sigma_1 = \forall x \forall y \forall z \dot{R}xy \to \dot{R}yz \to \dot{R}xz$
        \item $\sigma_2 = \forall x \forall y (\dot{R}xy \vee x=y \vee \dot{R}yx)$
        \item $\sigma_3 = \forall x \forall y (\dot{R}xy \to \neg\dot{R}yx)$
        \item $\sigma = \sigma_1 \wedge \sigma_2 \wedge \sigma_3$
    \end{itemize}
    Therefore $\sigma$ defines the set of all linearly ordered structures
    \item $\domain{R} = A$ iff $\sentSat{A}{\sigma}$ where $\sigma = \forall x \exists y \dot{R}xy$
    \item $\range{R} = A$ iff $\sentSat{A}{\sigma}$ where $\sigma = \forall y \exists x \dot{R}xy$
    \item $R$ is a (total) function iff $\sentSat{A}{\sigma}$ where
    \begin{itemize}
        \item $\sigma_4 =  \forall x \exists y \dot{R}xy$
        \item $\sigma_5 = \forall x \forall y \forall z \dot{R}xy \to \dot{R}xz \to x \doteq z$
        \item $\sigma = \sigma_4 \wedge \sigma_5$
    \end{itemize}
\end{itemize}

\subsection{Elementary Class}

\begin{definition}[Elementary Class]
    A set of strucutures $\mathcal{K}$ is an \textbf{elementary class} if there exists a sentence $\sigma$ s.t.
    \[ \mathcal{K} = \{ \frakA | \frakA \text{ is a model of $\sigma$} \} \]
    i.e.
    \[ \mathcal{K} = \{ \frakA | \sentSat{A}{\sigma} \} \]
\end{definition}

For example, the set of all graphs is an elementary class.

Let $\mathbb{L}$ be the language with a binary predicate symbol $\dot{E}$ and $\doteq$, and no other symbols. A structure $\mathfrak{G} = (G, E)$ for $\mathbb{L}$ is a graph if
\begin{itemize}
    \item $E$ is symmetric (undirected graph)
    \item For every $a \in G$, $(a, a) \notin E$ (no self-loops)
\end{itemize}

To show that the set of graphs defined above is an elementary class, we need to show that there is a sentence $\sigma$ s.t.
\[ \text{$G$ is a graph} \Leftrightarrow \sentSat{G}{\sigma} \]

Therefore $\sigma$ should be able to represent the symmetric and non-reflexible properties.

\[\sigma = (\forall x \forall y (\dot{E}xy \to \dot{E}yx)) \wedge (\forall x(\neg \dot{E}xx))\]

\subsubsection{In the Wider Sense}

\begin{definition}[Elementary Class in the Wider Sense]
    A set of structures $\mathcal{K}$ is an \textbf{elementary class in the wider sense} ($EC_\Delta$) if there is a set $\Sigma$ of sentences s.t.
    \[ \mathcal{K} = \{ \frakA | \frakA \text{ is a model of } \Sigma \} \]
    i.e.
    \[ \mathcal{K} = \{ \frakA | \sentSat{A}{\sigma} \text{ for every $\sigma \in \Sigma$} \} \]
\end{definition}

For example, we can use a set $\Sigma$ to define the set of strucutures whose univerise is infinite.

\[ \Sigma = \{ \lambda_2, \lambda_3, \dots, \lambda_n, \dots \} \]

where $\lambda_i$ denotes ``At least $i$ elements exists in $|\frakA|$'', for example $\lambda_2 = \exists x \exists y x \neq y$.

However, it is hard to decide whether there is a single sentence $\sigma$ s.t. $\frakA$ is a model of $\sigma$ iff $|\frakA|$ is infinite.

\section{Logical Implications and Satisfiability}

\subsection{Logical Implications}

\begin{definition}
    Let $\Gamma$ be a set of wffs and $\varphi$ be a wff. $\Gamma$ \textbf{logically implies} $\varphi$, written as
    \[ \Gamma \vDash \varphi \]
    if for every structure $\frakA$ and every assignment $s$, if $\frakA$ satisfies $\Gamma$ with $s$, then $\frakA$ satisfies $\varphi$ with $s$
\end{definition}

\begin{theorem}
    For a set of sentences $\Sigma$, and a sentence $\sigma$, $\Sigma\vDash\sigma$ iff for every model $\frakA$ of $\Sigma$, $\frakA$ is a model of $\sigma$.
\end{theorem}

As before, we denote $\{\alpha\}\vDash\beta$ by $\alpha\vDash\beta$

\subsection{Logical Equivalence}

\begin{definition}[Logical Equivalence]
    $\alpha$ and $\beta$ are logically equivalent, written as $\alpha\vDash\Dashv\beta$ if $\alpha\vDash\beta$ and $\beta\vDash\alpha$
\end{definition}

\subsection{Valid Formulas}

\begin{definition}[Valid WFFs]
    Let $\varphi$ be a wff in the language $\mathbb{L}$. $\varphi$ is \textbf{valid} if $\semanticalImply{\emptyset}{\varphi}$, written as $\tautology{\varphi}$.
\end{definition}

For example,

\begin{itemize}
    \item $x \doteq x$ is valid
    \item $\exists x \doteq x$ is valid
    \item $\forall x \exists y x\dot{\neq}y$ is not valid
    \item $\dot{P}x \vee \neg \dot{P}x$ is valid
    \item $\exists x(\dot{P}x \to \forall x \dot{P}x)$ is valid
\end{itemize}

We detail the proof of the last example

\begin{itemize}
    \item $\sat{A}{\exists x (Px\to\forall x Px)}{s}$
    \item[$\Leftrightarrow$] There is some $a \in |\frakA|$ s.t. $\sat{A}{Px \to \forall x Px}{s(x|a)}$
    \item[$\Leftrightarrow$] $\sat{A}{\dot{P}(x)}{s(x|a)} \Rightarrow \sat{A}{\forall x Px}{s(x|a)}$
    \item[$\Leftrightarrow$] $\sat{A}{\dot{P}(x)}{s(x|a)} \Rightarrow \sat{A}{\forall x Px}{s(x|a)}$
    \item[$\Leftrightarrow$] There is some $a$ s.t. if $a \in \dot{P}^{\frakA}$, then for every $b \in |\frakA|$, $b\in\dot{P}^{\frakA}$
    \item If there is some $a$ s.t. $a \notin \dot{P}^{\frakA}$, then the consequence holds trivially
    \item If there is no $a$ s.t. $a \notin \dot{P}^{\frakA}$, this means that for every $a \in |\frakA|$, $a\in\dot{P}^{\frakA}$. This is exactly the consequence so we are done
\end{itemize}

\subsection{Satisfiability}

\begin{definition}[Satisfiability]
    \begin{itemize}
        \item The wff $\varphi$ is \textbf{satisfiable} if there is some structure $\frakA$ and some assignment $s$ s.t. $\sat{A}{\varphi}{s}$.
        \item The set of wffs $\Gamma$ is \textbf{satisfiable} if there is some structure $\frakA$ and some assignment $s$ s.t. $\sat{A}{\varphi}{s}$ for every $\varphi$ in $\Gamma$.
    \end{itemize}
\end{definition}

\begin{theorem}
    $\varphi$ is not satisfiable iff $\neg\varphi$ is valid
\end{theorem}

\section{Definability}

\begin{definition}[Relations Defined by WFFs]
    Let $\frakA$ be a structure, and $\varphi$ be a wff, and $n$ be such that the variables occurring free in $\varphi$ are included among $v_1,\dots,v_n$

    The n-ary relation \textbf{defined by $\varphi$ in $\frakA$} is
    \[ \{ (a_1,\dots,a_n) | \assignSat{A}{\varphi}{a_1,\dots,a_n}\} \]
\end{definition}

Let $\mathfrak{N} = (\mathbb{N},\le, +, 1)$, the 2-ary relation $\{(a,b)|a < b\}$ is defined by
\[ v_1 \dot{+}\dot{1}\doteq v_2 \]

To show this, we will show that $(a,b) \in R \iff \assignSat{N}{\varphi}{a,b}$.

If $(a,b) \in R$, $\assignSat{N}{\varphi}{a,b} \iff a+1 \le b$

Conversely, if $\assignSat{N}{\varphi}{a,b}$, it's equivalent to $a+1 \le b$, which mathematically implies that $a < b$

\begin{definition}[Definability]
    The relation $R$ is \textbf{definable in the structure $\frakA$} if there is some wff $\varphi$ that defines it in $\frakA$.
\end{definition}

We show some examples for definability of functions. Let $\mathfrak{N} = (\mathbb{N}, <, +, x, 0, 1)$.

\begin{itemize}
    \item $v_1 \dot{+} v_2 \doteq v_3$ defines $\{ (a,b,c) | a + b = c\}$ which is the same as function $f$, where $f(a,b) = a + b$.
\end{itemize}

\subsection{Definable Relations}

\begin{definition}[Relations Defined by WFFs]
    Let
    \begin{itemize}
        \item $\frakA$ be a structure
        \item $\varphi$ be a wff and $n$ be such that the variables occurring free in $\varphi$ are included among $v_1,\dots,v_n$
    \end{itemize}

    The $n$-ary relation defined by $\varphi$ in $\frakA$ is
    \[ \{ (a_1,\dots,a_n) | \assignSat{A}{\varphi}{a_1,\dots,a_n} \} \]
\end{definition}

For example,

\begin{itemize}
    \item Let $\mathfrak{R} = (\mathbb{R},<,+,\times,0,1)$. The $1$-ary relation $\{ a \in \mathbb{R} | 0 \le a \}$ is defined by
    \[ \exists v_2, v_1 \doteq v_1 \times v_2 \]
    \item Let $\mathfrak{R} = (\mathbb{R},<,+,\times,0,1)$。 The $2$-ary relation $\{ (a,b) | a < b \}$ is defined by
    \[ \exists v_3 (v_1\dot{+}(\dot{1}\dot{+}v_3)\doteq v_2) \]
\end{itemize}

\begin{definition}[Definable Relations]
    The relation $R$ is \textbf{definable in structure} $\frakA$ if there is some wff that defines it in $\frakA$

    Let $f$ be a n-ary function $f$ whose domain is a subset $|\frakA| \times \dot \times |\frakA|$ and whose range is a subset of $|\frakA|$, $f$ is definable in $\frakA$ if the $(n+1)$-ary relation
    \[ \{ (a_1,\dots,a_n, b) | f(a_1,\dots,a_n) = b \} \]
    is definable in $\frakA$.
\end{definition}

Consider $\mathfrak{N} = (\mathbb{N},<,+,\times,0,1)$.
\begin{itemize}
    \item $v_1 + v_2 = v_3$ defines $\{ (a,b,c) | a + b = c \}$, which is the same as function $f(a,b) = a + b$
    \item $v_1 + v_3 = v_2$ defines $\{ (a,b,c) | a+c = b \}$, which is the same as function $f(a,b)$
    \[ f(a,b) = \begin{cases}
        b - a &\quad a \le b\\
        Undefined &\quad o.w.
    \end{cases} \]
\end{itemize}

\begin{lemma}
    Given a structure $\frakA$, the set of definable relations is \emph{enumerable}
\end{lemma}
\begin{lemma}
    Not every subset of $\mathbb{N}$ is definable.
\end{lemma}

The proof of the two lemmas are similar. Note that the set of wffs is enumerable, and every wff may define only one relation. And the set of all subsets of $\mathbb{N}$ is uncountable.

We now move from $\mathbb{N}$ to $\mathbb{R}$ and consider a more generall case. Consider whether the following subsets of $\mathbb{R}$ are definable in $\mathfrak{R} = (\mathbb{R}, <)$

\begin{itemize}
    \item $\emptyset$. Yes.
    \item $\mathbb{N}$. Yes.
    \item Anything else?
\end{itemize}

\section{Homomorphisms}

\begin{definition}[Homomorphism]
    Let $\mathfrak{A}$ and $\mathfrak{B}$ be structures for $\mathbb{L}$. A \textbf{homomorphism} from $\frakA$ to $\mathfrak{B}$ is a function $h:|\frakA| \mapsto |\mathfrak{B}|$ s.t.
    \begin{itemize}
        \item For every n-ary predicate symbol $R$, other than $\doteq$, and $a_1,\dots,a_n \in |\frakA|$,
        \[ (a_1,\dots,a_n) \in R^{\frakA} \iff (h(a_1),\dots,h(a_n))\in R^{\mathfrak{B}} \]
        \item For every n-ary function symbol $f$, and $a_1,\dots,a_n \in |\frakA|$,
        \[ h(f^\frakA(a_1,\dots,a_n)) = f^\mathfrak{B}(h(a_1),\dots,h(a_n)) \]
        \item For every constant symbol $c$
        \[ h(c^\frakA) = c^\mathfrak{B} \]
    \end{itemize}
\end{definition}

\begin{definition}[Onto of Homomorphism]
    $h$ is a homomorphism of $\frakA$ \textbf{onto} $\mathfrak{B}$ if $h$ is a homomorphism from $\mathfrak{A}$ to $\mathfrak{B}$ and $h$ maps $\frakA$ onto $\mathfrak{B}$.
\end{definition}

\begin{definition}[Isomorphism]
    A homomorphism $h$ from $\mathfrak{A}$ to $\mathfrak{B}$ is an \textbf{isomorphism} if $h$ is one-to-one
\end{definition}

\begin{definition}[Isomorphic]
    The structures $\frakA$ and $\mathfrak{B}$ are \textbf{isomorphic}, denoted by $\frakA \cong \mathfrak{B}$ if there is some \emph{isomorphism} of $\frakA$ \emph{onto} $\mathfrak{B}$. (One-to-one correspondence)
\end{definition}

\begin{definition}[Automorphism]
    An \textbf{automorphism} of $\frakA$ is an isomorphism of $\frakA$ onto $\frakA$
\end{definition}

\subsubsection{Examples of Homomorphism}

For example, let $\frakA = (\naturalSet, <^\naturalSet, +^\naturalSet)$, $\mathfrak{B} = (\mathbb{E}, <^\mathbb{E}, +^\mathbb{E})$, where $\mathbb{E}$ is the set of even non-negative integers. Then we claim that $h(n)=2n$ is an isomorphism of $\mathfrak{A}$ onto $\mathfrak{B}$

To do this, we show that 1) $h$ is a homomorphism; 2) $h$ is one-to-one; 3) $h$ is onto

\begin{itemize}
    \item[$\dot{<}$] $(a,b) \in <^\mathbb{N} \iff (h(a), h(b)) = (2a,2b) \in <^\mathbb{E}$
    \item[$\dot{+}$] $h(a +^\mathbb{N} b) = (h(a) +^\mathbb{E} h(b)) = (2a +^\mathbb{E} 2b)$
\end{itemize}

However, let $\mathfrak{C} = (\mathbb{O}, <^\mathbb{O}, +^\mathbb{O})$, where $\mathbb{O}$ is the set of all odd non-negative integers, then there is no isomorphism of $\frakA$ onto $\mathfrak{C}$. Infact $\mathfrak{C}$ is not even a valid strucutre because $+^\mathbb{O}$ is not closed.

\subsubsection{Automorphism of $\mathfrak{R}=(\realSet, <)$}

Consider which of the following $h$ are automorphisms of $\mathfrak{R}$. Note that to show this, we need to show that 1) $h$ is a homomorphism; 2) $h$ is one-to-one; 3) $h$ is onto; and 4) $h$ maps $\realSet$ to $\realSet$.

\begin{itemize}
    \item The identity function. Obviously yes.
    \item $h(a) = a + 3$. Yes.
    \item $h(a) = a - 4$. Yes.
    \item $h(a) = 2a$. Yes.
    \item $h(a) = -a$. Yes.
    \item $h(a) = ka + l$. Yes if $k>0$.
    \item $h(a) = a^3$. Yes.
    \item $h(a) = a^2$. No.
\end{itemize}

\subsubsection{Automorphism of $\mathfrak{N} = (\naturalSet, <)$}

Obviously the identity function is an automorphism. We consider other cases.

If we map $0$ to any $n>0$, i.e. $h(0) = n > 0$. Since $h$ is onto, there exists some $m > 0$ s.t. $h(m) = 0$. And here comes a problem
\[ m > 0 \iff h(m) = 0 > h(0) = n > 0 \]
Therefore $0$ can only be mapped to $0$, i.e. $h(0) = 0$

Similarly, $h(1)$ can only be mapped to $1$, and for each $n$, $h(n) = n$. Therefore the identity function $h(n) = n$ is the \emph{only} automorphism of $\naturalStruct$.

\subsection{Substructures}

We now consider a special kind of isomorphism

\begin{definition}[Substructures]
    Let $\frakA = (A,\dots)$ and $\frakB = (B,\dots)$ be structures for $\mathbb{L}$. $\frakA$ is a \textbf{substructure} of $\frakB$, denoted by $\frakA \subseteq \frakB$ if
    \begin{itemize}
        \item $A \subseteq B$
        \item For every $k$-ary predicate symbol,
        \[ P^\frakA = P^\frakB \cap A^k \]
        Note that this is to guarantee that $(a_1,\dots,a_k)\in P^\frakB \Longrightarrow (a_1,\dots,a_k)\in P^\frakA$
        \item For every $k$-ary function $f$ and every $k$-tuple of $A$
        \[ f^\frakA(a_1,\dots,a_k) = f^\frakB(a_1,\dots,a_k) \]
        \item For every constant $c$
        \[ c^\frakA = c^\frakB \]
    \end{itemize}
\end{definition}
\begin{remark}
    Substructures are defined under identity map $h(x)=x$. The identity map is an isomorphism of $\frakA$ into $\frakB$ iff
    \begin{enumerate}
        \item For each predicate $P$, $P^\frakA$ is the restriction of $P^\frakB$ to $A$
        \item For each function $f$, $f^\frakA$ is the restriction of $f^\frakB$ to $A$
        \item $c^\frakA = c^\frakB$.
    \end{enumerate}
    If these conditions are met, then $\frakA$ is called a \textbf{substructure} of $\frakB$.
\end{remark}

\subsection{Homomorphism Theorem}

\begin{lemma}
    \label{lem:HomomorphismLemma}
    Let $\frakA$ and $\frakB$ be structures for the language $\mathbb{L}$. Let $h$ be a homomorphism from $\frakA$ to $\frakB$, and $s:V\to|\frakA|$ be an assignment for $\frakA$. Then for every term $t$ of $\mathbb{L}$
    \[ h(\bar{s}(t)) = \overline{h\circ s}(t) \]
\end{lemma}
\begin{proof}
    Prove by induction.
    \begin{itemize}
        \item[] \textbf{Base Case.} If $t=c$, then
        \[ h(\bar{s}(c)) = h(c^\frakA) = c^\frakB = \overline{h\circ s(c)} \]
        If $t=v$, then
        \[ h(\bar{s}(x)) = h(s(x)) = h \circ s(x) \]
        \item[] \textbf{Inductive Case.} If $t = ft_1,\dots,t_n$,
        \[ h(\bar{s}(t)) = h(f^\frakA(\bar{s}(t_1),\dots,\bar{s}(t_n))) = f^\frakB(\bar{s}(t_1),\dots,\bar{s}(t_n)) \]
        \[ \overline{h\circ s} = f^\frakB(\overline{h\circ s}(t_1),\dots,\overline{h\circ s}(t_n)) \]
        From inductive hypothesis we know $\overline{h\circ s}(t_k) = h(\bar{s}(t_k))$, and we are done.
    \end{itemize}
\end{proof}

\begin{theorem}[Homomorphism Theorem]
    \label{thm:HomomorphismTheorem}
    Let $h$ be a homomorphism form $\frakA$ to $\frakB$ and $s$ be an assignment function for $\frakA$. The statement
    \[ \sat{A}{\varphi}{s} \iff \sat{B}{\varphi}{h \circ s} \]
    \begin{itemize}
        \item is true for every quantifier-free wff $\varphi$ not containing $\doteq$
        \item is true for every quantifier-free wff $\varphi$ if $h$ is one-to-one
        \item is true for every wff $\varphi$ wff $\varphi$ not containing $\doteq$ if $h$ is onto
        \item is true for every wff $\varphi$ if $h$ is an isomorphism of $\frakA$ onto $\frakB$ ($\frakA \cong \frakB$)
    \end{itemize}
\end{theorem}
\begin{proof}
    Prove by induction on $\varphi$.
    \begin{itemize}
        \item[] \textbf{Base Case.} Let $\varphi = Pt_1\dots t_n$.
         Since $h$ is a homomorphism, by definition we have
         \[ (\bar{s}(t_1),\dots,\bar{s}(t_n))\in P^\frakA \iff (h(\bar{s}(t_1)),\dots, h(\bar{s}(t_n))) \in P^\frakB \]
         Further, by Lemma~\ref{lem:HomomorphismLemma},
         \[ (\overline{h\circ s}(t_1),\dots,\overline{h\circ s}(t_2)) = (h(\bar{s}(t_1)),\dots, h(\bar{s}(t_n))) \in P^\frakB \]
         So we are done.
         \item[] \textbf{Inductive Case.}\begin{itemize}
             \item If $\varphi = \neg \alpha$, by induction hypothesis
             \[ \sat{A}{\alpha}{s} \iff \sat{B}{\alpha}{h \circ s} \]
             We can negate both sides
             \[ \unsat{A}{\alpha}{s} \iff \unsat{B}{\alpha}{h \circ s} \]
             Therefore,
             \[ \sat{A}{\neg\alpha}{s} \iff \sat{B}{\neg\alpha}{h\circ s} \]
             \item If $\varphi = \alpha \to \beta$.
             \[ \sat{A}{\alpha\to\beta}{s} \iff \sat{B}{\alpha\to\beta}{h\circ s} \]
             is equivalent to
             \[ \text{If} \sat{A}{\alpha}{s} \text{then} \sat{A}{\beta}{s} \iff \text{If} \sat{B}{\alpha}{h \circ s} \text{then} \sat{B}{\beta}{s} \]
             By induction hypothesis, we are done.
         \end{itemize}
    \end{itemize}
    Until now we have proved (a). We now further consider $\doteq$ and $\forall$

    We start from $\doteq$, which is a base case,
    If $\varphi = t_1\doteq t_2$
        \[ \sat{A}{t_1\doteq t_2}{s} \iff \sat{B}{t_1\doteq t_2}{h\circ s} \]
        \[ \bar{s}(t_1) = \bar{s}(t_2) \iff \overline{h\circ s}(t_1) = \overline{h\circ s}(t_2) \iff h(\bar{s}(t_1)) = h(\bar{s}(t_2)) \]
        $\Leftrightarrow$ always holds. However, $\Leftarrow$ requires an additional condition that $h$ is one-to-one.

    Finally, consider $\varphi=\forall x\alpha$ in the inductive case.
    \[ \sat{A}{\forall x \alpha}{s} \iff \sat{B}{\forall x \alpha}{h\circ s} \]
    \[ \text{For any $a\in|\frakA|$,} \sat{A}{\alpha}{s(x|a)} \iff \text{For any $b\in|\frakB|$,}\sat{B}{\alpha}{(h\circ s)(x|b)} \]
    Assume LHS, we prove RHS. Note that since $a$ and $b$ are arbitrary, we cannot relate them without additional conditions. Therefore to prove this we would require $h$ to be \emph{onto}. Then there is some $a'\in|\frakA|$ s.t. $h(a')=b$.
    \[ \sat{B}{\alpha}{(h\circ s)(x|h(a'))} \iff \sat{B}{\alpha}{h\circ(s(x|a'))} \]
    By LHS, we have $\sat{A}{\alpha}{s(x|a')}$, and by hypothesis, we have $\sat{B}{\alpha}{h\circ (s(x|a'))}$, so we are done.

    Assume RHS, we prove LHS. Let $b=h(a)$,
    \[ \sat{B}{\alpha}{(h\circ s)(x|h(a))} \iff \sat{B}{\alpha}{h\circ(s(x|a))} \]
    By hypothesis, we have $\sat{A}{\alpha}{s(x|a)}$. Done.
\end{proof}

\begin{corollary}
    \label{cor:HomomorphismToElementaryEquiv}
    If $\frakA \cong \frakB$, then $\frakA \equiv \frakB$. Recall that $\equiv$ means Elementary Equivalence (Def~\ref{def:ElementaryEquivalence}).
\end{corollary}

Corollary~\ref{cor:HomomorphismToElementaryEquiv} follows immediately from Homomorphism Theorem~\ref{thm:HomomorphismTheorem} because sentences do not care about assignments.

But the converse is not true. Take $\mathfrak{R} = (\realSet, <)$ and $\mathfrak{Q}=(\mathbb{Q}, <)$ as a counter example. We have claimed that they are elementary equivalent (though the proof is beyond the scope).

\begin{corollary}[Automorphism Theorem]
    \label{cor:AutomorphismTheorem}
    Let $h$ be an automorphism of $\frakA$. Let $R$ be an n-ary relation on $|\frakA|$ that is definable in $\frakA$. For every n-tuple $(a_1,\dots,a_n)$ of elements of $\frakA$,
    \[ (a_1,\dots,a_n) \in R \iff (h(a_1),\dots,h(a_n)) \in R \]
\end{corollary}
\begin{proof}
    Since $R$ is definable in $\frakA$,
    \begin{itemize}
        \item[] $(a_1,\dots,a_n) \in R$
        \item[$\iff$] $\assignSat{A}{\varphi}{a_1,\dots,a_n}$ 
        \item[$\iff$] $\assignSat{A}{\varphi}{h(a_1),\dots,h(a_n)}$ 
        \item[$\iff$] $(h(a_1),\dots,h(a_n))\in R$ 
    \end{itemize}
\end{proof}

Corollary~\ref{cor:AutomorphismTheorem} is often used for proving some relations are \emph{not definable}. For example, consider $\mathfrak{R} = (\realSet, <)$, its subset $\mathbb{N}$ is not definable in $\realStruct$.

Assume $\naturalSet$ is definable in $\realStruct$. Let $h(a)=a^3$. Obviously $h$ is an automorphism of $\realStruct$.

\begin{itemize}
    \item[] $a \in \mathbb{N}$
    \item[$\iff$] $\assignSat{R}{\varphi}{a}$
    \item[$\iff$] $\assignSat{R}{\varphi}{h(a)}$
    \item[$\iff$] $h(a)\in\naturalSet$
    \item[$\iff$] $a^3 \in \naturalSet$.  
\end{itemize}

This leads to a contradiction. If $a^3 = 2$, then no $a \in \naturalSet$.
