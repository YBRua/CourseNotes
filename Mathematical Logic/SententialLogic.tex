\chapter{Sentential Logic}

\section{Grammar}
\label{sec:Gramma}

\subsection{Symbols}
\label{sub:Symbols}
\begin{itemize}
    \item Logical Symbols
    \begin{itemize}
        \item Sentential Connectives
        \begin{itemize}
            \item $\neg$
            \item $\wedge$
            \item $\vee$
            \item $\rightarrow$
            \item $\leftrightarrow$
        \end{itemize}
        \item Parentheses
    \end{itemize}
    \item Non-logical Symbols: An enumerable set of elements
\end{itemize}

\subsection{Expressions}
\label{sub:Expressions}

\begin{definition}[Expression]
    \label{def:Expression}
    An expression is a finite sequence of symbols.
\end{definition}
\begin{remark}
    The set of all expressions is enumerable.
\end{remark}

We often use Greek alphabets $\alpha,\beta,\dots$ to represent expressions.

\subsection{Well-Formed Formulas}
\label{sub:WellFormedFormulas}

\begin{definition}[Well-Formed Formula]
    \label{def:WFF}
    A \textbf{well-formed formula} (or formula or wff) is an expression built up from sentence symbols by applying some finite times of \emph{formula building operations}
\end{definition}

\begin{definition}[Formula Building Operations]~{}
    \begin{itemize}
        \item $\xi_{\neg}(\alpha) = (\neg\alpha)$
        \item $\xi_{\wedge}(\alpha,\beta) = (\alpha \wedge \beta)$
        \item $\xi_{\vee}(\alpha,\beta) = (\alpha\vee\beta)$
        \item $\xi_{\rightarrow}(\alpha,\beta) = (\alpha\rightarrow\beta)$
        \item $\xi_{\leftrightarrow}(\alpha,\beta) = (\alpha\leftrightarrow\beta)$
    \end{itemize}
\end{definition}
\begin{remark}
    Do NOT omit the parentheses.
\end{remark}

\begin{definition}[Well-Formed Sequences of Expressions]
    \label{def:WellFormedSeqOfExpr}
    A \textbf{well-formed sequence of expressions} is a finite sequence $\alpha_1,\alpha_2,\dots,\alpha_n$ of expressions such that each $\alpha_i$ is either
    \begin{itemize}
        \item A sentence symbol
        \item $(\neg\alpha_j)$ for some $j < i$
        \item $(\alpha_j \wedge \beta_k)$ for some $j,k<i$
        \item $(\alpha_j \vee \beta_k)$ for some $j,k<i$
        \item $(\alpha_j \leftarrow \beta_k)$ for some $j,k<i$
        \item $(\alpha_j \leftrightarrow \beta_k)$ for some $j,k<i$
    \end{itemize}
\end{definition}

\begin{proposition}
    An expression $\alpha$ is a well-formed formula iff there is a well-formed sequence $(\alpha_1,\dots,\alpha_n)$ s.t. $\alpha = \alpha_n$
\end{proposition}

\subsection{The Induction Principle}
\label{sub:Induction}

Well-formed formulas are a form of inductive definitions with
\begin{itemize}
    \item Basic building blocks
    \item Closing operations
\end{itemize}

\begin{theorem}[The Induction Principle]
    \label{thm:InductionPrinciple}
    Let $S$ be a set of wffs ($S \subseteq W$), if
    \begin{enumerate}
        \item Every sentence symbol is in $S$
        \item For each wff $\alpha$ and $\beta$, if $\alpha$ and $\beta$ are in $S$ then each of the following are in $S$
        \begin{itemize}
            \item $(\neg \alpha)$
            \item $(\alpha \wedge \beta)$
            \item $(\alpha \vee \beta)$
            \item $(\alpha \rightarrow \beta)$
            \item $(\alpha \leftrightarrow \beta)$
        \end{itemize}
    \end{enumerate}
    Then $S$ is the set of \emph{all wffs} ($S = W$).
\end{theorem}

We can see an example of Indunction.

\begin{proposition}
    Every wff has the same number of left parentheses as right parenthesis
\end{proposition}
\begin{proof}
    Let $S\triangleq \{\alpha|\alpha\text{has equal number of left and right parentheses}\}$.
    \begin{itemize}
        \item[Base] $\alpha = A$. Straightforward. Sentense symbols do not have parenthesis
        \item[Step]
        \begin{enumerate}
            \item Let $\beta \in S$, $\alpha = (\neg \beta) \in S$.
            \item Let $\alpha_1, \alpha_2 \in S$, $\alpha = (\alpha_1 \wedge \alpha_2) \in S$.
            \item $\cdots$
        \end{enumerate} 
    \end{itemize}
\end{proof}

\subsection{Parsing Formulas}
\label{sub:ParsingFormulas}

The induction principle actually gives an algorithm for parsing formulas.

On input expression $\alpha$

\begin{enumerate}
    \item If is leaf node, we are done. Return.
    \item The first symbol must be `('.
    \item If the second symbol is `$\neg$', then expect an non-empty expression $\beta$ and parse $\beta$.
    \item If the second symbol is not `$\neg$', then expect a non-empty expression $\beta_1$, an operator and another expression $\beta_2$.
\end{enumerate}

\subsection{Abbreviations}
\label{sub:Abbreviations}

\begin{itemize}
    \item The outermost parentheses can be omitted.
    \item $\neg$ appplies to as little as possible, with the highest precedence.
    \item $\wedge$ and $\vee$ apply to as little as possible, subject to $\neg$.
    \item $\rightarrow$ and $\leftrightarrow$ apply to as little as possible, subject to other operators.
    \item When handling operators with the same precedence, grouping is always to the right. $A \rightarrow B \rightarrow C= (A \rightarrow (B\rightarrow C))$.
\end{itemize}

\section{Semantics}
\label{sec:Semantics}

\subsection{Truth Assignments}

Consider a math domain $\{T,F\}$ of truth values

\begin{itemize}
    \item T is called truth
    \item F is called falsity
\end{itemize}

\begin{definition}[Truth Assignment]
    A truth assignment for a set $\mathcal{S}$ of sentence symbols is a function
    \[ v:\mathcal{S}\mapsto\{T,F\} \]
\end{definition}

\begin{definition}[Extended Truth Assignment]
    Let $\bar{\mathcal{S}}$ be the set of wffs that can be built up from $\mathcal{S}$ by formula-building operations. Let $v$ be a truth assignment for $\mathcal{S}$. An \textbf{extension} $\bar{v}$ of $v$
    \[ \bar{v}:\bar{\mathcal{S}}\mapsto \{T,F\} \]
    assigns truth values to every wff in $\mathcal{S}$ s.t.
    \begin{itemize}
        \item $\bar{v}(\alpha) = v(\alpha)$ if $\alpha \in \mathcal{S}$
        \item $\bar{v}(\neg(\alpha))$ is T if $\bar{v}(\alpha)$ is F and F otherwise.
        \item $\bar{v}((\alpha\wedge\beta))$ is T if $\bar{v}(\alpha)$ is T and $\bar{v}(\beta)$ is T and F otherwise.
        \item $\bar{v}((\alpha\vee\beta))$ is T if $\bar{v}(\alpha)$ is T or $\bar{v}(\beta)$ is T and F otherwise.
        \item $\bar{v}((\alpha\to\beta))$ is F if $\bar{v}(\alpha)$ is T and $\bar{v}(\beta)$ is F and T otherwise.\footnote{Emphasizes the promise of a condition implying a consequence. If the condition is falsy then no guarantee for the consequence. \emph{“骗你是小狗”}}
        \item $\bar{v}((\alpha\leftrightarrow\beta))$ is T if $\bar{v}(\alpha) = \bar{v}(\beta)$ and is F otherwise.
    \end{itemize}
\end{definition}

\begin{theorem}[Determinacy of Truth Assignments]
    \label{thm:DeterminacyofTruthAssignments}
    For every $v_1$ and $v_2$ and wff $\alpha$, if
    \[ v_1(A) = v_2(A) \]
    for every sentence symbol that occurs in $\alpha$, then
    \[ \bar{v}_1(\alpha) = \bar{v}_2(\alpha) \]
\end{theorem}

\begin{remark}
    To determine the value of $\bar{v}(\alpha)$, we only need to know the value of $v$ on the sentence symbols that occur in $\alpha$. This leads to the method of \textbf{truth tables}.
\end{remark}

\subsection{Satisfiability}
\label{sub:Satisfiability}

We first introduce some new notations. We use captial Greek letters, $\Delta$, $\Sigma$, etc. to represent sets of wffs. And we use $\Sigma;\alpha$ to represent $\Sigma \cup \{\alpha\}$.

\begin{definition}~{}
    \begin{itemize}
        \item $v$ satisfies $\alpha$ if $\bar{v}(\alpha) = T$
        \item $v$ satisfies $\Sigma$ if $\bar{v}(\alpha) = T$ for every $\alpha \in \Sigma$.
    \end{itemize}
\end{definition}

\begin{definition}[Satisfiability]~{}
    \label{def:Satisfiability}
    \begin{itemize}
        \item $\alpha$ is satisfiable if there exists some $v$ that satisfies $\alpha$
        \item $\Sigma$ is satisfiable if there exists some $v$ that satisfies $\Sigma$
    \end{itemize}
\end{definition}

\begin{remark}
    Every $v$ satisfies $\emptyset$. Because $v$ satisfies $\emptyset$ iff
    \[ \forall \alpha, \alpha\in\emptyset \Longrightarrow \bar{v}(\alpha) = T \]
    The assumption itself is false, and therefore the consequence is always true.
\end{remark}

\subsection{Semantic Implications}
\label{sub:SemanticImplications}

\begin{definition}
    A set of wffs $\Sigma$ semantically implies $\alpha$ when every truth assignment satisfying $\Sigma$ also satisfies $\alpha$.
    \begin{itemize}
        \item $\Sigma \vDash \alpha$ denotes that $\Sigma$ implies $\alpha$
        \item $\alpha \vDash \beta$ denotes that $\{\alpha\} \vDash \beta$
    \end{itemize}
    If $\Sigma \vDash \alpha$, we call $\alpha$ a semantic consiquence of $\Sigma$.
\end{definition}

Semantic implication is also referred to as totallogical implication.

\subsection{Tautologies}
\label{sub:Tautologies}

\begin{definition}[Tautologies]
    $\alpha$ is a tautology if $\emptyset \vDash \alpha$, denoted by $\vDash \alpha$.
\end{definition}
\begin{remark}
    ~{}
    \begin{itemize}
        \item $\alpha$ is a tautology iff $\forall v$, $\bar{v}(\alpha)=T$.
        \item $\alpha$ is a tautology iff $\neg \alpha$ is \emph{not satisfiable}
        \item $\alpha$ is satisfiable iff $\neg \alpha$ is a tautology.
    \end{itemize}
\end{remark}
