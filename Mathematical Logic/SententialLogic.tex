\chapter{Sentential Logic}

\section{Grammar}
\label{sec:Gramma}

\subsection{Symbols}
\label{sub:Symbols}
\begin{itemize}
    \item Logical Symbols
    \begin{itemize}
        \item Sentential Connectives
        \begin{itemize}
            \item $\neg$
            \item $\wedge$
            \item $\vee$
            \item $\rightarrow$
            \item $\leftrightarrow$
        \end{itemize}
        \item Parentheses
    \end{itemize}
    \item Non-logical Symbols: An enumerable set of elements
\end{itemize}

\subsection{Expressions}
\label{sub:Expressions}

\begin{definition}[Expression]
    \label{def:Expression}
    An expression is a finite sequence of symbols.
\end{definition}
\begin{remark}
    The set of all expressions is enumerable.
\end{remark}

We often use Greek alphabets $\alpha,\beta,\dots$ to represent expressions.

\subsection{Well-Formed Formulas}
\label{sub:WellFormedFormulas}

\begin{definition}[Well-Formed Formula]
    \label{def:WFF}
    A \textbf{well-formed formula} (or formula or wff) is an expression built up from sentence symbols by applying some finite times of \emph{formula building operations}
\end{definition}

\begin{definition}[Formula Building Operations]~{}
    \begin{itemize}
        \item $\xi_{\neg}(\alpha) = (\neg\alpha)$
        \item $\xi_{\wedge}(\alpha,\beta) = (\alpha \wedge \beta)$
        \item $\xi_{\vee}(\alpha,\beta) = (\alpha\vee\beta)$
        \item $\xi_{\rightarrow}(\alpha,\beta) = (\alpha\rightarrow\beta)$
        \item $\xi_{\leftrightarrow}(\alpha,\beta) = (\alpha\leftrightarrow\beta)$
    \end{itemize}
\end{definition}
\begin{remark}
    Do NOT omit the parentheses.
\end{remark}

\begin{definition}[Well-Formed Sequences of Expressions]
    \label{def:WellFormedSeqOfExpr}
    A \textbf{well-formed sequence of expressions} is a finite sequence $\alpha_1,\alpha_2,\dots,\alpha_n$ of expressions such that each $\alpha_i$ is either
    \begin{itemize}
        \item A sentence symbol
        \item $(\neg\alpha_j)$ for some $j < i$
        \item $(\alpha_j \wedge \beta_k)$ for some $j,k<i$
        \item $(\alpha_j \vee \beta_k)$ for some $j,k<i$
        \item $(\alpha_j \leftarrow \beta_k)$ for some $j,k<i$
        \item $(\alpha_j \leftrightarrow \beta_k)$ for some $j,k<i$
    \end{itemize}
\end{definition}

\begin{proposition}
    An expression $\alpha$ is a well-formed formula iff there is a well-formed sequence $(\alpha_1,\dots,\alpha_n)$ s.t. $\alpha = \alpha_n$
\end{proposition}

\subsection{The Induction Principle}
\label{sub:Induction}

Well-formed formulas are a form of inductive definitions with
\begin{itemize}
    \item Basic building blocks
    \item Closing operations
\end{itemize}

\begin{theorem}[The Induction Principle]
    \label{thm:InductionPrinciple}
    Let $S$ be a set of wffs ($S \subseteq W$), if
    \begin{enumerate}
        \item Every sentence symbol is in $S$
        \item For each wff $\alpha$ and $\beta$, if $\alpha$ and $\beta$ are in $S$ then each of the following are in $S$
        \begin{itemize}
            \item $(\neg \alpha)$
            \item $(\alpha \wedge \beta)$
            \item $(\alpha \vee \beta)$
            \item $(\alpha \rightarrow \beta)$
            \item $(\alpha \leftrightarrow \beta)$
        \end{itemize}
    \end{enumerate}
    Then $S$ is the set of \emph{all wffs} ($S = W$).
\end{theorem}

We can see an example of Indunction.

\begin{proposition}
    Every wff has the same number of left parentheses as right parenthesis
\end{proposition}
\begin{proof}
    Let $S\triangleq \{\alpha|\alpha\text{has equal number of left and right parentheses}\}$.
    \begin{itemize}
        \item[Base] $\alpha = A$. Straightforward. Sentense symbols do not have parenthesis
        \item[Step]
        \begin{enumerate}
            \item Let $\beta \in S$, $\alpha = (\neg \beta) \in S$.
            \item Let $\alpha_1, \alpha_2 \in S$, $\alpha = (\alpha_1 \wedge \alpha_2) \in S$.
            \item $\cdots$
        \end{enumerate} 
    \end{itemize}
\end{proof}

\subsection{Parsing Formulas}
\label{sub:ParsingFormulas}

The induction principle actually gives an algorithm for parsing formulas.

On input expression $\alpha$

\begin{enumerate}
    \item If is leaf node, we are done. Return.
    \item The first symbol must be `('.
    \item If the second symbol is `$\neg$', then expect an non-empty expression $\beta$ and parse $\beta$.
    \item If the second symbol is not `$\neg$', then expect a non-empty expression $\beta_1$, an operator and another expression $\beta_2$.
\end{enumerate}

\subsection{Abbreviations}
\label{sub:Abbreviations}

\begin{itemize}
    \item The outermost parentheses can be omitted.
    \item $\neg$ appplies to as little as possible, with the highest precedence.
    \item $\wedge$ and $\vee$ apply to as little as possible, subject to $\neg$.
    \item $\rightarrow$ and $\leftrightarrow$ apply to as little as possible, subject to other operators.
    \item When handling operators with the same precedence, grouping is always to the right. $A \rightarrow B \rightarrow C= (A \rightarrow (B\rightarrow C))$.
\end{itemize}

\section{Semantics}
\label{sec:Semantics}

\subsection{Truth Assignments}

Consider a math domain $\{T,F\}$ of truth values

\begin{itemize}
    \item T is called truth
    \item F is called falsity
\end{itemize}

\begin{definition}[Truth Assignment]
    A truth assignment for a set $\mathcal{S}$ of sentence symbols is a function
    \[ v:\mathcal{S}\mapsto\{T,F\} \]
\end{definition}

\begin{definition}[Extended Truth Assignment]
    Let $\bar{\mathcal{S}}$ be the set of wffs that can be built up from $\mathcal{S}$ by formula-building operations. Let $v$ be a truth assignment for $\mathcal{S}$. An \textbf{extension} $\bar{v}$ of $v$
    \[ \bar{v}:\bar{\mathcal{S}}\mapsto \{T,F\} \]
    assigns truth values to every wff in $\mathcal{S}$ s.t.
    \begin{itemize}
        \item $\bar{v}(\alpha) = v(\alpha)$ if $\alpha \in \mathcal{S}$
        \item $\bar{v}(\neg(\alpha))$ is T if $\bar{v}(\alpha)$ is F and F otherwise.
        \item $\bar{v}((\alpha\wedge\beta))$ is T if $\bar{v}(\alpha)$ is T and $\bar{v}(\beta)$ is T and F otherwise.
        \item $\bar{v}((\alpha\vee\beta))$ is T if $\bar{v}(\alpha)$ is T or $\bar{v}(\beta)$ is T and F otherwise.
        \item $\bar{v}((\alpha\to\beta))$ is F if $\bar{v}(\alpha)$ is T and $\bar{v}(\beta)$ is F and T otherwise.\footnote{Emphasizes the promise of a condition implying a consequence. If the condition is falsy then no guarantee for the consequence. \emph{“骗你是小狗”}}
        \item $\bar{v}((\alpha\leftrightarrow\beta))$ is T if $\bar{v}(\alpha) = \bar{v}(\beta)$ and is F otherwise.
    \end{itemize}
\end{definition}

\begin{theorem}[Determinacy of Truth Assignments]
    \label{thm:DeterminacyofTruthAssignments}
    For every $v_1$ and $v_2$ and wff $\alpha$, if
    \[ v_1(A) = v_2(A) \]
    for every sentence symbol that occurs in $\alpha$, then
    \[ \bar{v}_1(\alpha) = \bar{v}_2(\alpha) \]
\end{theorem}

\begin{remark}
    To determine the value of $\bar{v}(\alpha)$, we only need to know the value of $v$ on the sentence symbols that occur in $\alpha$. This leads to the method of \textbf{truth tables}.
\end{remark}

\subsection{Satisfiability}
\label{sub:Satisfiability}

We first introduce some new notations. We use captial Greek letters, $\Delta$, $\Sigma$, etc. to represent sets of wffs. And we use $\Sigma;\alpha$ to represent $\Sigma \cup \{\alpha\}$.

\begin{definition}~{}
    \begin{itemize}
        \item $v$ satisfies $\alpha$ if $\bar{v}(\alpha) = T$
        \item $v$ satisfies $\Sigma$ if $\bar{v}(\alpha) = T$ for every $\alpha \in \Sigma$.
    \end{itemize}
\end{definition}

\begin{definition}[Satisfiability]~{}
    \label{def:Satisfiability}
    \begin{itemize}
        \item $\alpha$ is satisfiable if there exists some $v$ that satisfies $\alpha$
        \item $\Sigma$ is satisfiable if there exists some $v$ that satisfies $\Sigma$
    \end{itemize}
\end{definition}

\begin{remark}
    Every $v$ satisfies $\emptyset$. Because $v$ satisfies $\emptyset$ iff
    \[ \forall \alpha, \alpha\in\emptyset \Longrightarrow \bar{v}(\alpha) = T \]
    The assumption itself is false, and therefore the consequence is always true.
\end{remark}

\subsection{Semantic Implications}
\label{sub:SemanticImplications}

\begin{definition}
    A set of wffs $\Sigma$ semantically implies $\alpha$ when every truth assignment satisfying $\Sigma$ also satisfies $\alpha$.
    \begin{itemize}
        \item $\Sigma \vDash \alpha$ denotes that $\Sigma$ implies $\alpha$
        \item $\alpha \vDash \beta$ denotes that $\{\alpha\} \vDash \beta$
    \end{itemize}
    If $\Sigma \vDash \alpha$, we call $\alpha$ a semantic consiquence of $\Sigma$.
\end{definition}

Semantic implication is also referred to as tautological implication.

\begin{remark}
    Note that $\{\alpha,\neg\alpha\} \vDash \beta$ also holds, because the assumption does not hold, so the consequence trivially holds.
\end{remark}

\subsection{Tautologies}
\label{sub:Tautologies}

\begin{definition}[Tautologies]
    \label{def:Tautology}
    $\alpha$ is a tautology if $\emptyset \vDash \alpha$, denoted by $\vDash \alpha$.
\end{definition}
\begin{remark}
    ~{}
    \begin{itemize}
        \item $\alpha$ is a tautology iff $\forall v$, $\bar{v}(\alpha)=T$.
        \item $\alpha$ is a tautology iff $\neg \alpha$ is \emph{not satisfiable}
        \item $\alpha$ is satisfiable iff $\neg \alpha$ is not a tautology.
    \end{itemize}
\end{remark}

\subsection{Semantic Equivalence}
\label{sub:SemanticEquivalence}

\begin{definition}[Semantic Equivalence]
    Two wffs $\alpha$ and $\beta$ are semantically equivalent if both $\alpha\vDash\beta$ and $\beta\vDash\alpha$ hold. We use $\alpha\vDash\Dashv\beta$
\end{definition}

\begin{proposition}
    The following are equivalent
    \begin{itemize}
        \item $\alpha$ and $\beta$ are semantically equivalent
        \item For every $v$, $\bar{v}(\alpha)=\bar{v}(\beta)$
        \item $\alpha$ and $\beta$ have the same truth table
    \end{itemize}
\end{proposition}

We can use semantic equivalence to derive truthfulness of wffs. If $\alpha\vDash\models\beta$, we can freely exchange one for the other in deriving the truth of some formula $\sigma$ where $\alpha$ and/or $\beta$ occur.

Remember not to mix syntax and semantics

\begin{itemize}
    \item $\alpha=T$ is incorrect. Use $\bar{v}(\alpha)=T$.
    \item $v(\Sigma)=T$ is incorrect. Use $v$ satisfies $\Sigma$.
\end{itemize}

\subsection{Properties of Satisfaction and Implication}

\begin{itemize}
    \item If $\alpha$ is a tautology, then $\Sigma\vDash\alpha$ for every $\Sigma$
    \item If $\alpha\in\Sigma$ then $\Sigma\vDash\alpha$
    \item If $\Sigma\vDash\alpha$ and $\Sigma\vDash\alpha\to\beta$ then $\Sigma\vDash\beta$
    \item If $\Sigma\vDash\alpha$ and $\alpha\vDash\beta$ then $\Sigma\vDash\beta$.
    \item If $\Sigma\vDash\alpha$ then for all $\beta$, $\Sigma\vDash\beta\to\alpha$
    \item If $\Sigma\vDash\alpha$ and $\Sigma\vDash\beta$ then $\Sigma\vDash \alpha\wedge\beta$
    \item If $\Sigma\vDash\alpha$ or $\Sigma\vDash\beta$ then $\Sigma\vDash \alpha\vee\beta$
    \item If $\Sigma\nvDash\alpha$ iff $\Sigma\cup\{\neg\alpha\}$ is satisfiable
    \item If $\Sigma \vDash \alpha$ iff $\Sigma \cup \{\neg \alpha\}$ is not satisfiable
    \item If $\Sigma \vDash \alpha \to \beta$ iff $\Sigma;\alpha \vDash \beta$
    \item If $\Sigma$ is not satisfiable, then for every $\alpha$, $\Sigma\vDash\alpha$
    \item If $\Sigma\vDash\alpha$ and $\Sigma\subseteq\Delta$ then $\Delta\vDash\alpha$
    \item If $\Sigma$ is satisfiable then every subset of $\Sigma$ is satisfiable
    \item If every subset of $\Sigma$ is satisfiable then $\Sigma$ is satisfiable.
    \item If every finite subset of $\Sigma$ is satisfiable then $\Sigma$ is satisfiable
    \item If $\Sigma\vDash\alpha$ then there is a finite subset $\Delta$ of $\Sigma$ such that $\Delta\vDash\alpha$
\end{itemize}

\section{Normal Forms}
\label{sec:NormalForms}

\subsection{Disjunctive Normal Forms}
\label{sub:DisjunctiveNormalForms}

\begin{definition}[Disjunctive Normal Form]
    \label{def:DisjunctiveNormalForm}
    The wff $\alpha$ is in \textbf{disjunctive normal form} if $\alpha=\gamma_1\vee\gamma_2\vee\cdots\vee\gamma_k$ where each $\gamma_i$ is a conjunction
    \[ \gamma_i = \beta_{i1}\wedge\beta_{i2}\wedge\cdots\wedge\beta_{in_i} \]
    where each $\beta_{ij}$ is either a sentence symbol or the negation of a sentence symbol
\end{definition}

\subsection{Conjunctive Normal Forms}
\label{sub:ConjunctiveNormalForms}

\begin{definition}[Conjunctive Normal Form]
    \label{def:ConjunctiveNormalForm}
    The wff $\alpha$ is in \textbf{conjunctive normal form} if $\alpha=\gamma_1\wedge\gamma_2\wedge\cdots\wedge\gamma_k$ where each $\gamma_i$ is a conjunction
    \[ \gamma_i = \beta_{i1}\vee\beta_{i2}\vee\cdots\vee\beta_{in_i} \]
    where each $\beta_{ij}$ is either a sentence symbol or the negation of a sentence symbol
\end{definition}

\subsection{Completeness of Normal Forms}

\begin{theorem}[Completeness of DNFs]
    \label{thm:CompletenessOfDNF}
    Every wff is semantically equivalent to a wff in disjunctive normal form
\end{theorem}
\begin{proof}
    \begin{enumerate}
        \item Construct the disjunctive normal form truth tables
        \item Select all assignments $v$ s.t. $\bar{v}(\alpha)=T$
        \item Construct $\gamma_i$ where $\beta_{ij} = A_j$ if $A_j$ is assigned $T$ and $\beta_{ij} = \neg A_j$ otherwise
    \end{enumerate}
\end{proof}

\begin{theorem}[Completeness of CNFs]
    \label{thm:CompletenessOfCNF}
    Every wff is semantically equivalent to a wff in conjunctive normal form
\end{theorem}
\begin{proof}
    Given $\alpha=\gamma_1\vee\cdots\vee\gamma_n$ in DNF. If every $\gamma_i$ is a sentence symbol or the negation of a sentence symbol, we are done. Otherwise, there is some $\gamma_i = \beta_{i1}\wedge\beta_{i2}$. Then
    \[ \alpha \equiv (\beta_{i1} \wedge \beta_{i2}) \vee \alpha' \equiv (\beta_{i1}\vee\alpha') \wedge (\beta_{i2}\vee\alpha') \]
    where $\alpha'$ is the disjunction of $\{\gamma_k|k \neq i\}$

    And we recursively repeat the steps
\end{proof}

\section{Finite Satisfiability}
\label{sec:FiniteSatisfiability}

\begin{definition}[Finite Satisfiability]
    \label{def:FiniteSatisfiability}
    $\Sigma$ is \textbf{finitely satisfiable} if every finite subset of $\Sigma$ is satisfiable.
\end{definition}
\begin{remark}
    Suppose $\Delta$ is finitely satisfiable, and for every $\alpha$, $\alpha\in\Delta$ or $\neg\alpha\in\Delta$

    Then $\alpha\in\Delta$ iff $\neg\alpha\notin\Delta$
\end{remark}


\subsection{Compactness Theorem}

\begin{theorem}[Compactness Theorem]
    \label{thm:CompactnessTheorem}
    If $\Sigma$ is finitely satisfiable, then $\Sigma$ is satisfiable
\end{theorem}
\begin{sketchproof}
    We break down the proof into the following steps
    \begin{enumerate}
        \item From $\Sigma$, construct its superset $\Delta$ s.t.
        \begin{enumerate}
            \item $\Delta$ is finitely satisfiable
            \item For every wff $\alpha$, $\alpha\in\Delta$ or $\neg\alpha\in\Delta$
        \end{enumerate}
        \item Show that $\Delta$ is satisfiable, so that $\Sigma$ is also satisfiable
    \end{enumerate}
\end{sketchproof}

\begin{lemma}
    \label{lem:CompactnessLemma1}
    If $\Delta$ is finitely satisfiable, then for every wff $\alpha$,
    \begin{itemize}
        \item either $\Delta\cup\{\alpha\}$ is finitely satisfiable
        \item or $\Delta\cup\{\neg\alpha\}$ is finitely satisfiable
    \end{itemize}
\end{lemma}
\begin{proof}
    Prove by contradiction. Suppose neither $\Delta\cup\{\alpha\}$ nor $\Delta\cup\{\neg\alpha\}$ is finitely satisfiable. Then there are some finite subsets $\Delta_1 \subseteq \Delta\cup\{\alpha\}$ and $\Delta_2\subseteq\Delta\cup\{\neg\alpha\}$ which are not satisfiable. Notice that $\alpha$ must be in $\Delta_1$ and $\Delta_2$, or otherwise $\Delta_1$ and $\Delta_2$ would be satisfiable by finite satisfiability of $\Delta$. Therefore $\Delta_1 = \Delta'_1\cup\{\alpha\}$ and $\Delta_2'\cup\{\neg\alpha\}$.

    Then we construct $\Delta' = \Delta_1\cup\Delta_2 = \Delta_1'\cup\Delta_2'\cup\{\alpha,\neg\alpha\}$. Then there exists an assignment $v$ such that $v$ satisfies $\Delta_1'\cup\Delta_2'$, and that $v$ satisfies either $\alpha$ or $\neg\alpha$. Suppose $\bar{v}(\alpha)=T$, then $\Delta_1$ is satisfiable, which causes contradiction. Conversely, if $\bar{v}(\alpha) = F$ then $\bar{v}(\neg\alpha)=T$, then $\Delta_2$ is satisfiable, which is also a contradiction.
\end{proof}
\begin{remark}
    This lemma implies that we can expand $\Sigma$ for one step, by including $\alpha$ or $\neg\alpha$
\end{remark}


\begin{lemma}
    \label{lem:CompactnessLemma2}
    If $\Sigma$ is finitely satisfiable, then there is a $\Delta \supseteq \Sigma$ such that
    \begin{itemize}
        \item $\Delta$ is finitely satisfiable
        \item For each wff $\alpha$, $\alpha\in\Delta$ or $\neg\alpha\in\Delta$
    \end{itemize}
\end{lemma}
\begin{proof}
    The set of all wffs is enumerable, so we can write all wffs in a sequence
    \[ \alpha_1, \alpha_2,\dots,\alpha_n,\dots \]
    Then we can scan through the sequence and check if $\alpha_i$ can be added to $\Sigma$. Starting from $\Delta_0 = \Sigma$,
    \[\Delta_{i+1}\begin{cases}
        \Delta_i \cup \{\alpha_i\} &\quad\text{if it is finitely satisfiable}\\
        \Delta_i \cup \{\neg \alpha_i\} &\quad \text{otherwise}
    \end{cases}\]

    It can be proved by induction and Lemma~\ref{lem:CompactnessLemma1} that $\forall i,\Delta_i$ is finitely satisfiable.

    And $\Delta$ can be constructed by the union of all $\Delta_i$'s

    \[ \Delta = \bigcup_{i\in\mathbb{N}} \Delta_i \]

    $\Delta$ is finitely satisifiable because for each $\Delta'\subseteq\Delta$, $\Delta'\subseteq\Delta_i$ for some $\Delta_i$. Since $\Delta_i$ is finitely sat, $\Delta'$ is also satisfiable, and therefore $\Delta$ is finitely sat.

    For each wff $\alpha$, either $\alpha\in\Delta$ or $\neg\alpha\in\Delta$. Because $\alpha$ must exist as $\alpha_i$ in the sequence of all wffs, then either $\alpha_i\in\Delta_{i+1}$ or $\neg\alpha_i\in\Delta_{i+1}$.
\end{proof}

\begin{lemma}
    \label{lem:CompactnessLemma3}
    Let $\Delta$ be a set of wffs such that
    \begin{itemize}
        \item $\Delta$ is finitely satisifiable
        \item For every wff $\alpha$, $\alpha\in\Delta$ or $\neg\alpha\in\Delta$
    \end{itemize}
    Then $\Delta$ is satisfiable
\end{lemma}
\begin{proof}
    Consider the sentence symbols. All sentence symbols (or their negations) must be in $\Delta$ because they are also wffs. Therefore if there is an assignment $v$ that satisfies $\Delta$, its values is already determined by the sentence symbols in $\Delta$.

    \[v(A)=\begin{cases}
        T &\quad A\in\Delta\\
        F &\quad \neg A\in\Delta
    \end{cases}\]

    Then proving Lemma~\ref{lem:CompactnessLemma3} is equivalent to proving
    \[ \forall \alpha, \alpha\in\Delta\Leftrightarrow\bar{v}(\alpha) = T \]
    This can be proved by induction
    \begin{itemize}
        \item[base] Consider $\alpha=A$. Obviously it holds.
        \item[induction] \begin{enumerate}[(a)]
            \item $\alpha=\neg\beta$. The hypothesis is $\beta\in\Delta \Leftrightarrow \bar{v}(\beta)=T$. If $\neg\beta\in\Delta$, then $\beta$ cannot be in $\Delta$ due to the finite satisfiability of $\Delta$, and therefore $\bar{v}(\beta) = F$, and therefore $\bar{v}(\neg\beta) = T$. Conversely, If $\bar{v}(\neg\beta) = T$, then $\bar{v}(\beta) =F$, and therefore $\beta\notin\Delta$ and thus $\neg\beta$ must be in $\Delta$.
        \end{enumerate}
    \end{itemize}
\end{proof}

\begin{corollary}
    \label{coroll:CorollaryOfTheCompactnessTheorem}
    If $\Sigma\vDash\tau$, then there is a finite subset $\Delta$ of $\Sigma$ such that $\Delta\vDash\tau$
\end{corollary}
\begin{proof}
    $\Sigma\vDash\tau\Rightarrow\Sigma\cup\{\neg\tau\}$ is unsat. Therefore $\Sigma\cup\{\neg\tau\}$ is not finitely satisfiable by Compactness Theorem \ref{thm:CompactnessTheorem}. Therefore the exists a finite subset $\Delta$ of $\Sigma$ and $\Delta$ is unsat. Therefore $\Delta\vDash\tau$.
\end{proof}

\subsection{Decidability Results for Semantic Implications}

\begin{theorem}
    Given any finite set $\Sigma$ of wffs and any wff $\alpha$, there is an algorithm for deciding whether or not $\Sigma\vDash\alpha$.
\end{theorem}
\begin{enumerate}
    \item Collect all sentence symbols in $\Sigma$ and $\alpha$
    \item Use truth table
\end{enumerate}

\begin{corollary}
    Given a finite set of wffs $\Sigma$, the set of its semantic consequences is effectively decidable. In particular, the set of tautologies is effectively decidable.
\end{corollary}

\subsection{Enumerability Results for Semantic Implications}

\begin{theorem}
    If $\Sigma$ is an effectively enumerable set of wffs, then the set of semantic consequences of $\Sigma$ is effectively enumerable.
\end{theorem}
\begin{proof}
    Let $\beta_1,\dots,\beta_n,\dots$ be an effective enumeration of $\Sigma$. Let $\Delta_n=\beta_1,\dots,\beta_n$. Let $\alpha_1,\dots,\alpha_m,\dots$ be an effective enumeration of all wffs. We construct a table $T$ where $T_{ij} = \Delta_i \vDash \alpha_j$. Due to Corollary~\ref{coroll:CorollaryOfTheCompactnessTheorem}, if some $T_{ij}$ holds, then $\alpha_j$ is a semantic consequence of $\Sigma$.
\end{proof}