\chapter{Foundamentals of Speech Signal Processing}
The basic concepts of signal processing have been covered in EI015 Signals and Systems,  and therefore will not be mentioned in this note.

In other words, I am lazy.
\newpage
\section{Discrete Fourier Transform}
Similar to the continuous case, given a periodic discrete signal $\tilde{x}[n]$\footnote{The tilde sign here means that $\tilde{x}[n]$ is periodic.} with period $N$, it can be represented by \emph{a discrete sum of sinusoids}, rather than an integral (Recall the DTFT Synthesis Formula. Thank you, EI015!).

$\tilde{x}[n]$ can be represented as a sum of complex exponentials with radian frequency $(2\pi k / N)$, where $k=0,1,\dots,N-1$.
\[ \tilde{X}[k] = \sum_{n=0}^{N-1}\tilde{x}[n]e^{-j\frac{2\pi}{N}kn} \]

And the corresponding synthesis expression is
\[ \tilde{x}[n] = \frac{1}{N}\sum_{k=0}^{N-1}\tilde{X}[k]e^{j\frac{2\pi}{N}kn} \]

This representation of a periodic discrete signal is \emph{exact}. However, the DFT is generally used in another case, where $x[n]$ is a \emph{finite} sequence of signal. Since performing DFT only need $\tilde{x}[n]$ in a period $0 \le n \le N-1$, and whatever is out of this range does not matter, we may extend $x[n]$ and assume an ``implicit periodic sequence'' $\tilde{x}[n]$:
\[ \tilde{x}[n] = \sum_{r = -\infty}^{+\infty} x[n + rN] \]

And here comes our DFT.
\begin{definition}[Discrete Fourier Transform]
    \[ X[k] = \sum_{n=0}^{N-1} x[n]e^{-j\frac{2\pi}{N}kn} \quad k = 1,2,\dots,N-1 \]
    \[ x[n] = \sum_{k=0}^{N-1} X[k]e^{j\frac{2\pi}{N}kn} \quad n = 1,2,\dots,N-1 \]
\end{definition}
\begin{remark}
    Bear in mind that when using DFT representations, all signals behave as if they were implicitly \emph{periodic}, as the DFT is originally defined on periodic signals.
\end{remark}

\section{Voiced, Unvoiced and Silence}
The speech waveform can be classified into basically 3 stages.
\paragraph*{Unvoiced} Produced by creating a constriction somewhere in the vocal tract tube and forcing air through that constriction, thereby creating turbulent air flow, which acts as \emph{a broad-spectrum noise excitation of the vocal tract tube}. 
\paragraph*{Voiced} Produced by forcing air through the glottis with the tension of the vocal cords adjusted so that they vibrate in a relaxation oscillation, leading to \emph{quasi-periodic} waveforms.
\paragraph*{Silence} Usually occurs at the beginning or the end of speech, lacks characteristics of either voiced sounds or unvoiced sounds.