\chapter{Classification}
\newpage


\section{k Nearest Neighbours}
    kNN is a classical non-parametrized lazy-learning algorithm.

    \subsection{Distance}
        A distance function $d(x,y)$ is a function that satisfies
        \begin{itemize}
            \item Positive Definite. $d(x,y)>0$ and $d(x,y)=0$ iff $x=y$.
            \item Symmetric.
            \item Triangle Inequality. $d(x,y) \le d(x,z) + d(z,y)$.
        \end{itemize}
        Common distance functions include
        \begin{itemize}
            \item $l_2$ distance (Euclidean). $\|x-y\|_2 = \sqrt{\sum(x_i - y_i)^2}$.
            \item $l_1$ distance (Manhattan). $\|x-y\|_2 = \sum |x_i-y_i|$.
            \item $l_{\infty}$ distance. $\|x-y\|_{\infty} = \max\{|x_i-y_i|\}$.
        \end{itemize}

    \subsection{A general form of Euclidean Distance}
        Let $P$ be a projection matrix, we can generalize $l_2$ distance by
        \[\|Px-Py\|_2^2 = (P(x-y))^T(P(x-y)) = (x-y)^TP^TP(x-y)\]
        \[\Rightarrow Dist(x,y) = (x-y)^TA(x-y)\]
        \begin{remark}
            Actually $A$ can be learned.
        \end{remark}

    \subsection{kNN Algorithm}
        Given a dataset $\mathcal{D}$ and an input $x$, the kNN algorithm
        \begin{enumerate}
            \item Computes some kind of distance between $x$ and each $y$ in $\mathcal{D}$.
            \item Finds the nearest $k$ neighbours of $x$ in $\mathcal{D}$.
            \item Determines the class of $x$ by the label of neighbours.
        \end{enumerate}

    \subsection{Choosing k}
        \begin{itemize}
            \item For a binary classification task, $k$ should be an odd number.
            \item $k$ should not be too large or too small.
            \item Choosing a proper $k$ is a matter of parameter tuning.
        \end{itemize}

    \subsection{More on kNN}
        \paragraph{Complexity.}
        kNN requires
        \begin{itemize}
            \item An $O(n)$ time to calculate distances.
            \item An $O(n\log n)$ time to sort and find nearest neighbours.
        \end{itemize}
        \paragraph{Can it be faster?}
        Yes if we use parallelism.


\section{Naive Bayes Classifer}

    \subsection{Posterior, Prior, Likelihood and Evidence}
        Recall Bayes' Rule
        \[ p(w_j|x) = \frac{p(x|w_j) \cdot p(w_j)}{p(x)} \]
        \begin{itemize}
            \item $p(w_j|x)$ is called the \textbf{posterior}.
            \item $p(w_j)$ is called the \textbf{prior}.
            \item $p(x|w_j)$ is called the \textbf{likelihood}.
            \item $p(x)$ is called the \textbf{evidence}.
        \end{itemize}
        \[ Posterior = (Likelihood \cdot Prior) / Evidence \]
        Since optimizing w.r.t. $w$ has nothing to do with $p(x)$,
        \[ Posterior \sim (Likelihood \cdot Prior) \]

    \subsection{Bayesian Decision Theory}
        Suppose there are $N$ classes $c_1,c_2,\dots,c_N$. Let $\lambda_{ij}$ be the cost of (mistankenly) classifying a sample in $c_i$ into $c_j$.
        \begin{remark}
            In a more general sense, $c_1, c_2, \dots, c_N$ is more than just a set of classes. It can be seen as a set of ``actions''.
        \end{remark}

        \begin{definition}[Conditional Risk]
            The \textbf{expected loss}, or the \textbf{conditional risk} is defined as
            \[ R(c_i|x) = \sum_{j=1}^N \lambda_{ij}p(c_j|x) \]
        \end{definition}
        Our goal is to find a minimizer $h(x)$ of the general conditional risk $R(h) = \mathbb{E}[R(h(x)|x)]$.

        For each $x$, if $h(x)$ minimizes $R(h(x)|x)$, then obviously $h(x)$ also minimizes $R(h)$. The minimizing $h^*(x)$ is called the \textbf{Bayes optimal classifier}, and $R(h^*)$ is called the \textbf{Bayes risk}.

        If
        \[
        \lambda_{ij} = 
        \begin{cases}
            0 \quad & i = j\\
            1 \quad & otherwise
        \end{cases}    
        \]
        Then
        \[ R(c|x) = 1 - p(c|x) \]
        and
        \[ h^*(x) = \arg \max_{c_i} p(c_i|x) = \arg \max_{c_i} \frac{p(x|c_i)p(c_i)}{p(x)} = \arg\max_{c_i} p(x|c_i)p(c_i)\]
        So we choose the $x$ that \emph{maximizes posterior}.

        \begin{remark}
            A \textbf{discriminative model} directly models likelihood $p(c|x)$, while a \textbf{generative model} models joint distribution $p(x,c)$ by modeling $p(c) \cdot p(x|c)$.
        \end{remark}

        \subsubsection{Example.}
        For a binary classification (decision-making), if
        \[ R(c_1|x) < R(c_2|x) \]
        then we take $c_1$.

        This is equivalent to taking $c_1$ if
        \[ (\lambda_{21}-\lambda_{11})p(x|c_1)p(c_1) > (\lambda_{12}-\lambda_{22})p(x|c_2)p(c_2) \]
        or
        \[ \frac{p(x|c_1)}{p(x|c_2)} > \frac{\lambda_{12} - \lambda_{22}}{\lambda_{21} - \lambda_{11}} \cdot \frac{p(c_2)}{p(c_1)} \]

        If we further assume that $p(x|C_1)$ and $p(x|c_2)$ have Gaussian distributions with the same variance, then we test whether
        \[ \exp\left\{ -\frac{1}{2}(x-\mu_1)^2/\sigma^2 + \frac{1}{2}(x-\mu_2)^2/\sigma^2 \right\} \ge \lambda \]
        where $\lambda$ is a threshold.
        Taking logarithm and simplifying
        \[ (\mu_1/\sigma^2 - \mu_2/\sigma^2)x + constant \ge \log\lambda \]
        i.e.
        \[ ax+b > c \]
        or in higher dimensions
        \[ w^Tx > b \]
        It is a linear classifier.

    \subsection{Parameter Estimation}
    \subsubsection{Maximum Likelihood Estimation}
    Let $D$ be a dataset with $n$ i.i.d. samples $x_i$, MLE maximizes
    \[ p(D|\theta) = \prod_{i=1}^n p(x_i|\theta) \]
    to avoid underflow of floating point operations, we take the log likelihood
    \[ l(\theta) = \log(p(D|\theta)) = \sum_{i=1}^n \log(p(x_i|\theta)) \]
    
    \subsubsection{Example of MLE}
    Consider the multivariate Gaussian distribution with dimension $p$
    \[ f(x) = \frac{1}{(2\pi)^{p/2}|\Sigma|^{1/2}}\cdot \exp \left\{ -\frac{1}{2}(x-\mu)^T\Sigma^{-1}(x-\mu) \right\} \]
    The log likelihood is
    \[ \log\left( p(x_i|\theta) = -\frac{1}{2}\log\left[(2\pi)^p|\Sigma|\right] - \frac{1}{2}(x_i - \mu)^T\Sigma^{-1}(x_i - \mu)  \right) \]
    Taking gradient w.r.t. $\mu$
    \[ \nabla_{\mu} = \Sigma^{-1}(x_i-\mu) \]
    And the MLE estimator is
    \[ \hat{\mu} = \frac{1}{n}\sum_{i=1}^n x_i \]
    The covariance matrix can be estimated similarly.

    \subsubsection{Maximum-A-Posterior}
    Beyond the scope.

    \subsection{Naive Bayes}
        \paragraph{Notations.}
            \begin{itemize}
                \item $X$: $N \times d$ training set. 
                \item $x$: a $d$-dim data sample.
                \item $x^{(i)}$: the $i$-th sample in $X$.
                \item $y^{(i)}$: the label of $x^{(i)}$.
                \item $x_j$: the $j$-th feature in $x$.
                \item $c_i$'s: classes.
            \end{itemize}
        The naives Bayes classifier makes the attribute conditinoal independence assumption, aka. the naive Bayes assumption: \emph{all features of x are independent}.
        \[ p(x_1), x_2,\dots,x_d | y) = \prod_{i=1}^d p(x_i|y) \]
        Then we can compute the optimal classifier by
        \[ h_{nb}(x) = \arg\max_{c}p(c)\prod_{j=1}^d p(x_j|c) \]
        Therefore a naive Bayes classifier estimates $p(c)$ and $p(x_j|c)$.

        If we assume $p(x^{(i)}|c) \sim \mathcal{N}(\mu_{c,i}, \sigma^2_{c,i})$, then
        \[ p(c) = \frac{\sum_{i=1}^N\mathbb{I}[y^{(i)}=c]}{N} \]
        i.e. the frequency of $x$ with label $c$ in $X$.
        \[ p(x_j|c) = \frac{\sum_{i=1}^N\mathbb{I}[x^{(i)}_j = 1, y^{(i)}=c]}{\sum_{i=1}^N\mathbb{I}[y^{(i)}=c]} \]
        i.e. the frequency of those $x$ with feature $x_j$ in all $x$ with label $c$.

    \subsection{Laplace Smoothing}
        If an attribute value has never appeared in the training set, the output will either be $0$ or $\frac{0}{0}$.

        To fix this, we introduce \textbf{Laplace smoothing}.
        \[ p(y=i) = \frac{\sum_{j=1}^N\mathbb{I}[y^{(j)=i}] + 1}{N + k} \]
        where $k$ is the number of classes.


\section{Decision Trees}
    A decision tree is a tree-based classifier. It has a root, several leaf nodes and several internal nodes. Each leaf is a decision and in each internal node we perform a test on one or more attributes of input.

    \subsection{Entropy}
    \begin{definition}[Entropy]
        Given a dataset $D$ with $k$ classes, suppose the proportion of the $k$-th class in $D$ is $p_k$. The entropy is defined by
        \[ H(D) = -\sum_{i=1}^k p_k \log_2 p_k \]
    \end{definition}
    \begin{remark}~{}
            The smaller $H(D)$ is, the higher ``purity'' $D$ has.
    \end{remark}

    \begin{definition}[Conditional Entropy]
        Suppose an attribute $a$ has $V$ possible values. If we split $D$ based on the value of $a$, we will have $V$ branches.
        \[ H(Y|X) = \sum_{v=1}^V \mathbb{P}[X=x_v]\sum_{i=1}^k\mathbb{P}[Y=y_i|X=x_v]\log_2\mathbb{P}[Y=y_i|X=x_v] \]
        or
        \[ H(Y|X) = \sum_{v=1}^V\frac{|D_v|}{|D|}H(D_v) \]
    \end{definition}
    \begin{remark}
        Conditional Entropy is a weighted sum.
    \end{remark}

    \begin{definition}[Information Gain]
        Given a dataset $D$ and an attribute $a$, the information gain is defined by
        \[ IG(D,a) = H(D) - \sum_{v=1}^V\frac{|D_v|}{|D|}H(D_v) \]
    \end{definition}