\chapter{Regression}
Almost all of them have closed-form solutions.
\newpage
%%%%%%%%%%%%%%%%%%%%%%%%%%%%%%%%%%%%%%%%%%
%%%%%%%% Least Squares Regression %%%%%%%%
%%%%%%%%%%%%%%%%%%%%%%%%%%%%%%%%%%%%%%%%%%
\section{Least Squares Regression}
%%%%%%%% Notation
\subsection{Notations}
\begin{itemize}
    \item $X$: $N \times d$ matrix of data.
    \item $x^{(i)}$: $i$-th sample, $d$ dimensional feature vector, suppose $x_0 = 1$.
    \item $y$: $N$ dimensional output.
    \item  $w$: $d \times 1$ parameter.
\end{itemize}
%%%%%%%% Model
\subsection{Model}
\[ f(X;w) = w^T X \]
\begin{remark}
    If $X$ is invertible, then we are done.
    \[ w = X^{-1}y \]
    Unfortunately $X$ is usually not invertible, it is usually not even a square matrix. So we need optimizaiton-based approach to solve this.
\end{remark}
%%%%%%%% Loss Function
\subsection{Loss Function}\label{section:LeastSquareLossFunc}
\[ J(w)  = \frac{1}{2}\| Xw - y \|_2^2 = \frac{1}{2}\sum_{i=1}^N(w^Tx^{(i)} - y^{(i)})^2 \]
Our goal is to minimize the loss function
\[ \min_w \quad J(w) \]
%%%%%%%% Gradient Descent Update Rule
\subsection{Descent Method}
$J(w)$ is convex, and can be optimized by gradient descent.
\[ \nabla J(w) = X^T(Xw-y) \]
\[ w_{i+1} = w_i - \alpha \nabla J(w) \]
where $\alpha$ is the step size.
%%%%%%%% Closed Form Solution
\subsection{Closed Form Solution}
$J(w)$ is convex, and we can calculate the closed-form solution.

Let $\nabla J(w) = 0$.
\[ X^TXy = X^Ty \]
\[ w = (X^TX)^{-1}X^Ty \]
\begin{remark}
    If $X^TX$ is not invertible, we can use its pseudo-inverse, which is defined as
\end{remark}
\begin{definition}[Pseudo-Inverse]
    Let $A = U\Sigma V^T$ be the SVD of matrix $A \in \mathbb{R}^{m\times n}$, with $\mathbf{Rank}(A) = r$. The pseudo-inverse of $A$ is defined as
    \[ A^{\dagger} = V\Sigma^{-1}U^T \in \mathbb{R}^{n\times m} \]
\end{definition}
\begin{remark}
    Pseudo-Inverse does not necessarily give a good solution.
\end{remark}
%%%%%%%%
\subsection{Comparision}
\paragraph{Gradient Descent} is easy to implement and converges relatively fast.
\paragraph{Closed-form solution} is straightforward but may be numerically unstable.
%%%%%%%% Geometric Interpretation
\subsection{Geometric Interpretation}
\[ \hat{y} = Xw = X(X^TX)^{-1}X^Ty \]
\begin{align*}
    \hat{y} - y &= X(X^TX)^{-1}X^Ty - y \\
    &= (X(X^TX)^{-1}X^T - I)y
\end{align*}
Multiplying $X^T$ on both sides yields
\begin{align*}
    X^T(\hat{y}-y) &= X^T(X(X^TX)^{-1}X^T - I)y \\
    &= (X^TX(X^TX)^{-1}X^T - X^T)y \\
    &= 0
\end{align*}
The error $\hat{y} - y$ is perpendicular to the space spanned by samples $x^{(i)}$.
%%%%%%%% Probability Interpretation
\subsection{Probability Interpretation}\label{section:LeastSquareProbability}
We make the following assumptions
\begin{itemize}
    \item $y$ is linear with $x$ plus some random error $\varepsilon^{(i)}$.
    \[ y^{(i)} = w^Tx^{(i)} + \varepsilon \]
    \item $\varepsilon^{(i)}$ has a Gaussian distribution.
    \[ \varepsilon^{(i)} \sim \mathcal{N}(0,1) \]
\end{itemize}
Substitute $\varepsilon$ with $w^Tx-y$, and the probaility density function is
\[ p(y^{(i)}|x^{(i)};w) = \frac{1}{\sqrt{2\pi}}\exp\left(-\frac{(w^Tx^{(i)}-y^{(i)})^2}{2}\right) \]
We maximize the (log) likelihood
\[ L(w) = \prod_{i=1}^N \frac{1}{\sqrt{2\pi}}\exp\left(-\frac{\|w^Tx^{(i)}-y^{(i)}\|^2}{2}\right) \]
\[ l(w) = \log L(w) = constant - \sum_{i=1}^N \frac{1}{2}\|w^Tx^{(i)}-y^{(i)}\|^2\]
Maximizing the log likelihood is equivalent to minimizing the last term. So the problem becomes
\[ \min_{w} \quad \frac{1}{2}\sum_{i=1}^N\|w^Tx^{(i)} - y^{(i)}\|^2 \]
This is exactly the loss function in section \ref{section:LeastSquareLossFunc}.
%%%%%%%% Regularization
\subsection{Regularization}
\[ \min_w \| Xw - y \|^2 + \epsilon\|w\|^2 \]
where $\epsilon$ is a small positive value, and the closed form solution can be derived from
\[ (X^TX + \epsilon I)w = X^Ty \]
Adding $\epsilon$ sometimes makes the inverse of matrix more stable.
\begin{remark}
    Adds a contraint that the norm of $w$ cannot be too large. But regularization still does not guarantee a good solution when the matrix is not invertible.
\end{remark}

%%%%%%%%%%%%%%%%%%%%%%%%%%%%%%%%%%
%%%%%%%% Ridge Regression %%%%%%%%
%%%%%%%%%%%%%%%%%%%%%%%%%%%%%%%%%%
\section{Ridge Regression}
%%%%%%%% Model
\subsection{Model}
\begin{align*}
    \min_w \quad &\| Xw - y \|_2^2 \\
    \text{s.t.} \quad &\|w\|_2^2 \le t
\end{align*}
It can be rewritten in the form of Penalized Residual Sum of Squares, PRSS
\[ \textrm{PRSS}(w) = \|Xw-y\|_2^2 + \lambda\|w\|_2^2 \]

\subsection{Solving Ridge Regression}
Ridge regression is also a convex optimization problem. It can be solved by using numerical methods or deriving closed form solutions.
\[ \nabla\textrm{PRSS}(w) = -2X^T(y-Xw) + 2\lambda w \]
So
\[ w = (X^TX + \lambda I)^{-1}X^Ty \]
\begin{remark}
    Even if $X^TX$ is singular, by choosing a proper $\lambda$ we can make $(X^TX + \lambda I)$ nonsingular.
\end{remark}
%%%%%%%% Regularization Revisited
\subsection{Regularization Revisited}
$\lambda$ is a regularization hyperparameter.
\begin{itemize}
    \item When $\lambda = 0$, the penalty term has no effect and Ridge regression is equivalent to ordinary least squares.
    \item When $\lambda \to \infty$, $w\to 0 $.
    \item A proper $\lambda$ can be chosen via cross-validation.
\end{itemize}
%%%%%%%% Probability Interpretation
\subsection{Probability Interpretation}
In addtion to the assumptions made in Section \ref{section:LeastSquareProbability}, if we further assume
\[ w \sim \mathcal{N}(0,\sqrt{\lambda}) \]
Then by Bayes's Rule
\[ p(w|y) = \frac{p(y|w)\cdot p(w)}{p(y)} \]
By Maximum-A-Posteriori estimation
\[ \hat{w} = \arg \max_w p(w|y) = \arg \max_w \log p(y|w)\cdot p(w) \]
This yields the loss function of Ridge Regression.
%%%%%%%% Geometric Interpretation
\subsection{Geometric Interpretation}
The feasible region of Ridge Regression is a $l_2$-norm ball (circle in 2D and sphere in 3D). We are looking for the smallest value of objective function in the norm ball.

%%%%%%%%%%%%%%%%%%%%%%%%%%%%%%%%%%
%%%%%%%% LASSO Regression %%%%%%%%
%%%%%%%%%%%%%%%%%%%%%%%%%%%%%%%%%%
\section{LASSO}
\emph{Least Absolute Shrinkage and Selection Operator}.
%%%%%%%% Model
\subsection{Model}
\begin{align*}
    \min_w \quad &\| Xw - y \|_2^2 \\
    \text{s.t.} \quad &\|w\|_1 \le t
\end{align*}
The optimization problem can be reformulated as
\[ \min_w \quad \| Xw - y \|_2^2 + \lambda \|w\|_1 \]
%%%%%%%% Solving LASSO
\subsection{Solving LASSO}
LASSO can be solved, but solving it is nontrivial. There are a number of algorithms to solve LASSO.
%%%%%%%% Probabilistic Interpretation
\subsection{Probabilistic Interpretation}
In addition to the assumptions made in \ref{section:LeastSquareProbability}, we further assume that $w$ has a Gamma distribution.

The minimizing $w$ is the MAP estimator.
%%%%%%%% Geometric Interpretation
\subsection{Geometric Interpretation}
The feasible region of LASSO is a $l_1$-norm ball (diamond in 2D). We look for the smallest value in the diamond-shaped region.