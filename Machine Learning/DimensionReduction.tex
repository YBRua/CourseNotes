\chapter{Dimension Reduction}
\emph{“你们都不是49年的国军,你们是50年的国军。”}
\newpage


\section{Principal Component Analysis}
    \emph{aka. Discrete Karhunen Loeve Transform; Hotelling Transform}

    Let $y \in \mathbb{R}^K$ be a feature vector computed from $x \in \mathbb{R}^D$ where $K \ll D$. PCA determines an orthogonal matrix $W \in \mathbb{R}^{D \times K}$ s.t.
    \[ y = W^Tx \]

    Assume in this section that $x$ has a distribution with mean $\mu=0$ and covariance matrix $\Sigma$.

    \subsection{Objective Funtion}
        PCA minimizes the reconstruction error between original $x$ and reconstructed $\hat{x}$.

        Notice that $W$ is orthogonal, so $x$ can be reconstructed by
        \[ \hat{x} = Wy \]

        Let the reconstruction error be
        \[ \epsilon = x - \hat{x} = x - WW^Tx \]

        We want to minimize the reconstruction error
        \[ \|\epsilon\|^2 = \epsilon^T\epsilon = x^Tx - x^TWW^Tx \]

        The minimization is performed over all possible $x$
        \[ \int p(x)\epsilon(x)^T\epsilon(x)\mathrm{d}x \]

        \[ \mathbb{E}[\epsilon^T\epsilon] = \mathbb{E}[x^Tx] - \mathbb{E}[x^TWW^Tx]\]

    \subsection{Simple Case}
        If $K=1$, $w$ is a vector and $y$ is a scalar.
        \[\mathbb{E}[\epsilon^T\epsilon] = \mathbb{E}[x^Tx]-w^T\mathbb{E}[xx^T]w = \mathbb{E}[x^Tx] - w^T \Sigma w\]
        where $\Sigma = XX^T$ (recall that $X$ is assumed to have zero mean).

        We maximize the normalized second term
        \[ \max_w \quad J \triangleq \frac{w^TXX^Tw}{w^Tw} \]

        Or it can also be formulated as a equality constrained maximization and solved by Lagrangian.

        The solution is
        \[ XX^Tw=Jw \]

        So the problem becomes finding the eigenvalues of $XX^T$, to maximize $J$, we find the largest eigenvalue of $XX^T$.

        Other dimensions turn out to be the remaining eigenvectors.

    \subsection{Geometric Interpretation}
        \begin{itemize}
            \item PCA is a shift and rotation of axis.
            \item $w_1$ is the direction of greatest elongation
            \item $w_2$ is the second largest elongation that is orthogonal to $w_1$
        \end{itemize}


\section{Multi-Dimensional Scaling}
    MDS aims to keep the distances of samples in the mapped space that same as the distances in the original space.

    In this section we also assume the data samples have $\mu=0$.

    \[ \min_{\hat{x}} \left(\sum_{i<j}\left( \|\hat{x}_i-\hat{x}_j\| - d_{ij} \right)^2\right)^{1/2} \]
    where $\hat{x}$ are mapped (instead of reconstructed) samples.

    Let $X$ be the unknown \emph{dimension-reduced} samples. Let $T=XX^T$ and $t_{ij}=\hat{x}_i^T\hat{x}_j$.

    The distance between $\hat{x}_i$ and $\hat{x}_j$ is given by
    \[d_{ij}^2 = (\hat{x}_i-\hat{x}_j)^T(\hat{x}_i-\hat{x}_j)\]

    Solving for $t_{ij}$ yields
    \[ t_{ij} = -\frac{1}{2}(d_{ij}^2\hat{x}_i^2-\hat{x}_j^2) \]

    We also have
    \[\sum_j d_{ij}^2 = n\hat{x}_i^2 + \sum_j\hat{x}_j^2 - 2\hat{x}_i\sum_j\hat{x}_j = n\hat{x}_i^2 + \sum_j\hat{x}_j^2\]
    \[\sum_i d_{ij}^2 = n\hat{x}_j^2 + \sum_i\hat{x}_i^2 - 2\hat{x}_j\sum_i\hat{x}_i = n\hat{x}_j^2 + \sum_i\hat{x}_i^2\]
    \[\sum_{i}\sum_{j}d_{ij}^2=n\sum_i\hat{x}_i^2 + n\sum_j\hat{x}_j^2 \]
    Again recall that the data are assumed to have zero mean so the sum over all data samples yields zero.

    So we can solve for $t_{ij}$
    \[ t_{ij} = -\frac{1}{2}\left(d_{ij}^2-\frac{1}{n}\sum_kd_{ik}^2-\frac{1}{n}\sum_kd_{kj}^2-\frac{1}{n}\sum_{k}\sum_{l}d_{kl}^2\right) \]
    where $d_{ij}$ can be computed from the original dataset.

    Since $T=XX^T$, we can solve for $X$ by decomposing $T$.
    \[T = U\Lambda U^T = U\Lambda^{1/2}\Lambda^{1/2}U^T\]

    Therefore $X=U\Lambda^{1/2}$


\section{Non-Linear Dimension Reduction}
    \subsection{ISOmetric feature MAPping}
        Uses the geodesic distance (shortest distance on a plane) instead of Euclidean distance. The remaining work is similar to MDS and is pigeoned in this note.

    \subsection{Locally Linear Embedding}
        Assumes that a data sample can be represented by some linear combination of its local neighbors\footnote{This assumption is reasonable because the distribution can be somehow ``linearized'' in a sufficiently small area, just like derivatives.}, and that this relationship does not change after dimension reduction.

        LLE can basically be divided into two steps. We first solve for the locally linear combination relationship\footnote{Actually we will first have to find the local nearest neighbours for each data sample $x_i$.} in the original space (solve for a coefficient matrix $W$), and then reconstruct the relationship in a mapped space (solve for mapped data $Y$).

        \subsubsection{Solving Linear Combination}
        In the original space, let $x_j$ be some neighbors of $x_i$. Let $W \in \mathbb{R}^{N \times N}$.
        \[ \epsilon(W) = \sum_i \left|x_i - \sum_jW_{ij}x_j\right|^2 \]
        \begin{align*}
            \min_W &\quad \epsilon(W)\\
            \text{s.t.} &\quad W_{ij} = 0 \quad x_j \notin \mathcal{N}(x_i)\\
            &\quad \sum_j W_{ij}=1
        \end{align*}

        Can be solved by Lagrangian multipliers.

        \subsubsection{Solving Mapping}
        In the mapped space
        \[ \phi(y) = \sum_i\left| y_i - \sum_j W_{ij}y_j \right|^2 \]
        \begin{align*}
            \min_W &\quad \phi(y)\\
            \text{s.t.} &\quad \sum_i y_i = 0\\
            &\quad \frac{1}{N} YY^T = I
        \end{align*}
        where $Y$'s columns are $y_i$.

        To make the problem easier, we convert it into a quadratic form. Define an error matrix
        \[ \delta = Y-YW \triangleq [\Delta y_1,\Delta y_2,\cdots,\Delta y_N]\]

        The total error is given by
        \[ \sum_{i=1}^N \Delta y_i^T \Delta y_i \]

        Notice that
        \[ \delta^T\delta = \begin{bmatrix}
            \Delta y_1^T \Delta y_1 & \Delta y_1^T \Delta y_2 & \cdots & \Delta y_1^T \Delta y_N\\
            \Delta y_2^T \Delta y_1 & \Delta y_2^T \Delta y_2 & \cdots & \Delta y_2^T \Delta y_N\\
            \vdots & \vdots & \ddots & \vdots\\
            \Delta y_N^T \Delta y_1 & \Delta y_N^T \Delta y_2 & \cdots & \Delta y_N^T \Delta y_N
        \end{bmatrix} \]

        So the error can be expressed as
        \[ Tr[\delta^T\delta] = Tr[(Y(I-W))^T(Y(I-W))] = Tr[(I-W)^TY^TY(I-W)] = Tr[Y^T(I-W)^T(I-W)Y] \]

        Let $M = (I-W)^T(I-W)$, the objective function becomes
        \[ \min Tr(Y^TMY) \]

        So we can solve for $Y$ by Lagrangian multipliers.
