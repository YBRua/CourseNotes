\chapter{Introduction}
\emph{“你们CS比CS的强吗?那你们的EE比EE的强吗?”}
\newpage

\section{Basics}
The basic assumption of machine learning is that \textbf{data samples are i.i.d.}.

The goal of training a model is to \textbf{minimize the generalization error of the model}. Since we only have limited amount of data, what we can actually do is to minimize the emprical error.

However, we do not always want the emprical error to be as small as possible due to the risk of overfitting.

\section{Overfitting and Underfitting}
\paragraph{Overfitting.} High variance. The model performs well on training sets but performs poorly on new unseen samples. Using a high-order model to fit low-order distribution of data usually leads to overfitting.
\paragraph{Underfitting.} High bias. The model has not fully captured the underlying structure of the data. Conduct more training or change a more complicated model.

\section{Methods for Splitting data}
To train a model, we first need to divide data into training set and test set. Training set and test set should be disjoint.

\subsection{Hold-Out}
Divide dataset $\mathcal{D}$ into traning set $\mathcal{S}$ and test set $\mathcal{T}$ s.t.
\[ \mathcal{S} \cup \mathcal{T} = \mathcal{D} \quad \mathcal{S} \cap \mathcal{T} = \emptyset \]
Typical proportion of $\mathcal{S}$ and $\mathcal{T}$ is 30\% and 70\%.

\subsection{Cross-Validation}
Divide $\mathcal{D}$ into $k$ disjoint sets of similar size.
\[ \mathcal{D} = \mathcal{D}_1 \cup \mathcal{D}_2 \cup \dots \mathcal{D}_k \quad \text{s.t.} \quad \mathcal{D}_i \cap \mathcal{D}_j = \emptyset \]
Each time use $k-1$ sets for training and the remaining set for testing. 
A typical value of $k$ is $10$.

\subsection{Leave-One-Out}
A special case of cross-validation, wehre each set $\mathcal{D}_i$ contains only one sample.

\subsection{Bootstrapping}
Suppose $\mathcal{D}$ has $m$ samples. Randomly pick a sample from $\mathcal{D}$, copy it into some $\mathcal{D}'$ and put it back to $\mathcal{D}$. Repeat the process for $m$ times.
\[ \lim_{m\to\infty}(1-\frac{1}{m})^m = \frac{1}{e} \approx 0.368 \]
About $36.8\%$ samples in $\mathcal{D}$ will not be in $\mathcal{D}'$. So we can use $\mathcal{D}'$ for training and $\mathcal{D}\backslash\mathcal{D}'$ for testing.

\section{Performance Evaluation}
\subsection{Measure}
\paragraph{Regression} Common performance measure for a regression model is \textbf{Mean Squared Error}.
\[ E = \frac{1}{m}\sum_{i=1}^m(f(x^{(i)}) - y^{(i)})^2 \]
\paragraph{Classification} Common measure for a classification model is \textbf{Error Rate}
\[ E = \frac{1}{m}\sum_{i=1}^m\mathbb{I}[f(x^{(i)}) \neq y^{(i)}] \]

\subsection{TPR and FPR}
\begin{definition}[Sensitivity/TPR]
    \[ TPR = \frac{TP}{TP + FN} \]
\end{definition}
\begin{definition}[FPR]
    \[ FPR = \frac{FP}{TN + FP} \]
\end{definition}

\subsection{Receiver Operating Characteristic}
Many classification models output a real value and compare it to a certain threshold.

The \textbf{ROC Curve} uses $FPR$ as its $x$-axis, and $TPR$ as its $y$-axis. It can be plotted by setting different thresholds for dividing positive and negative samples.

The \textbf{Area Under Curve, AUC} is used to evaluate different models. Usually models with a larger AUC is considered to have better performance.

\subsection{Precision and Recall}
\begin{definition}[Precision]
    \[ P = \frac{TP}{TP + FP} \]
\end{definition}
\begin{definition}[Recall]
    \[ R = \frac{TP}{TP + FN} \]
\end{definition}
Similar to the ROC Curve, we can also plot the \textbf{P-R Curve}. And the \textbf{Break-Even Point, BEP}, defined as the value when $P = R$, is used to evaluate different models.

Another more common measure is the $F1$ rate
\begin{definition}[$F1$ Rate]
    \[ F1 = \frac{2 \times P \times R}{P + R} = \frac{2 \times TP}{\#Samples + TP - TN} \]
\end{definition}
\begin{remark}
    The $F1$ rate is defined by the harmonic mean of Precision and Recall.
\end{remark}

\begin{definition}[$F_{\beta}$ Rate]
    \[ F_{\beta} = \frac{(1+\beta^2)\times P \times R}{(\beta^2 \times P)+R} \]
\end{definition}
\begin{remark}
    $F_{\beta}$ is the weighted harmonic mean. When $\beta > 1$, precision has a higher weight. When $0 < \beta < 1$, recall has a higher weight.
\end{remark}

\section{Error Analysis}
\paragraph{Bias.} The \textbf{bias} is the difference between model prediction and ground truth. This is usually because the model is not well-trained, or because the model is not complex enough to fit the data distribution.
\paragraph{Variance.} The \textbf{variance} is the variance of outputs of the same model fitted different times. This is usually because the model is too complex and mistakenly fits the noise or specific features in the dataset.
\paragraph{Noise.} Noise.

High variance $\to$ Overfitting.

High bias $\to$ Underfitting.

\subsection{Bias-Variance Decomposition}
Let
\[ bias(x) = f(x) - y \]
\[ var(x) = \mathbb{E}_{\mathcal{D}}[(f(x;\mathcal{D}-f(x)))^2] \]
The generalization error of a model $f$ trained on $\mathcal{D}$ can be represented by
\[ E(f;\mathcal{D}) = bias^2(x) + var(x) + \varepsilon^2 \]