\chapter{Linear Discriminant Analysis}
\newpage

\section{Binary LDA}
    Linear Discriminant Analysis aims at projecting data samples onto a hyperplane, such that samples of the same class get as close to each other as possible, while samples of different classes get as far from each other as possible.

    \subsection{Within- and Between- class Distances}
        Let $\mu_1$ and $\mu_2$ be the mean value of two classes.
        \[ \mu_1 = \frac{1}{N_1}\sum_{i\in C_1}x^{(i)} \qquad \mu_2 = \frac{1}{N_2}\sum_{i\in C_2}x^{(i)} \]

        \subsubsection{Between-class Distance}
        Let $w$ be the linear mapping,
        \[ d_{between} = |w^Tu_1 - w^Tu_2| \]
        We want to maximize $d_{between}$, which is equivalent to maximizing
        \[ \|d_{between}\|^2 = w^T(\mu_1-\mu_2)(\mu_1-\mu_2)^Tw\]

        \subsubsection{Within-class Distance}
        We also want to minimize
        \[ d_{within}^{(i)} = \sum_{i\in C}|w^Tx^{(i)} -w^Tx^{(i)}| \]
        for all $i$, which is equivalent to minimizing
        \[ \|d_{within}^{(i)}\|^2 = \sum_{i\in C}w^T(x^{(i)}-\mu_i)(x^{(i)}-\mu_i)^Tw = w^T\Sigma_iw \]
        where $\Sigma_i$ is the covariance matrix of the $i$-th class.

        Therefore
        \[ d_{within} = \sum_i d_{within}^{(i)} \]

    \subsection{Opimization Formulation}
        Notice that we have two objective functions, the key is to combine the two objective into one single objective funtion.

        This can be done by simply setting the objective function to be
        \[ \max J = \frac{d_{between}}{d_{within}} = \frac{w^T(\mu_1-\mu_2)(\mu_1-\mu_2)^Tw}{w^T(\Sigma_1 + \Sigma_2)w} \]

        We define the \textbf{within-class scatter matrix} and \textbf{between-class scatter matrix} by
        \[ S_w = \Sigma_1 + \Sigma_2 \]
        \[ S_b = (\mu_1 - \mu_2)(\mu_1-\mu_2)^T \]

        Therefore
        \[ J = \frac{w^TS_bw}{w^TS_ww} \]
        this is also known as the \textbf{generalized Rayleight quotient}.

    \subsection{Solution}
        \[ J = \frac{w^TS_bw}{w^TS_ww} \]
        Notice that both the numerator and denominator are quadratic, so the norm of $w$ does not matter.

        \subsubsection{By Lagrangian}
        Since the norm of $w$ does not matter, the problem is equivalent to
        \begin{align*}
            \min &\quad -w^TS_bw\\
            \text{s.t.} &\quad w^TS_ww=1
        \end{align*}
        By Lagrangian multipliers
        \[ S_bw = \lambda S_ww \]

        Notice that $(\mu_1-\mu_2)^Tw$ is a scalar, and therefore $S_bw$ is parallel with $\mu_1-\mu_2$. WLOG let
        \[ S_bw = \lambda(\mu_1-\mu_2) \]

        And therefore
        \[ w = S_w^{-1}(\mu_0-\mu_1) \]

        \begin{remark}
            For numerical stability we usually compute the inverse by SVD.
        \end{remark}

        \subsubsection{By Gradient}
        We can directly take gradient w.r.t. $w$.
        \[ \nabla J = \frac{2(w^TS_ww)S_bw-2(w^TS_bw)S_ww}{(w^TS_ww)^2} \]
        which yields
        \[ S_bw=JS_ww \Longrightarrow S_w^{-1}S_bw = Jw \]
        and it becomes a eigenvalue decomposition problem.

        To solve this, we perform eigenvalue decomposition on $S_w^{-1}S_b$, and choose the eigenvector with the largest eigenvalue.

        Or alternatively, we can take more than one eigenvectors.


\section{Multi-class LDA}
    We introduce the \textbf{total scatter matrix} $S_T$
    \[ S_T = \sum_{i=1}^N(x^{(i)}-\mu)(x^{(i)}-\mu)^T \]
    where $\mu$ is the mean vector of all samples.

    \begin{align*}
        S_T &= \sum_{j=1}^{K}\sum_{i\in C_j}(x^{(i)}-\mu_j+\mu_j-\mu)(x^{(i)}-\mu_j+\mu_j=\mu)^T\\
        &= \sum_j\sum_{i\in C_j}(x^{(i)}-\mu_j)(x^{(i)}-\mu_j)^T\\
        &\quad + \sum_j\sum_i(\mu_j-\mu)(\mu_j-\mu)^T + 2 \sum_j\sum_i(x^{(i)}-\mu_j)(x^{(i)}-\mu_j)^T\\
        &= S_w + S_b
    \end{align*}
    Notice that the last term is zero.

    The solution is the same as the binary case.

    However, notice that
    \[ S_b = \sum_{j=1}^k N_j(\mu_j-\mu)(\mu_j-\mu)^T \]
    has at most $k-1$ positive eigenvalues, and therefore we can at most reduce the dimension to $k-1$, but not $k$.
