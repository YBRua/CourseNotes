\documentclass[oneside]{book}
\usepackage{xeCJK}
\usepackage{amsmath}
\usepackage{mathtools}
\usepackage{listings} % lstlist插入代码
\usepackage{booktabs}
\usepackage{ulem}
\usepackage{enumerate}
\usepackage{amsfonts}
\usepackage{amssymb}
\usepackage{amsthm}
\usepackage{setspace} % spacing环境设置行间距
\usepackage[ruled, vlined]{algorithm2e} % 算法与伪代码 
\usepackage{bm} % 数学公式中的加粗
\usepackage{pifont} % 打圈的数字。172-211。\ding
\usepackage{graphicx}
\usepackage{float}
\usepackage[dvipsnames]{xcolor}
%\usepackage{indentfirst}
\usepackage{ulem} %\sout{}打删除线
\normalem % 使用默认normalem
\usepackage{lmodern}
\usepackage{subcaption}
\usepackage[colorlinks, linkcolor=blue]{hyperref}
\usepackage{cleveref}
\usepackage[a4paper]{geometry}
\usepackage{titlesec}

\theoremstyle{definition}
\newtheorem{definition}{Definition}[section]
\newtheorem{theorem}{Theorem}[section]
\newtheorem*{optTheorem}{Theorem}
\newtheorem{proposition}{Proposition}[section]
\newtheorem{lemma}{Lemma}[section]
\newtheorem{corollary}{Corollary}[section]
\theoremstyle{remark}
\newtheorem*{remark}{Remark}
\newtheorem*{sketchproof}{Sketch of Proof}


\title{Notes to AI2613 Stochastic Processes}
\author{\textsc{YBiuR}}
\date{A long long time ago in a far far away SJTU}


\begin{document}
\begin{spacing}{1.2}
\setlength{\parskip}{1em}
\setlength{\parindent}{0em}

\frontmatter
\maketitle
\chapter*{Preface}
\paragraph{}Learning Convex Optimization is non-convex.
\paragraph{}Yet learning Stochastic Porcesses is indeed stochastic.
\mainmatter
\tableofcontents
\chapter{Discrete Markov Chains}
\emph{“和女朋友在商场走散了,是在原地等碰到的概率大还是随机走碰到的概率大?急,在线等。”}
\newpage


\section{Finite Space Markov Chain}

    \subsection{Basic definitions}
    % State Space
    \begin{definition}[State Space]
        A \textbf{state space} $\mathcal{S}$ is a finite or countable set of states, i.e. the values that random variables $X_i$ may take on.
    \end{definition}
    % Initial Distribution
    \begin{definition}[Initial Distribution]
        The \textbf{initial distribution} $\pi_0$ is the probability distribution of the Markov Chain at time $0$. Denote $\mathbb{P}[X_0 = i]$ by $\pi_0(i)$.
    \end{definition}
    \begin{remark}
        Formally, $\pi_0$ is a function from $\mathcal{S}$ to $[0,1]$ s.t.
        \[ \pi_0(i) \ge 0 \text{ for all $i\in\mathcal{S}$} \]
        \[ \sum_{i\in\mathcal{S}} \pi_0(i) = 1\]
    \end{remark}
    % Probability Transition Matrix
    \begin{definition}[Probability Transition Matrix]
        The \textbf{Transition Matrix} is a matrix $P = (p_{ij})$, where
        \[ p_{ij} = \mathbb{P}[X_{n+1} = j | X_n = i] \]
        i.e. the probability given that the chain is at state $i$ at $T=n$ that jumps to $j$ at $T=n+1$
    \end{definition}
    \begin{remark}
        ~{}
        \begin{itemize}
            \item The \emph{rows} of $P$ sum up to $1$.
            \item The entries in $P$ are all non-negative.
        \end{itemize}
    \end{remark}

    %%%%%%%% Markov Property
    \subsection{The Markov Property}
    % Markov Property
    \begin{definition}[Markov Property]
        We say a stochastic process $X_1, X_2, \dots $ satisfies the \textbf{Markov property} if 
        \[ \mathbb{P}[X_{n+1} = i_{n+1} | X_n = i_n, \dots, X_0 = i_0] = \mathbb{P}[X_{n+1} = i_{n+1} | X_n = i_n] \]
    \end{definition}
    That is, the \emph{next} state $X_{n+1}$ depends only on the \emph{current} state $X_n$.
    % Time-homogeneity
    \begin{definition}[Time-homogeneous Markov Chain]
        A Markov Chain is said to be \textbf{time homogeneous} if
        \[ \forall t \quad \mathbb{P}[X_{t+1} = j | X_t = i] = P(i,j) \]
    \end{definition}
    For now, we only consider \emph{time homogeneous} Markov Chains.

    %%%%%%%% Multistep Transition by Matrix Algebra
    \subsection{Matrix Interpretation of Markov Chains}
    We now computes the probability distribution at $T=n+1$, denoted by $\pi_{n+1}$.
    \[ \pi_{n+1}(j) = \mathbb{P}[X_{n+1} = j] = \sum_{i=1}^N \mathbb{P}[X_n=i]\mathbb{P}[X_{n+1} = j | X_n = i] = \sum_{i=i}^N \pi_n(i)P(i,j)\]
    Therefore
    \[ \pi_{n+1}^T = \pi_n^T P \]
    \[ \pi_n^T = \pi_0^T P^n \]
    We use $P(i,j)$ to denote the element $(i,j)$ of $P$, we use $P^n$ to denote the $n$-th power of $P$, and we assume that $\pi_i$'s are column vectors.
    % Chapman-Kolmogorov
    \begin{theorem}[Chapman-Kolmogorov Equality]\label{thm:ChapmanKolmogorovEquality}
        \[ P^{m+n}(i,j) = \sum_k P^m(i,k)P^n(k,j) \]
    \end{theorem}


\section{Stationary Distribution}

    % Stationary Distribution
    \begin{definition}[Stationary Distribution]
        $\pi$ is called a \textbf{stationary distribution} of a Markov Chain if
        \[ \pi^TP = \pi \]
    \end{definition}
    \begin{remark}
        A Markov Chain may have 0, 1 or infinitely many stationary distribution.
    \end{remark}

    \emph{Here comes the question: When does a stationary distribution exist? If it exists, is it unique? If it is unique, does the chain converges to it?}


\section{Irreducibility, Aperiodicity and Recurrence}
For convenience, we will use $\mathbb{P}_i[A]$ to denote $\mathbb{P}[A | X_0 = i]$, and use $\mathbb{E}_i$ to denote expectation in an analogous way.

    %%%%%%%% Irreducibility
    \subsection{Irreducibility}
    % Accessible
    \begin{definition}[Accessibility]
        Let $i$, $j$ be two states, we say $j$ is \textbf{accessible from} $i$ if it is possible (with positive probability) for the chain to ever visit $j$ if the chain starts from $i$.
        \[ \mathbb{P}_i[\bigcup_{n=0}^\infty\{X_n = j\}] > 0 \]
        or equivalently
        \[ \sum_{n=0}^\infty P^n(i,j) = \sum_{n=0}^\infty \mathbb{P}_i[X_n = j] > 0 \]
    \end{definition}
    % Communicate
    \begin{definition}[Communication]
        We say $i$ \textbf{communicates with} $j$ if $j$ is accessible from $i$ and $i$ is accessible from $j$.
    \end{definition}
    % Irreducible
    \begin{definition}[Irreducibility]
        We say a Markov Chain is \textbf{irreducible} if all pairs of states communicate. And it is reducible otherwise.
    \end{definition}
    The relation \emph{communicate with} is an equivalent relation, and irreducible simply means the number of equivalent classes is 1.

    %%%%%%%% Aperiodicity
    \subsection{Aperiodicity}
    % Period
    \begin{definition}[Period]
        Given a Markov Chain, its \textbf{period} of state $i$ is defined to be the greatest common divisor $d_i$ of the lengths of loops starting from $i$.
        \[ d_i = \gcd\{n|P^n(i,i) > 0\} \]
    \end{definition}
    % Period is a class property
    \begin{theorem}
        If states $i$ and $j$ communicate, then $d_i = d_j$
    \end{theorem}
    \begin{sketchproof}
        ~{}
        \begin{itemize}
            \item $P^{n_1}(i,j) > 0$ and $P^{n_2}(j,i) > 0$.
            \item $P^{n_1+n_2}(i,i)>0 \Rightarrow d_i | n_1 + n_2$.
            \item Suppose $P^n{j,j} > 0$, then $P^{n+n_1+n_2}(i,i) > 0 \Rightarrow d_i | n + n_1 + n_2$.
            \item $d_i | n \Rightarrow d_j \ge d_i$.
            \item Similarly $d_i \ge d_j$. We are done.
        \end{itemize}
    \end{sketchproof}
    \begin{remark}
        Therefore all states in a communicating class have the same period, and all states in an irreducible Markov chain have the same period.
    \end{remark}
    % Aperiodic
    \begin{definition}[Aperiodicity]
        An irreducible Markov chain is said to be \textbf{aperiodic} if its period is 1, and periodic otherwise.
    \end{definition}
    \begin{proposition}
        \normalfont
        If $P(i,i) > 0$, then the Markov Chain is aperiodic.
    \end{proposition}
    \begin{remark}
        This is a sufficient but not necessary condition.
    \end{remark}

    %%%%%%%% Recurrence
    \subsection{Recurrence}
    We define the \textbf{First Hitting Time} $T_i$ of the state $i$ by
    \[ T_i = \inf\{n>0 | X_n = i\} \]
    and we can define recurrence as follows
    % Recurrent
    \begin{definition}[Recurrence]
        The state $i$ is \textbf{recurrent} if $\mathbb{P}_i[T_i<\infty] = 1$, and is transient if it is not recurrent.
    \end{definition}
    \begin{remark}
        Recurrence means that starting from state $i$ at $T=0$, the chain \emph{is sure to} return to $i$ eventually.
    \end{remark}
    % Recurrence is a class property
    \begin{theorem}
        \normalfont
        Let $i$ be a recurrent state, and let $j$ be accessible from $i$, then all of the following hold:
        \begin{enumerate}
            \item $\mathbb{P}_i[T_j < \infty] = 1$.
            \item $\mathbb{P}_j[T_i < \infty] = 1$.
            \item The state $j$ is recurrent.
        \end{enumerate}
    \end{theorem}
    \begin{sketchproof}
        ~{}
        \begin{itemize}
            \item The paths starting from $i$ can be seen as infinitely many \emph{cycles}.
            \item Whether the chain visits $p$ in the cycles can be seen as a Bernoulli distribution with probability $p>0$.
            \item The probability of not visiting $j$ in the first $n$ cycles is $(i-p)^n$, which goes to $0$ as $n\to\infty$. So (1)hold.
            \item (2) can be proved by contradiction. $\mathbb{P}_j[T_i < \infty] < 1$ will lead to contradiction against the fact that $i$ is recurrent.
            \item (1)(2) implies (3).
        \end{itemize}
    \end{sketchproof}
    \begin{corollary}
        If $\mathbb{P}_i[T_j < \infty] > 0$ but $\mathbb{P}_j[T_i < \infty] < 1$, then $i$ is transient.
    \end{corollary}
    % Equivalent Statement of Recurrence
    \begin{theorem}[Equivalent Statement of Recurrence]
        \normalfont
        The state $i$ is recurrent if and only if $\mathbb{E}_i[N_i] = \infty$, where $N_i = \sum_{i=0}^\infty\mathbb{I}\{X_n = i\}$.
    \end{theorem}
    \begin{sketchproof}
        ~{}
        \begin{itemize}
            \item Recurrence $\Rightarrow \mathbb{P}_i[N_i = \infty] = 1 \Rightarrow \mathbb{E}_i[N_i] = \infty$.
            \item The converse is proved by contradiction.
            \item If $i$ is transient, there is a chance of $p$ that the chain never return to $i$.
            \item So $N_i$ is distributed geometrically, and $\mathbb{E}_i[N_i]$ will be finite. Contradiction.
        \end{itemize}
    \end{sketchproof}
    \begin{remark}
        By taking expectation on $N_i$, we have:
        \[ \mathbb{E}_i[N_j] = \sum_{i=0}^\infty P^n(i,j) \]
    \end{remark}

    \begin{corollary}\label{TransientStateGoesToZero}
        If $j$ is transient, then $\lim_{n\to\infty}P^n(i,j) = 0$ for all states $i$.
    \end{corollary}
    \begin{sketchproof}
        ~{}
        \begin{itemize}
            \item $\mathbb{E}_j[N_j] < \infty$.
            \item $\mathbb{E}_i[N_j] = \mathbb{P}_i[T_j < \infty]\mathbb{E}_i[N_j | T_j < \infty]$.
            \item $\mathbb{E}_i[N_j] \le \mathbb{E}_i[N_j | T_j] = \mathbb{E}_j[N_j] < \infty$ since the probability \emph{restarts} once the chain visits $j$ again.
            \item $\mathbb{E}_i[N_j] = \sum_{i=0}^\infty P^n(i,j) < \infty$ and this implies our conclusion.
        \end{itemize}
    \end{sketchproof}

    \begin{corollary}
        If $i$ is recurrent, then $\sum_{n=1}^{\infty}p^n(i,i) = \infty$;

        If $i$ is transient, then $\sum_{n=1}^{\infty}p^n(i,i) < \infty$.
    \end{corollary}

    \begin{proposition}
        Suppose a Markov Chain has a stationary distribution $\pi$, if the state $j$ is transient, then $\pi(j) = 0$.
    \end{proposition}
    The last proposition follows from Corollary \ref{TransientStateGoesToZero}.

    \begin{corollary}
        If an irreducible Markov Chain has a stationary distribution, then the chain is recurrent.
    \end{corollary}
    \begin{sketchproof}
        The chain cannot be transient, or otherwise all $\pi(j)$ would be 0 and the sum of $\pi$ does not equal to 1.
    \end{sketchproof}
    \begin{remark}
        The converse is not true!
    \end{remark}

    \begin{proposition}
        A drunk man will find his way home, but a drunk bird may get lost forever.
    \end{proposition}
    This is because the random walk on $\mathbb{Z}$ and $\mathbb{Z}^2$ is recurrent, while the walks on higher dimensions are transient.

    %%%%%%%% More on Recurrence
    \subsection{More on Recurrence}
    % Null Recurrence
    \begin{definition}[Null Recurrence]
        The state $i$ is \textbf{null recurrent} if it is recurrent and $\mathbb{E}_i[T_i] = \infty$.
    \end{definition}
    % Positive Recurrence
    \begin{definition}[Positive Recurrence]
        The state $i$ is \textbf{positive reccurent} it is recurrent and $\mathbb{E}_i[T_i] < \infty$.
    \end{definition}

    \begin{proposition}
        Given an irreducible Markov Chain, it is either transient, null recurrent or positive recurrent.
    \end{proposition}



\section{Strong Law of Large Numbers of Markov Chains}
% SLLN of Markov Chains
\begin{theorem}[SLLN of Markov Chains]\label{SLLN}
    Let $X_0, X_1, \dots$ be a Markov Chain starting in the state $X_0=i$. Suppose state $i$ communicates with state $j$. The limiting fraction of time that the chain spends in $j$ is $\frac{1}{\mathbb{E_j}[T_j]}$.
    \[ \mathbb{P}_i[\lim_{n\to\infty}\frac{1}{n}\sum_{t=1}^n\mathbb{I}\{X_t=j\} = \frac{1}{\mathbb{E_j}[T_j]}] = 1 \]
\end{theorem}
\begin{sketchproof}
    Read the book. I'm not going to write this because I am lazy.
\end{sketchproof}


\section{Basic Limit Theorem}\label{BasicLimitTheorem}
aka. Foundamental Theorem of Markov Chains

    %%%%%%%% Finite Case
    \subsection{Finite State Space}
    % Spectral Radius
    \begin{definition}[Spectral Radius]
        Given a non-negative matrix $A$, the spectral radius $\rho(A)$ is the maximum norm of its eigenvalues
        \[ \rho(A) = \max\{ \lambda(A) \} \]
    \end{definition}
    \begin{proposition}
        Let $A$ be a non-negative matrix, then
        \[ \min_{1\le i \le n} \sum_{j=1}^N a_{i,j} \le \rho(A) \le \max_{1\le i \le n} \sum_{j=1}^N a_{i,j} \]
    \end{proposition}
    % Perron-Frobenius Theorem
    \begin{lemma}[Perron-Frobenius Theorem]\label{PFT}
        Let $A$ be a non-negative matrix with spectral radius $\rho(A) = \alpha$, then $\alpha$ is an eigenvalue of $A$, and has both left and right non-negative eigenvectors.
    \end{lemma}
    \begin{remark}
        Lemma \ref{PFT} implies that for a finite probability transition matrix $P$, it always has at least one stationary distribution, because it always has eigenvalue $1$ ($P\mathbf{1} = \mathbf{1}$) and a corresponding eigenvector $\pi$ s.t. $\pi^TP = \pi$.
    \end{remark}

    \begin{lemma}
        Suppose a $k \times k$ matrix $P$ is irreducible. Then there exists a unique solution to $\pi P = \pi$.
    \end{lemma}
    % Foundamental Theorem, Finite Case
    \begin{theorem}[Foundamental Theorem, Finite Case] If a finite Markov Chain is \emph{irreducible} and \emph{aperiodic}, then it has a \emph{unique stationary distribution} $\pi$ any initial distribution \emph{converges} to it:
        \[ \forall \pi_0,\quad \lim_{t\to\infty}\pi_0^T P^t = \pi^T \]
    \end{theorem}

    %%%%%%%% Countable Case
    \subsection{Countable Case}
    % Basic Limit Theorem
    \begin{theorem}[Basic Limit Theorem]
        Let $X_1, X_2, \dots$ be an \emph{irreducible aperiodic} Markov Chain that has a \emph{stationary distribution} $\pi$. Then
        \[ \lim_{n\to\infty}\pi_n(i) = \pi(i) \]
        for any state $i$ and for any initial distribution $\pi_0$.
    \end{theorem}
    % Bounded Convergence Theorem
    \begin{lemma}[Bounded Convergence Theorem]\label{BCT}
        If $X_n \to X$ with probability $1$ and there is a finite number $b$ such that $|X_n| < b$ for all $n$, then
        \[ \mathbb{E}[X_n] \to \mathbb{E}[X] \]
    \end{lemma}
    % Corollary of SLLN
    Recall SLLN \ref{SLLN}, and the following is a corollary.
    \begin{corollary}\label{SLLNCoro}
        For an \emph{irreducible} Markov Chain, we have
        \[ \lim_{n\to\infty} \frac{1}{n}\sum_{t=1}^n P^t(i,j) = \frac{1}{\mathbb{E}_j[T_j]} \]
    \end{corollary}
    \begin{sketchproof}
        Take expectations on both sides of Theorem \ref{SLLN} yields the conclusion.
    \end{sketchproof}
    % Cesaro Average
    \begin{proposition}[Cesaro Average]
        If a sequence of numbers $a_n$ converges to a value $a$, then the \textbf{Cesaro Average} $(1/n)\sum_{t=1}^na_t$ also converges to it.
    \end{proposition}
    Corollary \ref{CesaroCoro} follows immediately from the proposition,
    \begin{corollary}\label{CesaroCoro}
        For an \emph{irreducible aperiodic} Markov Chain with a \emph{stationary distribution},
        \[ \lim_{n\to\infty}\frac{1}{n}\sum_{t=1}^nP^t(i,j) \to \pi(j) \]
    \end{corollary}
    \begin{theorem}
        An \emph{irreducible, aperiodic} Markov Chain with a \emph{stationary distribution} has a stationary distribution given by
        \[ \pi(j) = \frac{1}{\mathbb{E}_j[T_j]} \]
    \end{theorem}
    \begin{sketchproof}
        The proof of the theorem is trivial: compare Corollary \ref{SLLNCoro} and \ref{CesaroCoro}.
    \end{sketchproof}
    % Foundamental Theorem, Countable Case
    \begin{theorem}[Foundamental Theorem, Countable Case]\label{theorem:FTMC-Countable}
        An \emph{irreducible} Markov Chain has a \emph{unique stationary distribution} given by
        \[ \pi(j) = \frac{1}{\mathbb{E}_j[T_j]} \]
        if and only if it is \emph{positive recurrent}.
    \end{theorem}

    %%%%%%%% Other Conclusions
    \subsection{Other Conclusions}
    \emph{Since they are not covered in class, no proof will be provided. I'm lazy you know.}
    \begin{definition}[Doubly Stochastic Chains]
        A transition matrix $P$ is said to be \textbf{doubly stochastic} if all of its columns sum up to 1.
    \end{definition}
    \begin{optTheorem}
        For a doubly stochastic Markov Chain with $N$ states, the uniform distribution $\pi(i) = 1/N$ is a stationary distribution.
    \end{optTheorem}
    \begin{optTheorem}
        For a Markov Chain with symmetric $P$ and $N$ states, the uniform distribution $\pi(i) = 1/N$ is a stationary distribution.
    \end{optTheorem}
    \begin{optTheorem}
        Let $T_j^k = \min\{ n > T_y^{k-1} | X_n = y \}$ be the time of the $k$-th visit to $j$, then by the Markov property,
        \[ \mathbb{T}_x[T_y^k < \infty] = \mathbb{P}_x[T_y < \infty]\cdot\mathbb{P}_y[T_y<\infty] \]
        Notice that
        \[ \mathbb{E}[X] = \sum_{k=0}^{\infty}\mathbb{P}[X \le k] \]
        Using the two equations, we have
        \[ \mathbb{E}_x[N_y] = \frac{\mathbb{P}_x[T_y < \infty]}{1-\mathbb{P}_y[T_y < \infty]} \]
    \end{optTheorem}
    \begin{remark}
        This theorem gives us more insights into the results in Section \ref{BasicLimitTheorem}.
    \end{remark}



\section{Examples}

    %%%%%%%% Galton-Watson Process
    \subsection{Galton-Watson Process}\label{sub:BranchingProcess}
    \emph{aka. Branching Process}
    \paragraph*{Problem.} What is the probability that a family name eventually extincts?
    \paragraph*{Notations.}
    \begin{itemize}
        \item $G_t$ denotes the number of males in generation $t$.
        \item $X_{tk}$ denotes the number of sons fathered by the $k$-th father in the $t$-th generation. Assume $X_i$'s are \emph{iid} with probability mass function $p(\cdot)$.
        \item $\rho = \mathbb{P}[extinction] = \mathbb{P}[\cup_{k\ge 1}\{ G_k=0 \}]$ denotes the probability that the family name eventually goes extinct.
        \item $q_t = \mathbb{P}[G_t = 0]$ denotes the probability that the family name goes extinct at the $t$-th generation.
    \end{itemize}
    The problem is trivial when $p(0) = 0$ or $p(0) + p(1) = 1$. Either the family name never goes extinct or it goes extinct almost surely. We consider $p(0) > 0$ and $p(0) + p(1) < 1$.

    By conditioning on what happens at the first step, we can calculate $\rho$ by
    \[ G_{t+1} = \sum_{i=1}^{G_t} X_{ti} \]
    \begin{align}
    \rho &= \sum_{k=0}^{\infty}\mathbb{P}[extinction \wedge  G_1 = k] \notag \\
    &= \sum_{k=0}^{\infty}\mathbb{P}[extinction | G_1 = k]\cdot\mathbb{P}[G_1 = k | G_0 = 1] \quad \text{(by 全概率公式)} \notag \\
    &= \sum_{k=0}^{\infty}f(k)\rho^k \triangleq \psi(\rho) \notag
    \end{align}
    The last step used the fact that all males have sons independently. $\psi(\rho)$ is called the \textbf{Probability Generating Function} of $p(\cdot)$.

    The probability of eventual extinction satisfies $\psi(\rho) = \rho$.
    \[ \psi'(z) = \sum_{k=1}^{\infty}kp(k)z^{k-1} > 0\]
    \[ \psi''(z) = \sum_{k=2}^{\infty}k(k-1)p(k)z^{k-2} > 0 \]
    So $\psi(z)$ is a strictly increasing convex function with the following properties.
    \begin{itemize}
        \item $\psi(1) = 1$. So $\rho = 1$ is always a solution.
        \item $\psi'(1) = \sum_{k=1}^{\infty}kp(k) = \mathbb{E}[X] \triangleq \mu$
    \end{itemize}
    \begin{enumerate}[(i)]
        \item If $\mu \le 1$, $\rho = 1$ is the only solution. Therefore the family name will go extinct.
        \item If $\mu > 1$, there exists another solution $r<1$.
        
        Observe that $\{ G_t = 0 \} \subseteq \{ G_{t+1} = 0 \}$ and $q_t \le t_{t+1}$. Since $q_t$ has a upper bound $\rho$, it will always converge.
        \[ q_t \uparrow \rho \]
        So we only need to prove $q_t < r \quad \forall t$. By induction,
        \begin{itemize}
            \item[base.] $q_0 = 0 < r$.
            \item[hypo.] $q_t < r$.
            \item[step.] $q_{t+1} = \sum_{i=0}^{\infty}\mathbb{P}[G_1 = i]\mathbb{P}[G_{t+1} = 0| G_1 = i] = \sum_{i=0}^{\infty}p(i)(q_t)^i$. The last step used the iid assuption of $X$. 
            Therefore $q_{t+1} = \psi(q_t) \le \psi(r) = r$. Done.
        \end{itemize} 
    \end{enumerate}

    %%%%%%%% Gambler's Ruin
    \subsection{Gambler's Ruin}\label{GamblerRuin}
    \paragraph*{Problem.} Consider a gambler in a casino, who has probability $p$ to win \$$1$ and $1-p$ to lose \$$1$. The process stops when the gambler gets \$$N$ or goes to $0$. Assume the gambler starts the game with \$$i$.
    \paragraph*{Notations.}
    \begin{itemize}
        \item $P_i$ denotes the probability of the gambler gets \$$N$ and wins, starting with \$$i$.
        \item $Z_i \in \{1, -1\}$ denotes whether the gambler wins or loses in round $i$.
        \[ X_t = X_0 + \sum_{i=0}^{t-1}Z_i \]
    \end{itemize}
    Obviously,
    \[ P_N = 1 \quad P_0 = 0 \]
    When $1 \le i \le N-1$, $P_i$ can be calculated by
    \begin{align*}
        P_i &= \mathbb{P}[win | X_0 = i] \\
        &= \mathbb{P}[win \wedge Z_0 = 1 | X_0 = i] + \mathbb{P}[win \wedge Z_0 = -1 | X_0 = i] \quad \text{(Again by 全概率公式 on $Z_i$)} \\
        &= \mathbb{P}[win | X_0=i, Z_0 = 1]\cdot\mathbb{P}[Z_0=1|X_0=i] + \mathbb{P}[win|X_0=1, Z_0=-1]\cdot\mathbb{P}[Z_0=-1|X_0=i] \\
        &= p\cdot P_{i+1} + (1-p)\cdot P_{i-1}
    \end{align*}
    Rearranging,
    \[ p(P_{i+1} - P_i) = (1-p)(P_i - P_{i-1}) \]
    \[ P_{i+1} -P_i = \frac{1-p}{p}(P_i - P_{i-1}) \]
    Let $\theta = \frac{1-p}{p}$, by high school mafs
    \[ P_{i+1} - P_i = \theta^i(P_1 - P_{0}) \]
    \begin{enumerate}[(i)]
        \item Assume for now that $p \neq \frac{1}{2}$ so $\theta \neq 1$, then summing over $i$ yields
        \[ P_N - P_0 = 1 = \frac{1-\theta^N}{1-\theta} P_1 \]
        Therefore
        \[ P_1 = \frac{1-\theta}{1-\theta^N} \]
        Summing from $0$ to $i$ yields
        \[ P_i = \frac{1-\theta^i}{1-\theta^N} \]
        \item If $p=\frac{1}{2}$,
        \[ P_{i+1} - P_i = P_i - P_{i-1} \]
        This is a arithmetic progress, and again by high school mafs
        \[ P_i = \frac{i}{N} \]
    \end{enumerate}
    To sum up
    \[ 
        P_i = 
        \begin{cases}
            \frac{i}{N} &(i = \frac{1}{2})\\
            \frac{1-\theta^i}{1-\theta^N} &(\text{o.w.})
        \end{cases}    
    \]

    \subsection{Drug Test}\label{DrugTest}
    This example is based on results from the previous subsection \ref{GamblerRuin}.
    \paragraph*{Problem.} We want to test the cure rate $p_1$ of a drug \textrm{Drug}1. We already have a \textrm{Drug}2 with known cure rate $p_2$. We want to know whether $p_1 > p_2$ or not. To do this, we find $t$ pairs of patients $(X_i, Y_i)$ and conduct tests using the two drugs on $X$ and $Y$ respectively. Once the number of patients who are cured by \textrm{Drug}1 but not \textrm{Drug}2 exceeds a certain threshold $M$, we can claim that $p_1 \ge p_2$.
    \paragraph*{Notations.}
    \begin{itemize}
        \item $X_i,Y_i \in \{0,1\}$ is a boolean denoting whether the first or the second drug cured patient $i$.
        \item $Z_i = X_i - Y_i$.
        \item $p_1, p_2$ denotes the probability that the two drugs cure a patient, respectively.
    \end{itemize}

    The value of $Z_i$ falls into 3 cases:
    \[ Z_i = 
        \begin{cases}
            1 &p_1(1-p_2)\\
            -1 &p_2(1-p_1)\\
            0 &(\text{o.w.})
        \end{cases}
    \]
    If we ignore the cases where $Z_i = 0$, then we can model this problem as a gambler's ruin, with $p = \frac{p_1(1-p_2)}{p_1(1-p_2)+p_2(1-p_1)}$.

    And
    \[ \mathbb{P}[TestWrong] = 1 - \frac{1-\theta^M}{1-\theta^{2M}} = \frac{1}{\theta^{-M} + 1} \]
    The probability above drops exponentially with $M$, so the threshold does not need to be very large to achieve accurate results.

    %%%%%%%% Another random walk on N
    \subsection{Another Random Walk}\label{AnotherRandWalk}
    \paragraph*{Problem.} Consider a random walk on $\mathbb{N}$ where
    \[ P(0,1) = 1 \quad P(N,N-1) = 1 \]
    We want to compute how many steps we need to reach $N$ starting from $i$.
    \paragraph*{Notations.}
    \begin{itemize}
        \item $h_i$ denotes the number of steps to reach $N$ starting from $X_0=i$.
        \item $Y_i$ denotes the number of steps from $i$ to $i+1$ for the first time.
        \item $g_j \triangleq \mathbb{E}[Y_j]$.
    \end{itemize}
    Obviously
    \[ \mathbb{E}[h_0] = 1 + \mathbb{E}[h_1] \quad \mathbb{E}[h_N] = 0 \]
    Similar to subsection \ref{DrugTest},
    \[ h_i = 1+ (1-p)h_{i-1} + ph_{i+1} \]
    This can be calculated using \emph{the linearity of expectation}. 

    Notice that
    \[ h_i = \sum_{j=i}^{N-1}Y_j \]
    Taking expectations on both sides
    \[ \mathbb{E}[h_i] = \sum_{j=i}^{N-1}\mathbb{E}[Y_j] \]
    So we only need to caculate $\mathbb{E}[Y_j]$.
    \begin{itemize}
        \item $g_0 = 1$.
        \item $g_i = 1 + 0 \cdot p + (1-p)(g_{i-1}+g_i)$.
    \end{itemize}
    \begin{enumerate}
        \item Again assume $p \neq \frac{1}{2}$. 
        Rearranging yields
        \[ g_i = \frac{1}{p} + \theta g_{i-1} \]
        and after a few steps of arithmetics
        \[ g_i = \sum_{i=0}^{t-1}\frac{1}{p}\theta^i + \theta^t \]
        Summing over $g_i$ is a sum of geometric progress, easy.
        \item If $p = \frac{1}{2}$, then
        \[ g_{i} - g_{i-1} = \frac{1}{p} = 2 \]
        so
        \[ g_t = 2t + 1 \]
        and
        \[ \mathbb{E}[h_0] = \sum_{t=0}^{N-1}(2t+1) = N^2 \]
    \end{enumerate}
    \begin{remark}
        As the converse of the conclusion, if we take $N$ steps, the farthest distance we can go is $\sqrt{N}$.
    \end{remark}



\section{Coupling and Stochastical Dominance}
In this section we assume the sample space $\Omega$ is at most countable, but the results in this section can be generalized to continuous case.

    %%%%%%%% Motivating Example: 2-SAT
    \subsection{Motivating Example: 2-SAT}
        Recall the SAT problem in \textsc{AI2615 Design and Analysis of Algorithms}. A 2-SAT is a special case of SAT where each \verb|or| expression has at most 2 terms
        \[ \varphi = (x_1 \vee y_1) \wedge (x_2 \vee y_2) \cdots \]
        The problem is RP and can be solved using the following algorithm.

        \begin{algorithm}
            \caption{2-SAT Solver}
            \KwIn{CNF with $n$ terms}
            Random intialize a solution $\sigma$.\\
            \For{$i = 1:100n^2$}
            {
                \If{$\sigma$ satisfies CNF}
                {
                    \KwRet{$\sigma$}
                }
                \Else
                {
                    Randomly choose one term $(X_i \vee Y_i)$ that is false.\\
                    Randomly flip $X_i$ or $Y_i$.
                }
            }
            \KwRet{Not satisfiable}
            \label{2SATAlgo}
        \end{algorithm}

        To prove its correctness, we only need to consider the cases where the CNF is indeed satisfiable with solution $\sigma$. We denote the sequence of attempts produced by the algorithm by
        \[ \sigma_0 \to \sigma_1 \to \cdots \to \sigma_{100n^2} \]
        Let $X_i$ denotes the number of terms in $\sigma$ and $\sigma_i$ that are exactly the same.

        You should convince yourself that $X_0, X_1, \dots$ is not a Markov Chain. \sout{So we get stuck}. However we can still informally show the correctness.

        We first introduce some basic concepts.

    %%%%%%%% Stochastic Dominance
    \subsection{Stochastic Dominance}
        \begin{definition}[Stochastic Dominance]
            A distribution $\mu$ is said to \textbf{stochasitcally dominate} $\nu$ if
            \[ \forall x\quad \mathbb{P}_{X\sim\mu}[X \ge x] \ge \mathbb{P}_{Y\sim\nu}[Y \ge x] \]
        \end{definition}
        \begin{remark}
            Let $F$ and $G$ be the cumulative distribution function of $X$ and $Y$, we have $F(x) \le G(x)$, i.e. the probability of $X$ concentrates on larger values.
        \end{remark}
        \paragraph{Some Examples.}
        \begin{itemize}
            \item 2-SAT.
            \item Binomial distribution. $B(n,p)$ and $B(n,q)$.
            \item Erd$\mathrm{\ddot{o}}$s-R$\mathrm{\acute{e}}$nyi Random Graph $\mathcal{G}(n,p)$ is a graph with $n$ nodes and the edges have a probability of $p$ to exist. We sample $G$ randomly from $\mathcal{G}$ and test if $G$ is connected. If $p>q$ then $G_p$ stocahstically dominates $G_q$.
        \end{itemize}

    %%%%%%%% Coupling
    \subsection{Coupling}
        \begin{definition}[Coupling]
            $(X',Y')$ is the coupling of random variables $(X,Y)$ if $X'$ has the same distribution of $X$ and $Y'$ has the same of $Y$. i.e. A coupling $C$ of $X$ and $Y$ is a joint distribution of $X$ and $Y$.
        \end{definition}
        \begin{remark}~{}
            \begin{itemize}
                \item The coupling of $X$ and $Y$ is not unique.
                \item If $X$ has distribution $\mu$ and $Y$ has distribution $\nu$, then any joint distribution $(X,Y)$ having marginal distribution $X\sim\mu$ and $Y\sim\nu$ is a coupling of $X$ and $Y$.
                \[ \forall x, \quad \mathbb{P}_{(X,Y) \sim C}[X=x] = \mu(x) \]
                \[ \forall y, \quad \mathbb{P}_{(X,Y) \sim C}[Y=y] = \nu(y) \]
                \item Constructing a coupling can be intuitively seen as a process of filling a table of the joint distribution of $X$ and $Y$.
            \end{itemize}
        \end{remark}
        \begin{definition}[Monotone Coupling]
            Let $C$ be a coupling of $X$ and $Y$. $C$ is a \textbf{monotone coupling} if
            \[ \mathbb{P}_{(X,Y)\sim C}[X\ge Y] = 1\footnote{“它总是大”——Prof.Zhang} \]
        \end{definition}
        \begin{theorem}\label{MonoCouplingAndDominance}
            There is a monotone coupling of $X$ and $Y$ \emph{if and only if} $X$ stochastically doniminates $Y$.
        \end{theorem}
        \begin{proof}
            ~{}\\
            $\Rightarrow$.
            By definition of coupling
            \[ \mathbb{P}_{Y\sim\nu}[Y \ge a] = \mathbb{P}_{(X,Y)\sim C}[Y\ge a] \]
            By the Law of Total Probability
            \[ \mathbb{P}_{Y\sim\nu}[Y \ge a] = \mathbb{P}_{(X,Y)\sim C}[Y \ge a \wedge X \ge Y]\mathbb{P}[X \ge Y] + \mathbb{P}_{(X,Y)\sim C}[Y \ge a \wedge X < Y]\mathbb{P}[X < Y] \]
            The latter term is $0$ by definition of monotone coupling, so
            \[ \mathbb{P}_{Y\sim\nu}[Y \ge a] = \mathbb{P}_{(X,Y)\sim C}[X \ge Y \ge a] \le \mathbb{P}_{(X,Y)\sim C}[X \ge a] = \mathbb{P}_{X\sim\mu}[X \ge a] \]

            $\Leftarrow$. Suppose $X$, $Y$ are two random variables having  distribution $F(x)$ and $G(x)$ respectively. By stochastic dominance we have $F(x) \le G(x)$ for any $x$. Let $U$ be a random variable having a uniform distribution over $(0,1)$, since $F(x)$ is increasing
            \[ F^{-1}(U) \le x \Leftrightarrow F(F^{-1}(U)) \le F(x) \]
            Therefore
            \[ \mathbb{P}[F^{-1}(U) \le x] = \mathbb{P}[F(F^{-1}(U)) \le F(x)] = \mathbb{P}[U \le F(x)] = F(x) \]
            Similarly
            \[ \mathbb{P}[G^{-1}(U) \le x] = G(x) \]
            Since $F(x) \le G(x)$, we have $F^{-1}(x) \ge G^{-1}(x)$ and therefore we obtained a coupling $C = (X' = F^{-1}(U), Y' = G^{-1}(U))$ for which $X' \ge Y'$.
        \end{proof}

    %%%%%%%% 2-SAT Revisited
    \subsection{2-SAT: Proof of Correctness}
        \begin{theorem}[Markov's Inequality]\label{MarkovIneq}
            Let $X$ be a nonnegative random varialbe and let $a > 0$, then
            \[ \mathbb{P}[X \ge a] \le \frac{\mathbb{E}[X]}{a} \]
        \end{theorem}
        We can now prove the correctness of Algorithm \ref{2SATAlgo}.
        \begin{sketchproof}
            Note that $X_{i+1}$ differs from $X_i$ in at most 1 operand, and that
            \[ \mathbb{P}[X_{i+1} = X_i + 1] \ge \frac{1}{2} \]
            \[ \mathbb{P}[X_{i+1} = X_i - 1] \le \frac{1}{2} \]
            We can introduce a random walk described in subsection \ref{AnotherRandWalk} $Y_1, Y_2, \dots$ with $p = \frac{1}{2}$.

            $X_i$ stochatically dominates $Y_i$. This is intuitive, and can be proved using Theorem \ref{MonoCouplingAndDominance} and constructing a monotone coupling:

            We introduce a random variable $U$ that has a uniform distribution $U \sim U(0,1)$. Let $q$ be the probability of $X_{i+1} = X_i + 1$, from the arguments above we already have $p < q$. Then we construct a coupling of $X$ and $Y$ by
            \[ X_i = \mathbb{I}[U_i < q] \quad Y_i = \mathbb{I}[U_i < p] \]
            Since $p < q$, clearly $Y_i \le X_i$, and thus
            \[ X_0 + X_1 + \dots + X_i \ge Y_0 + Y_1 + \dots + Y_i \]
            So we have constructed a monotone coupling, and it follows that $X$ stocahstically dominates $Y$.
            
            Since $Y_i$ has an expectation of $n^2$ to get to state $n$, it follows intuitively that $X_i$ has a smaller expectation of steps to get to $n$.

            We have chosen the max iteration to be $100n^2$, by Markov's Inequality \ref{MarkovIneq}, the probability that we take $100n^2$ iterations without finding a feasible solution is at most $1/100$.
        \end{sketchproof}
        \begin{remark}~{}
            \begin{itemize}
                \item \sout{A formal proof of correctness will be given in the next lecture.}
                \item The rigorous proof has been updated.
                \item The idea of coupling can be used to prove other examples of stochastic dominance.
            \end{itemize}
        \end{remark}

    %%%%%%%% Maximum Couplings
    \subsection{Maximum Couplings}
        \begin{definition}[Maximum Coupling]
            Suppose $X$ has distribution function $\mu$, $Y$ has distribution function $\nu$. Let $C$ denote the set of all couplings of $(X,Y)$. $(\hat{X},\hat{Y})$ is a \textbf{maximum $\mu$,$\nu$ couple} if
            \[ \mathbb{P}[\hat{X}=\hat{Y}] = \max_{(X,Y)\in C}\mathbb{P}[X=Y] \]
            i.e. among all such couples its components are most likely to be equal.
        \end{definition}
        \begin{proposition}
            A maximum coupling always exists.
            \[ \mathbb{P}[\hat{X} = \hat{Y}] = \sum_{z\in\Omega}m(z) \]
            where $m(z) = \min\{ \mu(z), \nu(z) \}$.
        \end{proposition}
        \begin{proof}
            Let $p = \sum_{z\in\Omega m(z)}$. Let $A = \{x| \mu(x) > \nu(x)\}$.
            
            Notice that
            \[ m(x) = 
                \begin{cases}
                    \mu(x) \quad &(x \in A)\\
                    \nu(x) \quad &(x \notin A)
                \end{cases}
            \]
            
            \begin{align*}
                \mathbb{P}[X=Y] &= \mathbb{P}[X=Y\in A] + \mathbb{P}[X = Y \notin A]\\
                &\le \mathbb{P}[Y \in A] + \mathbb{P}[X \notin A]\\
                &= \sum_{z\in A}\nu(z) + \sum_{z\notin A}\mu(z)\\
                &= \sum_{z\in A}m(z) + \sum_{z \notin A}m(z)\\
                &= p
            \end{align*}
        \end{proof}
        \begin{remark}
            For continuous cases, the sum becomes integral.
        \end{remark}

    %%%%%%%% Total Variance Distance
    \subsection{Total Variance Distance}
        \begin{definition}[Total Variation Distance]
            Given two distributions $\mu$ and $\nu$ defined on a sample space $\Omega$, the \textbf{total variance distance} of $\mu$ and $\nu$ is defined as
            \[ \|\mu - \nu \|_{TV} \triangleq \frac{1}{2}\sum_{x\in\Omega}|\mu(x)-\nu(x)| = \max_{A\subseteq\Omega} \mu(A)-\nu(A) = 1-\min_{z\in\Omega}\{ \mu(z), \nu(z) \} \]
            Geometrically, this is the half of size of the area between the pdf of $\mu$ and $\nu$.
        \end{definition}
        \begin{lemma}[The Coupling Lemma]\label{CouplingLemma}
            For any coupling $C$ of $\mu$ and $\nu$,
            \[ \mathbb{P}_{(X,Y)\sim C}[X \neq Y] \ge \|\mu - \nu \|_{TV} \]
            and there exists a coupling $C^{*}$ that achieves the equality.
        \end{lemma}
        \begin{proof}
            If we view the coupling as a way to fill the table, then $\mathbb{P}_{(X,Y) \sim C}[X=Y]$ is the sum of all values on the diagonal of the table. Intuitively, the value on $i$-th row and $j$-th column is upper bounded by $\mu(i)$ and $\nu(j)$, so the sum is upper bounded by $\sum_{z\in\Omega}\min\{\mu(z), \nu(z)\}$.
            \begin{align*}
                \mathbb{P}[X \neq Y] &= 1 - \mathbb{P}_{(X,Y) \sim C}[X = Y] \\
                &= 1 - \sum_{z \in \Omega}\mathbb{P}[X=Y=z] \\
                &\ge 1 - \sum_{z \in \Omega}\min \{\mu(z), \nu(z)\}\\
                &= \sum_{z \in \Omega}\left( \mu(z) - \min\{\mu(z),\nu(z)\} \right)\\
                &= \| \mu - \nu \|_{TV}
            \end{align*}
            The previous part showed that $\mathbb{P}_{(X,Y)\sim C}[X \neq Y]$ is always lower bounded by $\|\mu - \nu\|_{TV}$, we further show that the equality is achieved by the maximum coupling. Let $A = \{x| \mu(x) > \nu(x)\}$
            \begin{align*}
                \|\mu - \nu\|_{TV} &= \mu(A) - \nu(A) = 1 - \mu(A^{c}) - \nu(A)\\
                &= 1 - \sum_{z \notin A}\mu(z) - \sum_{z \in A}\nu(z)\\
                &= 1 - \sum_{z \notin A}m(z) - \sum_{z \in A}m(z)\\
                &= 1 - \mathbb{P}[X=Y]\\
                &= \mathbb{P}[X \neq Y]
            \end{align*}
        \end{proof}
        \begin{remark}
            The $C^*$ here is the maximal coupling.
        \end{remark}



\section{Proof of the Foundamental Theorem}
    We are going to prove that:
    \paragraph{}
        \emph{An irreducible, aperiodic, positive recurrent Markov Chain has a unique stationary distribution, and it converges to the distribution.}

    \begin{proof}
        We only consider the finite case. The proof of Theorem \ref{theorem:FTMC-Countable} has already shown that an Irreducible Positive Recurrent Markov chain has a Unique Stationary Distribution, so we only prove the convergence part here. Suppose there are two Markov Chains $X$ and $Y$
        \[ X_0 \to X_1 \to \dots \to X_t \to \dots \]
        \[ Y_0 \to Y_1 \to \dots \to Y_t \to \dots \]
        Let the initial distribution of $Y$ be the stationary distribution $\pi$, so each $Y_i$ has distribution $\pi$. Let the initial distribution of $X$ be $\mu_0$, and $X_i$ has the distribution given by $\mu_{i} = P^T\mu_{i-1}$.

        We define \textbf{convergence} by $\lim_{i\to\infty}\|\mu_{i} - \pi\|_{TV} = 0$.

        For each $(X_t, Y_t)$, we construct a coupling $C_t$ such that each chain is runned independently, and once $X_t = Y_t$ for some $t$, then $X_{t'} = Y_{t'}$ for all $t' \ge t$.

        Irreducibility guarantees that 
        \[ \forall i,j \quad \exists n \quad P^n(i,j) > 0 \]
        And Aperiodicity further guarantees that
        \begin{equation}\label{FTMCEq1}
            \exists n \quad \forall i,j \quad P^n(i,j) > 0
        \end{equation}
        To prove (\ref{FTMCEq1}), we are going to show that
        \begin{equation}\label{FTMCEq2}
            \forall i,j \quad \exists t_{ij} \quad \forall t>t_{i,j} \quad P^t(i,j) > 0
        \end{equation}
        Then (\ref{FTMCEq1}) can be proved by setting $n = \max_{i,j}\{t_{ij}\}$. Note that this result only holds in finite case.

        We visualize the Markov Chain from $i$ to $j$ as many cycles starting from and ending at $i$, of length $c_1,c_2,\dots$. Since the chain is Aperiodic,
        \[ \gcd(c_1, c_2,\dots,c_s) = 1 \]
        And by Lemma \ref{theorem:Bezout}, there exist integers $x_1,x_2,\dots,x_s$ such that
        \[ c_1x_1 + c_2x_2 + \dots + c_sx_s = 1 \]
        And it follows immediately that there exist $y_1,y_2,\dots,y_s$ such that
        \[ c_1y_1 + c_2y_2 + \dots + c_sy_x = b \]
        for all sufficiently large $b$.

        \begin{lemma}[B$\mathrm{\acute{e}}$zout's Theorem]\label{theorem:Bezout}
            Let $a,b$ be two integers, then there exist integer $\mu, \nu$ such that
            \[ a\mu + b\nu = \gcd(a,b) \]
        \end{lemma}

        Now consider the coupling.
        \[ \mathbb{P}[X_n = Y_n] \ge \mathbb{P}[X_n=Y_n=x] \ge SomeConstant~{} \alpha > 0 \]
        This is guaranteed by the irreducibility of the chain.
        So
        \[ \mathbb{P}[X_n \neq Y_n] \le 1 - \alpha \]
        Now condider the $2n$-th step, by the Law of Total Probability,
        \[ \mathbb{P}[X_{2n} \neq Y_{2n}] = \mathbb{P}[X_{2n} \neq Y_{2n} \wedge X_n \neq Y_n] + \mathbb{P}[X_{2n} \neq Y_{2n} \wedge X_n = Y_n] \]
        The latter is $0$ because it is impossible by our construction of coupling.
        
        By conditional probability
        \[ \mathbb{P}[X_{2n} \neq Y_{2n} \wedge X_n \neq Y_n] \le \mathbb{P}[X_{2n} \neq Y_{2n} | X_n \neq Y_n](1-\alpha) \le (1-\alpha)^2 \]

        It follows immediately that $\mathbb{P}[X_t \neq Y_t] \to 0$ as $t \to \infty$. So $\lim_{t\to\infty}\|\mu_t - \pi\|_{TV} = 0$, and we are done.
    \end{proof}


\section{Applications of Markov Chain}
    We assume the state space $\mathcal{S}$ is finite, but in most cases the applications can be generalized to countably infinite cases. And we assume that the Markov chains are Aperiodic and Irreducible.

    \subsection{Time Reversibility}
        \begin{definition}[Time Reversibility]
            A Markov chain is said to be time reversible if there exists a distribution $\pi$ such that
            \begin{equation}\label{def:DetailedBalanceCondition}
                \forall x,y \in \mathcal{S} \quad \pi(x)P(x,y) = \pi(y)P(y,x)
            \end{equation}
        \end{definition}
        \begin{remark} ~{}
            \begin{itemize}
                \item The $\pi$ here is actually the stationay distribution. because
                \[ (\pi^TP)(y) = \sum_{x\in\mathcal{S}}\pi(x)P(x,y) = \sum_{x\in\mathcal{S}}\pi(y)P(y,x) = \pi(y)\sum_{x\in\mathcal{S}}P(y,x) = \pi(y) \]
                \item (\ref{def:DetailedBalanceCondition}) is also called the \textbf{detailed balance condition}.
                \item The detailed balance condition implies that $(X_0,X_1,\dots,X_n)$ and $X_n,X_{n-1},\dots,X_0$ have \emph{the same distribution}, and this is the reason why such chains are called time-reversible.
            \end{itemize}
        \end{remark}

    \subsection{Metropolis Algorithm}
        If we want to sample from some distribution $\mu$, one way of doing this is to design a random walk (Markov Chain) with stationary distribution $\mu$ and run the random walk for a sufficiently long period of time. We now discuss how to design such random walks. Recall the following proposition:
        \begin{proposition}
            A random walk on graph $\mathcal{G}$ with probability transition matrix
            \[P(i,j) = 
            \begin{cases}
                \frac{1}{d_i} \quad &j \in \mathcal{N}(i)\\
                0 \quad &j otherwise
            \end{cases}
            \]
            has stationary distribution
            \[ \pi(i) = \frac{d_i}{\sum_{j\in\mathcal{S}}d_j} \]
        \end{proposition}
        Consider a random walk on graph $\mathcal{G} = (V, E)$. Given a distribution $\mu$, we want to design a random walk such that the stationay distribution is $\mu$. Let $d_i$ denote the degree of the $i$-th node in $\mathcal{G}$.\\
        Let
        \[ \Delta = \max_{i \in V} d_i \]
        For any node $i$, for all $j \in \mathcal{N}(i)$, we ``propose'' to move to $j$ with probability $1/\Delta$. And $j$ ``accepts'' $i$ with probability $\min \{\frac{\mu(j)}{\mu(i)}, 1\}$. If the move is rejected, we stay in $i$. We define the probability transition matrix as follows
        \begin{equation}\notag
            P(i, j) = 
            \begin{cases}
                0 \quad & j \notin \mathcal{N}(i)\\
                \frac{1}{\Delta}\min\{1, \frac{\mu(j)}{\mu(i)}\} \quad &j \in \mathcal{N}(i)\\
                1 - \sum_{k\in\mathcal{N}(i)} P(i,k) \quad & i = j
            \end{cases}
        \end{equation}
        \begin{proposition}
            The $P$ defined above satisfies
            \[ \mu P = \mu \]
            i.e. it is a Markov chain with stationay distribution $\pi$.
        \end{proposition}
        \begin{sketchproof}~{}
            \begin{itemize}
                \item Trivial if $j \notin \mathcal{N}(i)$.
                \item If $j \in \mathcal{N}(i)$. WLOG assume $\mu(j) \ge \mu(i)$.
                \[ \mu(i)P(i,j) = \mu(i)\frac{1}{\Delta}\min\{1,\frac{\mu(j)}{\mu(i)}\} = \frac{\mu(i)}{\Delta} \]
                and
                \[ \mu(j)P(j,i) = \mu(j)\frac{1}{\Delta}\min\{1,\frac{\mu(i)}{\mu(j)}\} = \frac{\mu(i)}{\Delta} \]
                So we are done.
            \end{itemize}
        \end{sketchproof}
        \begin{remark}
            If we run the algorithm sufficiently long, the sequence we get will have a distribution that is approximately the same as $\mu$. What's more, in this algorithm, we only need to know $\frac{\mu(j)}{\mu(i)}$, instead of knowing $\mu$. And this is useful in practice.
        \end{remark}

    \subsection{Simulated Annealing}
        \par Simulated Annealing is a stochastic optimization algorithm. Given a set $\mathcal{S}$, and a function $w(x)$. We want to minimize (or maximize) $w(x)$.
        \paragraph{Notations.}
        \begin{itemize}
            \item $\mathcal{S}^* = \{x\in\mathcal{S}|w(x) = \min_{y\in\mathcal{S}} w(y)\}$: the set of global minimizers of $w(x)$.
            \item $\mu^*$: a uniform distribution on $\mathcal{S}^*$.
            \item $T$: The ``Temperature'' parameter.
        \end{itemize}
        \par We define a distribution on $\mathcal{S}$, and we want to make the probability concentrate on $x$ where $w(x)$ is small. However sampling from $\mathcal{S}$ is usually infeasible because $\mathcal{S}$ is too large.
        \par Instead, we introduce a parameter $T$, and let
        \[ \mu_T(x) \sim e^{-\frac{w(x)}{T}} \]
        \begin{proposition}
            As $T \to 0$, $\mu_T(x) \to \mu^*(x)$.
        \end{proposition}
        \par However, if we start with a very small $T$, we may get stuck in a local extremum. So we start with a large $T_1$ and gradually decrease $T$.
        \begin{theorem}
            If $T_1,T_2,\dots$ satisify
            \begin{enumerate}
                \item $\{T_i\} \to 0$.
                \item $T_i$ decreases slow enough. (``Slow'' means that the sum of a certain series defined by $\mathcal{S}$ diverges, but it is so complicated that it is beyond the scope.)
            \end{enumerate}
            Then let $P^{(n)} \triangleq \prod_{k=1}^n P_{T_k}$.
            \[ \forall \mu \quad \mu^TP^{(n)} \to \mu^* \]
        \end{theorem}
        That is, as long as $T$ decreases slowly enough, we only need to run each temperature $T_i$ \emph{for one step with the Metropolis algorithm}.


\section{Convergence Analysis of Markov Chains}
    By designing a proper coupling $C$, we can upper bound the convergence speed of a Markov Chain.

    \subsection{Mixing Time}
        \begin{definition}[Mixing Time]\label{def:MixingTime}
            The \textbf{mixing time} $\tau_{mix}(\varepsilon)$ is defined as
            \[ \tau_{mix}(\varepsilon) = \max_{a} \min_{t} \| \mu_t^a - \pi \|_{TV} \le \varepsilon \]
            It represents the minimal time for $\mu$ to converge to $\pi$ starting from the worst initial distribution $\mu^a$.
        \end{definition}

    \subsection{Some Examples}

        \subsubsection{Random Walk on Hypercube.}
            A \textbf{hypercube} is a $\{0,1\}$ string $\{0,1\}^n$, and there exists an edge between $x$ and $y$ if and only if $\sum_{i=1}^n|x(i)-y(i)| = 1$, i.e. $x$ and $y$ differs in only one bit.
            Let
            \[
                X_{t+1} = 
                \begin{cases}
                    X_t \quad & w.p.\frac{1}{2}\\
                    \textit{Move into one of the neighbours} & w.p.\frac{1}{2}
                \end{cases}
            \]
            i.e. we have probability $1/2$ to stay still and probability $1/2$ to move into a neighbour.

            If we start from the worst case, then $\mathbb{P}[X_t \neq Y_t]$ can give an upper bound of $\tau_{mix}$.

            We construct the coupling as follows. Suppose we are at $X_t$,
            \begin{enumerate}
                \item Uniformly pick $i \in n$.
                \item Uniformly pick $c \in \{0, 1\}$.
                \item Set the $i$-th bit of $X_t$ to $c$ and yield $X_{t+1}$.
            \end{enumerate}
            We choose \emph{the same} $i$ and $c$ for both $X_t$ and $Y_t$.

            It can be proved that if we run $n\log n + cn$ iterations, the probability that $X_t \neq Y_t$ is at most $e^{-c}$. i.e.
            If
            \[ t \ge n\log n + cn \]
            Then
            \[ \mathbb{P}[X_t \ne Y_t] \le e^{-c} = \varepsilon \]
            Therefore 
            \[ \tau_{mix} \le n\log n + n\log \frac{1}{\varepsilon} \le \varepsilon\]
            So we need approximately at most $n\log n$ steps.

        \subsubsection{Coupon Collection}
            Given a set of $n$ coupons, we pick $X$ times. And we want to know the total time we need to collect all $n$ coupons.

            Let $X_i$ denotes the number of gachas needed to get the $i$-th coupon.
            \[ \mathbb{E}[X] = \mathbb{E}[\sum_{i=1}^n X_i] = \sum_{i=i}^n \mathbb{E}[X_i] \]
            Notice that $X_i$ has a geometric distribution with parameter $\frac{n-i+1}{n}$. Therefore
            \[ \mathbb{E}[X] = \sum_{i=1}^n \frac{n}{n-i+1} = n\sum_{i=1}^n \frac{1}{i} \approx n(\log n + \gamma) \]

        \subsubsection{Card Shuffling}
            Suppose we shuffle a deck of 52 cards using a ``top-in-at-random'' method: we pick the card on the top and insert it into one of 52 positions (including the top) at random. This Markov chain has the stationary distribution $\pi = Uniform(\mathcal{S}_{52})$. By FTMC, $\|\pi_n - pi\|_{TV} \to 0$ as $n \to \infty$.

            It is possible to find a random variable $T$ which denotes the time at which the deck becomes exactly uniformly distributed. We first introduce some notations.

            \begin{itemize}
                \item $U_i$ is uniformly distributed on $1,2,\dots,52$ representing that the $i$-th shuffle moves the card on the top to position $U_i$.
                \item $T_1 = \inf\{ n: U_n = 52 \}$ denotes the first time a card is moved below card 52.
                \item $T_2 = \inf\{ n>T_1: U_n \ge 51 \}$ denotes the second time a card is moved below card 52.
                \item $T_{51} = \inf\{ n>T_50: U_n \ge 2 \}$ denotes the 51st time a card is moved below card 52.
            \end{itemize}

            Then $T = T_{52} = T_{51} + 1$.

            Clearly we have
            \begin{itemize}
                \item $T_1 \sim Geometry(1/51)$
                \item $T_2 - T_1 \sim Geometry(2/52)$
                \item \dots
                \item $T_{52} - T_{51} \sim Geometry(52/52) = 1$
            \end{itemize}

            We can calculate $\mathbb{E}[T]$. By linearity of Expectation,
            \[ \mathbb{E}[T] = \mathbb{E}[T_1] + \mathbb{E}[T_2-T_1] + \dots + \mathbb{E}[T_52 - T_51] \approx 52\log 52 \]
            Similarly, if we are shuffling $d$ cards, we have
            \[ \mathbb{E}[T] = d\log{d} \]

    \subsection{Strong Stationary Times}
        \begin{definition}[Stopping Time]
            The time $T$ is called the \textbf{stopping time} if for each $n$, one can tell whether $T=n$ by just looking at $X_0, X_1, \dots, X_n$, i.e. it is not necessary to know any other future values of $X_{n+1}, X_{n+2}, \dots$.
        \end{definition}
        \begin{remark}
            In the card-shuffling example, $T$ is a stopping time.
        \end{remark}
        \begin{definition}[Strong Stationary Time]
            A random variable $T$ is called a \textbf{strong stationary time} if it satisfies
            \begin{enumerate}
                \item $T$ is a stopping time
                \item $X_T$ is distributed as $\pi$
                \item $X_T$ is independent of $T$
            \end{enumerate}
        \end{definition}
        \begin{lemma}\label{Lemma:StrongStationaryTime}
            If $T$ is a strong stationary time for the Markov Chain $\{X_n\}$, then $\|\pi_n - \pi\| \le \mathbb{P}[T > n]$ for all $n$.
        \end{lemma}

    \subsection{Threshold Pheonomenon}
        In card shuffling and many other Markov Chains, $\|\pi_n - \pi\|$ stays close to $1$ for a while, and suddenly drop to nearly $0$ at approximately $n = d\log{d}$

        \begin{theorem}
            For $T$ defined in the card shuffling, let $\Delta(n) = \|\pi_n - \pi\|$, we have
            \[ \Delta(d\log{d} + cd) \le \mathbb{P}[T > d\log{d} + cd] \le e^{-c} \]
            for all $c \ge 0$
        \end{theorem}
        \begin{sketchproof} ~{}
            \begin{enumerate}
                \item The first inequality follows from Lemma \ref{Lemma:StrongStationaryTime}.
                \item To prove the second inequality, we introduce the coupon collector's problem.
                \item Suppose we want to collect $d$ uniforly distributed coupons, the \sout{gachas} boxes we need to open to collect all $d$ coupons have the distribution
                \[ 1 + Geometry((d-1)/d) + \dots + Geometry(1/d) \]
                \item Let $B_i = \{ \text{coupon $i$ does not appear in the first $n$ cereal boxes} \}$
                \item Then $T_i > n$ is the union $\bigcup_{i=1}^dB_i$
                \item $\mathbb{P}[T>n] \le \sum_{i=1}^d\mathbb{P}[B_i] = d\left(1 - \frac{1}{d}\right)^n \le de^{-n/d}$
                \item Take $n = d\log{d} + cd$ yield the result
            \end{enumerate}
        \end{sketchproof}

\chapter{Poison Processes}
\emph{“好想退休哦。”}
\newpage

\section{Poisson Distribution}
    \subsection{Poisson Distibution}
    \begin{definition}[Poisson Distribution]
        $X$ has a \textbf{Poisson distribution} with rate $\lambda$, denoted by $Poisson(\lambda)$, if
        \[ \mathbb{P}[X=k] = \frac{\lambda^k}{k!}e^{-\lambda} \]
    \end{definition}
    \begin{remark}
        If we view $k$ as a constant w.r.t. $n$, then the Poisson distribution can be seen as the limitation of a Bernoulli distribution $Bermoulli(n, p)$ with $p = \frac{\lambda}{n}$, as $n\to\infty$.
    \end{remark}
    \begin{theorem}
        If $n$ is large, then $Bernoulli(n, \lambda/n)$ is approximately $Poisson(\lambda)$.
    \end{theorem}

    \subsection{Expectation and Variance of Poisson Distribution}
    Let $X$ be a random variable with a Poisson distribution, then
    \[ \mathbb{E}[X] = \lambda \]
    \[ Var[X] = \lambda \]

    \subsection{Additivity of Poisson Distributions}
    \begin{theorem}[Additivity of Poisson Distributions]
        Let $X_1$, $X_2$ be two independent random variables, $X_1\sim Poisson(\lambda_1)$, $X_2\sim Poisson(\lambda_2)$, then
        \[ X_1 + X_2 \sim Poisson(\lambda_1 + \lambda_2) \]
    \end{theorem}
    \begin{proof}
        \begin{align*}
            \mathbb{P}[X_1=X_2=n] &=\sum_{m=0}^n \mathbb{P}[X_1=m, X_2=n-m]\\
            &= \sum_{m=0}^n\mathbb{P}[X_1=m]\cdot\mathbb{P}[X_2=n-m]\\
            &= \sum_{m=0}^n\frac{\lambda_1^m}{m!}e^{-\lambda_1}\cdot\frac{\lambda_2^{n-m}}{(n-m)!}e^{-\lambda_2}\\
            &= \frac{e^{-(\lambda_1+\lambda_2)}}{n!}\sum_{m=0}^n\frac{n!}{n!(n-m)!}\lambda_1^m\lambda_2^{n-m}\\
            &= \frac{e^{-(\lambda_1+\lambda_2)}}{n!}\sum_{m=0}^n\mathrm{C}_m^n \lambda_1^m \lambda_2^{n-m}\\
            &= \frac{(\lambda_1+\lambda_2)^n}{n!}e^{-(\lambda_1+\lambda_2)}
        \end{align*}
    \end{proof}


\section{Exponential Distributions}

    \subsection{Definition of Exponential Distributions}
    \begin{definition}[Exponential Distribution]
        A random variable $X$ is said to have an \textbf{exponential distribution} with rate $\lambda$, $X \sim Exp(\lambda)$ if
        \[ \forall t \ge 0 \quad \mathbb{P}[X \le t] = 1 - e^{-\lambda t} \]
    \end{definition}

    \subsection{Properties of Exponential Distribution}

        \subsubsection{Probability Density Function}
        The exponential distribution $Exp(\lambda)$ has probability density function
        \[ p(x) = \lambda e^{-\lambda x} \]

        \subsubsection{Expectation}
        \[ \mathbb{E}[X] = \int_0^{+\infty} t\lambda e^{-\lambda t}\mathrm{d}t = \frac{1}{\lambda} \]

        \subsubsection{Variance}
        \[ Var[X] = \mathbb{E}[X^2] - \mathbb{E}[X] = \frac{1}{\lambda^2} \]

        \subsubsection{Lack of Memory}
        \[ \mathbb{P}[X>t+s|X>s] = \mathbb{P}[X \ge t] \]
        \begin{remark}
            If we have waited for $s$ units of time, then the probability that we have to wait for $t$ more units is the same as that of we have not waited at all.\footnote{白~{}等~{}了}
        \end{remark}

        \subsubsection{Exponential Races}
        Let $X_1 \sim Exp(\lambda_1)$, $X_2 \sim Exp(\lambda_2)$ be two independent random variables. Let $Y = \min(X_1, X_2)$, then
        \[ Y \sim Exp(\lambda_1 + \lambda_2) \]
        \begin{proof}
            \begin{align*}
                \mathbb{P}[Y \ge k] &= \mathbb{P}[\min\{X_1,X_2\} \ge k]\\
                &= \mathbb{P}[X_1 \ge k, X_2 \ge k]\\
                &= \mathbb{P}[X_1 \ge k]\cdot \mathbb{P}[X_2 \ge k]\\
                &= e^{-(\lambda_1 + \lambda_2)k}
            \end{align*}
        \end{proof}

        If we consider the problem of ``Who finishes first between $X_1$ and $X_2$'', then
        \[ \mathbb{P}[Y=X_1]=\sum_{s=0}^\infty \mathbb{P}[X_1=s,X_2>s] = \int_0^{+\infty}p_1(s)\mathbb{P}[X_2>s]\mathrm{d}s = \frac{\lambda_1}{\lambda_1 + \lambda_2} \]


\section{Poisson Processes}

    \subsection{Definition of Poisson Processes}
    \begin{definition}[Poisson Process]\label{Def:PoissonProcess}
        $\{N(s)|s \ge 0\}$ is a \textbf{Poisson Process}, if
        \begin{enumerate}
            \item $N(0)=0$
            \item $\forall t, s \ge 0 \quad N(t+s) - N(s) \sim Poisson(\lambda t)$
            \item $n$ has \textbf{independent increments}: $\forall t_0 \le t_1 \le \dots \le t_n$,
            \[ N(t_1)-N(t_0), N(t_2)-N(t_1), \dots, N(t_n)-N(t_{n-1}) \]
            are \textbf{mutually independent}.
        \end{enumerate}
    \end{definition}

    \subsection{Alternative Interpretation of Poisson Process}
    We can construct a Poisson process as the sum of multiple random variables with Exponential distributions.
    \begin{proposition}\label{Prop:AltDefOfPoissonProcess}
        Let $\tau_1,\tau_2,\dots,\tau_n$ be independent random variables with exponential distribution $Exp(\lambda)$. Let $T_n=\sum_{i=1}^n\tau_i$, $N(s)=\max\{n|T_n \le s\}$

        Then $N(s)$ is a Poisson process.
    \end{proposition}
    \begin{remark}
        If we think of $\tau_i$ as the time interval between the arrival time of customers in a store, then $N(s)$ is the number of customer arrivals before time $s$.
    \end{remark}

    To prove Proposition \ref{Prop:AltDefOfPoissonProcess}, we first introduce a theorem.

    \begin{theorem}\label{Thm:SumOfExpHasGammaDistribution}
        Let $\tau_1, \tau_2, \dots, \tau_n$ be independent random variables with Exponential distribution $Exp(\lambda)$. Let $T_n = \sum_{i=1}^n\tau_n$, then $T_n$ has a \emph{Gamma distribution} $\Gamma(n, \lambda n)$
        \[ f_{T_n}(t) = \lambda e^{-\lambda t}\frac{(\lambda t)^{n-1}}{(n-1)!} \quad \forall t \ge 0 \]
    \end{theorem}
    \begin{proof}
        Prove by induction on $n$.
        \begin{itemize}
            \item \textbf{Base. } $n=1$. $f_{T_1}(t) = \lambda e^{-\lambda t}$ obviouly holds.
            \item \textbf{Hypothesis. } Suppose it holds for $n$.
            \item \textbf{Step. }
            \begin{align*}
                f_{T_{n+1}}(t) &= \int_0^t f_{T_n}(s)\cdot\lambda e^{-\lambda(t-s)}\mathrm{d}s\\
                &= \int_0^t \lambda e^{-\lambda s}\frac{(\lambda s)^{n-1}}{(n-1)!}\lambda e^{-\lambda(t-s)}\mathrm{d}s \quad \text{(Plug in the hypothesis)}\\
                &= \frac{\lambda^{n+1}}{(n-1)!}e^{-\lambda t}\int_0^t s^{n-1}\mathrm{d}s \quad \text{(Move the constants out)}\\
                &= \frac{\lambda^{n+1}}{n!}e^{-\lambda t}t^n
            \end{align*}
            which is exactly $\Gamma(n+1, \lambda(n+1))$, so we are done.
        \end{itemize}
    \end{proof}
    \begin{remark}
        Theorem \ref{Thm:SumOfExpHasGammaDistribution} states that the arrival time of the $n$-th customer has a Gamma distribution.
    \end{remark}

    We can now prove Proposition \ref{Prop:AltDefOfPoissonProcess}.
    \subsection{Proof of Equivalence of the two Definitions}
    \begin{proof}
        To show that the two definitions are equivalent, we need to show that the three requirements in Definition \ref{Def:PoissonProcess} can be satisfied by the alternative definition \ref{Prop:AltDefOfPoissonProcess}.
        \begin{enumerate}
            \item $N(0)=0$ is trivial.
            \item We begin proving (2) of Definition \ref{Def:PoissonProcess} from the case $s=0$. Notice that $N(t)=n \Leftrightarrow T_n \le t \wedge T_{n+1} > t$, i.e. the $n$-th customer has arrived, but the $(n+1)$-th has not. So we only need to prove the latter has Poisson distribution.
            \begin{align*}
                \mathbb{P}[N(t)=n] &= \mathbb{P}[T_n \le t, T_{n+1} > t]\\
                &= \int_0^tf_{T_n}(s)\cdot\mathbb{P}[\tau_{n+1} > t-s]\mathrm{d}s \quad \text{(Enumerate all possible values of $t$)}\\
                &= \int_0^t\lambda e^{-\lambda s}\frac{(\lambda s)^{n-1}}{(n-1)!}\cdot e^{-\lambda(t-s)}\mathrm{d}s \quad \text{(Theorem \ref{Thm:SumOfExpHasGammaDistribution})}\\
                &= \frac{\lambda^n}{(n-1)!}e^{-\lambda t}\int_0^t s^{n-1}\mathrm{d}s \quad \text{(Move out the constants)}\\
                &= \frac{(t\lambda)^n}{n!}\cdot e^{-\lambda t}
            \end{align*}
            which is exactly the expression of $Poisson(\lambda t)$, so we are done.

            When $s > 0$, by the lack of memory property, $N(t+s)-N(s)$ must have the same distribution as $N(t) - N(0) = N(t)$. Of course this can also be verified by some moderate calculations.
            \item (3) of Definitiion \ref{Def:PoissonProcess} can be proved using a similar argument. Notice that the lack-of-memory property and (2) implies that the ``number of arrivals'' after $s$ is independent of the arrivals before $s$. Therefore $N(t_n) - N(t_{n-1})$ is independent of $N(r)$ for all $r < t_{n-1}$. So the result can be proved by induction.
        \end{enumerate}
    \end{proof}


\section{Associating Another R.V. with a Poisson Process}

    \subsection{A Property of Poisson Process}
        We can associate some i.i.d. random variables $Y_i$ with each arrival. For example, $Y_i$ can be the gender of the $i$-th customer arriving at some restaurant; or if customers arrive in cars, $Y_i$ can be the number of passengers in a car. Let $P_j = \mathbb{P}[Y_i = j]$. Let $N_j(t)$ be the total number of $i \le N(t)$ with $Y_i = j$.

        \begin{theorem}\label{Thm:ThinningOfPoissonProcess}
            $N_j(t)$ are \emph{independent} Poisson Processes with rate $\lambda P_j$.
        \end{theorem}
        \begin{proof}
            We will prove the result using the first definition of Poisson Distribution \ref{Def:PoissonProcess}. We only need to consider the simplest cases where $Y_i \in \{0,1\}$.
            \begin{align*}
                \mathbb{P}[N_0(t) = j, N_1(t) = k] &= e^{-\lambda t}\frac{(\lambda t)^{j+k}}{(j+k)!}\cdot C^{j+k}_j P_0^j P_1^k \\
                &= e^{-\lambda t}\frac{(\lambda t)^{j+k}}{(j+k)!}\frac{(j+k)!}{j! \cdot k!}P_0^j P_1^k\\
                &= e^{-P_0\lambda t}\frac{(P_0\lambda t)^j}{j!} \cdot e^{-P_1\lambda t} \frac{(P_1 \lambda t)^k}{k!}
            \end{align*}
            This is exactly the product of two Poisson Processes, with rate $P_0\lambda$ and $P_1\lambda$ respectively. Therefore we have completed our proof.
        \end{proof}
        \begin{remark}
            One tricky fact behind this theorem is that, suppose we are interested in the gender of customers arriving at a restaurant. Assume that the probability that the customer is male or female are equal. If one day 40 males came to the restaurant, this does not give any implication to the number of females coming to the restaurant because they are independent.
        \end{remark}

    \subsection{Examples}
        \subsubsection{Finding typos in a book.} Two editors read a 300-page manuscript. Editor $A$ finds $100$ typos, Editor $B$ finds $120$ typos, and $80$ of these typos are the same. Suppose $A$ and $B$ have probability $P_A$ and $P_B$ of discovering a typo, and suppose the typos in the book is a rate $\lambda$ Poisson Process. How can we estimate $\lambda$, $P_A$ and $P_B$?
    
        We can associate a random variable with each typos appearing in the manuscript.
        \begin{enumerate}
            \item Neither $A$ or $B$ found it. $(1-P_A)(1-P_B)$
            \item Only $A$ found it. $P_A(1-P_B)$
            \item Only $B$ found it. $P_B(1-P_A)$
            \item Both $A$ and $B$ found it. $P_AP_B$
        \end{enumerate}
        From Theorem \ref{Thm:ThinningOfPoissonProcess}, we know that the four cases follow four independent Poisson Processes. So we can estimate the parameters by solving
        \[
        \begin{cases}
            300P_A(1-P_B)\lambda &= 20\\
            300P_B(1-P_B)\lambda &= 40\\
            \lambda P_AP_B &= 80
        \end{cases}
        \quad \Longrightarrow \quad
        \begin{cases}
            P_A &= 2/3\\
            P_B &= 4/5\\
            \lambda &= 1/2
        \end{cases}
        \]
    
    \subsubsection{Coupon Collector Once Again.}
    We revisit the Coupon Collector's problem, but this time we assume the probability of the $j$-th coupon is $P_j$.
    \begin{itemize}
        \item Let $N_j$ be the number of gachas before we first get a coupon of the $j$-th category.
        \item Let $N = \max_{1 \le j \le n}\{N_j\}$ is the total number of gachas we need to collect all categories of coupons.
    \end{itemize}
    We want to find $\mathbb{E}[N]$.

    We re-formulate the problem into the following: suppose we owns a restaurant, and the customers coming to the restaurant is a Poisson Process of rate $\lambda = 1$. Each customer brings a coupon, let $Y_j$ be the category of the coupon brought by the $j$-th customer. Therefore $N_j(t)$ is a Poisson Process with rate $P_j$. Let $X_j$ be the random variable \emph{in this Poisson Process setting} denoting the first time to meet a customer with a category $j$ coupon, and we now alternatively consider $X = \max \{X_j\}$.

    We are interested in $\mathbb{P}[X \le t]$, i.e. the probability that we have got all categories of coupons before time $t$.
    \begin{align*}
        \mathbb{P}[X \le t] &= \mathbb{P}[X_1 \le t, X_2 \le t, \dots X_n \le t]\\
        &= \mathbb{P}[X_1 \le t] \cdot \mathbb{P}[X_2 \le t] \cdot \cdots \cdot \mathbb{P}[X_n \le t]
    \end{align*}
    By the alternative definition of Poisson Process \ref{Prop:AltDefOfPoissonProcess}, $X_j$ has an Exponential Distribution with rate $P_j$. Therefore
    \[ \mathbb{P}[X \le t] = \prod_{i=1}^n (1-e^{-P_jt}) \]
    \[ \mathbb{P}[X > t] = 1 - \prod_{i=1}^n (1-e^{-P_jt}) \]
    \begin{proposition}\label{Prop:AltComputationOfExpectation}
        If $X \ge 0$, then
        \[ \mathbb{E}[X] = \sum_{t=0}^{\infty} \mathbb{P}[X \ge t] = \sum_{i=0}^{\infty}i\cdot\mathbb{P}[X=i] \]
        \[ \mathbb{E}[X] = \int_0^{\infty} \mathbb{P}[X > t]\mathrm{d}t\]
    \end{proposition}
    By Proposition \ref{Prop:AltComputationOfExpectation},
    \[ \mathbb{E}[X] = \int_0^{\infty} \mathbb{P}[X > t]\mathrm{d}t = \int_0^{\infty}1 - \prod_i (1-e^{-P_it})\mathrm{d}t \]
    We have figured out $\mathbb{E}[X]$, but how is $\mathbb{E}[X]$ related to $\mathbb{E}[N]$?

    Notice that $X$ denotes the exact time at which the customer carrying the $N$-th coupon arrives. If $\tau_i$ denotes the time interval between the arrival of the $(i-1)$-th and the $i$-th customers, then
    \[ X = \sum_{i=1}^N \tau_i \]
    Take expectation on both sides
    \[ \mathbb{E}[X] = \mathbb{E}\left[ \sum_{i=1}^N \tau_i \right] = \mathbb{E}[N]\cdot\mathbb{E}[\tau_1] = \mathbb{E}[N] \]
    Notice that $N$ is a random variable, but we somehow still exchanged the summation and expectation in the second equation, this holds because of Wald's Equation, which will be covered in the future. For now, notice that once $N$ is determined, $\mathbb{E}[X|N]$ is determined.
    \[ \mathbb{E}[X|N] = N\mathbb{E}[\tau_1] = N \]
    Take expectations on both sides again
    \[ \mathbb{E}[\mathbb{E}[X|N]] = \mathbb{E}[X] = \mathbb{E}[N] \]
    \begin{remark}
        The original Coupon Collector's problem is a special case where $P_j$ are uniform. It can be proved\footnote{“我也不会积,但是我用Mathematica跑了一下他们确实相等” -- Chihao} that the integral equals to the sum of harmonic series.
    \end{remark}


\section{Conditioning}

    \subsection{Conditioning}
        Let $T_1, T_2, \dots, T_n$ be the arrival times of a Poisson Process with rate $\lambda$. Let $U_1, U_2, \dots, U_n$ be independent uniform random variables on $[0,t]$. Let $V_1, V_2, \dots, V_n$ be $U_i$'s re-arranged in increasing order, then
        \begin{theorem}[Conditioning]\label{Thm:ConditioningOfPoissonProcess}
            If we condition on $N(t)=n$, then the distribution of $T_1, \dots, T_n$ is the same as the distribution of $V_1, \dots, V_n$.
        \end{theorem}
        \begin{sketchproof}~{}
            \begin{itemize}
                \item The joint distribution of $T_1,\dots,T_n$ given $N(t)=n$ is $n!/t^n$ (by brutal force calculation).
                \item The resulting distribution is uniform over $[0,t]$ because the space has volume $t^n$ and $n!$ possible orderings.
            \end{itemize}
        \end{sketchproof}
        \begin{remark}
            This property is useful when computing $\mathbb{P}[N(s)=m|N(t)=n]$
            \[ \mathbb{P}[N(s)=m|N(t)=n] = C^n_m \left(\frac{s}{t}\right)^m\left(1-\frac{s}{t}\right)^{n-m} \]
        \end{remark}


\section{Poisson Approximation}
    \subsection{Motivating Example: m-balls-in-n-bins}\label{subsec:m-balls-in-n-bins}
        Suppose we throw balls randomly into several bins, and we want to know how many balls there are in the bin with the most balls.

        If we consider the number of balls in each bin, the number follows a Bernoulli distribution. However, since the number of balls in different bins are not independent, analyzing the problem in this way can be complicated.

        Alternatively, we can analyze the problem by constructing a Poisson Process.

        \begin{itemize}
            \item Let $X_i$ be the number of balls in the $i$-th bin.
            \item $X_i \sim Bernoulli(m, \frac{1}{n})$.
            \item $\sum_i X_i = m$.
            \item We are interested in $X = \max X_i$.
        \end{itemize}

    \subsection{Poisson Approximation}
        \begin{theorem}[Poisson Approximation]\label{Thm:PoissonApproximation}
            Let $(X_1,\dots,X_n)$ be a sequence of random variables where each $X_i$ has a Bernoulli distribution, then its distribution is the same as $(Y_1, \dots, Y_n)$ conditioned on $\sum_i Y_i = m$, where $Y_i \sim Poisson(\lambda)$ are \emph{independent} Poisson variables with rate $\lambda$.
        \end{theorem}

    \subsection{m-balls-in-n-bins: MaxLoad Revisited}
        With Theorem \ref{Thm:PoissonApproximation}, we can solve the motivating example in Section \ref{subsec:m-balls-in-n-bins}.
        \begin{theorem}
            Let $X = \max_i X_i$. Then there exist constants $c_1$, $c_2$ such that
            \[ \mathbb{P}\left[\frac{c_1\log n}{\log\log n} < X < \frac{c_2\log n}{\log\log n}\right] = 1 - O\left(\frac{1}{n}\right) \]
        \end{theorem}
        \begin{proof}
            We first calculate $\mathbb{P}\left[ X \ge \frac{c_1\log n}{\log\log n} \right]$.
            \begin{align*}
                \mathbb{P}\left[ X \ge \frac{c_1\log n}{\log\log n} \right] &= \mathbb{P}[\exists i: X_i \ge k]\\
                &\le n\mathbb{P}[X_1 \ge k] \quad \text{(Union Bound)}\\
                &\le n C^k_n\left(\frac{1}{n}\right)^k\\
                &\le n\cdot \left(\frac{en}{k}\right)^k n^{-k} \quad \text{(Mafs)}\\
                &= n\cdot\left(\frac{e}{k}\right)^k
            \end{align*}
        \end{proof}
        Using the theorem above, we can derive an important tool.
        \begin{theorem}
            Let $X_i$ be Bernoulli random variables, and let $Y_i$ be independent Poisson random variables with rate $\lambda = m/n$. For any function $f$ from $\mathbb{N}^n$ to $\mathbb{N}$,
            \[ \mathbb{E}[f(X_1, \dots, X_n)] \le e\sqrt{m}\cdot\mathbb{E}[f(Y_1,\dots,Y_m)] \]
        \end{theorem}
        \begin{proof}
            \begin{align*}
                \mathbb{E}[f(Y_1,\dots,Y_n)] &= \sum_{k=0}^{\infty}\mathbb{E}[f(Y_1,\dots,Y_n) | \sum_i Y_i = k] \cdot \mathbb{P}[\sum_i Y_i = k] \quad \text{(Total Probability)}\\
                &\ge \mathbb{E}[f(Y_1,\dots,Y_n)|\sum Y_i = m]\cdot \mathbb{P}[\sum Y_i = m]\\
                &= \mathbb{E}[f(X_1,\dots,X_m)]\cdot\mathbb{P}[\sum Y_i = m] \quad \text{(Thm \ref{Thm:PoissonApproximation})}
            \end{align*}
            If we let each $Y_i$ has a Poisson distribution of rate $m/n$, by additivity of Poisson variables we have $\sum Y_i \sim Poisson(m)$, and then
            \begin{align*}
                \mathbb{E}[f(Y_1,\dots,Y_n)] &= \mathbb{E}[f(X_1,\dots,X_n)]\cdot e^{-m}\frac{m^m}{m!}\\
                &\ge \mathbb{E}[f(X_1,\dots,X_n)]\cdot\frac{1}{e\sqrt{m}}
            \end{align*}
        \end{proof}

\chapter{Continuous Time Markov Chain}
\emph{“为了把连续的情况说清楚它还要发明一堆黑话。”}
\newpage


\section{Definition of Continuous Time Markov Chains}
    \subsection{Continuous Time Markov Chain}
        In the case of continuous time, it is technically difficult to specify the ``conditional probability given all of $X_r$ for all $r<s$''. Therefore instead, the Continuous Time Markov Chain is defined by

        \begin{definition}[Continuous Time Markov Chain]\label{Def:ContinuousTimeMarkovChain}
            A stochastic process $X(t)$ is a \textbf{Continuous Time Markov Chain} if $\forall s,t \ge 0$, $\forall 0 \le s_0 < \cdots < s_n < s$
            \[ \mathbb{P}[X_{s+t} = j | X_s = i, X_{s_n} = i_n,\dots,X_{s_0}=i_0] = \mathbb{P}[X_t=j | X_0 = i] \]
        \end{definition}

        Given the current state, the previous states in the past is irrelevant for predicting the future, so we can simply ``throw away'' the previous states before $s$.

    \subsection{Transition Probability}
        In continuous case, the matrix-multiplication version of multistep transition probability cannot be applied directly. Instead we define a transition probability for each $t>0$.
        \[ p_t(i,j) = \mathbb{P}[X_t=j|X(0)=i] \]

        Recall the Chapman-Kolmogorov Equality \ref{Thm:ChapmanKolmogorovEquality}, it still holds in the continuous case.

        \begin{theorem}[Chapman-Kolmogorov Equality, Continuous Case]\label{Thm:ContinuousChapmanKolmogorovEquality}
            \[ \sum_{k}p_s(i,k)p_t(k,j) = p_{s+t}(i,j) \]
        \end{theorem}
        \begin{remark}
            This sugguests that if we know the transition probability for all $t<t_0$ (for some $t_0>0$), then we will be able to know the transition probability for all $t' \in \mathbb{R}$, by using the equality for sufficiently many times and keep doing the summation until $s+t=t'$.
        \end{remark}

        This further suggests that $p_t$ can be determined by its derivative at $t=0$.
        \begin{definition}[Jump Rate]\label{Def:JumpRate}
            If the limit exists (assume it always does),
            \[ q(i,j) = \lim_{h \to 0}\frac{p_h(i,j)}{h} \quad \text{for $j \neq i$} \]
            then $q(i,j)$ is defined as the \textbf{jump rate} from $i$ to $j$.
        \end{definition}

    \subsection{A VERY Important Example}\label{Sub:CTMCCoreExample}
        Let\footnote{“因为这个例子太重要了所以我给它起了个名字叫Example Star”--Chihao}
        \begin{itemize}
            \item $Y_n$ is a Markov Chain with transition matrix $u(i,j)$.
            \item $N(t)$ be a Poison Process with rate $\lambda$.
            \item $X(t)$ be a random variable defined as $X(t)=Y_{N(t)}$
        \end{itemize}
        That is, the discrete Markov chain $Y_n$ takes a jump according to the transition probability at each new arrival of the Poisson process $N(t)$.

        \subsubsection{Jump Rate of VERY Important Example.}
        By enumerating over all number of arrivals $n$
        \[ p_h(i,j) = \sum_{n=0}^\infty e^{-\lambda h}\frac{(\lambda h)^n}{n!} \cdot u^n(i,j) \]

        Notice that the probability of at least 2 jumps before time $h$ is 1 minus the probability of 0 and 1 jump,
        \[ 1 - (e^{-\lambda h}+\lambda h e^{-\lambda h}) \approx (\lambda h)^2/2! = o(h) \]
        So it converges to 0 when divided by $h$ and as $h \to 0$.
        Therefore
        \[ \frac{p_h(i,j)}{h} \approx \lambda e^{-\lambda h}u(i,j) \to \lambda u(i,j) \]


\section{More Examples}
    \subsection{The Poisson Process}
        Let $X(t)$ be the number of arrivals up to time $t$ in $Poisson(\lambda)$.

        Notice that at each new arrival, $X$ goes from $i$ to $i+1$ (with probability 1).
        
        Therefore $\forall n$
        \[ q(n, n+1) = \lambda \]

    \subsection{M/M/s Queue}
        We now consider a queue. Customers arrive according to a Poisson process of rate $\lambda$; the time to serve a customer is modeled by another Poisson process of rate $\mu$; there are only $s$ counters.

        Since the customers come in rate $\lambda$, we have
        \[ q(n, n+1) = \lambda \]

        Since the customers leave in rate $\mu$, we can use an exponential race in \ref{Subs:ExponentialRace} to model the process, and therefore
        \[ q(n, n-1) = \begin{cases}
            \mu n & \quad n < s\\
            \mu s & \quad n \ge s
        \end{cases} \]


\section{Constructing a CT Markov Chain}
    Given a jump rate $q(i,j)$, let
    \[ \lambda_i = \sum_{j \neq i}q(i,j) \]
    $\lambda_i$ is the rate that the chain leaves $i$.
    \begin{itemize}
        \item If $\lambda_i = \infty$, then the chain leaves $i$ immediately.
        \item If $\lambda_i = 0$, then the chain will never leave $i$.
    \end{itemize}
    If $\lambda_i > 0$, let
    \[ u(i,j) = \frac{q(i,j)}{\lambda_i} \]
    be the probability that the chain goes to $j$ when it leaves $i$.

    \subsection{Informal Construction}
        \begin{itemize}
            \item If $X_t$ is in a state $i$ with $\lambda_i=0$, then the chain never leaves and we are done.
            \item If $X_t$ is in a state $i$ with $\lambda_i > 0$, then we first let the chain stay at $i$ according to an Exponential distribution with rate $\lambda_i$, then choose a destination according to $u(i,j)$.
        \end{itemize}


\section{Computing the Transition Probability}
    Given the jump rate $q(i,j)$, the transition probability can be calculated By
    \begin{align*}
        p_{t+h}(i,j) - p_t(i,j) &= \sum_k p_h(i,k)p_t(k,j) - p_t(i,j) \quad \text{(CK Equation \ref{Thm:ContinuousChapmanKolmogorovEquality})}\\
        &= \sum_{k \neq i} p_h(i,k)p_t(k,j) + (p_h(i,i)-1)p_t(i,j)\\
        &\triangleq A + B
    \end{align*}

    Multiply by $1/h$ and take limitation, then Expression A becomes
    \begin{align*}
        \lim_{h \to 0} \frac{1}{h}A &= \sum_{k \neq i}\lim_{h \to 0}\frac{1}{h}p_h(i,k)p_t(k,j)\\
        &= \sum_{k \neq i}q(i,k)p_t(k,j)
    \end{align*}

    and Expression B becomes
    \begin{align*}
        \lim_{h \to 0} \frac{1}{h}B &= \lim_{h \to 0}\frac{\sum_{k \neq i}p_h(i,k)}{h}p_t(i,j)\\
        &= \left(\sum_{k \neq i}q(i,k)\right)p_t(i,j)\\
        &= -\lambda_i p_t(i,j)
    \end{align*}

    And therefore
    \begin{equation}\label{Eq:TransitionProbabilityKeyEq}
         \lim_{h \to 0}\frac{1}{h}\left(p_{t+h}(i,j) - p_t(i,j)\right) = \sum_{k \neq i}q_(i,k)p_t(k,j) - \lambda_i p_t(i,j) 
    \end{equation}

    Notice that the LHS of equation (\ref{Eq:TransitionProbabilityKeyEq}) is the derivative of $p_t(i,j)$, and therefore
    \[ p'_t(i,j) = \sum_{k \neq i}q_(i,k)p_t(k,j) - \lambda_i p_t(i,j) \]

    Further notice that  hte first term $\sum_{k \neq i}q_(i,k)p_t(k,j)$ in RHS of (\ref{Eq:TransitionProbabilityKeyEq}) can be re-written in a matrix multiplication. Let
    \[ Q(i,j) = \begin{cases}
        q(i,j) &\quad i \neq j\\
        -\lambda_i &\quad i = j
    \end{cases} \]

    And therefore
    \begin{equation}\label{Eq:KolmogorovBackwardEquation}
        P'_t = QP_t
    \end{equation}
    This equation is also known as the \textbf{Kolmogorov Backward Equation}.

    This looks similar to a differential equation and the solution is
    \[ P_t = e^{Qt} = \sum_{n=0}^{\infty}\frac{(tQ)^n}{n!} \]
    where $e^{Qt}$ is defined as
    \[ e^Q = \sum_{n=0}^{\infty} \frac{Q^n}{n!} \]

    Furthermore, if we write $\sum_{k}p_t(i,k)p_h(k,j)$ instead of $\sum_{k}p_h(i,k)p_t(k,j)$ in the first step of derivation, we will end up with the \textbf{Kolmogorov Forward Equation}.
    \begin{equation}\label{Eq:KolmogorovForwardEquation}
        P'_t = P_tQ
    \end{equation}

    \subsection{Example: Poisson Process Revisited}
        In the previous sections, we know that given the jump rate $q(i,j)$, we are able to compute the transition probability and simulate the continuous time Markov chain.

        Let $X(t)$ be the number of arrivals of a Poisson process up to time $t$. $p_t(i,j)$ is the probability that the number rises from $i$ to $j$ in time interval $t$. By definition of the Poisson process, the arrival between $0$ and $t$ has a Poisson distribution of rate $\lambda t$, and $p_t(i,j)$ is simply the probability that the number of arrivals between $0$ and $t$ is $j-i$.
        \[ p_t(i,j) = e^{-\lambda t}\frac{(\lambda t)^{j-i}}{(j-i)!} \]

        We can use the result to verify the forward and backward equation \ref{Eq:KolmogorovForwardEquation} and \ref{Eq:KolmogorovBackwardEquation}.

    \subsection{Example: Two State Markov Chain}
        Suppose there are only two states $\Omega = \{1,2\}$. The two-state continuous-time Markov chain can be specified by
        \[ q(1,2) = \lambda \quad q(2,1) = \mu \]
        and
        \[ Q = \begin{bmatrix}
            -\lambda & \lambda\\
            \mu & -\mu
        \end{bmatrix} \]

        By Kolmogorov backward equation \ref{Eq:KolmogorovBackwardEquation},
        \[
        \begin{bmatrix}
            p'_t(1,1) & p'_t(1,2)\\
            p'_t(2,1) & p'_t(2,2)
        \end{bmatrix} = 
        \begin{bmatrix}
            -\lambda & \lambda\\
            \mu & -\mu
        \end{bmatrix}
        \begin{bmatrix}
            p_t(1,1) & p_t(1,2)\\
            p_t(2,1) & p_t(2,2)
        \end{bmatrix}
        \]

        Since there are only two states, it suffices to compute $p_t(1,1)$ and $p_t(2,1)$ only.
        \[ p'_t(1,1) = -\lambda p_t(1,1) + \lambda p_t(2,1) \]
        \[ p'_t(2,1) = \mu p_t(1,1) - \mu p_t(2,1) \]

        Therefore
        \[ \left(p_t(1,1)-p_t(2,1)\right)' = -(\lambda + \mu)\left(p_t(1,1)-p_t(2,1)\right) \]

        Solving the differential equation gives us
        \[ p_t(1,1) - p_t(2,1) = e^{-(\lambda + \mu)t} \]

        Plug the result back into the expression for $p'_t(1,1)$,
        \[ p'_t(1,1) = -\lambda e^{-(\lambda + \mu)t} \]

        Integrate over $t$
        \[ p_t(1,1) = p_o(1,1) - \lambda\int_0^t e^{-(\lambda + \mu)s}\mathrm{d}s = \frac{\mu}{\mu + \lambda}+\frac{\lambda}{\mu + \lambda} e^{-(\lambda + \mu)t} \]


\section{Limiting Behaviour: Continuous Time}
    \subsection{Irreducibility, Continuous Case}
        The intuition remains the same: the chain can jump from any state $i$ to any state $j$ in finite steps.
        \begin{definition}[Continuous Time Irreducibility]
            A continuous-time Markov chain is \textbf{irreducible} if there exists a sequence $k_0 = i, k_1, \dots, k_n=j$ such that
            \[ q(k_{m-1},k_m) > 0 \quad \forall 1 \le m \le n \]
        \end{definition}

        \begin{lemma}\label{Lem:NecessaryConditionOfIrreducibility}
            If $X(t)$ is irreducible and $t>0$, then
            \[ p_t(i,j) > 0 \quad \forall i,j \]
        \end{lemma}

    \subsection{Stationary Distribution}
        \begin{lemma}[Stationary Distribution, Continuous Case]\label{Lem:ContinuousTimeStationaryDistribution}
            A distribution $\pi$ is a stationary distribution if and only if
            \[ \pi^T Q = 0 \]
        \end{lemma}
        \begin{remark}
            Using the definition of $Q$ we can prove that
            \[ \sum_{i \neq j}\pi(i)q(i,j) = \lambda_j\pi(j) \]
            The LHS is the rate that the chain ``gets into'' state $j$, and the RHS is the rate that the chain ``gets out of'' state $j$. So the intuition of continuous-time stationary distribution is that the rate that a chain goes in and goes out of a state are equal.
        \end{remark}

    \subsection{Convergence}
        \begin{theorem}[Convergence of CT Markov Chain]\label{Thm:ConvergenceOfCTMarkovChain}
            If a continuous time Markov chain is irreducible and has a stationary distribution $\pi$, then
            \[ \lim_{t\to\infty}p_t(i,j) = \pi(j) \]
        \end{theorem}

    \subsection{Detailed Balance Condition, Continuous Case}
        \begin{theorem}[Detailed Balance Condition]\label{Thm:DetailedBalanceConditionOfCTMarkovChain}
            If
            \[ \forall i \neq j \quad \pi(i)q(i,j) = \pi(j)q(j,i) \]
            then $\pi$ is a stationary distribution.
        \end{theorem}


\begin{thebibliography}{9}
    \bibitem{eosp} Durrett, Richard, and R. Durrett.\textit{Essentials of stochastic processes}. Vol. 1. New York: Springer, 1999.
    \bibitem{ITPM} Ross, Sheldon M. \textit{Introduction to probability models}. Academic press, 2014.
    \bibitem{Chang} J.Chang, \textit{Lecture Notes to Stochastic Processes}, 2007
\end{thebibliography}

\end{spacing}
\end{document}