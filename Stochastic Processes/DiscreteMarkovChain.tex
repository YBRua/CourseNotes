\chapter{Discrete Markov Chains}
\emph{“和女朋友在商场走散了,是在原地等碰到的概率大还是随机走碰到的概率大?急,在线等。”}
\newpage


\section{Finite Space Markov Chain}

    \subsection{Basic Definitions}
    % State Space
    \begin{definition}[State Space]
        A \textbf{state space} $\mathcal{S}$ is a finite or countable set of states, i.e. the values that random variables $X_i$ may take on.
    \end{definition}
    % Initial Distribution
    \begin{definition}[Initial Distribution]
        The \textbf{initial distribution} $\pi_0$ is the probability distribution of the Markov Chain at time $0$. Denote $\mathbb{P}[X_0 = i]$ by $\pi_0(i)$.
    \end{definition}
    \begin{remark}
        Formally, $\pi_0$ is a function from $\mathcal{S}$ to $[0,1]$ s.t.
        \[ \pi_0(i) \ge 0 \text{ for all $i\in\mathcal{S}$} \]
        \[ \sum_{i\in\mathcal{S}} \pi_0(i) = 1\]
    \end{remark}
    % Probability Transition Matrix
    \begin{definition}[Probability Transition Matrix]
        The \textbf{Transition Matrix} is a matrix $P = (p_{ij})$, where
        \[ p_{ij} = \mathbb{P}[X_{n+1} = j | X_n = i] \]
        i.e. the probability given that the chain is at state $i$ at $T=n$ that jumps to $j$ at $T=n+1$
    \end{definition}
    \begin{remark}
        ~{}
        \begin{itemize}
            \item The \emph{rows} of $P$ sum up to $1$.
            \item The entries in $P$ are all non-negative.
        \end{itemize}
    \end{remark}

    %%%%%%%% Markov Property
    \subsection{The Markov Property}
    % Markov Property
    \begin{definition}[Markov Property]
        We say a stochastic process $X_1, X_2, \dots $ satisfies the \textbf{Markov property} if 
        \[ \mathbb{P}[X_{n+1} = i_{n+1} | X_n = i_n, \dots, X_0 = i_0] = \mathbb{P}[X_{n+1} = i_{n+1} | X_n = i_n] \]
    \end{definition}
    That is, the \emph{next} state $X_{n+1}$ depends only on the \emph{current} state $X_n$.
    % Time-homogeneity
    \begin{definition}[Time-homogeneous Markov Chain]
        A Markov Chain is said to be \textbf{time homogeneous} if
        \[ \forall t \quad \mathbb{P}[X_{t+1} = j | X_t = i] = P(i,j) \]
    \end{definition}
    For now, we only consider \emph{time homogeneous} Markov Chains.

    %%%%%%%% Multistep Transition by Matrix Algebra
    \subsection{Matrix Interpretation of Markov Chains}
    We now computes the probability distribution at $T=n+1$, denoted by $\pi_{n+1}$.
    \[ \pi_{n+1}(j) = \mathbb{P}[X_{n+1} = j] = \sum_{i=1}^N \mathbb{P}[X_n=i]\mathbb{P}[X_{n+1} = j | X_n = i] = \sum_{i=i}^N \pi_n(i)P(i,j)\]
    Therefore
    \[ \pi_{n+1}^T = \pi_n^T P \]
    \[ \pi_n^T = \pi_0^T P^n \]
    We use $P(i,j)$ to denote the element $(i,j)$ of $P$, we use $P^n$ to denote the $n$-th power of $P$, and we assume that $\pi_i$'s are column vectors.
    % Chapman-Kolmogorov
    \begin{theorem}[Chapman-Kolmogorov Equality]\label{Thm:ChapmanKolmogorovEquality}
        \[ P^{m+n}(i,j) = \sum_k P^m(i,k)P^n(k,j) \]
    \end{theorem}


\section{Stationary Distribution}

    % Stationary Distribution
    \begin{definition}[Stationary Distribution]
        $\pi$ is called a \textbf{stationary distribution} of a Markov Chain if
        \[ \pi^TP = \pi \]
    \end{definition}
    \begin{remark}
        A Markov Chain may have 0, 1 or infinitely many stationary distribution.
    \end{remark}

    \emph{Here comes the question: When does a stationary distribution exist? If it exists, is it unique? If it is unique, does the chain converges to it?}


\section{Irreducibility, Aperiodicity and Recurrence}
For convenience, we will use $\mathbb{P}_i[A]$ to denote $\mathbb{P}[A | X_0 = i]$, and use $\mathbb{E}_i$ to denote expectation in an analogous way.

    %%%%%%%% Irreducibility
    \subsection{Irreducibility}
    % Accessible
    \begin{definition}[Accessibility]
        Let $i$, $j$ be two states, we say $j$ is \textbf{accessible from} $i$ if it is possible (with positive probability) for the chain to ever visit $j$ if the chain starts from $i$.
        \[ \mathbb{P}_i[\bigcup_{n=0}^\infty\{X_n = j\}] > 0 \]
        or equivalently
        \[ \sum_{n=0}^\infty P^n(i,j) = \sum_{n=0}^\infty \mathbb{P}_i[X_n = j] > 0 \]
    \end{definition}
    % Communicate
    \begin{definition}[Communication]
        We say $i$ \textbf{communicates with} $j$ if $j$ is accessible from $i$ and $i$ is accessible from $j$.
    \end{definition}
    % Irreducible
    \begin{definition}[Irreducibility]
        We say a Markov Chain is \textbf{irreducible} if all pairs of states communicate. And it is reducible otherwise.
    \end{definition}
    The relation \emph{communicate with} is an equivalent relation, and irreducible simply means the number of equivalent classes is 1.

    %%%%%%%% Aperiodicity
    \subsection{Aperiodicity}
    % Period
    \begin{definition}[Period]
        Given a Markov Chain, its \textbf{period} of state $i$ is defined to be the greatest common divisor $d_i$ of the lengths of loops starting from $i$.
        \[ d_i = \gcd\{n|P^n(i,i) > 0\} \]
    \end{definition}
    % Period is a class property
    \begin{theorem}
        If states $i$ and $j$ communicate, then $d_i = d_j$
    \end{theorem}
    \begin{sketchproof}
        ~{}
        \begin{itemize}
            \item $P^{n_1}(i,j) > 0$ and $P^{n_2}(j,i) > 0$.
            \item $P^{n_1+n_2}(i,i)>0 \Rightarrow d_i | n_1 + n_2$.
            \item Suppose $P^n{j,j} > 0$, then $P^{n+n_1+n_2}(i,i) > 0 \Rightarrow d_i | n + n_1 + n_2$.
            \item $d_i | n \Rightarrow d_j \ge d_i$.
            \item Similarly $d_i \ge d_j$. We are done.
        \end{itemize}
    \end{sketchproof}
    \begin{remark}
        Therefore all states in a communicating class have the same period, and all states in an irreducible Markov chain have the same period.
    \end{remark}
    % Aperiodic
    \begin{definition}[Aperiodicity]
        An irreducible Markov chain is said to be \textbf{aperiodic} if its period is 1, and periodic otherwise.
    \end{definition}
    \begin{proposition}
        \normalfont
        If $P(i,i) > 0$, then the Markov Chain is aperiodic.
    \end{proposition}
    \begin{remark}
        This is a sufficient but not necessary condition.
    \end{remark}

    %%%%%%%% Recurrence
    \subsection{Recurrence}
    We define the \textbf{First Hitting Time} $T_i$ of the state $i$ by
    \[ T_i = \inf\{n>0 | X_n = i\} \]
    and we can define recurrence as follows
    % Recurrent
    \begin{definition}[Recurrence]
        The state $i$ is \textbf{recurrent} if $\mathbb{P}_i[T_i<\infty] = 1$, and is transient if it is not recurrent.
    \end{definition}
    \begin{remark}
        Recurrence means that starting from state $i$ at $T=0$, the chain \emph{is sure to} return to $i$ eventually.
    \end{remark}
    % Recurrence is a class property
    \begin{theorem}
        \normalfont
        Let $i$ be a recurrent state, and let $j$ be accessible from $i$, then all of the following hold:
        \begin{enumerate}
            \item $\mathbb{P}_i[T_j < \infty] = 1$.
            \item $\mathbb{P}_j[T_i < \infty] = 1$.
            \item The state $j$ is recurrent.
        \end{enumerate}
    \end{theorem}
    \begin{sketchproof}
        ~{}
        \begin{itemize}
            \item The paths starting from $i$ can be seen as infinitely many \emph{cycles}.
            \item Whether the chain visits $p$ in the cycles can be seen as a Bernoulli distribution with probability $p>0$.
            \item The probability of not visiting $j$ in the first $n$ cycles is $(i-p)^n$, which goes to $0$ as $n\to\infty$. So (1)hold.
            \item (2) can be proved by contradiction. $\mathbb{P}_j[T_i < \infty] < 1$ will lead to contradiction against the fact that $i$ is recurrent.
            \item (1)(2) implies (3).
        \end{itemize}
    \end{sketchproof}
    \begin{corollary}
        If $\mathbb{P}_i[T_j < \infty] > 0$ but $\mathbb{P}_j[T_i < \infty] < 1$, then $i$ is transient.
    \end{corollary}
    % Equivalent Statement of Recurrence
    \begin{theorem}[Equivalent Statement of Recurrence]
        \normalfont
        The state $i$ is recurrent if and only if $\mathbb{E}_i[N_i] = \infty$, where $N_i = \sum_{i=0}^\infty\mathbb{I}\{X_n = i\}$.
    \end{theorem}
    \begin{sketchproof}
        ~{}
        \begin{itemize}
            \item Recurrence $\Rightarrow \mathbb{P}_i[N_i = \infty] = 1 \Rightarrow \mathbb{E}_i[N_i] = \infty$.
            \item The converse is proved by contradiction.
            \item If $i$ is transient, there is a chance of $p$ that the chain never return to $i$.
            \item So $N_i$ is distributed geometrically, and $\mathbb{E}_i[N_i]$ will be finite. Contradiction.
        \end{itemize}
    \end{sketchproof}
    \begin{remark}
        By taking expectation on $N_i$, we have:
        \[ \mathbb{E}_i[N_j] = \sum_{i=0}^\infty P^n(i,j) \]
    \end{remark}

    \begin{corollary}\label{TransientStateGoesToZero}
        If $j$ is transient, then $\lim_{n\to\infty}P^n(i,j) = 0$ for all states $i$.
    \end{corollary}
    \begin{sketchproof}
        ~{}
        \begin{itemize}
            \item $\mathbb{E}_j[N_j] < \infty$.
            \item $\mathbb{E}_i[N_j] = \mathbb{P}_i[T_j < \infty]\mathbb{E}_i[N_j | T_j < \infty]$.
            \item $\mathbb{E}_i[N_j] \le \mathbb{E}_i[N_j | T_j] = \mathbb{E}_j[N_j] < \infty$ since the probability \emph{restarts} once the chain visits $j$ again.
            \item $\mathbb{E}_i[N_j] = \sum_{i=0}^\infty P^n(i,j) < \infty$ and this implies our conclusion.
        \end{itemize}
    \end{sketchproof}

    \begin{corollary}
        If $i$ is recurrent, then $\sum_{n=1}^{\infty}p^n(i,i) = \infty$;

        If $i$ is transient, then $\sum_{n=1}^{\infty}p^n(i,i) < \infty$.
    \end{corollary}

    \begin{proposition}
        Suppose a Markov Chain has a stationary distribution $\pi$, if the state $j$ is transient, then $\pi(j) = 0$.
    \end{proposition}
    The last proposition follows from Corollary \ref{TransientStateGoesToZero}.

    \begin{corollary}
        If an irreducible Markov Chain has a stationary distribution, then the chain is recurrent.
    \end{corollary}
    \begin{sketchproof}
        The chain cannot be transient, or otherwise all $\pi(j)$ would be 0 and the sum of $\pi$ does not equal to 1.
    \end{sketchproof}
    \begin{remark}
        The converse is not true!
    \end{remark}

    \begin{proposition}
        A drunk man will find his way home, but a drunk bird may get lost forever.
    \end{proposition}
    This is because the random walk on $\mathbb{Z}$ and $\mathbb{Z}^2$ is recurrent, while the walks on higher dimensions are transient.

    %%%%%%%% More on Recurrence
    \subsection{More on Recurrence}
    % Null Recurrence
    \begin{definition}[Null Recurrence]
        The state $i$ is \textbf{null recurrent} if it is recurrent and $\mathbb{E}_i[T_i] = \infty$.
    \end{definition}
    % Positive Recurrence
    \begin{definition}[Positive Recurrence]
        The state $i$ is \textbf{positive reccurent} it is recurrent and $\mathbb{E}_i[T_i] < \infty$.
    \end{definition}

    \begin{proposition}
        Given an irreducible Markov Chain, it is either transient, null recurrent or positive recurrent.
    \end{proposition}



\section{Strong Law of Large Numbers of Markov Chains}
% SLLN of Markov Chains
\begin{theorem}[SLLN of Markov Chains]\label{SLLN}
    Let $X_0, X_1, \dots$ be a Markov Chain starting in the state $X_0=i$. Suppose state $i$ communicates with state $j$. The limiting fraction of time that the chain spends in $j$ is $\frac{1}{\mathbb{E_j}[T_j]}$.
    \[ \mathbb{P}_i[\lim_{n\to\infty}\frac{1}{n}\sum_{t=1}^n\mathbb{I}\{X_t=j\} = \frac{1}{\mathbb{E_j}[T_j]}] = 1 \]
\end{theorem}
\begin{sketchproof}
    Read the book. I'm not going to write this because I am lazy.
\end{sketchproof}


\section{Basic Limit Theorem}\label{BasicLimitTheorem}
aka. Foundamental Theorem of Markov Chains

    %%%%%%%% Finite Case
    \subsection{Finite State Space}
    % Spectral Radius
    \begin{definition}[Spectral Radius]
        Given a non-negative matrix $A$, the spectral radius $\rho(A)$ is the maximum norm of its eigenvalues
        \[ \rho(A) = \max\{ \lambda(A) \} \]
    \end{definition}
    \begin{proposition}
        Let $A$ be a non-negative matrix, then
        \[ \min_{1\le i \le n} \sum_{j=1}^N a_{i,j} \le \rho(A) \le \max_{1\le i \le n} \sum_{j=1}^N a_{i,j} \]
    \end{proposition}
    % Perron-Frobenius Theorem
    \begin{lemma}[Perron-Frobenius Theorem]\label{PFT}
        Let $A$ be a non-negative matrix with spectral radius $\rho(A) = \alpha$, then $\alpha$ is an eigenvalue of $A$, and has both left and right non-negative eigenvectors.
    \end{lemma}
    \begin{remark}
        Lemma \ref{PFT} implies that for a finite probability transition matrix $P$, it always has at least one stationary distribution, because it always has eigenvalue $1$ ($P\mathbf{1} = \mathbf{1}$) and a corresponding eigenvector $\pi$ s.t. $\pi^TP = \pi$.
    \end{remark}

    \begin{lemma}
        Suppose a $k \times k$ matrix $P$ is irreducible. Then there exists a unique solution to $\pi P = \pi$.
    \end{lemma}
    % Foundamental Theorem, Finite Case
    \begin{theorem}[Foundamental Theorem, Finite Case] If a finite Markov Chain is \emph{irreducible} and \emph{aperiodic}, then it has a \emph{unique stationary distribution} $\pi$ any initial distribution \emph{converges} to it:
        \[ \forall \pi_0,\quad \lim_{t\to\infty}\pi_0^T P^t = \pi^T \]
    \end{theorem}

    %%%%%%%% Countable Case
    \subsection{Countable Case}
    % Basic Limit Theorem
    \begin{theorem}[Basic Limit Theorem]
        Let $X_1, X_2, \dots$ be an \emph{irreducible aperiodic} Markov Chain that has a \emph{stationary distribution} $\pi$. Then
        \[ \lim_{n\to\infty}\pi_n(i) = \pi(i) \]
        for any state $i$ and for any initial distribution $\pi_0$.
    \end{theorem}
    % Bounded Convergence Theorem
    \begin{lemma}[Bounded Convergence Theorem]\label{BCT}
        If $X_n \to X$ with probability $1$ and there is a finite number $b$ such that $|X_n| < b$ for all $n$, then
        \[ \mathbb{E}[X_n] \to \mathbb{E}[X] \]
    \end{lemma}
    % Corollary of SLLN
    Recall SLLN \ref{SLLN}, and the following is a corollary.
    \begin{corollary}\label{SLLNCoro}
        For an \emph{irreducible} Markov Chain, we have
        \[ \lim_{n\to\infty} \frac{1}{n}\sum_{t=1}^n P^t(i,j) = \frac{1}{\mathbb{E}_j[T_j]} \]
    \end{corollary}
    \begin{sketchproof}
        Take expectations on both sides of Theorem \ref{SLLN} yields the conclusion.
    \end{sketchproof}
    % Cesaro Average
    \begin{proposition}[Cesaro Average]
        If a sequence of numbers $a_n$ converges to a value $a$, then the \textbf{Cesaro Average} $(1/n)\sum_{t=1}^na_t$ also converges to it.
    \end{proposition}
    Corollary \ref{CesaroCoro} follows immediately from the proposition,
    \begin{corollary}\label{CesaroCoro}
        For an \emph{irreducible aperiodic} Markov Chain with a \emph{stationary distribution},
        \[ \lim_{n\to\infty}\frac{1}{n}\sum_{t=1}^nP^t(i,j) \to \pi(j) \]
    \end{corollary}
    \begin{theorem}
        An \emph{irreducible, aperiodic} Markov Chain with a \emph{stationary distribution} has a stationary distribution given by
        \[ \pi(j) = \frac{1}{\mathbb{E}_j[T_j]} \]
    \end{theorem}
    \begin{sketchproof}
        The proof of the theorem is trivial: compare Corollary \ref{SLLNCoro} and \ref{CesaroCoro}.
    \end{sketchproof}
    % Foundamental Theorem, Countable Case
    \begin{theorem}[Foundamental Theorem, Countable Case]\label{theorem:FTMC-Countable}
        An \emph{irreducible} Markov Chain has a \emph{unique stationary distribution} given by
        \[ \pi(j) = \frac{1}{\mathbb{E}_j[T_j]} \]
        if and only if it is \emph{positive recurrent}.
    \end{theorem}

    %%%%%%%% Other Conclusions
    \subsection{Other Conclusions}
    \emph{Since they are not covered in class, no proof will be provided. I'm lazy you know.}
    \begin{definition}[Doubly Stochastic Chains]
        A transition matrix $P$ is said to be \textbf{doubly stochastic} if all of its columns sum up to 1.
    \end{definition}
    \begin{optTheorem}
        For a doubly stochastic Markov Chain with $N$ states, the uniform distribution $\pi(i) = 1/N$ is a stationary distribution.
    \end{optTheorem}
    \begin{optTheorem}
        For a Markov Chain with symmetric $P$ and $N$ states, the uniform distribution $\pi(i) = 1/N$ is a stationary distribution.
    \end{optTheorem}
    \begin{optTheorem}
        Let $T_j^k = \min\{ n > T_y^{k-1} | X_n = y \}$ be the time of the $k$-th visit to $j$, then by the Markov property,
        \[ \mathbb{T}_x[T_y^k < \infty] = \mathbb{P}_x[T_y < \infty]\cdot\mathbb{P}_y[T_y<\infty] \]
        Notice that
        \[ \mathbb{E}[X] = \sum_{k=0}^{\infty}\mathbb{P}[X \le k] \]
        Using the two equations, we have
        \[ \mathbb{E}_x[N_y] = \frac{\mathbb{P}_x[T_y < \infty]}{1-\mathbb{P}_y[T_y < \infty]} \]
    \end{optTheorem}
    \begin{remark}
        This theorem gives us more insights into the results in Section \ref{BasicLimitTheorem}.
    \end{remark}



\section{Examples}

    %%%%%%%% Galton-Watson Process
    \subsection{Galton-Watson Process}\label{Sub:BranchingProcess}
    \emph{aka. Branching Process}
    \paragraph*{Problem.} What is the probability that a family name eventually extincts?
    \paragraph*{Notations.}
    \begin{itemize}
        \item $G_t$ denotes the number of males in generation $t$.
        \item $X_{tk}$ denotes the number of sons fathered by the $k$-th father in the $t$-th generation. Assume $X_i$'s are \emph{iid} with probability mass function $p(\cdot)$.
        \item $\rho = \mathbb{P}[extinction] = \mathbb{P}[\cup_{k\ge 1}\{ G_k=0 \}]$ denotes the probability that the family name eventually goes extinct.
        \item $q_t = \mathbb{P}[G_t = 0]$ denotes the probability that the family name goes extinct at the $t$-th generation.
    \end{itemize}
    The problem is trivial when $p(0) = 0$ or $p(0) + p(1) = 1$. Either the family name never goes extinct or it goes extinct almost surely. We consider $p(0) > 0$ and $p(0) + p(1) < 1$.

    By conditioning on what happens at the first step, we can calculate $\rho$ by
    \[ G_{t+1} = \sum_{i=1}^{G_t} X_{ti} \]
    \begin{align}
    \rho &= \sum_{k=0}^{\infty}\mathbb{P}[extinction \wedge  G_1 = k] \notag \\
    &= \sum_{k=0}^{\infty}\mathbb{P}[extinction | G_1 = k]\cdot\mathbb{P}[G_1 = k | G_0 = 1] \quad \text{(by 全概率公式)} \notag \\
    &= \sum_{k=0}^{\infty}f(k)\rho^k \triangleq \psi(\rho) \notag
    \end{align}
    The last step used the fact that all males have sons independently. $\psi(\rho)$ is called the \textbf{Probability Generating Function} of $p(\cdot)$.

    The probability of eventual extinction satisfies $\psi(\rho) = \rho$.
    \[ \psi'(z) = \sum_{k=1}^{\infty}kp(k)z^{k-1} > 0\]
    \[ \psi''(z) = \sum_{k=2}^{\infty}k(k-1)p(k)z^{k-2} > 0 \]
    So $\psi(z)$ is a strictly increasing convex function with the following properties.
    \begin{itemize}
        \item $\psi(1) = 1$. So $\rho = 1$ is always a solution.
        \item $\psi'(1) = \sum_{k=1}^{\infty}kp(k) = \mathbb{E}[X] \triangleq \mu$
    \end{itemize}
    \begin{enumerate}[(i)]
        \item If $\mu \le 1$, $\rho = 1$ is the only solution. Therefore the family name will go extinct.
        \item If $\mu > 1$, there exists another solution $r<1$.
        
        Observe that $\{ G_t = 0 \} \subseteq \{ G_{t+1} = 0 \}$ and $q_t \le t_{t+1}$. Since $q_t$ has a upper bound $\rho$, it will always converge.
        \[ q_t \uparrow \rho \]
        So we only need to prove $q_t < r \quad \forall t$. By induction,
        \begin{itemize}
            \item[base.] $q_0 = 0 < r$.
            \item[hypo.] $q_t < r$.
            \item[step.] $q_{t+1} = \sum_{i=0}^{\infty}\mathbb{P}[G_1 = i]\mathbb{P}[G_{t+1} = 0| G_1 = i] = \sum_{i=0}^{\infty}p(i)(q_t)^i$. The last step used the iid assuption of $X$. 
            Therefore $q_{t+1} = \psi(q_t) \le \psi(r) = r$. Done.
        \end{itemize} 
    \end{enumerate}

    %%%%%%%% Gambler's Ruin
    \subsection{Gambler's Ruin}\label{GamblerRuin}
    \paragraph*{Problem.} Consider a gambler in a casino, who has probability $p$ to win \$$1$ and $1-p$ to lose \$$1$. The process stops when the gambler gets \$$N$ or goes to $0$. Assume the gambler starts the game with \$$i$.
    \paragraph*{Notations.}
    \begin{itemize}
        \item $P_i$ denotes the probability of the gambler gets \$$N$ and wins, starting with \$$i$.
        \item $Z_i \in \{1, -1\}$ denotes whether the gambler wins or loses in round $i$.
        \[ X_t = X_0 + \sum_{i=0}^{t-1}Z_i \]
    \end{itemize}
    Obviously,
    \[ P_N = 1 \quad P_0 = 0 \]
    When $1 \le i \le N-1$, $P_i$ can be calculated by
    \begin{align*}
        P_i &= \mathbb{P}[win | X_0 = i] \\
        &= \mathbb{P}[win \wedge Z_0 = 1 | X_0 = i] + \mathbb{P}[win \wedge Z_0 = -1 | X_0 = i] \quad \text{(Again by 全概率公式 on $Z_i$)} \\
        &= \mathbb{P}[win | X_0=i, Z_0 = 1]\cdot\mathbb{P}[Z_0=1|X_0=i] + \mathbb{P}[win|X_0=1, Z_0=-1]\cdot\mathbb{P}[Z_0=-1|X_0=i] \\
        &= p\cdot P_{i+1} + (1-p)\cdot P_{i-1}
    \end{align*}
    Rearranging,
    \[ p(P_{i+1} - P_i) = (1-p)(P_i - P_{i-1}) \]
    \[ P_{i+1} -P_i = \frac{1-p}{p}(P_i - P_{i-1}) \]
    Let $\theta = \frac{1-p}{p}$, by high school mafs
    \[ P_{i+1} - P_i = \theta^i(P_1 - P_{0}) \]
    \begin{enumerate}[(i)]
        \item Assume for now that $p \neq \frac{1}{2}$ so $\theta \neq 1$, then summing over $i$ yields
        \[ P_N - P_0 = 1 = \frac{1-\theta^N}{1-\theta} P_1 \]
        Therefore
        \[ P_1 = \frac{1-\theta}{1-\theta^N} \]
        Summing from $0$ to $i$ yields
        \[ P_i = \frac{1-\theta^i}{1-\theta^N} \]
        \item If $p=\frac{1}{2}$,
        \[ P_{i+1} - P_i = P_i - P_{i-1} \]
        This is a arithmetic progress, and again by high school mafs
        \[ P_i = \frac{i}{N} \]
    \end{enumerate}
    To sum up
    \[ 
        P_i = 
        \begin{cases}
            \frac{i}{N} &(i = \frac{1}{2})\\
            \frac{1-\theta^i}{1-\theta^N} &(\text{o.w.})
        \end{cases}    
    \]

    \subsection{Drug Test}\label{DrugTest}
    This example is based on results from the previous subsection \ref{GamblerRuin}.
    \paragraph*{Problem.} We want to test the cure rate $p_1$ of a drug \textrm{Drug}1. We already have a \textrm{Drug}2 with known cure rate $p_2$. We want to know whether $p_1 > p_2$ or not. To do this, we find $t$ pairs of patients $(X_i, Y_i)$ and conduct tests using the two drugs on $X$ and $Y$ respectively. Once the number of patients who are cured by \textrm{Drug}1 but not \textrm{Drug}2 exceeds a certain threshold $M$, we can claim that $p_1 \ge p_2$.
    \paragraph*{Notations.}
    \begin{itemize}
        \item $X_i,Y_i \in \{0,1\}$ is a boolean denoting whether the first or the second drug cured patient $i$.
        \item $Z_i = X_i - Y_i$.
        \item $p_1, p_2$ denotes the probability that the two drugs cure a patient, respectively.
    \end{itemize}

    The value of $Z_i$ falls into 3 cases:
    \[ Z_i = 
        \begin{cases}
            1 &p_1(1-p_2)\\
            -1 &p_2(1-p_1)\\
            0 &(\text{o.w.})
        \end{cases}
    \]
    If we ignore the cases where $Z_i = 0$, then we can model this problem as a gambler's ruin, with $p = \frac{p_1(1-p_2)}{p_1(1-p_2)+p_2(1-p_1)}$.

    And
    \[ \mathbb{P}[TestWrong] = 1 - \frac{1-\theta^M}{1-\theta^{2M}} = \frac{1}{\theta^{-M} + 1} \]
    The probability above drops exponentially with $M$, so the threshold does not need to be very large to achieve accurate results.

    %%%%%%%% Another random walk on N
    \subsection{Another Random Walk}\label{AnotherRandWalk}
    \paragraph*{Problem.} Consider a random walk on $\mathbb{N}$ where
    \[ P(0,1) = 1 \quad P(N,N-1) = 1 \]
    We want to compute how many steps we need to reach $N$ starting from $i$.
    \paragraph*{Notations.}
    \begin{itemize}
        \item $h_i$ denotes the number of steps to reach $N$ starting from $X_0=i$.
        \item $Y_i$ denotes the number of steps from $i$ to $i+1$ for the first time.
        \item $g_j \triangleq \mathbb{E}[Y_j]$.
    \end{itemize}
    Obviously
    \[ \mathbb{E}[h_0] = 1 + \mathbb{E}[h_1] \quad \mathbb{E}[h_N] = 0 \]
    Similar to subsection \ref{DrugTest},
    \[ h_i = 1+ (1-p)h_{i-1} + ph_{i+1} \]
    This can be calculated using \emph{the linearity of expectation}. 

    Notice that
    \[ h_i = \sum_{j=i}^{N-1}Y_j \]
    Taking expectations on both sides
    \[ \mathbb{E}[h_i] = \sum_{j=i}^{N-1}\mathbb{E}[Y_j] \]
    So we only need to caculate $\mathbb{E}[Y_j]$.
    \begin{itemize}
        \item $g_0 = 1$.
        \item $g_i = 1 + 0 \cdot p + (1-p)(g_{i-1}+g_i)$.
    \end{itemize}
    \begin{enumerate}
        \item Again assume $p \neq \frac{1}{2}$. 
        Rearranging yields
        \[ g_i = \frac{1}{p} + \theta g_{i-1} \]
        and after a few steps of arithmetics
        \[ g_i = \sum_{i=0}^{t-1}\frac{1}{p}\theta^i + \theta^t \]
        Summing over $g_i$ is a sum of geometric progress, easy.
        \item If $p = \frac{1}{2}$, then
        \[ g_{i} - g_{i-1} = \frac{1}{p} = 2 \]
        so
        \[ g_t = 2t + 1 \]
        and
        \[ \mathbb{E}[h_0] = \sum_{t=0}^{N-1}(2t+1) = N^2 \]
    \end{enumerate}
    \begin{remark}
        As the converse of the conclusion, if we take $N$ steps, the farthest distance we can go is $\sqrt{N}$.
    \end{remark}



\section{Coupling and Stochastical Dominance}
In this section we assume the sample space $\Omega$ is at most countable, but the results in this section can be generalized to continuous case.

    %%%%%%%% Motivating Example: 2-SAT
    \subsection{Motivating Example: 2-SAT}
        Recall the SAT problem in \textsc{AI2615 Design and Analysis of Algorithms}. A 2-SAT is a special case of SAT where each \verb|or| expression has at most 2 terms
        \[ \varphi = (x_1 \vee y_1) \wedge (x_2 \vee y_2) \cdots \]
        The problem is RP and can be solved using the following algorithm.

        \begin{algorithm}
            \caption{2-SAT Solver}
            \KwIn{CNF with $n$ terms}
            Random intialize a solution $\sigma$.\\
            \For{$i = 1:100n^2$}
            {
                \If{$\sigma$ satisfies CNF}
                {
                    \KwRet{$\sigma$}
                }
                \Else
                {
                    Randomly choose one term $(X_i \vee Y_i)$ that is false.\\
                    Randomly flip $X_i$ or $Y_i$.
                }
            }
            \KwRet{Not satisfiable}
            \label{2SATAlgo}
        \end{algorithm}

        To prove its correctness, we only need to consider the cases where the CNF is indeed satisfiable with solution $\sigma$. We denote the sequence of attempts produced by the algorithm by
        \[ \sigma_0 \to \sigma_1 \to \cdots \to \sigma_{100n^2} \]
        Let $X_i$ denotes the number of terms in $\sigma$ and $\sigma_i$ that are exactly the same.

        You should convince yourself that $X_0, X_1, \dots$ is not a Markov Chain. \sout{So we get stuck}. However we can still informally show the correctness.

        We first introduce some basic concepts.

    %%%%%%%% Stochastic Dominance
    \subsection{Stochastic Dominance}
        \begin{definition}[Stochastic Dominance]
            A distribution $\mu$ is said to \textbf{stochasitcally dominate} $\nu$ if
            \[ \forall x\quad \mathbb{P}_{X\sim\mu}[X \ge x] \ge \mathbb{P}_{Y\sim\nu}[Y \ge x] \]
        \end{definition}
        \begin{remark}
            Let $F$ and $G$ be the cumulative distribution function of $X$ and $Y$, we have $F(x) \le G(x)$, i.e. the probability of $X$ concentrates on larger values.
        \end{remark}
        \paragraph{Some Examples.}
        \begin{itemize}
            \item 2-SAT.
            \item Binomial distribution. $B(n,p)$ and $B(n,q)$.
            \item Erd$\mathrm{\ddot{o}}$s-R$\mathrm{\acute{e}}$nyi Random Graph $\mathcal{G}(n,p)$ is a graph with $n$ nodes and the edges have a probability of $p$ to exist. We sample $G$ randomly from $\mathcal{G}$ and test if $G$ is connected. If $p>q$ then $G_p$ stocahstically dominates $G_q$.
        \end{itemize}

    %%%%%%%% Coupling
    \subsection{Coupling}
        \begin{definition}[Coupling]
            $(X',Y')$ is the coupling of random variables $(X,Y)$ if $X'$ has the same distribution of $X$ and $Y'$ has the same of $Y$. i.e. A coupling $C$ of $X$ and $Y$ is a joint distribution of $X$ and $Y$.
        \end{definition}
        \begin{remark}~{}
            \begin{itemize}
                \item The coupling of $X$ and $Y$ is not unique.
                \item If $X$ has distribution $\mu$ and $Y$ has distribution $\nu$, then any joint distribution $(X,Y)$ having marginal distribution $X\sim\mu$ and $Y\sim\nu$ is a coupling of $X$ and $Y$.
                \[ \forall x, \quad \mathbb{P}_{(X,Y) \sim C}[X=x] = \mu(x) \]
                \[ \forall y, \quad \mathbb{P}_{(X,Y) \sim C}[Y=y] = \nu(y) \]
                \item Constructing a coupling can be intuitively seen as a process of filling a table of the joint distribution of $X$ and $Y$.
            \end{itemize}
        \end{remark}
        \begin{definition}[Monotone Coupling]
            Let $C$ be a coupling of $X$ and $Y$. $C$ is a \textbf{monotone coupling} if
            \[ \mathbb{P}_{(X,Y)\sim C}[X\ge Y] = 1\footnote{“它总是大”——Prof.Zhang} \]
        \end{definition}
        \begin{theorem}\label{MonoCouplingAndDominance}
            There is a monotone coupling of $X$ and $Y$ \emph{if and only if} $X$ stochastically doniminates $Y$.
        \end{theorem}
        \begin{proof}
            ~{}\\
            $\Rightarrow$.
            By definition of coupling
            \[ \mathbb{P}_{Y\sim\nu}[Y \ge a] = \mathbb{P}_{(X,Y)\sim C}[Y\ge a] \]
            By the Law of Total Probability
            \[ \mathbb{P}_{Y\sim\nu}[Y \ge a] = \mathbb{P}_{(X,Y)\sim C}[Y \ge a \wedge X \ge Y]\mathbb{P}[X \ge Y] + \mathbb{P}_{(X,Y)\sim C}[Y \ge a \wedge X < Y]\mathbb{P}[X < Y] \]
            The latter term is $0$ by definition of monotone coupling, so
            \[ \mathbb{P}_{Y\sim\nu}[Y \ge a] = \mathbb{P}_{(X,Y)\sim C}[X \ge Y \ge a] \le \mathbb{P}_{(X,Y)\sim C}[X \ge a] = \mathbb{P}_{X\sim\mu}[X \ge a] \]

            $\Leftarrow$. Suppose $X$, $Y$ are two random variables having  distribution $F(x)$ and $G(x)$ respectively. By stochastic dominance we have $F(x) \le G(x)$ for any $x$. Let $U$ be a random variable having a uniform distribution over $(0,1)$, since $F(x)$ is increasing
            \[ F^{-1}(U) \le x \Leftrightarrow F(F^{-1}(U)) \le F(x) \]
            Therefore
            \[ \mathbb{P}[F^{-1}(U) \le x] = \mathbb{P}[F(F^{-1}(U)) \le F(x)] = \mathbb{P}[U \le F(x)] = F(x) \]
            Similarly
            \[ \mathbb{P}[G^{-1}(U) \le x] = G(x) \]
            Since $F(x) \le G(x)$, we have $F^{-1}(x) \ge G^{-1}(x)$ and therefore we obtained a coupling $C = (X' = F^{-1}(U), Y' = G^{-1}(U))$ for which $X' \ge Y'$.
        \end{proof}

    %%%%%%%% 2-SAT Revisited
    \subsection{2-SAT: Proof of Correctness}
        \begin{theorem}[Markov's Inequality]\label{MarkovIneq}
            Let $X$ be a nonnegative random varialbe and let $a > 0$, then
            \[ \mathbb{P}[X \ge a] \le \frac{\mathbb{E}[X]}{a} \]
        \end{theorem}
        We can now prove the correctness of Algorithm \ref{2SATAlgo}.
        \begin{sketchproof}
            Note that $X_{i+1}$ differs from $X_i$ in at most 1 operand, and that
            \[ \mathbb{P}[X_{i+1} = X_i + 1] \ge \frac{1}{2} \]
            \[ \mathbb{P}[X_{i+1} = X_i - 1] \le \frac{1}{2} \]
            We can introduce a random walk described in subsection \ref{AnotherRandWalk} $Y_1, Y_2, \dots$ with $p = \frac{1}{2}$.

            $X_i$ stochatically dominates $Y_i$. This is intuitive, and can be proved using Theorem \ref{MonoCouplingAndDominance} and constructing a monotone coupling:

            We introduce a random variable $U$ that has a uniform distribution $U \sim U(0,1)$. Let $q$ be the probability of $X_{i+1} = X_i + 1$, from the arguments above we already have $p < q$. Then we construct a coupling of $X$ and $Y$ by
            \[ X_i = \mathbb{I}[U_i < q] \quad Y_i = \mathbb{I}[U_i < p] \]
            Since $p < q$, clearly $Y_i \le X_i$, and thus
            \[ X_0 + X_1 + \dots + X_i \ge Y_0 + Y_1 + \dots + Y_i \]
            So we have constructed a monotone coupling, and it follows that $X$ stocahstically dominates $Y$.
            
            Since $Y_i$ has an expectation of $n^2$ to get to state $n$, it follows intuitively that $X_i$ has a smaller expectation of steps to get to $n$.

            We have chosen the max iteration to be $100n^2$, by Markov's Inequality \ref{MarkovIneq}, the probability that we take $100n^2$ iterations without finding a feasible solution is at most $1/100$.
        \end{sketchproof}
        \begin{remark}~{}
            \begin{itemize}
                \item \sout{A formal proof of correctness will be given in the next lecture.}
                \item The rigorous proof has been updated.
                \item The idea of coupling can be used to prove other examples of stochastic dominance.
            \end{itemize}
        \end{remark}

    %%%%%%%% Maximum Couplings
    \subsection{Maximum Couplings}
        \begin{definition}[Maximum Coupling]
            Suppose $X$ has distribution function $\mu$, $Y$ has distribution function $\nu$. Let $C$ denote the set of all couplings of $(X,Y)$. $(\hat{X},\hat{Y})$ is a \textbf{maximum $\mu$,$\nu$ couple} if
            \[ \mathbb{P}[\hat{X}=\hat{Y}] = \max_{(X,Y)\in C}\mathbb{P}[X=Y] \]
            i.e. among all such couples its components are most likely to be equal.
        \end{definition}
        \begin{proposition}
            A maximum coupling always exists.
            \[ \mathbb{P}[\hat{X} = \hat{Y}] = \sum_{z\in\Omega}m(z) \]
            where $m(z) = \min\{ \mu(z), \nu(z) \}$.
        \end{proposition}
        \begin{proof}
            Let $p = \sum_{z\in\Omega m(z)}$. Let $A = \{x| \mu(x) > \nu(x)\}$.
            
            Notice that
            \[ m(x) = 
                \begin{cases}
                    \mu(x) \quad &(x \in A)\\
                    \nu(x) \quad &(x \notin A)
                \end{cases}
            \]
            
            \begin{align*}
                \mathbb{P}[X=Y] &= \mathbb{P}[X=Y\in A] + \mathbb{P}[X = Y \notin A]\\
                &\le \mathbb{P}[Y \in A] + \mathbb{P}[X \notin A]\\
                &= \sum_{z\in A}\nu(z) + \sum_{z\notin A}\mu(z)\\
                &= \sum_{z\in A}m(z) + \sum_{z \notin A}m(z)\\
                &= p
            \end{align*}
        \end{proof}
        \begin{remark}
            For continuous cases, the sum becomes integral.
        \end{remark}

    %%%%%%%% Total Variance Distance
    \subsection{Total Variance Distance}
        \begin{definition}[Total Variation Distance]
            Given two distributions $\mu$ and $\nu$ defined on a sample space $\Omega$, the \textbf{total variance distance} of $\mu$ and $\nu$ is defined as
            \[ \|\mu - \nu \|_{TV} \triangleq \frac{1}{2}\sum_{x\in\Omega}|\mu(x)-\nu(x)| = \max_{A\subseteq\Omega} \mu(A)-\nu(A) = 1-\min_{z\in\Omega}\{ \mu(z), \nu(z) \} \]
            Geometrically, this is the half of size of the area between the pdf of $\mu$ and $\nu$.
        \end{definition}
        \begin{lemma}[The Coupling Lemma]\label{CouplingLemma}
            For any coupling $C$ of $\mu$ and $\nu$,
            \[ \mathbb{P}_{(X,Y)\sim C}[X \neq Y] \ge \|\mu - \nu \|_{TV} \]
            and there exists a coupling $C^{*}$ that achieves the equality.
        \end{lemma}
        \begin{proof}
            If we view the coupling as a way to fill the table, then $\mathbb{P}_{(X,Y) \sim C}[X=Y]$ is the sum of all values on the diagonal of the table. Intuitively, the value on $i$-th row and $j$-th column is upper bounded by $\mu(i)$ and $\nu(j)$, so the sum is upper bounded by $\sum_{z\in\Omega}\min\{\mu(z), \nu(z)\}$.
            \begin{align*}
                \mathbb{P}[X \neq Y] &= 1 - \mathbb{P}_{(X,Y) \sim C}[X = Y] \\
                &= 1 - \sum_{z \in \Omega}\mathbb{P}[X=Y=z] \\
                &\ge 1 - \sum_{z \in \Omega}\min \{\mu(z), \nu(z)\}\\
                &= \sum_{z \in \Omega}\left( \mu(z) - \min\{\mu(z),\nu(z)\} \right)\\
                &= \| \mu - \nu \|_{TV}
            \end{align*}
            The previous part showed that $\mathbb{P}_{(X,Y)\sim C}[X \neq Y]$ is always lower bounded by $\|\mu - \nu\|_{TV}$, we further show that the equality is achieved by the maximum coupling. Let $A = \{x| \mu(x) > \nu(x)\}$
            \begin{align*}
                \|\mu - \nu\|_{TV} &= \mu(A) - \nu(A) = 1 - \mu(A^{c}) - \nu(A)\\
                &= 1 - \sum_{z \notin A}\mu(z) - \sum_{z \in A}\nu(z)\\
                &= 1 - \sum_{z \notin A}m(z) - \sum_{z \in A}m(z)\\
                &= 1 - \mathbb{P}[X=Y]\\
                &= \mathbb{P}[X \neq Y]
            \end{align*}
        \end{proof}
        \begin{remark}
            The $C^*$ here is the maximal coupling.
        \end{remark}



\section{Proof of the Foundamental Theorem}
    We are going to prove that:
    \paragraph{}
        \emph{An irreducible, aperiodic, positive recurrent Markov Chain has a unique stationary distribution, and it converges to the distribution.}

    \begin{proof}
        We only consider the finite case. The proof of Theorem \ref{theorem:FTMC-Countable} has already shown that an Irreducible Positive Recurrent Markov chain has a Unique Stationary Distribution, so we only prove the convergence part here. Suppose there are two Markov Chains $X$ and $Y$
        \[ X_0 \to X_1 \to \dots \to X_t \to \dots \]
        \[ Y_0 \to Y_1 \to \dots \to Y_t \to \dots \]
        Let the initial distribution of $Y$ be the stationary distribution $\pi$, so each $Y_i$ has distribution $\pi$. Let the initial distribution of $X$ be $\mu_0$, and $X_i$ has the distribution given by $\mu_{i} = P^T\mu_{i-1}$.

        We define \textbf{convergence} by $\lim_{i\to\infty}\|\mu_{i} - \pi\|_{TV} = 0$.

        For each $(X_t, Y_t)$, we construct a coupling $C_t$ such that each chain is runned independently, and once $X_t = Y_t$ for some $t$, then $X_{t'} = Y_{t'}$ for all $t' \ge t$.

        Irreducibility guarantees that 
        \[ \forall i,j \quad \exists n \quad P^n(i,j) > 0 \]
        And Aperiodicity further guarantees that
        \begin{equation}\label{FTMCEq1}
            \exists n \quad \forall i,j \quad P^n(i,j) > 0
        \end{equation}
        To prove (\ref{FTMCEq1}), we are going to show that
        \begin{equation}\label{FTMCEq2}
            \forall i,j \quad \exists t_{ij} \quad \forall t>t_{i,j} \quad P^t(i,j) > 0
        \end{equation}
        Then (\ref{FTMCEq1}) can be proved by setting $n = \max_{i,j}\{t_{ij}\}$. Note that this result only holds in finite case.

        We visualize the Markov Chain from $i$ to $j$ as many cycles starting from and ending at $i$, of length $c_1,c_2,\dots$. Since the chain is Aperiodic,
        \[ \gcd(c_1, c_2,\dots,c_s) = 1 \]
        And by Lemma \ref{theorem:Bezout}, there exist integers $x_1,x_2,\dots,x_s$ such that
        \[ c_1x_1 + c_2x_2 + \dots + c_sx_s = 1 \]
        And it follows immediately that there exist $y_1,y_2,\dots,y_s$ such that
        \[ c_1y_1 + c_2y_2 + \dots + c_sy_x = b \]
        for all sufficiently large $b$.

        \begin{lemma}[B$\mathrm{\acute{e}}$zout's Theorem]\label{theorem:Bezout}
            Let $a,b$ be two integers, then there exist integer $\mu, \nu$ such that
            \[ a\mu + b\nu = \gcd(a,b) \]
        \end{lemma}

        Now consider the coupling.
        \[ \mathbb{P}[X_n = Y_n] \ge \mathbb{P}[X_n=Y_n=x] \ge SomeConstant~{} \alpha > 0 \]
        This is guaranteed by the irreducibility of the chain.
        So
        \[ \mathbb{P}[X_n \neq Y_n] \le 1 - \alpha \]
        Now condider the $2n$-th step, by the Law of Total Probability,
        \[ \mathbb{P}[X_{2n} \neq Y_{2n}] = \mathbb{P}[X_{2n} \neq Y_{2n} \wedge X_n \neq Y_n] + \mathbb{P}[X_{2n} \neq Y_{2n} \wedge X_n = Y_n] \]
        The latter is $0$ because it is impossible by our construction of coupling.
        
        By conditional probability
        \[ \mathbb{P}[X_{2n} \neq Y_{2n} \wedge X_n \neq Y_n] \le \mathbb{P}[X_{2n} \neq Y_{2n} | X_n \neq Y_n](1-\alpha) \le (1-\alpha)^2 \]

        It follows immediately that $\mathbb{P}[X_t \neq Y_t] \to 0$ as $t \to \infty$. So $\lim_{t\to\infty}\|\mu_t - \pi\|_{TV} = 0$, and we are done.
    \end{proof}


\section{Applications of Markov Chain}
    We assume the state space $\mathcal{S}$ is finite, but in most cases the applications can be generalized to countably infinite cases. And we assume that the Markov chains are Aperiodic and Irreducible.

    \subsection{Time Reversibility}
        \begin{definition}[Time Reversibility]
            A Markov chain is said to be time reversible if there exists a distribution $\pi$ such that
            \begin{equation}\label{Def:DetailedBalanceCondition}
                \forall x,y \in \mathcal{S} \quad \pi(x)P(x,y) = \pi(y)P(y,x)
            \end{equation}
        \end{definition}
        \begin{remark} ~{}
            \begin{itemize}
                \item The $\pi$ here is actually the stationay distribution. because
                \[ (\pi^TP)(y) = \sum_{x\in\mathcal{S}}\pi(x)P(x,y) = \sum_{x\in\mathcal{S}}\pi(y)P(y,x) = \pi(y)\sum_{x\in\mathcal{S}}P(y,x) = \pi(y) \]
                \item (\ref{Def:DetailedBalanceCondition}) is also called the \textbf{detailed balance condition}.
                \item The detailed balance condition implies that $(X_0,X_1,\dots,X_n)$ and $X_n,X_{n-1},\dots,X_0$ have \emph{the same distribution}, and this is the reason why such chains are called time-reversible.
            \end{itemize}
        \end{remark}

    \subsection{Metropolis Algorithm}
        If we want to sample from some distribution $\mu$, one way of doing this is to design a random walk (Markov Chain) with stationary distribution $\mu$ and run the random walk for a sufficiently long period of time. We now discuss how to design such random walks. Recall the following proposition:
        \begin{proposition}
            A random walk on graph $\mathcal{G}$ with probability transition matrix
            \[P(i,j) = 
            \begin{cases}
                \frac{1}{d_i} \quad &j \in \mathcal{N}(i)\\
                0 \quad &j otherwise
            \end{cases}
            \]
            has stationary distribution
            \[ \pi(i) = \frac{d_i}{\sum_{j\in\mathcal{S}}d_j} \]
        \end{proposition}
        Consider a random walk on graph $\mathcal{G} = (V, E)$. Given a distribution $\mu$, we want to design a random walk such that the stationay distribution is $\mu$. Let $d_i$ denote the degree of the $i$-th node in $\mathcal{G}$.\\
        Let
        \[ \Delta = \max_{i \in V} d_i \]
        For any node $i$, for all $j \in \mathcal{N}(i)$, we ``propose'' to move to $j$ with probability $1/\Delta$. And $j$ ``accepts'' $i$ with probability $\min \{\frac{\mu(j)}{\mu(i)}, 1\}$. If the move is rejected, we stay in $i$. We define the probability transition matrix as follows
        \begin{equation}\notag
            P(i, j) = 
            \begin{cases}
                0 \quad & j \notin \mathcal{N}(i)\\
                \frac{1}{\Delta}\min\{1, \frac{\mu(j)}{\mu(i)}\} \quad &j \in \mathcal{N}(i)\\
                1 - \sum_{k\in\mathcal{N}(i)} P(i,k) \quad & i = j
            \end{cases}
        \end{equation}
        \begin{proposition}
            The $P$ defined above satisfies
            \[ \mu P = \mu \]
            i.e. it is a Markov chain with stationay distribution $\pi$.
        \end{proposition}
        \begin{sketchproof}~{}
            \begin{itemize}
                \item Trivial if $j \notin \mathcal{N}(i)$.
                \item If $j \in \mathcal{N}(i)$. WLOG assume $\mu(j) \ge \mu(i)$.
                \[ \mu(i)P(i,j) = \mu(i)\frac{1}{\Delta}\min\{1,\frac{\mu(j)}{\mu(i)}\} = \frac{\mu(i)}{\Delta} \]
                and
                \[ \mu(j)P(j,i) = \mu(j)\frac{1}{\Delta}\min\{1,\frac{\mu(i)}{\mu(j)}\} = \frac{\mu(i)}{\Delta} \]
                So we are done.
            \end{itemize}
        \end{sketchproof}
        \begin{remark}
            If we run the algorithm sufficiently long, the sequence we get will have a distribution that is approximately the same as $\mu$. What's more, in this algorithm, we only need to know $\frac{\mu(j)}{\mu(i)}$, instead of knowing $\mu$. And this is useful in practice.
        \end{remark}

    \subsection{Simulated Annealing}
        \par Simulated Annealing is a stochastic optimization algorithm. Given a set $\mathcal{S}$, and a function $w(x)$. We want to minimize (or maximize) $w(x)$.
        \paragraph{Notations.}
        \begin{itemize}
            \item $\mathcal{S}^* = \{x\in\mathcal{S}|w(x) = \min_{y\in\mathcal{S}} w(y)\}$: the set of global minimizers of $w(x)$.
            \item $\mu^*$: a uniform distribution on $\mathcal{S}^*$.
            \item $T$: The ``Temperature'' parameter.
        \end{itemize}
        \par We define a distribution on $\mathcal{S}$, and we want to make the probability concentrate on $x$ where $w(x)$ is small. However sampling from $\mathcal{S}$ is usually infeasible because $\mathcal{S}$ is too large.
        \par Instead, we introduce a parameter $T$, and let
        \[ \mu_T(x) \sim e^{-\frac{w(x)}{T}} \]
        \begin{proposition}
            As $T \to 0$, $\mu_T(x) \to \mu^*(x)$.
        \end{proposition}
        \par However, if we start with a very small $T$, we may get stuck in a local extremum. So we start with a large $T_1$ and gradually decrease $T$.
        \begin{theorem}
            If $T_1,T_2,\dots$ satisify
            \begin{enumerate}
                \item $\{T_i\} \to 0$.
                \item $T_i$ decreases slow enough. (``Slow'' means that the sum of a certain series defined by $\mathcal{S}$ diverges, but it is so complicated that it is beyond the scope.)
            \end{enumerate}
            Then let $P^{(n)} \triangleq \prod_{k=1}^n P_{T_k}$.
            \[ \forall \mu \quad \mu^TP^{(n)} \to \mu^* \]
        \end{theorem}
        That is, as long as $T$ decreases slowly enough, we only need to run each temperature $T_i$ \emph{for one step with the Metropolis algorithm}.


\section{Convergence Analysis of Markov Chains}
    By designing a proper coupling $C$, we can upper bound the convergence speed of a Markov Chain.

    \subsection{Mixing Time}
        \begin{definition}[Mixing Time]\label{def:MixingTime}
            The \textbf{mixing time} $\tau_{mix}(\varepsilon)$ is defined as
            \[ \tau_{mix}(\varepsilon) = \max_{a} \min_{t} \| \mu_t^a - \pi \|_{TV} \le \varepsilon \]
            It represents the minimal time for $\mu$ to converge to $\pi$ starting from the worst initial distribution $\mu^a$.
        \end{definition}

    \subsection{Some Examples}

        \subsubsection{Random Walk on Hypercube.}
            A \textbf{hypercube} is a $\{0,1\}$ string $\{0,1\}^n$, and there exists an edge between $x$ and $y$ if and only if $\sum_{i=1}^n|x(i)-y(i)| = 1$, i.e. $x$ and $y$ differs in only one bit.
            Let
            \[
                X_{t+1} = 
                \begin{cases}
                    X_t \quad & w.p.\frac{1}{2}\\
                    \textit{Move into one of the neighbours} & w.p.\frac{1}{2}
                \end{cases}
            \]
            i.e. we have probability $1/2$ to stay still and probability $1/2$ to move into a neighbour.

            If we start from the worst case, then $\mathbb{P}[X_t \neq Y_t]$ can give an upper bound of $\tau_{mix}$.

            We construct the coupling as follows. Suppose we are at $X_t$,
            \begin{enumerate}
                \item Uniformly pick $i \in n$.
                \item Uniformly pick $c \in \{0, 1\}$.
                \item Set the $i$-th bit of $X_t$ to $c$ and yield $X_{t+1}$.
            \end{enumerate}
            We choose \emph{the same} $i$ and $c$ for both $X_t$ and $Y_t$.

            It can be proved that if we run $n\log n + cn$ iterations, the probability that $X_t \neq Y_t$ is at most $e^{-c}$. i.e.
            If
            \[ t \ge n\log n + cn \]
            Then
            \[ \mathbb{P}[X_t \ne Y_t] \le e^{-c} = \varepsilon \]
            Therefore 
            \[ \tau_{mix} \le n\log n + n\log \frac{1}{\varepsilon} \le \varepsilon\]
            So we need approximately at most $n\log n$ steps.

        \subsubsection{Coupon Collection}
            Given a set of $n$ coupons, we pick $X$ times. And we want to know the total time we need to collect all $n$ coupons.

            Let $X_i$ denotes the number of gachas needed to get the $i$-th coupon.
            \[ \mathbb{E}[X] = \mathbb{E}[\sum_{i=1}^n X_i] = \sum_{i=i}^n \mathbb{E}[X_i] \]
            Notice that $X_i$ has a geometric distribution with parameter $\frac{n-i+1}{n}$. Therefore
            \[ \mathbb{E}[X] = \sum_{i=1}^n \frac{n}{n-i+1} = n\sum_{i=1}^n \frac{1}{i} \approx n(\log n + \gamma) \]

        \subsubsection{Card Shuffling}
            Suppose we shuffle a deck of 52 cards using a ``top-in-at-random'' method: we pick the card on the top and insert it into one of 52 positions (including the top) at random. This Markov chain has the stationary distribution $\pi = Uniform(\mathcal{S}_{52})$. By FTMC, $\|\pi_n - pi\|_{TV} \to 0$ as $n \to \infty$.

            It is possible to find a random variable $T$ which denotes the time at which the deck becomes exactly uniformly distributed. We first introduce some notations.

            \begin{itemize}
                \item $U_i$ is uniformly distributed on $1,2,\dots,52$ representing that the $i$-th shuffle moves the card on the top to position $U_i$.
                \item $T_1 = \inf\{ n: U_n = 52 \}$ denotes the first time a card is moved below card 52.
                \item $T_2 = \inf\{ n>T_1: U_n \ge 51 \}$ denotes the second time a card is moved below card 52.
                \item $T_{51} = \inf\{ n>T_50: U_n \ge 2 \}$ denotes the 51st time a card is moved below card 52.
            \end{itemize}

            Then $T = T_{52} = T_{51} + 1$.

            Clearly we have
            \begin{itemize}
                \item $T_1 \sim Geometry(1/51)$
                \item $T_2 - T_1 \sim Geometry(2/52)$
                \item \dots
                \item $T_{52} - T_{51} \sim Geometry(52/52) = 1$
            \end{itemize}

            We can calculate $\mathbb{E}[T]$. By linearity of Expectation,
            \[ \mathbb{E}[T] = \mathbb{E}[T_1] + \mathbb{E}[T_2-T_1] + \dots + \mathbb{E}[T_52 - T_51] \approx 52\log 52 \]
            Similarly, if we are shuffling $d$ cards, we have
            \[ \mathbb{E}[T] = d\log{d} \]

    \subsection{Strong Stationary Times}
        \begin{definition}[Stopping Time]
            The time $T$ is called the \textbf{stopping time} if for each $n$, one can tell whether $T=n$ by just looking at $X_0, X_1, \dots, X_n$, i.e. it is not necessary to know any other future values of $X_{n+1}, X_{n+2}, \dots$.
        \end{definition}
        \begin{remark}
            In the card-shuffling example, $T$ is a stopping time.
        \end{remark}
        \begin{definition}[Strong Stationary Time]
            A random variable $T$ is called a \textbf{strong stationary time} if it satisfies
            \begin{enumerate}
                \item $T$ is a stopping time
                \item $X_T$ is distributed as $\pi$
                \item $X_T$ is independent of $T$
            \end{enumerate}
        \end{definition}
        \begin{lemma}\label{Lemma:StrongStationaryTime}
            If $T$ is a strong stationary time for the Markov Chain $\{X_n\}$, then $\|\pi_n - \pi\| \le \mathbb{P}[T > n]$ for all $n$.
        \end{lemma}

    \subsection{Threshold Pheonomenon}
        In card shuffling and many other Markov Chains, $\|\pi_n - \pi\|$ stays close to $1$ for a while, and suddenly drop to nearly $0$ at approximately $n = d\log{d}$

        \begin{theorem}
            For $T$ defined in the card shuffling, let $\Delta(n) = \|\pi_n - \pi\|$, we have
            \[ \Delta(d\log{d} + cd) \le \mathbb{P}[T > d\log{d} + cd] \le e^{-c} \]
            for all $c \ge 0$
        \end{theorem}
        \begin{sketchproof} ~{}
            \begin{enumerate}
                \item The first inequality follows from Lemma \ref{Lemma:StrongStationaryTime}.
                \item To prove the second inequality, we introduce the coupon collector's problem.
                \item Suppose we want to collect $d$ uniforly distributed coupons, the \sout{gachas} boxes we need to open to collect all $d$ coupons have the distribution
                \[ 1 + Geometry((d-1)/d) + \dots + Geometry(1/d) \]
                \item Let $B_i = \{ \text{coupon $i$ does not appear in the first $n$ cereal boxes} \}$
                \item Then $T_i > n$ is the union $\bigcup_{i=1}^dB_i$
                \item $\mathbb{P}[T>n] \le \sum_{i=1}^d\mathbb{P}[B_i] = d\left(1 - \frac{1}{d}\right)^n \le de^{-n/d}$
                \item Take $n = d\log{d} + cd$ yield the result
            \end{enumerate}
        \end{sketchproof}
