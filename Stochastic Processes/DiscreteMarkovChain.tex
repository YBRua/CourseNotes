\chapter{Discrete Markov Chains}
\emph{“和女朋友在商场走散了,是在原地等碰到的概率大还是随机走碰到的概率大?急,在线等。”}
\newpage

%%%%%%%%%%%%%%%%%%%%%%%%%%%%%%%%%%%%%%%%%%%
%%%%%%%% Finite Space Markov Chain %%%%%%%%
%%%%%%%%%%%%%%%%%%%%%%%%%%%%%%%%%%%%%%%%%%%
\section{Finite Space Markov Chain}

%%%%%%%% Basic Definitions
\subsection{Basic Definitions}
% State Space
\begin{definition}[State Space]
    A \textbf{state space} $\mathcal{S}$ is a finite or countable set of states, i.e. the values that random variables $X_i$ may take on.
\end{definition}
% Initial Distribution
\begin{definition}[Initial Distribution]
    The \textbf{initial distribution} $\pi_0$ is the probability distribution of the Markov Chain at time $0$. Denote $\mathbb{P}[X_0 = i]$ by $\pi_0(i)$.
\end{definition}
\begin{remark}
    Formally, $\pi_0$ is a function from $\mathcal{S}$ to $[0,1]$ s.t.
    \[ \pi_0(i) \ge 0 \text{ for all $i\in\mathcal{S}$} \]
    \[ \sum_{i\in\mathcal{S}} \pi_0(i) = 1\]
\end{remark}
% Probability Transition Matrix
\begin{definition}[Probability Transition Matrix]
    The \textbf{Transition Matrix} is a matrix $P = (p_{ij})$, where
    \[ p_{ij} = \mathbb{P}[X_{n+1} = j | X_n = i] \]
    i.e. the probability given that the chain is at state $i$ at $T=n$ that jumps to $j$ at $T=n+1$
\end{definition}
\begin{remark}
    ~{}
    \begin{itemize}
        \item The \emph{rows} of $P$ sum up to $1$.
        \item The entries in $P$ are all non-negative.
    \end{itemize}
\end{remark}

%%%%%%%% Markov Property
\subsection{The Markov Property}
% Markov Property
\begin{definition}[Markov Property]
    We say a stochastic process $X_1, X_2, \dots $ satisfies the \textbf{Markov property} if 
    \[ \mathbb{P}[X_{n+1} = i_{n+1} | X_n = i_n, \dots, X_0 = i_0] = \mathbb{P}[X_{n+1} = i_{n+1} | X_n = i_n] \]
\end{definition}
That is, the \emph{next} state $X_{n+1}$ depends only on the \emph{current} state $X_n$.
% Time-homogeneity
\begin{definition}[Time-homogeneous Markov Chain]
    A Markov Chain is said to be \textbf{time homogeneous} if
    \[ \forall t \quad \mathbb{P}[X_{t+1} = j | X_t = i] = P(i,j) \]
\end{definition}
For now, we only consider \emph{time homogeneous} Markov Chains.

%%%%%%%% Multistep Transition by Matrix Algebra
\subsection{Matrix Interpretation of Markov Chains}
We now computes the probability distribution at $T=n+1$, denoted by $\pi_{n+1}$.
\[ \pi_{n+1}(j) = \mathbb{P}[X_{n+1} = j] = \sum_{i=1}^N \mathbb{P}[X_n=i]\mathbb{P}[X_{n+1} = j | X_n = i] = \sum_{i=i}^N \pi_n(i)P(i,j)\]
Therefore
\[ \pi_{n+1}^T = \pi_n^T P \]
\[ \pi_n^T = \pi_0^T P^n \]
We use $P(i,j)$ to denote the element $(i,j)$ of $P$, we use $P^n$ to denote the $n$-th power of $P$, and we assume that $\pi_i$'s are column vectors.
% Chapman-Kolmogorov
\begin{theorem}[Chapman-Kolmogorov Equality]
    \[ P^{m+n}(i,j) = \sum_k P^m(i,k)P^n(k,j) \]
\end{theorem}

%%%%%%%%%%%%%%%%%%%%%%%%%%%%%%%%%%%%%%%%%
%%%%%%%% Stationary Distribution %%%%%%%%
%%%%%%%%%%%%%%%%%%%%%%%%%%%%%%%%%%%%%%%%%
\section{Stationary Distribution}
% Stationary Distribution
\begin{definition}[Stationary Distribution]
    $\pi$ is called a \textbf{stationary distribution} of a Markov Chain if
    \[ \pi^TP = \pi \]
\end{definition}
\begin{remark}
    A Markov Chain may have 0, 1 or infinitely many stationary distribution.
\end{remark}

\emph{Here comes the question: When does a stationary distribution exist? If it exists, is it unique? If it is unique, does the chain converges to it?}

%%%%%%%%%%%%%%%%%%%%%%%%%%%%%%%%%%%%%%%%%%%%%%%%%%%%%%%%%%%%%
%%%%%%%% Irreducibility, Aperiodicity and Recurrence %%%%%%%%
%%%%%%%%%%%%%%%%%%%%%%%%%%%%%%%%%%%%%%%%%%%%%%%%%%%%%%%%%%%%%
\section{Irreducibility, Aperiodicity and Recurrence}
For convenience, we will use $\mathbb{P}_i[A]$ to denote $\mathbb{P}[A | X_0 = i]$, and use $\mathbb{E}_i$ to denote expectation in an analogous way.

%%%%%%%% Irreducibility
\subsection{Irreducibility}
% Accessible
\begin{definition}[Accessibility]
    Let $i$, $j$ be two states, we say $j$ is \textbf{accessible from} $i$ if it is possible (with positive probability) for the chain to ever visit $j$ if the chain starts from $i$.
    \[ \mathbb{P}_i[\bigcup_{n=0}^\infty\{X_n = j\}] > 0 \]
    or equivalently
    \[ \sum_{n=0}^\infty P^n(i,j) = \sum_{n=0}^\infty \mathbb{P}_i[X_n = j] > 0 \]
\end{definition}
% Communicate
\begin{definition}[Communication]
    We say $i$ \textbf{communicates with} $j$ if $j$ is accessible from $i$ and $i$ is accessible from $j$.
\end{definition}
% Irreducible
\begin{definition}[Irreducibility]
    We say a Markov Chain is \textbf{irreducible} if all pairs of states communicate. And it is reducible otherwise.
\end{definition}
The relation \emph{communicate with} is an equivalent relation, and irreducible simply means the number of equivalent classes is 1.

%%%%%%%% Aperiodicity
\subsection{Aperiodicity}
% Period
\begin{definition}[Period]
    Given a Markov Chain, its \textbf{period} of state $i$ is defined to be the greatest common divisor $d_i$ of the lengths of loops starting from $i$.
    \[ d_i = \gcd\{n|P^n(i,i) > 0\} \]
\end{definition}
% Period is a class property
\begin{theorem}
    If states $i$ and $j$ communicate, then $d_i = d_j$
\end{theorem}
\begin{sketchproof}
    ~{}
    \begin{itemize}
        \item $P^{n_1}(i,j) > 0$ and $P^{n_2}(j,i) > 0$.
        \item $P^{n_1+n_2}(i,i)>0 \Rightarrow d_i | n_1 + n_2$.
        \item Suppose $P^n{j,j} > 0$, then $P^{n+n_1+n_2}(i,i) > 0 \Rightarrow d_i | n + n_1 + n_2$.
        \item $d_i | n \Rightarrow d_j \ge d_i$.
        \item Similarly $d_i \ge d_j$. We are done.
    \end{itemize}
\end{sketchproof}
\begin{remark}
    Therefore all states in a communicating class have the same period, and all states in an irreducible Markov chain have the same period.
\end{remark}
% Aperiodic
\begin{definition}[Aperiodicity]
    An irreducible Markov chain is said to be \textbf{aperiodic} if its period is 1, and periodic otherwise.
\end{definition}
\begin{proposition}
    \normalfont
    If $P(i,i) > 0$, then the Markov Chain is aperiodic.
\end{proposition}
\begin{remark}
    This is a sufficient but not necessary condition.
\end{remark}

%%%%%%%% Recurrence
\subsection{Recurrence}
We define the \textbf{First Hitting Time} $T_i$ of the state $i$ by
\[ T_i = \inf\{n>0 | X_n = i\} \]
and we can define recurrence as follows
% Recurrent
\begin{definition}[Recurrence]
    The state $i$ is \textbf{recurrent} if $\mathbb{P}_i[T_i<\infty] = 1$, and is transient if it is not recurrent.
\end{definition}
\begin{remark}
    Recurrence means that starting from state $i$ at $T=0$, the chain \emph{is sure to} return to $i$ eventually.
\end{remark}
% Recurrence is a class property
\begin{theorem}
    \normalfont
    Let $i$ be a recurrent state, and let $j$ be accessible from $i$, then all of the following hold:
    \begin{enumerate}
        \item $\mathbb{P}_i[T_j < \infty] = 1$.
        \item $\mathbb{P}_j[T_i < \infty] = 1$.
        \item The state $j$ is recurrent.
    \end{enumerate}
\end{theorem}
\begin{sketchproof}
    ~{}
    \begin{itemize}
        \item The paths starting from $i$ can be seen as infinitely many \emph{cycles}.
        \item Whether the chain visits $p$ in the cycles can be seen as a Bernoulli distribution with probability $p>0$.
        \item The probability of not visiting $j$ in the first $n$ cycles is $(i-p)^n$, which goes to $0$ as $n\to\infty$. So (1)hold.
        \item (2) can be proved by contradiction. $\mathbb{P}_j[T_i < \infty] < 1$ will lead to contradiction against the fact that $i$ is recurrent.
        \item (1)(2) implies (3).
    \end{itemize}
\end{sketchproof}
\begin{corollary}
    If $\mathbb{P}_i[T_j < \infty] > 0$ but $\mathbb{P}_j[T_i < \infty] < 1$, then $i$ is transient.
\end{corollary}
% Equivalent Statement of Recurrence
\begin{theorem}[Equivalent Statement of Recurrence]
    \normalfont
    The state $i$ is recurrent if and only if $\mathbb{E}_i[N_i] = \infty$, where $N_i = \sum_{i=0}^\infty\mathbb{I}\{X_n = i\}$.
\end{theorem}
\begin{sketchproof}
    ~{}
    \begin{itemize}
        \item Recurrence $\Rightarrow \mathbb{P}_i[N_i = \infty] = 1 \Rightarrow \mathbb{E}_i[N_i] = \infty$.
        \item The converse is proved by contradiction.
        \item If $i$ is transient, there is a chance of $p$ that the chain never return to $i$.
        \item So $N_i$ is distributed geometrically, and $\mathbb{E}_i[N_i]$ will be finite. Contradiction.
    \end{itemize}
\end{sketchproof}
\begin{remark}
    By taking expectation on $N_i$, we have:
    \[ \mathbb{E}_i[N_j] = \sum_{i=0}^\infty P^n(i,j) \]
\end{remark}

\begin{corollary}\label{TransientStateGoesToZero}
    If $j$ is transient, then $\lim_{n\to\infty}P^n(i,j) = 0$ for all states $i$.
\end{corollary}
\begin{sketchproof}
    ~{}
    \begin{itemize}
        \item $\mathbb{E}_j[N_j] < \infty$.
        \item $\mathbb{E}_i[N_j] = \mathbb{P}_i[T_j < \infty]\mathbb{E}_i[N_j | T_j < \infty]$.
        \item $\mathbb{E}_i[N_j] \le \mathbb{E}_i[N_j | T_j] = \mathbb{E}_j[N_j] < \infty$ since the probability \emph{restarts} once the chain visits $j$ again.
        \item $\mathbb{E}_i[N_j] = \sum_{i=0}^\infty P^n(i,j) < \infty$ and this implies our conclusion.
    \end{itemize}
\end{sketchproof}

\begin{corollary}
    If $i$ is recurrent, then $\sum_{n=1}^{\infty}p^n(i,i) = \infty$;

    If $i$ is transient, then $\sum_{n=1}^{\infty}p^n(i,i) < \infty$.
\end{corollary}

\begin{proposition}
    Suppose a Markov Chain has a stationary distribution $\pi$, if the state $j$ is transient, then $\pi(j) = 0$.
\end{proposition}
The last proposition follows from Corollary \ref{TransientStateGoesToZero}.

\begin{corollary}
    If an irreducible Markov Chain has a stationary distribution, then the chain is recurrent.
\end{corollary}
\begin{sketchproof}
    The chain cannot be transient, or otherwise all $\pi(j)$ would be 0 and the sum of $\pi$ does not equal to 1.
\end{sketchproof}
\begin{remark}
    The converse is not true!
\end{remark}

\begin{proposition}
    A drunk man will find his way home, but a drunk bird may get lost forever.
\end{proposition}
This is because the random walk on $\mathbb{Z}$ and $\mathbb{Z}^2$ is recurrent, while the walks on higher dimensions are transient.

%%%%%%%% More on Recurrence
\subsection{More on Recurrence}
% Null Recurrence
\begin{definition}[Null Recurrence]
    The state $i$ is \textbf{null recurrent} if it is recurrent and $\mathbb{E}_i[T_i] = \infty$.
\end{definition}
% Positive Recurrence
\begin{definition}[Positive Recurrence]
    The state $i$ is \textbf{positive reccurent} it is recurrent and $\mathbb{E}_i[T_i] < \infty$.
\end{definition}

\begin{proposition}
    Given an irreducible Markov Chain, it is either transient, null recurrent or positive recurrent.
\end{proposition}

%%%%%%%%%%%%%%%%%%%%%%
%%%%%%%% SLLN %%%%%%%%
%%%%%%%%%%%%%%%%%%%%%%
\section{Strong Law of Large Numbers of Markov Chains}
% SLLN of Markov Chains
\begin{theorem}[SLLN of Markov Chains]\label{SLLN}
    Let $X_0, X_1, \dots$ be a Markov Chain starting in the state $X_0=i$. Suppose state $i$ communicates with state $j$. The limiting fraction of time that the chain spends in $j$ is $\frac{1}{\mathbb{E_j}[T_j]}$.
    \[ \mathbb{P}_i[\lim_{n\to\infty}\frac{1}{n}\sum_{t=1}^n\mathbb{I}\{X_t=j\} = \frac{1}{\mathbb{E_j}[T_j]}] = 1 \]
\end{theorem}
\begin{sketchproof}
    Read the book. I'm not going to write this because I am lazy.
\end{sketchproof}

%%%%%%%%%%%%%%%%%%%%%%%%%%%%%%%%%%%%%
%%%%%%%% Basic Limit Theorem %%%%%%%%
%%%%%%%%%%%%%%%%%%%%%%%%%%%%%%%%%%%%%
\section{Basic Limit Theorem}\label{BasicLimitTheorem}
aka. Foundamental Theorem of Markov Chains

%%%%%%%% Finite Case
\subsection{Finite State Space}
% Spectral Radius
\begin{definition}[Spectral Radius]
    Given a non-negative matrix $A$, the spectral radius $\rho(A)$ is the maximum norm of its eigenvalues
    \[ \rho(A) = \max\{ \lambda(A) \} \]
\end{definition}
\begin{proposition}
    Let $A$ be a non-negative matrix, then
    \[ \min_{1\le i \le n} \sum_{j=1}^N a_{i,j} \le \rho(A) \le \max_{1\le i \le n} \sum_{j=1}^N a_{i,j} \]
\end{proposition}
% Perron-Frobenius Theorem
\begin{lemma}[Perron-Frobenius Theorem]\label{PFT}
    Let $A$ be a non-negative matrix with spectral radius $\rho(A) = \alpha$, then $\alpha$ is an eigenvalue of $A$, and has both left and right non-negative eigenvectors.
\end{lemma}
\begin{remark}
    Lemma \ref{PFT} implies that for a finite probability transition matrix $P$, it always has at least one stationary distribution, because it always has eigenvalue $1$ ($P\mathbf{1} = \mathbf{1}$) and a corresponding eigenvector $\pi$ s.t. $\pi^TP = \pi$.
\end{remark}

\begin{lemma}
    Suppose a $k \times k$ matrix $P$ is irreducible. Then there exists a unique solution to $\pi P = \pi$.
\end{lemma}
% Foundamental Theorem, Finite Case
\begin{theorem}[Foundamental Theorem, Finite Case] If a finite Markov Chain is \emph{irreducible} and \emph{aperiodic}, then it has a \emph{unique stationary distribution} $\pi$ any initial distribution \emph{converges} to it:
    \[ \forall \pi_0,\quad \lim_{t\to\infty}\pi_0^T P^t = \pi^T \]
\end{theorem}

%%%%%%%% Countable Case
\subsection{Countable Case}
% Basic Limit Theorem
\begin{theorem}[Basic Limit Theorem]
    Let $X_1, X_2, \dots$ be an \emph{irreducible aperiodic} Markov Chain that has a \emph{stationary distribution} $\pi$. Then
    \[ \lim_{n\to\infty}\pi_n(i) = \pi(i) \]
    for any state $i$ and for any initial distribution $\pi_0$.
\end{theorem}
% Bounded Convergence Theorem
\begin{lemma}[Bounded Convergence Theorem]\label{BCT}
    If $X_n \to X$ with probability $1$ and there is a finite number $b$ such that $|X_n| < b$ for all $n$, then
    \[ \mathbb{E}[X_n] \to \mathbb{E}[X] \]
\end{lemma}
% Corollary of SLLN
Recall SLLN \ref{SLLN}, and the following is a corollary.
\begin{corollary}\label{SLLNCoro}
    For an \emph{irreducible} Markov Chain, we have
    \[ \lim_{n\to\infty} \frac{1}{n}\sum_{t=1}^n P^t(i,j) = \frac{1}{\mathbb{E}_j[T_j]} \]
\end{corollary}
\begin{sketchproof}
    Take expectations on both sides of Theorem \ref{SLLN} yields the conclusion.
\end{sketchproof}
% Cesaro Average
\begin{proposition}[Cesaro Average]
    If a sequence of numbers $a_n$ converges to a value $a$, then the \textbf{Cesaro Average} $(1/n)\sum_{t=1}^na_t$ also converges to it.
\end{proposition}
Corollary \ref{CesaroCoro} follows immediately from the proposition,
\begin{corollary}\label{CesaroCoro}
    For an \emph{irreducible aperiodic} Markov Chain with a \emph{stationary distribution},
    \[ \lim_{n\to\infty}\frac{1}{n}\sum_{t=1}^nP^t(i,j) \to \pi(j) \]
\end{corollary}
\begin{theorem}
    An \emph{irreducible, aperiodic} Markov Chain with a \emph{stationary distribution} has a stationary distribution given by
    \[ \pi(j) = \frac{1}{\mathbb{E}_j[T_j]} \]
\end{theorem}
\begin{sketchproof}
    The proof of the theorem is trivial: compare Corollary \ref{SLLNCoro} and \ref{CesaroCoro}.
\end{sketchproof}
% Foundamental Theorem, Countable Case
\begin{theorem}[Foundamental Theorem, Countable Case]
    An \emph{irreducible} Markov Chain has a \emph{unique stationary distribution} given by
    \[ \pi(j) = \frac{1}{\mathbb{E}_j[T_j]} \]
    if and only if it is \emph{positive recurrent}.
\end{theorem}

%%%%%%%% Other Conclusions
\subsection{Other Conclusions}
\emph{Since they are not covered in class, no proof will be provided. I'm lazy you know.}
\begin{definition}[Doubly Stochastic Chains]
    A transition matrix $P$ is said to be \textbf{doubly stochastic} if all of its columns sum up to 1.
\end{definition}
\begin{optTheorem}
    For a doubly stochastic Markov Chain with $N$ states, the uniform distribution $\pi(i) = 1/N$ is a stationary distribution.
\end{optTheorem}
\begin{optTheorem}
    For a Markov Chain with symmetric $P$ and $N$ states, the uniform distribution $\pi(i) = 1/N$ is a stationary distribution.
\end{optTheorem}
\begin{optTheorem}
    Let $T_j^k = \min\{ n > T_y^{k-1} | X_n = y \}$ be the time of the $k$-th visit to $j$, then by the Markov property,
    \[ \mathbb{T}_x[T_y^k < \infty] = \mathbb{P}_x[T_y < \infty]\cdot\mathbb{P}_y[T_y<\infty] \]
    Notice that
    \[ \mathbb{E}[X] = \sum_{k=0}^{\infty}\mathbb{P}[X \le k] \]
    Using the two equations, we have
    \[ \mathbb{E}_x[N_y] = \frac{\mathbb{P}_x[T_y < \infty]}{1-\mathbb{P}_y[T_y < \infty]} \]
\end{optTheorem}
\begin{remark}
    This theorem gives us more insights into the results in Section \ref{BasicLimitTheorem}.
\end{remark}

%%%%%%%%%%%%%%%%%%%%%%%%%%
%%%%%%%% Examples %%%%%%%%
%%%%%%%%%%%%%%%%%%%%%%%%%%
\section{Examples}
%%%%%%%% Galton-Watson Process
\subsection{Galton-Watson Process}
\emph{aka. Branching Process}
\paragraph*{Problem.} What is the probability that a family name eventually extincts?
\paragraph*{Notations.}
\begin{itemize}
    \item $G_t$ denotes the number of males in generation $t$.
    \item $X_{tk}$ denotes the number of sons fathered by the $k$-th father in the $t$-th generation. Assume $X_i$'s are \emph{iid} with probability mass function $p(\cdot)$.
    \item $\rho = \mathbb{P}[extinction] = \mathbb{P}[\cup_{k\ge 1}\{ G_k=0 \}]$ denotes the probability that the family name eventually goes extinct.
    \item $q_t = \mathbb{P}[G_t = 0]$ denotes the probability that the family name goes extinct at the $t$-th generation.
\end{itemize}
The problem is trivial when $p(0) = 0$ or $p(0) + p(1) = 1$. Either the family name never goes extinct or it goes extinct almost surely. We consider $p(0) > 0$ and $p(0) + p(1) < 1$.

By conditioning on what happens at the first step, we can calculate $\rho$ by
\[ G_{t+1} = \sum_{i=1}^{G_t} X_{ti} \]
\begin{align}
\rho &= \sum_{k=0}^{\infty}\mathbb{P}[extinction \wedge  G_1 = k] \notag \\
&= \sum_{k=0}^{\infty}\mathbb{P}[extinction | G_1 = k]\cdot\mathbb{P}[G_1 = k | G_0 = 1] \quad \text{(by 全概率公式)} \notag \\
&= \sum_{k=0}^{\infty}f(k)\rho^k \triangleq \psi(\rho) \notag
\end{align}
The last step used the fact that all males have sons independently. $\psi(\rho)$ is called the \textbf{Probability Generating Function} of $p(\cdot)$.

The probability of eventual extinction satisfies $\psi(\rho) = \rho$.
\[ \psi'(z) = \sum_{k=1}^{\infty}kp(k)z^{k-1} > 0\]
\[ \psi''(z) = \sum_{k=2}^{\infty}k(k-1)p(k)z^{k-2} > 0 \]
So $\psi(z)$ is a strictly increasing convex function with the following properties.
\begin{itemize}
    \item $\psi(1) = 1$. So $\rho = 1$ is always a solution.
    \item $\psi'(1) = \sum_{k=1}^{\infty}kp(k) = \mathbb{E}[X] \triangleq \mu$
\end{itemize}
\begin{enumerate}[(i)]
    \item If $\mu \le 1$, $\rho = 1$ is the only solution. Therefore the family name will go extinct.
    \item If $\mu > 1$, there exists another solution $r<1$.
    
    Observe that $\{ G_t = 0 \} \subseteq \{ G_{t+1} = 0 \}$ and $q_t \le t_{t+1}$. Since $q_t$ has a upper bound $\rho$, it will always converge.
    \[ q_t \uparrow \rho \]
    So we only need to prove $q_t < r \quad \forall t$. By induction,
    \begin{itemize}
        \item[base.] $q_0 = 0 < r$.
        \item[hypo.] $q_t < r$.
        \item[step.] $q_{t+1} = \sum_{i=0}^{\infty}\mathbb{P}[G_1 = i]\mathbb{P}[G_{t+1} = 0| G_1 = i] = \sum_{i=0}^{\infty}p(i)(q_t)^i$. The last step used the iid assuption of $X$. 
        Therefore $q_{t+1} = \psi(q_t) \le \psi(r) = r$. Done.
    \end{itemize} 
\end{enumerate}

%%%%%%%% Gambler's Ruin
\subsection{Gambler's Ruin}\label{GamblerRuin}
\paragraph*{Problem.} Consider a gambler in a casino, who has probability $p$ to win \$$1$ and $1-p$ to lose \$$1$. The process stops when the gambler gets \$$N$ or goes to $0$. Assume the gambler starts the game with \$$i$.
\paragraph*{Notations.}
\begin{itemize}
    \item $P_i$ denotes the probability of the gambler gets \$$N$ and wins, starting with \$$i$.
    \item $Z_i \in \{1, -1\}$ denotes whether the gambler wins or loses in round $i$.
    \[ X_t = X_0 + \sum_{i=0}^{t-1}Z_i \]
\end{itemize}
Obviously,
\[ P_N = 1 \quad P_0 = 0 \]
When $1 \le i \le N-1$, $P_i$ can be calculated by
\begin{align*}
    P_i &= \mathbb{P}[win | X_0 = i] \\
    &= \mathbb{P}[win \wedge Z_0 = 1 | X_0 = i] + \mathbb{P}[win \wedge Z_0 = -1 | X_0 = i] \quad \text{(Again by 全概率公式 on $Z_i$)} \\
    &= \mathbb{P}[win | X_0=i, Z_0 = 1]\cdot\mathbb{P}[Z_0=1|X_0=i] + \mathbb{P}[win|X_0=1, Z_0=-1]\cdot\mathbb{P}[Z_0=-1|X_0=i] \\
    &= p\cdot P_{i+1} + (1-p)\cdot P_{i-1}
\end{align*}
Rearranging,
\[ p(P_{i+1} - P_i) = (1-p)(P_i - P_{i-1}) \]
\[ P_{i+1} -P_i = \frac{1-p}{p}(P_i - P_{i-1}) \]
Let $\theta = \frac{1-p}{p}$, by high school mafs
\[ P_{i+1} - P_i = \theta^i(P_1 - P_{0}) \]
\begin{enumerate}[(i)]
    \item Assume for now that $p \neq \frac{1}{2}$ so $\theta \neq 1$, then summing over $i$ yields
    \[ P_N - P_0 = 1 = \frac{1-\theta^N}{1-\theta} P_1 \]
    Therefore
    \[ P_1 = \frac{1-\theta}{1-\theta^N} \]
    Summing from $0$ to $i$ yields
    \[ P_i = \frac{1-\theta^i}{1-\theta^N} \]
    \item If $p=\frac{1}{2}$,
    \[ P_{i+1} - P_i = P_i - P_{i-1} \]
    This is a arithmetic progress, and again by high school mafs
    \[ P_i = \frac{i}{N} \]
\end{enumerate}
To sum up
\[ 
    P_i = 
    \begin{cases}
        \frac{i}{N} &(i = \frac{1}{2})\\
        \frac{1-\theta^i}{1-\theta^N} &(\text{o.w.})
    \end{cases}    
\]

\subsection{Drug Test}\label{DrugTest}
This example is based on results from the previous subsection \ref{GamblerRuin}.
\paragraph*{Problem.} We want to test the cure rate $p_1$ of a drug \textrm{Drug}1. We already have a \textrm{Drug}2 with known cure rate $p_2$. We want to know whether $p_1 > p_2$ or not. To do this, we find $t$ pairs of patients $(X_i, Y_i)$ and conduct tests using the two drugs on $X$ and $Y$ respectively. Once the number of patients who are cured by \textrm{Drug}1 but not \textrm{Drug}2 exceeds a certain threshold $M$, we can claim that $p_1 \ge p_2$.
\paragraph*{Notations.}
\begin{itemize}
    \item $X_i,Y_i \in \{0,1\}$ is a boolean denoting whether the first or the second drug cured patient $i$.
    \item $Z_i = X_i - Y_i$.
    \item $p_1, p_2$ denotes the probability that the two drugs cure a patient, respectively.
\end{itemize}

The value of $Z_i$ falls into 3 cases:
\[ Z_i = 
    \begin{cases}
        1 &p_1(1-p_2)\\
        -1 &p_2(1-p_1)\\
        0 &(\text{o.w.})
    \end{cases}
\]
If we ignore the cases where $Z_i = 0$, then we can model this problem as a gambler's ruin, with $p = \frac{p_1(1-p_2)}{p_1(1-p_2)+p_2(1-p_1)}$.

And
\[ \mathbb{P}[TestWrong] = 1 - \frac{1-\theta^M}{1-\theta^{2M}} = \frac{1}{\theta^{-M} + 1} \]
The probability above drops exponentially with $M$, so the threshold does not need to be very large to achieve accurate results.

%%%%%%%% Another random walk on N
\subsection{Another Random Walk}\label{AnotherRandWalk}
\paragraph*{Problem.} Consider a random walk on $\mathbb{N}$ where
\[ P(0,1) = 1 \quad P(N,N-1) = 1 \]
We want to compute how many steps we need to reach $N$ starting from $i$.
\paragraph*{Notations.}
\begin{itemize}
    \item $h_i$ denotes the number of steps to reach $N$ starting from $X_0=i$.
    \item $Y_i$ denotes the number of steps from $i$ to $i+1$ for the first time.
    \item $g_j \triangleq \mathbb{E}[Y_j]$.
\end{itemize}
Obviously
\[ \mathbb{E}[h_0] = 1 + \mathbb{E}[h_1] \quad \mathbb{E}[h_N] = 0 \]
Similar to subsection \ref{DrugTest},
\[ h_i = 1+ (1-p)h_{i-1} + ph_{i+1} \]
This can be calculated using \emph{the linearity of expectation}. 

Notice that
\[ h_i = \sum_{j=i}^{N-1}Y_j \]
Taking expectations on both sides
\[ \mathbb{E}[h_i] = \sum_{j=i}^{N-1}\mathbb{E}[Y_j] \]
So we only need to caculate $\mathbb{E}[Y_j]$.
\begin{itemize}
    \item $g_0 = 1$.
    \item $g_i = 1 + 0 \cdot p + (1-p)(g_{i-1}+g_i)$.
\end{itemize}
\begin{enumerate}
    \item Again assume $p \neq \frac{1}{2}$. 
    Rearranging yields
    \[ g_i = \frac{1}{p} + \theta g_{i-1} \]
    and after a few steps of arithmetics
    \[ g_i = \sum_{i=0}^{t-1}\frac{1}{p}\theta^i + \theta^t \]
    Summing over $g_i$ is a sum of geometric progress, easy.
    \item If $p = \frac{1}{2}$, then
    \[ g_{i} - g_{i-1} = \frac{1}{p} = 2 \]
    so
    \[ g_t = 2t + 1 \]
    and
    \[ \mathbb{E}[h_0] = \sum_{t=0}^{N-1}(2t+1) = N^2 \]
\end{enumerate}
\begin{remark}
    As the converse of the conclusion, if we take $N$ steps, the farthest distance we can go is $\sqrt{N}$.
\end{remark}

%%%%%%%% 2-SAT
\subsection{2-SAT}
Recall the SAT problem in \textsc{AI2615 Design and Analysis of Algorithms}. A 2-SAT is a special case of SAT where each \verb|or| expression has at most 2 terms
\[ \varphi = (x_1 \vee y_1) \wedge (x_2 \vee y_2) \cdots \]
The problem is RP and can be solved using the following algorithm.

\begin{algorithm}
    \caption{2-SAT Solver}
    \KwIn{CNF with $n$ terms}
    Random intialize a solution $\sigma$.\\
    \For{$i = 1:100n^2$}
    {
        \If{$\sigma$ satisfies CNF}
        {
            \KwRet{$\sigma$}
        }
        \Else
        {
            Randomly choose one term $(X_i \vee Y_i)$ that is false.\\
            Randomly flip $X_i$ or $Y_i$.
        }
    }
    \KwRet{Not satisfiable}
\end{algorithm}

To prove its correctness, we only need to consider the cases where the CNF is indeed satisfiable with solution $\sigma$. We denote the sequence of attempts produced by the algorithm by
\[ \sigma_0 \to \sigma_1 \to \cdots \to \sigma_{100n^2} \]
Let $X_i$ denotes the number of exactly the same terms between $\sigma$ and $\sigma_i$.

$X_0, X_1, \dots$ is not a Markov Chain. \sout{So we get stuck}. However we can still informally show the correctness.

We first introduce some basic concepts.
\begin{definition}[Stochasitcal Dominance]
    A random variable $X$ has first-order stochastic dominance over random variable $Y$ if
    \[ \mathbb{P}[X \ge x] \ge \mathbb{P}[Y \ge x] \quad \forall x \]
    \[ \mathbb{P}[X > x] > \mathbb{P}[Y > x] \quad \exists x \]
\end{definition}
\begin{theorem}[Markov's Inequality]\label{MarkovIneq}
    Let $X$ be a nonnegative random varialbe and let $a > 0$, then
    \[ \mathbb{P}[X \ge a] \le \frac{\mathbb{E}[X]}{a} \]
\end{theorem}
\begin{sketchproof}
    Note that $X_{i+1}$ differs from $X_i$ in at most 1 operand, and that
    \[ \mathbb{P}[X_{i+1} = X_i + 1] \ge \frac{1}{2} \]
    \[ \mathbb{P}[X_{i+1} = X_i - 1] \le \frac{1}{2} \]
    We can introduce a random walk described in subsection \ref{AnotherRandWalk} $Y_1, Y_2, \dots$ with $p = \frac{1}{2}$.

    $X_i$ stochatically dominates $Y_i$. Since $Y_i$ has an expectation of $n^2$ to get to state $n$, it follows intuitively that $X_i$ has a smaller expectation of steps to get to $n$.

    We have chosen the max iteration to be $100n^2$, by Markov's Inequality \ref{MarkovIneq}, the probability that we take $100n^2$ iterations without finding a feasible solution is at most $1/100$.
\end{sketchproof}
\begin{remark}
    A formal proof of correctness will be given in the next lecture.
\end{remark}