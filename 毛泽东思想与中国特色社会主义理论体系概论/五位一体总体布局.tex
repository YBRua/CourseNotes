\chapter{“五位一体”总体布局}

\section{建设现代化经济体系}
    \subsection{贯彻新发展理念}
        党的十八届五中全会坚持以人民为中心的发展思想,鲜明提出了创新、协调、绿色、开放、共享的新发展理念。

        \begin{itemize}
            \item 创新是引领发展的第一动力。
            \item 协调是持续健康发展的内在要求。
            \item 绿色是永续发展的必要条件。
            \item 开放是国家繁荣发展的必由之路。
            \item 共享是中国特色社会主义的本质要求。
        \end{itemize}

        \firstOrder{创新、协调、绿色、开放、共享的新发展理念,相互贯通、相互促进,是具有内在联系的集合体。}\footnote{本段以理解为主。}
        \begin{itemize}
            \item 创新注重的是解决发展动力问题
            \item 协调注重的是解决发展不平衡问题
            \item 绿色注重的是解决人与自然和谐的问题
            \item 开放注重的是解决发展内外联动问题
            \item 共享注重的是解决社会公平正义问题
        \end{itemize}

    \subsection{深化供给侧结构性改革}
        贯彻新发展理念、建设现代化经济体系必须坚持供给侧结构性改革。坚持质量第一、效益优先,以供给侧结构性改革为主线,推动经济发展质量变革、效率变革、动力变革,提高全要素生产率。

        \begin{enumerate}
            \item 推进增长动能转换,以加快发展先进制造业为重点全面提升实体经济。
            \item 深化要素市场化配置改革,实现由以价取胜向以质取胜的转变。
            \item 加大人力资本培育力度,更加注重调动和保护人的积极性。
            \item 持续推进“三去一降一补”(去产能、去库存、去杠杆、降成本、补短板),优化市场供求结构。
        \end{enumerate}

    \subsection{建设现代化经济体系的主要任务}
        建设现代化经济体系,需要扎实惯用的政策举措和行动。当前,要突出抓好以下几个方面工作。
        \begin{enumerate}
            \item \firstOrder{大力发展实体经济。}实体经济是一国经济的立身之本,是财富创造的根本源泉,是国家强盛的重要支柱,是现代化经济体系的坚实基础。
            \item \firstOrder{加快实施创新驱动发展战略。}深入实施科教兴国战略、人才强国战略、创新驱动发展战略。
            \item 激发各类市场主体活力。
            \item 积极推动城乡区域协调发展。
            \item 着力发展开放型经济。
            \item \firstOrder{加快完善社会主义市场经济体制。}
            \begin{itemize}
                \item 毫不动摇地巩固和发展公有制经济
                \item 毫不动摇鼓励、支持、引导非公有制经济发展
            \end{itemize}
        \end{enumerate}


\section{发展社会主义民主政治}
    \subsection{坚持中国特色社会主义政治发展道路}
        \begin{itemize}
            \item \firstOrder{走中国特色社会主义政治发展道路,必须坚持党的领导、人民当家作主、依法治国有机统一。}
            \begin{itemize}
                \item 党的领导是人民当家作主和依法治国的根本保证
                \item 人民当家作主是社会主义民主政治的本质特征
                \item 依法治国是党领导人民治理国家的基本方式
                \item 三者统一于我国社会主义民主政治伟大实践
            \end{itemize}
            \item 走中国特色社会主义政治发展道路,必须坚持正确政治方向。
        \end{itemize}

    \subsection{健全人民当家做主制度体系}
        我国是工人阶级领导的、以工农联盟为基础的人民民主专政的社会主义国家。
        \begin{itemize}
            \item \firstOrder{人民代表大会制度是我国根本政治制度。}是坚持党的领导、人民当家作主、依法治国有机统一的根本政治制度安排,必须长期坚持、不断完善。
            \item 发挥社会主义协商民主的重要作用。实行人民民主,保证人民当家作主,要求治国理政大政方针在人民内部各方面进行广泛商量。统筹推进政党协商、人大协商、政府协商、政协协商、人民团体协商、基层协商及社会组织协商。
            \item 中国共产党领导的多党合作和政治协商制度是我国的一项基本政治制度,人民政协是具有中国特色的制度安排,是社会主义协商民主的重要渠道和专门协商机构。把协商民主贯穿政治协商、民主监督、参政议政全过程。
            \item 民族区域自治制度是我国的一项基本政治制度。
            \item 基层群众自治制度是我国的一项基本政治制度。
        \end{itemize}

    \subsection{巩固和发展爱国统一战线}
        \begin{enumerate}
            \item 坚持长期共存、互相监督、肝胆相照、荣辱与共,支持民主党派按照中国特色社会主义参政党要求更好履行职能
            \item 深化民族团结进步教育,筑牢中华民族共同体意识
            \item 全面贯彻党的宗教工作基本方针,坚持我国宗教的中国化方向,积极引导宗教与社会主义相适应
            \item 牢牢把握大团结大联合的主题,做好统战工作
        \end{enumerate}

    \subsection{坚持“一国两制”,推进祖国统一}
        似乎不考。
        \begin{enumerate}
            \item 全面准确贯彻一国两制方针
            \item 扎实推进祖国和平统一进程
        \end{enumerate}


\section{推动社会主义文化繁荣昌盛}
    \subsection{牢牢掌握意识形态工作领导权}
        \begin{enumerate}
            \item 要旗帜鲜明坚持马克思主义指导地位
            \item 要加快构建中国特色哲学社会科学
            \item 要坚持正确的舆论导向
            \item 要建设好网络空间
            \item 要落实好意识形态工作责任制
        \end{enumerate}

    \subsection{培育和践行社会主义核心价值观}
        富强民主文明和谐、自由平等公正法治、爱国敬业诚信友善。
        \begin{enumerate}
            \item 要把社会主义核心价值观融入社会生活各个方面
            \item 要坚持全民行动、干部带头,从家庭做起、从娃娃抓起
            \item 必须立足中华优秀传统文化和革命文化
            \item 必须发痒中国人民在长期奋斗中培育、继承、发展起来的伟大民族精神:伟大创造精神、伟大奋斗精神、伟大团结精神和伟大梦想精神。
        \end{enumerate}

    \subsection{坚定文化自信,建设社会主义文化强国}
        \begin{enumerate}
            \item 必须培养高度的文化自信
            \item 必须大力发展文化事业和文化产业
            \item 必须提高国家文化软实力
        \end{enumerate}


\section{坚持在发展中保障和改善民生}
    \subsection{提高保障和改善民生水平}
        \begin{enumerate}
            \item 优先发展教育事业
            \item 提高就业质量和人民收入水平
            \item 加强社会保障体系建设
            \item 坚决打赢脱贫攻坚战
            \item 实施健康中国战略
        \end{enumerate}

    \subsection{加强和创新社会治理}
        \begin{enumerate}
            \item 创新社会治理体质
            \item 改进社会治理方式
            \item 加强预防和化解社会矛盾机制建设
            \item 加强社会心理服务建设体系
            \item 加强社区治理体系建设
        \end{enumerate}

    \subsection{坚持总体国家安全观}
        \begin{enumerate}
            \item 完善国家安全体系
            \item 健全公共安全体系
            \item 推进平安中国建设
            \item 加强国家安全能力建设
            \item 加强国家安全教育
        \end{enumerate}


\section{建设美丽中国}
    \subsection{坚持人与自然和谐共生}
        \firstOrder{生态文明的核心是坚持人与自然和谐共生。}
        \begin{enumerate}
            \item 尊重自然,是人与自然相处时应秉持的首要态度。要求人对自然怀有敬畏之心、感恩之情、报恩之一,尊重自然界的创造和存在,决不能凌驾于自然之上
            \item 顺应自然,是人与自然相处时应遵循的基本原则。要求人顺应自然的客观规律,按自然规律办事
            \item 保护自然,是人与自然相处时应承担的重要责任,要求人发挥主观能动性,在向自然界索取生存发展之需的同时,呵护自然,回报自然,保护自然界的生态系统
        \end{enumerate}

    \subsection{形成人与自然和谐发展新格局}
        \begin{enumerate}
            \item 把节约资源放在首位
            \item 坚持保护优先、自然恢复为主
            \item 着力推进绿色发展、循环发展、低碳发展
            \item 形成节约资源和保护环境的空间格局、产业结构、生产方式、生活方式。
        \end{enumerate}

    \subsection{加快生态文明体制改革}
        \begin{enumerate}
            \item 推进绿色发展
            \item 着力解决突出环境问题
            \item 加大生态系统保护力度
            \item 改革生态环境监管体制
        \end{enumerate}
