\documentclass[oneside]{book}
\usepackage{xeCJK}
\usepackage{amsmath}
\usepackage{mathtools}
\usepackage{listings} % lstlist插入代码
\usepackage{booktabs}
\usepackage{ulem}
\usepackage{enumerate}
\usepackage{amsfonts}
\usepackage{amssymb}
\usepackage{setspace} % spacing环境设置行间距
\usepackage[ruled, vlined]{algorithm2e} % 算法与伪代码 
\usepackage{bm} % 数学公式中的加粗
\usepackage{pifont} % 打圈的数字。172-211。\ding
\usepackage{graphicx}
\usepackage{float}
\usepackage[dvipsnames]{xcolor}
\usepackage{indentfirst}
\usepackage{ulem} %\sout{}打删除线
\normalem % 使用默认normalem
\usepackage{lmodern}
\usepackage{subcaption}
\usepackage[colorlinks, linkcolor=blue]{hyperref}
\usepackage{cleveref}
\usepackage[a4paper]{geometry}
\usepackage{titlesec}

\usepackage{zhnumber}
\renewcommand{\thesubsection}{\zhnum{subsection}}
\renewcommand{\thesection}{\zhnum{section}}
\renewcommand{\thechapter}{\zhnum{chapter}}
\renewcommand{\thepart}{\zhnum{part}}

\renewcommand{\contentsname}{目录}

\titleformat{\part}{\centering\Huge\bfseries}{第\thepart 部分}{1em}{}
\titleformat{\chapter}{\centering\Huge\bfseries}{第\thechapter 章}{1em}{}
\titleformat{\section}{\LARGE\bfseries}{第\thesection 节}{1em}{}
\titleformat{\subsection}{\Large\bfseries}{\thesubsection 、}{1em}{}

\usepackage{fancyhdr}
\pagestyle{plain}

\title{\Huge\textbf{毛泽东思想和中国特色\\社会主义理论体系概论}}
\author{背不完书的\textsc{YBiuR}}
\date{2020-2021学年{} 春季学期}


\begin{document}
\begin{spacing}{1.2}
% \setlength{\leftskip}{2em}
\setlength{\parindent}{0em}
\setlength{\parskip}{1em}

\frontmatter
\maketitle
\chapter*{前言}
\section*{碎碎念}
\paragraph{}\verb|aReasonablyLongCourseName|: \textsc{TH029 Introduction to Mao Zedong's Thoughts and Theoretical System of Socialism with Chinese Characteristics}.
\paragraph{}排版中文太难力。
\paragraph{}讲正事了。


\section*{马克思主义中国化}
    \textbf{马克思主义中国化}就是把马克思主义基本原理同中国具体实际和时代特征结合起来,运用马克思主义的立场、观点、方法研究和解决中国革命、建设、改革中的实际问题,使之成为具有中国特色、中国风格、中国气派的马克思主义。

    在中国革命、建设、改革的历史进程中,马克思主义中国化实现了\textbf{两次历史性飞跃}:
    \begin{enumerate}
        \item 第一次历史性飞跃发生在\emph{新民主主义革命时期},形成了\emph{毛泽东思想}。
        \item 第二次历史性飞跃发生在社会主义进入\emph{改革开放}的新时期,形成了\emph{中国特色社会主义理论体系}。
    \end{enumerate}


\tableofcontents

\mainmatter
\part{毛泽东思想}
\chapter{毛泽东思想及其历史地位}
\par 毛泽东思想是在革命和建设的长期实践中,以毛泽东为主要代表的中国共产党人,根据马克思列宁主义基本原理,形成的适合中国情况的科学指导思想,是被实践证明了的关于中国革命和建设的正确的理论原则和经验总结,是中国共产党集体智慧的结晶。

\par 毛泽东思想是马克思主义中国化的第一个重大理论成果,是马克思列宁主义在中国的运用和发展,是被实践证明了的关于中国革命和建设的正确的理论原则和经验总结,是中国国共产党集体智慧的结晶,是党必须长期坚持的指导思想。


\section{毛泽东思想的形成和发展}

    \subsection{毛泽东思想形成发展的历史条件}
    Nothing here.

    \subsection{毛泽东思想形成发展的过程}
    \firstOrder{毛泽东思想的形成。}
    书本中萌芽时期和形成时期统称为形成时期。
    \begin{enumerate}
        \item \firstOrder{萌芽.} 第一次国内革命战争时期。这一时期的著作
        \begin{itemize}
            \item 《中国社会各阶级的分析》
            \item 《湖南农民运动考察报告》
        \end{itemize}
        分析了中国社会各阶级在革命中的地位和作用,提出了\emph{新民主主义革命}的基本思想。
        \item \firstOrder{形成.} 土地革命战争时期。著作:
        \begin{itemize}
            \item 《中国的红色政权为什么能够存在?》
            \item 《井冈山的斗争》
            \item 《星星之火,可以燎原》
            \item 《反对本本主义》
        \end{itemize}
        \par 在同党内一度盛行的把马克思主义教条化、把共产国际决议和苏联经验神圣化的错误倾向的斗争中,提出并阐述了\emph{农村包围城市}、\emph{武装夺取政权}的思想,标志着毛泽东思想的初步形成。
        \item \firstOrder{成熟.} 著作:
        \begin{itemize}
            \item 《实践论》、《矛盾论》
            \item 《<共产党人>发刊词》
            \item 《中国革命和中国共产党》
            \item 《新民主主义论》
            \item 《论联合政府》
        \end{itemize}
        \par 运用马克思主义的认识论和辩证法,系统分析了党内“左”的和右的错误地思想根源。
        \par 新民主主义革命理论的系统阐述,实现了马克思主义与中国革命实践相结合的历史性飞跃,标志着毛泽东思想得到多方面展开而趋于成熟。
        \par \emph{1945年党的七大将毛泽东思想写入党章,确立为党必须长期坚持的指导思想。}
        \item \firstOrder{继续发展.} 著作:
        \begin{itemize}
            \item 《在中国共产党第七届中央委员会第二次全体会议上的报告》
            \item 《论人民民主专政》
            \item 《论十大关系》
            \item 《关于正确处理人民内部矛盾的问题》
        \end{itemize}
        先后提出\emph{人民民主专政理论}、\emph{社会主义改造理论}、\emph{关于严格区分和正确处理两类矛盾的学说}特别是正确处理人民内部矛盾的理论。
    \end{enumerate}


\section{毛泽东思想的主要内容和活的灵魂}

    \subsection{毛泽东思想的主要内容}
    \begin{enumerate}
        \item 新民主主义革命理论:第\ref{Chapter:新民主主义革命理论}章节
        \item 社会主义革命和社会主义建设理论:第\ref{Chapter:社会主义改造理论}章节
        \item 革命军队建设和军事战略的理论
        \item 政策和策略的理论
        \item 思想政治工作和文化工作的理论
        \item 党的建设理论
    \end{enumerate}

    \subsection{毛泽东思想活的灵魂}
    \firstOrder{贯穿于毛泽东思想各个组成部分的立场、观点和方法,是毛泽东思想的活的灵魂,有三个基本方面,即实事求是,群众路线,独立自主。}

    \subsubsection{1. 实事求是} 实事求是,就是一切从实际出发,理论联系实际,坚持在实践中检验真理和发展真理。

    \subsubsection{2. 群众路线} 群众路线,就是一切为了群众,一切依靠群众,从群众中来,到群众中去,把党的正确主张变为群众的自觉行动。

    \subsubsection{3. 独立自主} 独立自主,就是坚持独立思考,走自己的路,就是坚定不移地维护民族独立、捍卫国家主权,把立足点放在依靠自己力量的基础上,同时积极争取外援,开展国际经济文化交流,学习外国一切对我们有益的先进事物。


\section{毛泽东思想的历史地位}

    \subsection{马克思主义中国化的第一个重大理论成果}
    毛泽东思想是马克思主义中国话第一次历史性飞跃的理论成果。在马克思主义中国话的历史进程中,毛泽东思想为中国特色社会主义理论体系的形成奠定了理论基础。

    \subsection{中国革命和建设的科学指南}

    \subsection{中国共产党和中国人民宝贵的精神财富}
\chapter{新民主主义革命理论}\label{Chapter:新民主主义革命理论}


\section{新民主主义革命理论形成的依据}

    \subsection{近代中国国情和中国革命的时代特征}
    近代中国占支配地位的主要矛盾是帝国主义和中华民族的矛盾、封建主义和人民大众的矛盾,而\emph{帝国主义和中华民族的矛盾是各种矛盾中最主要的矛盾}。这决定了近代中国革命的\emph{根本任务是推翻帝国主义、封建主义和官僚资本主义的统治(三座大山)}。

    以五四运动的爆发为标志,中国资产阶级民主革命进入新民主主义革命的崭新阶段\footnote{新民主主义革命自1919年开始,旧民主主义到1922年结束。}。

    无产阶级开始以独立的政治力量登上历史舞台,由自在阶级转变为自为的阶级。马克思主义在中国得到广泛传播,逐步成为中国革命的指导思想。

    它推翻帝国主义、封建主义和官僚资本主义的反动统治,但不破坏参加反帝反封建的资本主义成分。

    中国革命要分两步走,第一步是完成反帝反封建的新民主主义革命任务,第二步是完成社会主义革命任务,这是性质不同但又相互联系的两个革命过程。

    \subsection{新民主主义革命理论的实践基础}
    不考。


\section{新民主主义革命的总路线和基本纲领}

    \subsection{新民主主义革命的总路线}
    1939年,毛泽东在《中国革命和中国共产党》一文中首次提出“新民主主义革命”的科学概念。

    1948年,他在《在晋绥干部会议上的讲话》中完整地表述了总路线的内容:

    \begin{center}
        \emph{无产阶级领导的,人民大众的,反对帝国主义、封建主义和官僚资本主义的革命。}
    \end{center}

    \begin{enumerate}
        \item \textbf{新民主主义革命的对象.}
        分清敌友,这是革命的\textbf{首要问题}
        \begin{itemize}
            \item 帝国主义。是中国革命的\textbf{首要对象}。
            \item 封建地主阶级。
            \item 官僚资本主义。
        \end{itemize}
        \item \textbf{新民主主义革命的动力.}
        新民主主义革命的动力包括无产阶级、农民阶级、城市小资产阶级和民族资产阶级。
        \begin{itemize}
            \item 无产阶级是中国革命\textbf{最基本的动力}。
            \item 农民是中国革命的主力军。
            \item 城市小资产阶级是无产阶级的可靠同盟者。包括广大的知识分子、小商人、手工业者和自由职业者。
            \item 民族资产阶级也是中国革命的动力之一。\footnote{注意中国革命的对象不是所有剥削阶级。城市小资产阶级和民族资产阶级不在革命对象范围内。}
        \end{itemize}
        \item \textbf{新民主主义革命的领导力量.}
        无产阶级的领导权是中国革命的\textbf{中心问题},也是新民主主义革命理论的\textbf{核心问题}。

        区别新旧两种不同范畴的民主主义革命的根本标准是,\emph{革命的领导权掌握在无产阶级手中还是掌握在资产阶级手中}。

        中国无产阶级具有一些特点和优点
        \begin{enumerate}
            \item 从诞生之日起,就身受外国资本主义、本国封建势力和资产阶级的三重压迫。
            \item 分布集中,有利于无产阶级队伍的组织和团结,有利于革命思想的传播和强大革命力量的形成。
            \item 和农民有着天然的联系,这是的无产阶级便于和农民结成亲密的联盟。
        \end{enumerate}
        无产阶级及其政党——中国共产党的领导,是中国革命取得胜利的根本保证。

        无产阶级及其政党实现对各革命阶级的领导,必须建立以工农联盟为基础的广泛地统一战线,这是实现领导权的关键。

        \item \textbf{新民主主义革命的性质和前途.}
        近代中国半殖民地半封建社会的性质和中国革命的历史人物,决定了中国革命的性质不是无产阶级社会主义革命,而是资产阶级民主主义革命。

        革命的前途是社会主义而不是资本主义。
    \end{enumerate}

    \subsection{新民主主义的基本纲领}
    \emph{只需要知道有三个纲领就行。}
    \begin{enumerate}
        \item \textbf{政治纲领}:推翻帝国主义和封建主义的统治,建立一个无产阶级领导的,以工农联盟为基础的、各革命阶级联合专政的新民主主义的共和国。
        \item \textbf{经济纲领}:没收封建地主阶级的土地归农民所有,没收官僚资产阶级的垄断资本归新民主主义的国家所有,保护民族工商业。
        \item \textbf{文化纲领}:无产阶级领导的人民大众的反帝反封建的文化,即民族的科学的大众的文化。
    \end{enumerate}


\section{新民主主义革命的道路和基本经验}

    \subsection{新民主主义革命的道路}
    即农村包围城市、武装夺取政权的革命道路。
    \paragraph{新民主主义革命道路的内容及意义.} 中国革命走农村包围城市、武装夺取政权的道路,根本在于处理好\textbf{土地革命}、\textbf{武装斗争}、\textbf{农村革命根据地建设}三者之间的关系。
    \begin{enumerate}
        \item 土地革命是民主革命的基本内容。
        \item 武装斗争是中国革命的主要形式,是农村根据地建设和土地革命的强有力保证。
        \item 农村革命根据地是中国革命的战略阵地,是进行武装斗争和开展土地革命的依托。
    \end{enumerate}

    \subsection{新民主主义革命的三大法宝}
    毛泽东在《<共产党人>发刊词》中,把\textbf{统一战线}、\textbf{武装斗争}、\textbf{党的建设}比作党在中国革命中战胜敌人的三个主要的法宝。
    \begin{enumerate}
        \item 统一战线
        \item 武装斗争
        \item 党的建设
    \end{enumerate}
\chapter{社会主义改造理论}\label{Chapter:社会主义改造理论}


\section{从新民主主义到社会主义的转变}

    \subsection{新民主主义社会是一个过渡性的社会}
    新民主主义社会不是一个独立的社会形态,而是由新民主主义向社会主义转变的过渡性社会形态。\footnote{持续时间自新中国成立到社会主义改造基本完成。}

    \subsection{党在过渡时期的总路线及其理论依据}
        \subsubsection{1. 党在过渡时期的总路线的提出}
        党在过渡时期总路线的主要内容被概括为\textbf{“一化三改”}:一化即社会主义工业化;三改即对个体农业、手工业和资本主义工商业的社会主义改造。

        这是一条社会主义建设和社会主义改造同时并举的路线,体现了社会主义工业化和社会主义改造的紧密结合。


\section{社会主义改造道路和历史经验}

    \subsection{适合中国特点的社会主义改造道路}
        \subsubsection{农业、手工业的社会主义改造}
        农业:
        \begin{enumerate}
            \item 积极引导农民组织起来,走互助合作道路。
            \item 遵循自愿互利、典型示范和国家帮助的原则,以互助合作的优越性吸引农民走互助合作道路。
            \item 正确分析农村的阶级和阶层状况,制定正确的阶级政策。
            \item 坚持积极引导、稳步前进的方针,采取循序渐进的步骤。
        \end{enumerate}

        手工业:采取积极领导、稳步前进的方针。

        \subsubsection{资本主义工商业的社会主义改造}
        \begin{enumerate}
            \item 用和平赎买的方法改造资本主义工商业。
            \item 采取从低级到高级的国家资本主义的过渡形式。\footnote{这些企业的利润,按国家所得税、企业公积金、工人福利费、资方红利这四个方面进行分配,即当时所说的“四马分肥”。}
            \item 把资本主义工商业者改造为自食其力的社会主义劳动者。
        \end{enumerate}

    \subsection{社会主义改造的历史经验}
    \begin{enumerate}
        \item 坚持社会主义工业化建设与社会主义改造同时并举。
        \item 采取积极引导、逐步过渡的方式。
        \item 用和平方法进行改造。
    \end{enumerate}


\section{社会主义制度在中国的确立}

    \subsection{社会主义制度的基本确立及其理论依据}
    1956年底,我国对农业、手工业和资本主义工商业的社会主义改造基本完成,社会主义公有制已成为我国社会的经济基础,标志着中国历史上长达数千年的阶级剥削制度的结束和社会主义基本制度的确立。

    1954年9月,《中华人民共和国宪法》制定并颁布施行,明确规定了我国人民民主专政的国体和人民代表大会的政体。人民代表大会制度这一根本政治制度、中国共产党领导的多党合作和政治协商制度、民族区域自治制度这些基本政治制度的确立,表明我国有一个新民主主义的国家转变为社会主义国家。

    \subsection{确立社会主义基本制度额重大意义}
    \verb|NotImplementedError:| 没讲。
\chapter{社会主义建设道路初步探索的理论成果}

\section{初步探索的重要理论成果}
\begin{enumerate}
    \item 调动一切积极因素为社会主义事业服务。
    \item 正确认识和处理社会主义矛盾的思想。
    \item 走中国工业化道路的思想。
\end{enumerate}

    \subsection{调动一切积极因素为社会主义事业服务}
    1956年4月和5月,毛泽东先后在中央政治局扩大会议和最高国务会议上,做了《论十大关系》的报告。

    《论十大关系》标志着党探索中国社会主义建设道路的良好开端。

    《论十大关系》确立了一个基本方针:\emph{努力把党内党外国内国外的一切积极的因素,直接的、间接的积极因素全部调动起来,为社会主义建设服务}。

    \subsection{正确认识和处理社会主义社会矛盾的思想}
    有两类社会矛盾:\textbf{敌我矛盾}和\textbf{人民内部矛盾},这是两类性质完全不同的矛盾。敌我之间和人民内部这两类矛盾的性质不同,解决的方法也不同:采用\textbf{专政}和\textbf{民主}两种不同的方法。

    \subsection{走中国工业化道路的思想}
    毛泽东在《论十大关系》中论述的第一大关系,便是\emph{重工业、轻工业和农业的关系}。要走一条有别于苏联的中国工业化道路。

    提出了\emph{以农业为基础,以工业为主导,以农轻重魏旭发展国民经济的总方针}。


\section{初步探索的意义和经验教训}

    \subsection{初步探索的意义}
    \begin{enumerate}
        \item 巩固和发展了我国的社会主义制度。
        \item 为开创中国特色社会主义提供了宝贵经验、理论准备、物质基础。
        \item 丰富了科学社会主义的理论和实践。
    \end{enumerate}

    \subsection{初步探索的经验教训}
    \begin{enumerate}
        \item 必须把马克思主义与中国实际相结合,探索符合中国特点的社会主义建设道路。
        \item 必须正确认识社会主义社会的主要矛盾和根本任务,集中力量发展生产力。
        \item 必须从实际出发进行社会主义建设,建设规模和速度要和国力相适应,不能急于求成。
        \item 必须发展社会主义民主,健全社会主义法制。
        \item 必须坚持党的民主集中制和集体领导制度,加强执政党建设。
        \item 必须坚持对外开放,借鉴和吸收人类文明成果建设社会主义,不能关起门来搞建设。
    \end{enumerate}

\part{邓小平理论、三个代表重要思想、科学发展观}
\chapter{邓小平理论}
\emph{“什么是社会主义、怎样建设社会主义。”}

\section{邓小平理论的形成}
    \subsection{邓小平理论的形成条件}
        \begin{itemize}
            \item 和平与发展成为时代主题是邓小平理论形成的时代背景。
            \item 社会主义建设的经验教训是邓小平理论形成的历史根据
            \item 改革开放和现代化建设的实践是邓小平理论形成的现实依据。
        \end{itemize}

    \subsection{邓小平理论的形成过程}
        \begin{enumerate}
            \item 1978年12月召开的党的十一届三中全会
            \begin{itemize}
                \item \firstOrder{重新确立了解放思想、实事求是的思想路线}
                \item \firstOrder{停止使用以阶级斗争为纲的错误提法}
                \item \firstOrder{确定把全党工作重点转移到社会主义现代化建设上}
                \item 作出实行改革开放的重大决策
            \end{itemize}
            \item \firstOrder{1982年在党的十二大开幕词中明确指出建设有中国特色的社会主义。从此,中国特色社会主义成为党的全部理论和实践创新的主题。}
            \item \firstOrder{1987年党的十三大,第一次比较系统地论述了我国社会主义初级阶段理论。明确概括和全面阐发了党的“一个中心、两个基本点”的基本路线。}
            \item 1992年南方谈话重申了深化改革、加速发展的必要性和重要性。提出了一系列重要论断。
            \item 1992年党的十四大第一次比较系统地初步回答了什么是社会主义、怎样建设社会主义。
            \item \firstOrder{1997年党的十五大正式提出邓小平理论这一概念,同马克思列宁主义、毛泽东思想一起确立为党的指导思想并写入党章。}
            \item 1999年正式将邓小平理论载入宪法。
        \end{enumerate}


\section{邓小平理论的基本问题和主要内容}
    \subsection{邓小平理论回答的基本问题}
        \firstOrder{什么是社会主义、怎样建设社会主义},是邓小平在领导改革开放和现代化建设这一新的革命过程中,不断提出和反复思考的首要基本理论问题。

    \subsubsection{社会主义的本质} 1992年初,邓小平在南方谈话中对社会主义的本质作了总结性理论概括:\firstOrder{解放生产力,发展生产力,消灭剥削,消除两极分化,最终达到共同富裕。}

    \subsection{邓小平理论的主要内容}
        \subsubsection{解放思想、实事求是} \firstOrder{解放思想、实事求是的思想路线:是邓小平理论活的灵魂,是邓小平理论的精髓。}

        \subsubsection{社会主义初级阶段理论} 
        党的十三大系统的论述了社会主义初级阶段理论。

        \textbf{社会主义初级阶段},就是指我国在生产力落后、商品经济不发达条件下建设社会主义必然要经历的特定阶段,即从我国进入社会主义到基本实现社会主义现代化的整个历史阶段。
        \begin{enumerate}
            \item 我国已经进入社会主义社会,必须坚持而不能离开社会主义社会。
            \item 我国的社会主义社会还处在不发达的阶段,必须正视而不能超越初级阶段。
        \end{enumerate}

        党的十五大进一步阐述了社会主义初级阶段的基本特征(P99)。

        \subsubsection{党的基本路线}
        领导和团结全国各族人民,\emph{以经济建设为中心,坚持四项基本原则,坚持改革开放},自力更生,艰苦创业,为把我国建设成为富强、民主、文明(、和谐、美丽)的社会主义现代化国家(强国)而奋斗。
        \begin{enumerate}
            \item 建设富强民主文明和谐美丽的社会主义现代化国家。是党在社会主义初级阶段的奋斗目标,体现了社会主义社会全面发展的要求。
            \item 一个中心、两个基本点。是基本路线最主要的内容,是实现社会主义现代化奋斗目标的基本路径
            \item 领导和团结全国各族人民。是实现社会主义现代化奋斗目标的领导力量和依靠力量。
            \item 自力更生艰苦创业。是党的优良传统,也是实现社会主义初级阶段奋斗目标的根本立足点。
        \end{enumerate}

        \subsubsection{社会主义的根本任务}
        生产力是社会发展的最根本的决定性因素,社会主义的根本任务是\textbf{发展生产力}。社会主义革命是为了解放生产力、发展生产力。

        \emph{贫穷不是社会主义,社会主义要消灭贫穷。}

        \subsubsection{三步走战略}
        1987年4月,邓小平第一次提出了分三步走基本实现现代化的战略。同年10月,党的十三大把邓小平三步走的发展战略构想确定下来。
        \begin{enumerate}
            \item 从1981年到1990年实现国民生产总值比1980年翻一番,解决人民的温饱问题。
            \item 从1991年到20世纪末,使国民生产总值再翻一番,达到小康水平。
            \item \firstOrder{到21世纪中叶,国民生产总值再翻两番,达到中等发达国家水平,基本实现现代化。}\footnote{在习近平新时代中国特色社会主义思想中,这一任务目标提前至2035年。}
        \end{enumerate}

        为了顺利实现现代化发展战略,邓小平提出了台阶式发展的思想,要求抓住机遇,加快发展,争取隔几年是国民经济上一个新台阶。

        为了顺利实现现代化发展战略,邓小平还提出允许和鼓励一部分地区、一部分人先富起来逐步达到共同富裕的思想。

        \subsubsection{改革开放理论}
        (P105)。我国的渐进式增量改革,有别于苏联和俄罗斯的改革模式。

        新时期最鲜明的特点是改革开放。改革开放是中国的第二次革命。作为一次新的革命,不是也不允许否定和抛弃社会主义基本制度,它是社会主义制度的自我完善和发展。
        
        改革开放的实质和目标,是要从根本上改变舒服我国生产力发展的经济体制,建立充满生机和活力的社会主义新经济体制,同时相应地改革政治体制和其他方面的体制,以实现中国的社会主义现代化。

        判断改革和各方面工作的是非得失,归根到底,要以是否有利于发展社会主义的生产力,是否有利于增强社会主义国家的综合国力,是否有利于提高人民的生活水平为标准。

        \subsubsection{社会主义市场经济理论}
        \firstOrder{农村家庭联产承包责任制。}

        十二届三中全会通过的《中共中央关于经济体制改革的决定》提出了社会主义经济是公有制基础上有计划的商品经济的论断。

        在南方谈话中,邓小平指出,计划经济不等于社会主义,资本主义也有计划;市场经济不等于资本主义,社会主义也有市场。从根本上解除了把计划经济和市场经济看做社会基本制度范畴的思想束缚。

        社会主义市场经济理论的要点有
        \begin{enumerate}
            \item 计划经济和市场经济不是划分社会制度的标志
            \item 计划和市场都是经济手段,对经济活动的调节各有优势和长处,社会主义实行市场经济要把两者结合起来。
            \item 市场经济作为资源配置的一种方式本身不具有制度属性,可以和不同社会制度结合。
        \end{enumerate}

        \subsubsection{两手抓,两手都要硬}
        \begin{itemize}
            \item 一手抓物质文明,一手抓精神文明
            \item 一手抓建设,一手抓法制
            \item 一手抓改革开放,一手抓惩治腐败
        \end{itemize}

        \subsubsection{一国两制}
        和平统一、一国两制。(不考)

        \subsubsection{中国问题的关键在于党}
        建设中国特色社会主义,关键在于坚持、加强和改善党的领导。除了改善党的组织状况以外,还要改善党的领导工作状况、改善党的领导制度。


\section{邓小平理论的历史地位}
    \subsection{马克思列宁主义、毛泽东思想的继承和发展}
        是马克思列宁主义基本原理与当代中国实际和时代特征相结合的产物,是马克思列宁主义、毛泽东思想的继承和发展,是全党全国人民集体智慧的结晶。

    \subsection{中国特色社会主义理论体系的开山之作}
        响亮提出走自己的路、建设有中国特色的社会主义的伟大号召,从此中国特色社会主义称为我们党全部理论和实践一以贯之的主题。

    \subsection{改革开放和社会主义现代化建设的科学指南}
        邓小平理论是中国共产党和中国人民宝贵的精神财富,是改革开放和社会主义现代化建设的科学指南,是党和国家必须长期坚持的指导思想。
\chapter{“三个代表”重要思想}
\emph{以江泽民为主要代表的中国共产党人,在世界社会主义陷入低谷时,坚决捍卫中国特色社会主义,并成功推向21世纪。}


\section{“三个代表”重要思想的形成}
    \subsection{“三个代表”重要思想的形成条件}
        \begin{enumerate}
            \item 在对冷战结束后国际局势科学判断的基础上形成的:世界多极化和经济全球化的趋势在曲折中发展,和平与发展仍是时代的主题。
            \item 在科学判断党的历史方位和总结历史经验的基础上提出的
            \item 在建设中国特色社会主义伟大实践的基础上形成的
        \end{enumerate}

    \subsection{“三个代表”重要思想的形成过程}
        \begin{itemize}
            \item 2000年2月25日,江泽民在广东考察工作时,从全面总结当的历史经验和如何适应新形势新任务的要求出发,首次对“三个代表”进行了比较全面的阐述。
            \item 2001年7月1日,江泽民在庆祝中国共产党成立80周年大会上的讲话中全面阐述了“三个代表”重要思想的科学内涵和基本内容。
            \item 党的十六大将“三个代表”重要思想与马克思列宁主义、毛泽东思想和邓小平理论一道确立为党必须长期坚持的指导思想,并写入党章。
        \end{itemize}


\section{“三个代表”重要思想的核心观点和主要内容}
    \emph{“中国共产党必须始终代表中国先进生产力的发展要求,代表中国先进文化的前进方向,代表中国最广大人民的根本利益。”}

    \subsection{“三个代表”重要思想的核心观点}
        \begin{itemize}
            \item 始终代表中国先进生产力的发展要求
            \item 始终代表中国先进文化的前进方向
            \item 始终代表中国最广大人民的根本利益
        \end{itemize}

    \subsection{“三个代表”重要思想的主要内容}
        \begin{itemize}
            \item 发展是党执政兴国的第一要务
            \item 建立社会主义市场经济体制
            \item 全面建设小康社会
            \item 建设社会主义政治文明
            \item 推进党的建设新的伟大工程
        \end{itemize}


\section{“三个代表”重要思想的历史地位}
    “三个代表”重要思想在邓小平理论的基础上,进一步回答了什么是社会主义、怎样建设社会主义的问题,创造性地回答了建设什么样的党、怎样建设党的问题,深化了对中国特色社会主义的认识。
    \subsection{中国特色社会主义理论体系的接续发展}
        这一思想把新时期党的建设目标、任务和要求,提到了一个新的高度,具有鲜明的时代特征,从根本上回答了在充满希望与挑战的新世纪,\emph{要把我们党建设成为一个什么样的党和怎样建设党}这样一个重大历史性问题。

    \subsection{加强和改进党的建设,推进中国特色社会主义视野的强大理论武器}

\chapter{科学发展观}

\section{科学发展观的形成}
    \subsection{科学发展观的形成条件}
        \begin{enumerate}
            \item 科学发展观是在深刻把握我国基本国情和新的阶段性特征的基础上形成和发展的。
            \item 科学发展观是在深入总结改革开放以来特别是党的十六大依赖实践经验的基础上形成和发展的
            \item 科学发展观是在深刻分析国际形势、顺应世界发展趋势、借鉴国外发展经验的基础上形成和发展的
        \end{enumerate}

    \subsection{科学发展观的形成过程}
        科学发展观是在\textbf{抗击非典疫情}和\textbf{探索完善社会主义市场经济体制}的过程中逐步形成。


\section{科学发展观的科学内涵和主要内容}
    \subsection{科学发展观的科学内涵}
        \firstOrder{科学发展观,第一要义是发展,核心立场是以人为本,基本要求是全面协调可持续,根本方法是统筹兼顾。}
        \begin{enumerate}
            \item 推动经济社会发展是科学发展观的第一要义
            \item 以人为本是科学发展观的核心立场
            \item 全面协调可持续是科学发展观的基本要求
            \item 统筹兼顾是科学发展观的根本方法
        \end{enumerate}

    \subsection{科学发展观的主要内容}
        \begin{enumerate}
            \item 加快转变经济发展方式
            \item 发展社会主义民主政治
            \item 推进社会主义文化强国建设
            \item 构建社会主义和谐社会
            \item 推进生态文明建设
            \item 全面提高党的建设科学化水平
        \end{enumerate}


\section{科学发展观的历史地位}
    \subsection{中国特色社会主义理论体系的接续发展}
    \subsection{发展中国特色社会主义必须长期坚持的指导思想}


\part{习近平新时代中国特色社会主义思想}
\chapter{习近平新时代中国特色社会主义思想及其历史地位}


\section{中国特色社会主义进入新时代}
    \subsection{历史性成就和历史性变革}
        党的十八大以来,解决了许多长期想解决而没有解决的难题,办成了许多过去想办而没有办成的大事,推动党和国家事业取得了全方位的、开创性的历史性成就。
        \subsubsection{历史性成就}
        \begin{enumerate}
            \item 经济建设取得重大成就
            \item 全面深化改革取得重大突破
            \item 民主法治建设迈出重大步伐
            \item 思想文化建设取得重大进展
            \item 人民生活不断改善
            \item 生态文明建设成效显著
            \item 强军兴军开创新局面
            \item 港澳台工作取得新进展
            \item 全方位外交布局深入展开
            \item 全面从严治党成效显著
        \end{enumerate}

        \subsubsection{历史性变革}
        \begin{enumerate}
            \item 党的领导得到全面加强
            \item 贯彻新发展理念,发展观念不正确、发展方式粗放的状况得到明显改变
            \item 全面深化改革,各方面体制机制弊端阻碍发展活力和社会活力的状况得到明显改变
            \item 推进依法治国,有法不依、执法不严、司法不公问题严重的状况得到明显改变
            \item 党对意识形态工作的领导,社会思想舆论环境混乱的状况得到明显改变
            \item 推进生态文明建设,忽视生态环境保护、生态环境恶化的状况得到明显改变
            \item 推进国防和军队现代化,人民军队中一度存在的不良政治状况得到明显改变
            \item 推进中国特色大国外交,在国际力量对比中面临的不利状况得到明显改变
            \item 推进全面从严治党,管党治党宽松状况得到明显改变。
        \end{enumerate}

    \subsection{社会主要矛盾的变化}
        \subsubsection{旧版本}
        \begin{itemize}
            \item 1956年党的八大指出,我国的主要矛盾是人民对于建立先进的工业国的要求同落后的农业国的现实之间的矛盾;是人民对于经济文化迅速发展的需要同当前经济文化不能满足人民需要的状况之间的矛盾。
            \item 1981年十一届六中全会指出,我国所要解决的主要矛盾,是人民日益增长的物质文化需要同落后的社会生产之间的矛盾。
        \end{itemize}

        党的十九大明确指出,我国社会主要矛盾已经转化为人民日益增长的美好生活需要和不平衡不充分的发展之间的矛盾。

        主要依据有以下三个方面:
        \begin{enumerate}
            \item 经过改革开放40年的发展,我国社会生产力水平总体上显著提高,很多方面进入世界前列。这说明,我国进入社会主义初级阶段以来的“落后的社会生产”已经发生了新的阶段性变化。
            \item 人民生活水平显著提高,对美好生活的向往更加强烈,不仅对物质文化生活提出了更高要求,而且在民主、政治、公平、正义、安全、环境等方面的要求日益增长。这说明,人民群众对于日益增长的“物质文化需要”层次更高、内容范围更广,出现了阶段性的新特征。
            \item 影响满足人们们好生活需要的因素很多,但主要是发展的不平衡不充分问题。
            \begin{itemize}
                \item 不平衡,主要指各区域各领域各方面发展不平衡,制约了全国发展水平提升。
                \item 不充分,主要指一些地区、一些领域、一些方面还存在发展不足的问题,发展的任务仍然很重。
            \end{itemize}
        \end{enumerate}

        我国社会主要矛盾的变化,没有改变我们对我国社会主义所处历史阶段的判断,\emph{我国仍处于并将长期处于社会主义初级阶段的基本国情没有变,我国是世界最大发展中国家的国际地位没有变。}

    \subsection{新时代的内涵和意义}
        经过长期努力,中国特色社会主义进入了新时代,这是我国发展新的历史方位。
        \subsubsection{内涵}
        \begin{enumerate}
            \item 这个新时代是承前启后、继往开来,在新的历史条件下继续夺取中国特色社会主义伟大胜利的时代。
            \item 这个新时代是决胜全面建成小康社会、进而全面建设社会主义现代化强国的时代。
            \item 这个新时代是全国各族人民团结奋斗、不断创造美好生活、逐步实现全体人民共同富裕的时代。
            \item 这个新时代是全体中华儿女勠力同心、奋力实现中华民族伟大复兴中国梦的时代。
            \item 这个新时代是我国日益走近世界舞台中央、不断为人类作出更大贡献的时代。
        \end{enumerate}

        \subsubsection{意义}
        \begin{enumerate}
            \item 从中华民族复兴的历史进程来看,进入新时代意味着近代以来久经磨难的中华民族迎来了从站起来、富起来到强起来的伟大飞跃,迎来了实现中华民族伟大复兴的光明前景。新中国的成立使中国人民站起来,改革开放使中国人民逐步富起来,新时代中华民族要实现强起来的宏伟目标。
            \item 从科学社会主义发展进程看,进入新时代意味着科学社会主义在21世纪的中国焕发出强大生机活力,在世界上高高举起了中国特色社会主义伟大旗帜。
            \item 从人类文明进程看,进入新时代意味着中国特色社会主义道路、理论、制度、文化不断发展,为解决人类问题贡献了中国智慧和中国方案。
        \end{enumerate}


\section{新时代中国特色社会主义思想的主要内容}
    \subsection{核心要义和丰富内涵}
        \subsubsection{核心要义}
        \emph{坚持和发展中国特色社会主义,是改革开放以来我们党全部理论和实践的鲜明主题,也是习近平新时代中国特色社会主义思想的核心要义。}

        \subsubsection{丰富内涵}
        八个明确。似乎不是重点。
        \begin{enumerate}
            \item 明确坚持和发展中国特色社会主义:总任务是实现社会主义现代化和中华民族伟大复兴,在全面建成小康社会的基础上,分两步走在本世纪中叶建成富强民主文明和谐美丽的社会主义现代化强国
            \item 明确新时代我国社会主要矛盾,必须坚持以人民为中心的发展思想,不断促进人的全面发展、全体人民共同富裕
            \item 明确中国特色社会主义事业总体布局是“五位一体”,战略布局是“四个全面”,强调坚定道路自信、理论自信、制度自信、文化自信
            \item 明确全面深化改革总目标:完善和发展中国特色社会主义制度、推进国家治理体系和治理能力现代化
            \item 明确全面推进全面依法治国总目标:建设中国特色社会主义法治体系、建设社会主义法治国家
            \item 明确党在新时代的强军目标:建设一支听党指挥、能打胜仗、作风优良的人民军队,把人民军队建设成为世界一流军队
            \item 明确中国特色大国外交要推动构建新型国际关系,推动构建人类命运共同体
            \item 明确中国特色社会主义最本质的特征是中国共产党的领导,中国特色社会主义制度最大的优势是中国共产党的领导,提出新时代党的建设总要求,突出政治建设在党的建设中的重要地位
        \end{enumerate}

    \subsection{坚持和发展中国特色社会主义的基本方略}
        \subsubsection{十四个坚持}
        似乎不是重点。
        \begin{enumerate}
            \item 坚持党对一切工作的领导。
            \item 坚持以人民为中心。
            \item 坚持全面深化改革。
            \item 坚持新发展理念:创新协调绿色开放共享。
            \item 坚持人民当家做主。
            \item 坚持全面依法治国。
            \item 坚持社会主义核心价值体系。
            \item 坚持在发展中保障和改善民生。
            \item 坚持人与自然和谐共生。
            \item 坚持总体国家安全观。
            \item 坚持党对人民军队的绝对领导。
            \item 坚持一国两制和推进祖国统一。
            \item 坚持推动构建人类命运共同体。
            \item 坚持全面从严治党。
        \end{enumerate}

    八个明确是指导思想层面的表述,重点讲的是怎么看,回答的是新时代坚持和发展什么样的中国特色社会主义的问题;十四个坚持是行动纲领,重点讲的是怎么办,回答的是新时代怎样坚持和发展中国特色社会主义的问题。八个明确和十四个坚持体现了习近平新时代中国特色社会主义思想理论与实践的统一。


\section{新时代中国特色社会主义思想的历史地位}
    \emph{习近平新时代中国特色社会主义思想是马克思主义中国化的最新成果,是中国特色社会主义理论体系的重要组成部分,是当代中国马克思主义、21世纪马克思主义,是党和国家必须长期坚持并不断发展的指导思想,是全党全国人民为实现中华民族伟大复兴而奋斗的行动指南。}
    \subsection{马克思主义中国化的最新成果}
        与马克思列宁主义、毛泽东思想、邓小平理论、“三个代表”重要思想、科学发展观既一脉相承又与时俱进,是马克思主义中国化的飞跃,是当代中国马克思主义、21世纪马克思主义。

    \subsection{新时代的精神旗帜}
        着眼统揽伟大斗争、伟大工程、伟大事业、伟大梦想。

        党的十九大把习近平新时代中国特色社会主义思想确立为党的指导思想;十三届全国人大一次会议把这一思想载入宪法。

    \subsection{实现中华民族伟大复兴的行动指南}

\chapter{坚持和发展中国特色社会主义的总任务}

\section{是吸纳中华民族伟大复兴的中国梦}
    \subsection{中华民族近代以来最伟大的梦想}
        坚持和发展中国特色社会主义的总任务,是实现社会主义现代化和中华民族伟大复兴,在全面建成小康社会的基础上,分两步走在本世纪中叶建成富强民主文明和谐美丽的社会主义现代化强国。中国梦是中华民族伟大复兴的形象表达。

    \subsection{中国梦的科学内涵}
        \emph{中国梦的本质是国家富强、民族振兴、人民幸福。}
        \begin{itemize}
            \item 国家富强,是指我国综合国力进一步增强,中国特色社会主义事业进一步完善和发展。(经济、科技、政治、文化、社会、生态)
            \item 民族振兴,就是通过自身地不断发展与强大,继承并创造中华民族的优秀文化以及先进的文明成果,进而使中华民族再次处于世界领先的地位,再次以高昂的姿态屹立于世界民族之林。
            \item 人民幸福,就是人民权利保障更加充分、人人得享共同发展。
        \end{itemize}

        国家富强、民族振兴是人民幸福的基础和保障。人民幸福是国家富强、民族振兴的根本出发点和落脚点。

    \subsection{奋力实现中国梦}
        \emph{实现中国梦必须走中国道路、弘扬中国精神、凝聚中国力量}
        \begin{itemize}
            \item 实现中国梦必须走中国道路,这就是中国特色社会主义道路。
            \item 实现中国梦必须弘扬中国精神,这就是以爱国主义为核心的民族精神和以改革创新为核心的时代精神。
            \item 实现中国梦必须凝聚中国力量,这就是全国各族人民大团结的力量。
            \item 实现中华民族伟大复兴是海内外中华儿女的共同梦想。
            \item 实干才能梦想成真。
            \item 实现中国梦任重而道远,需要锲而不舍、驰而不息的艰苦努力。
            \item 实现中国梦需要和平,只有和平才能实现梦想。
        \end{itemize}


\section{建成社会主义现代化抢过的战略安排}
    \subsection{开启全面建设社会主义现代化强国的新征程}
    \subsection{实现社会主义现代化强国“两步走”战略}
        \begin{enumerate}
            \item 从2020年到2035年,基本实现社会主义现代化的目标要求
            \begin{itemize}
                \item 在经济建设方面,我国经济实力、科技实力将大幅跃升,跻身创新型国家前列。
                \item 在政治建设方面,人民平等参与、平等发展权利得到充分保障,法治国家、法治政府、法治社会基本建成,各方面制度更加完善,国家治理体系和治理能力现代化基本实现。
                \item 在文化建设方面,社会文明程度达到新高度,国家文化软实力显著增强,中华文化影响更加广泛深入。
                \item 在民生和社会建设方面,人民生活更为宽裕,中等收入群体比例明显提高,城乡区域发展差距和居民生活水平差距显著缩小,基本公共服务均等化基本实现,全体人民共同富裕迈出坚实步伐。
                \item 在生态文明建设方面,生态环境根本好转,美丽中国目标基本实现。
            \end{itemize}
            \item 从2035年到本世纪中叶,建成社会主义现代化强国的目标要求
        \end{enumerate}


\end{spacing}
\end{document}