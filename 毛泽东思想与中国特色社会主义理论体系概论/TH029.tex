\documentclass[oneside]{book}
\usepackage{xeCJK}
\usepackage{amsmath}
\usepackage{mathtools}
\usepackage{listings} % lstlist插入代码
\usepackage{booktabs}
\usepackage{ulem}
\usepackage{enumerate}
\usepackage{amsfonts}
\usepackage{amssymb}
\usepackage{setspace} % spacing环境设置行间距
\usepackage[ruled, vlined]{algorithm2e} % 算法与伪代码 
\usepackage{bm} % 数学公式中的加粗
\usepackage{pifont} % 打圈的数字。172-211。\ding
\usepackage{graphicx}
\usepackage{float}
\usepackage[dvipsnames]{xcolor}
\usepackage{indentfirst}
\usepackage{ulem} %\sout{}打删除线
\normalem % 使用默认normalem
\usepackage{lmodern}
\usepackage{subcaption}
\usepackage[colorlinks, linkcolor=blue]{hyperref}
\usepackage{cleveref}
\usepackage[a4paper]{geometry}
\usepackage{titlesec}

\usepackage{zhnumber}
\renewcommand{\thesubsection}{\zhnum{subsection}}
\renewcommand{\thesection}{\zhnum{section}}
\renewcommand{\thechapter}{\zhnum{chapter}}
\renewcommand{\thepart}{\zhnum{part}}

\renewcommand{\contentsname}{目录}

\titleformat{\part}{\centering\Huge\bfseries}{第\thepart 部分}{1em}{}
\titleformat{\chapter}{\centering\Huge\bfseries}{第\thechapter 章}{1em}{}
\titleformat{\section}{\LARGE\bfseries}{第\thesection 节}{1em}{}
\titleformat{\subsection}{\Large\bfseries}{\thesubsection 、}{1em}{}

\usepackage{fancyhdr}
\pagestyle{plain}

\title{\Huge\textbf{毛泽东思想和中国特色\\社会主义理论体系概论}}
\author{背不完书的\textsc{YBiuR}}
\date{2020-2021学年{} 春季学期}


\begin{document}
\begin{spacing}{1.5}
% \setlength{\leftskip}{2em}
\setlength{\parindent}{2em}

\frontmatter
\maketitle
\chapter*{前言}
\section*{碎碎念}
\paragraph{}\verb|aReasonablyLongCourseName|: \textsc{TH029 Introduction to Mao Zedong's Thoughts and Theoretical System of Socialism with Chinese Characteristics}.
\paragraph{}排版中文太难力。
\paragraph{}讲正事了。


\section*{马克思主义中国化}
    \textbf{马克思主义中国化}就是把马克思主义基本原理同中国具体实际和时代特征结合起来,运用马克思主义的立场、观点、方法研究和解决中国革命、建设、改革中的实际问题,使之成为具有中国特色、中国风格、中国气派的马克思主义。

    在中国革命、建设、改革的历史进程中,马克思主义中国化实现了\textbf{两次历史性飞跃}:
    \begin{enumerate}
        \item 第一次历史性飞跃发生在\emph{新民主主义革命时期},形成了\emph{毛泽东思想}。
        \item 第二次历史性飞跃发生在社会主义进入\emph{改革开放}的新时期,形成了\emph{中国特色社会主义理论体系}。
    \end{enumerate}


\tableofcontents

\mainmatter
\part{毛泽东思想}
\chapter{毛泽东思想及其历史地位}
\par 毛泽东思想是在革命和建设的长期实践中,以毛泽东为主要代表的中国共产党人,根据马克思列宁主义基本原理,形成的适合中国情况的科学指导思想,是被实践证明了的关于中国革命和建设的正确的理论原则和经验总结,是中国共产党集体智慧的结晶。

\par 毛泽东思想是马克思主义中国化的第一个重大理论成果,是马克思列宁主义在中国的运用和发展,是被实践证明了的关于中国革命和建设的正确的理论原则和经验总结,是中国国共产党集体智慧的结晶,是党必须长期坚持的指导思想。


\section{毛泽东思想的形成和发展}

    \subsection{毛泽东思想形成发展的历史条件}
    Nothing here.

    \subsection{毛泽东思想形成发展的过程}
    \firstOrder{毛泽东思想的形成。}
    书本中萌芽时期和形成时期统称为形成时期。
    \begin{enumerate}
        \item \firstOrder{萌芽.} 第一次国内革命战争时期。这一时期的著作
        \begin{itemize}
            \item 《中国社会各阶级的分析》
            \item 《湖南农民运动考察报告》
        \end{itemize}
        分析了中国社会各阶级在革命中的地位和作用,提出了\emph{新民主主义革命}的基本思想。
        \item \firstOrder{形成.} 土地革命战争时期。著作:
        \begin{itemize}
            \item 《中国的红色政权为什么能够存在?》
            \item 《井冈山的斗争》
            \item 《星星之火,可以燎原》
            \item 《反对本本主义》
        \end{itemize}
        \par 在同党内一度盛行的把马克思主义教条化、把共产国际决议和苏联经验神圣化的错误倾向的斗争中,提出并阐述了\emph{农村包围城市}、\emph{武装夺取政权}的思想,标志着毛泽东思想的初步形成。
        \item \firstOrder{成熟.} 著作:
        \begin{itemize}
            \item 《实践论》、《矛盾论》
            \item 《<共产党人>发刊词》
            \item 《中国革命和中国共产党》
            \item 《新民主主义论》
            \item 《论联合政府》
        \end{itemize}
        \par 运用马克思主义的认识论和辩证法,系统分析了党内“左”的和右的错误地思想根源。
        \par 新民主主义革命理论的系统阐述,实现了马克思主义与中国革命实践相结合的历史性飞跃,标志着毛泽东思想得到多方面展开而趋于成熟。
        \par \emph{1945年党的七大将毛泽东思想写入党章,确立为党必须长期坚持的指导思想。}
        \item \firstOrder{继续发展.} 著作:
        \begin{itemize}
            \item 《在中国共产党第七届中央委员会第二次全体会议上的报告》
            \item 《论人民民主专政》
            \item 《论十大关系》
            \item 《关于正确处理人民内部矛盾的问题》
        \end{itemize}
        先后提出\emph{人民民主专政理论}、\emph{社会主义改造理论}、\emph{关于严格区分和正确处理两类矛盾的学说}特别是正确处理人民内部矛盾的理论。
    \end{enumerate}


\section{毛泽东思想的主要内容和活的灵魂}

    \subsection{毛泽东思想的主要内容}
    \begin{enumerate}
        \item 新民主主义革命理论:第\ref{Chapter:新民主主义革命理论}章节
        \item 社会主义革命和社会主义建设理论:第\ref{Chapter:社会主义改造理论}章节
        \item 革命军队建设和军事战略的理论
        \item 政策和策略的理论
        \item 思想政治工作和文化工作的理论
        \item 党的建设理论
    \end{enumerate}

    \subsection{毛泽东思想活的灵魂}
    \firstOrder{贯穿于毛泽东思想各个组成部分的立场、观点和方法,是毛泽东思想的活的灵魂,有三个基本方面,即实事求是,群众路线,独立自主。}

    \subsubsection{1. 实事求是} 实事求是,就是一切从实际出发,理论联系实际,坚持在实践中检验真理和发展真理。

    \subsubsection{2. 群众路线} 群众路线,就是一切为了群众,一切依靠群众,从群众中来,到群众中去,把党的正确主张变为群众的自觉行动。

    \subsubsection{3. 独立自主} 独立自主,就是坚持独立思考,走自己的路,就是坚定不移地维护民族独立、捍卫国家主权,把立足点放在依靠自己力量的基础上,同时积极争取外援,开展国际经济文化交流,学习外国一切对我们有益的先进事物。


\section{毛泽东思想的历史地位}

    \subsection{马克思主义中国化的第一个重大理论成果}
    毛泽东思想是马克思主义中国话第一次历史性飞跃的理论成果。在马克思主义中国话的历史进程中,毛泽东思想为中国特色社会主义理论体系的形成奠定了理论基础。

    \subsection{中国革命和建设的科学指南}

    \subsection{中国共产党和中国人民宝贵的精神财富}
\chapter{新民主主义革命理论}\label{Chapter:新民主主义革命理论}


\section{新民主主义革命理论形成的依据}

    \subsection{近代中国国情和中国革命的时代特征}
    近代中国占支配地位的主要矛盾是帝国主义和中华民族的矛盾、封建主义和人民大众的矛盾,而\emph{帝国主义和中华民族的矛盾是各种矛盾中最主要的矛盾}。这决定了近代中国革命的\emph{根本任务是推翻帝国主义、封建主义和官僚资本主义的统治(三座大山)}。

    以五四运动的爆发为标志,中国资产阶级民主革命进入新民主主义革命的崭新阶段\footnote{新民主主义革命自1919年开始,旧民主主义到1922年结束。}。

    无产阶级开始以独立的政治力量登上历史舞台,由自在阶级转变为自为的阶级。马克思主义在中国得到广泛传播,逐步成为中国革命的指导思想。

    它推翻帝国主义、封建主义和官僚资本主义的反动统治,但不破坏参加反帝反封建的资本主义成分。

    中国革命要分两步走,第一步是完成反帝反封建的新民主主义革命任务,第二步是完成社会主义革命任务,这是性质不同但又相互联系的两个革命过程。

    \subsection{新民主主义革命理论的实践基础}
    不考。


\section{新民主主义革命的总路线和基本纲领}

    \subsection{新民主主义革命的总路线}
    1939年,毛泽东在《中国革命和中国共产党》一文中首次提出“新民主主义革命”的科学概念。

    1948年,他在《在晋绥干部会议上的讲话》中完整地表述了总路线的内容:

    \begin{center}
        \emph{无产阶级领导的,人民大众的,反对帝国主义、封建主义和官僚资本主义的革命。}
    \end{center}

    \begin{enumerate}
        \item \textbf{新民主主义革命的对象.}
        分清敌友,这是革命的\textbf{首要问题}
        \begin{itemize}
            \item 帝国主义。是中国革命的\textbf{首要对象}。
            \item 封建地主阶级。
            \item 官僚资本主义。
        \end{itemize}
        \item \textbf{新民主主义革命的动力.}
        新民主主义革命的动力包括无产阶级、农民阶级、城市小资产阶级和民族资产阶级。
        \begin{itemize}
            \item 无产阶级是中国革命\textbf{最基本的动力}。
            \item 农民是中国革命的主力军。
            \item 城市小资产阶级是无产阶级的可靠同盟者。包括广大的知识分子、小商人、手工业者和自由职业者。
            \item 民族资产阶级也是中国革命的动力之一。\footnote{注意中国革命的对象不是所有剥削阶级。城市小资产阶级和民族资产阶级不在革命对象范围内。}
        \end{itemize}
        \item \textbf{新民主主义革命的领导力量.}
        无产阶级的领导权是中国革命的\textbf{中心问题},也是新民主主义革命理论的\textbf{核心问题}。

        区别新旧两种不同范畴的民主主义革命的根本标准是,\emph{革命的领导权掌握在无产阶级手中还是掌握在资产阶级手中}。

        中国无产阶级具有一些特点和优点
        \begin{enumerate}
            \item 从诞生之日起,就身受外国资本主义、本国封建势力和资产阶级的三重压迫。
            \item 分布集中,有利于无产阶级队伍的组织和团结,有利于革命思想的传播和强大革命力量的形成。
            \item 和农民有着天然的联系,这是的无产阶级便于和农民结成亲密的联盟。
        \end{enumerate}
        无产阶级及其政党——中国共产党的领导,是中国革命取得胜利的根本保证。

        无产阶级及其政党实现对各革命阶级的领导,必须建立以工农联盟为基础的广泛地统一战线,这是实现领导权的关键。

        \item \textbf{新民主主义革命的性质和前途.}
        近代中国半殖民地半封建社会的性质和中国革命的历史人物,决定了中国革命的性质不是无产阶级社会主义革命,而是资产阶级民主主义革命。

        革命的前途是社会主义而不是资本主义。
    \end{enumerate}

    \subsection{新民主主义的基本纲领}
    \emph{只需要知道有三个纲领就行。}
    \begin{enumerate}
        \item \textbf{政治纲领}:推翻帝国主义和封建主义的统治,建立一个无产阶级领导的,以工农联盟为基础的、各革命阶级联合专政的新民主主义的共和国。
        \item \textbf{经济纲领}:没收封建地主阶级的土地归农民所有,没收官僚资产阶级的垄断资本归新民主主义的国家所有,保护民族工商业。
        \item \textbf{文化纲领}:无产阶级领导的人民大众的反帝反封建的文化,即民族的科学的大众的文化。
    \end{enumerate}


\section{新民主主义革命的道路和基本经验}

    \subsection{新民主主义革命的道路}
    即农村包围城市、武装夺取政权的革命道路。
    \paragraph{新民主主义革命道路的内容及意义.} 中国革命走农村包围城市、武装夺取政权的道路,根本在于处理好\textbf{土地革命}、\textbf{武装斗争}、\textbf{农村革命根据地建设}三者之间的关系。
    \begin{enumerate}
        \item 土地革命是民主革命的基本内容。
        \item 武装斗争是中国革命的主要形式,是农村根据地建设和土地革命的强有力保证。
        \item 农村革命根据地是中国革命的战略阵地,是进行武装斗争和开展土地革命的依托。
    \end{enumerate}

    \subsection{新民主主义革命的三大法宝}
    毛泽东在《<共产党人>发刊词》中,把\textbf{统一战线}、\textbf{武装斗争}、\textbf{党的建设}比作党在中国革命中战胜敌人的三个主要的法宝。
    \begin{enumerate}
        \item 统一战线
        \item 武装斗争
        \item 党的建设
    \end{enumerate}
\chapter{社会主义改造理论}\label{Chapter:社会主义改造理论}


\section{从新民主主义到社会主义的转变}

    \subsection{新民主主义社会是一个过渡性的社会}
    新民主主义社会不是一个独立的社会形态,而是由新民主主义向社会主义转变的过渡性社会形态。\footnote{持续时间自新中国成立到社会主义改造基本完成。}

    \subsection{党在过渡时期的总路线及其理论依据}
        \subsubsection{1. 党在过渡时期的总路线的提出}
        党在过渡时期总路线的主要内容被概括为\textbf{“一化三改”}:一化即社会主义工业化;三改即对个体农业、手工业和资本主义工商业的社会主义改造。

        这是一条社会主义建设和社会主义改造同时并举的路线,体现了社会主义工业化和社会主义改造的紧密结合。


\section{社会主义改造道路和历史经验}

    \subsection{适合中国特点的社会主义改造道路}
        \subsubsection{农业、手工业的社会主义改造}
        农业:
        \begin{enumerate}
            \item 积极引导农民组织起来,走互助合作道路。
            \item 遵循自愿互利、典型示范和国家帮助的原则,以互助合作的优越性吸引农民走互助合作道路。
            \item 正确分析农村的阶级和阶层状况,制定正确的阶级政策。
            \item 坚持积极引导、稳步前进的方针,采取循序渐进的步骤。
        \end{enumerate}

        手工业:采取积极领导、稳步前进的方针。

        \subsubsection{资本主义工商业的社会主义改造}
        \begin{enumerate}
            \item 用和平赎买的方法改造资本主义工商业。
            \item 采取从低级到高级的国家资本主义的过渡形式。\footnote{这些企业的利润,按国家所得税、企业公积金、工人福利费、资方红利这四个方面进行分配,即当时所说的“四马分肥”。}
            \item 把资本主义工商业者改造为自食其力的社会主义劳动者。
        \end{enumerate}

    \subsection{社会主义改造的历史经验}
    \begin{enumerate}
        \item 坚持社会主义工业化建设与社会主义改造同时并举。
        \item 采取积极引导、逐步过渡的方式。
        \item 用和平方法进行改造。
    \end{enumerate}


\section{社会主义制度在中国的确立}

    \subsection{社会主义制度的基本确立及其理论依据}
    1956年底,我国对农业、手工业和资本主义工商业的社会主义改造基本完成,社会主义公有制已成为我国社会的经济基础,标志着中国历史上长达数千年的阶级剥削制度的结束和社会主义基本制度的确立。

    1954年9月,《中华人民共和国宪法》制定并颁布施行,明确规定了我国人民民主专政的国体和人民代表大会的政体。人民代表大会制度这一根本政治制度、中国共产党领导的多党合作和政治协商制度、民族区域自治制度这些基本政治制度的确立,表明我国有一个新民主主义的国家转变为社会主义国家。

    \subsection{确立社会主义基本制度的重大意义}
    社会主义基本制度的确立是中国历史上最深刻最伟大的社会变革,为当代中国一切发展进步奠定了制度基础,也为中国特色社会主义制度的创新和发展提供了重要前提。
\chapter{社会主义建设道路初步探索的理论成果}

\section{初步探索的重要理论成果}
\begin{enumerate}
    \item 调动一切积极因素为社会主义事业服务。
    \item 正确认识和处理社会主义矛盾的思想。
    \item 走中国工业化道路的思想。
\end{enumerate}

    \subsection{调动一切积极因素为社会主义事业服务}
    1956年4月和5月,毛泽东先后在中央政治局扩大会议和最高国务会议上,做了《论十大关系》的报告。

    《论十大关系》标志着党探索中国社会主义建设道路的良好开端。

    《论十大关系》确立了一个基本方针:\emph{努力把党内党外国内国外的一切积极的因素,直接的、间接的积极因素全部调动起来,为社会主义建设服务}。

    \subsection{正确认识和处理社会主义社会矛盾的思想}
    毛泽东在1957年2月所作的《关于正确处理人民内部矛盾的问题》的报告中,系统论述了社会主义社会矛盾的理论。
    
    有两类社会矛盾:\textbf{敌我矛盾}和\textbf{人民内部矛盾},这是两类性质完全不同的矛盾。敌我之间和人民内部这两类矛盾的性质不同,解决的方法也不同:采用\textbf{专政}和\textbf{民主}两种不同的方法。

    \subsection{走中国工业化道路的思想}
    毛泽东在《论十大关系》中论述的第一大关系,便是\emph{重工业、轻工业和农业的关系}。要走一条有别于苏联的中国工业化道路。

    提出了\emph{以农业为基础,以工业为主导,以农轻重魏旭发展国民经济的总方针}。


\section{初步探索的意义和经验教训}

    \subsection{初步探索的意义}
    \begin{enumerate}
        \item 巩固和发展了我国的社会主义制度。
        \item 为开创中国特色社会主义提供了宝贵经验、理论准备、物质基础。
        \item 丰富了科学社会主义的理论和实践。
    \end{enumerate}

    \subsection{初步探索的经验教训}
    \begin{enumerate}
        \item 必须把马克思主义与中国实际相结合,探索符合中国特点的社会主义建设道路。
        \item 必须正确认识社会主义社会的主要矛盾和根本任务,集中力量发展生产力。
        \item 必须从实际出发进行社会主义建设,建设规模和速度要和国力相适应,不能急于求成。
        \item 必须发展社会主义民主,健全社会主义法制。
        \item 必须坚持党的民主集中制和集体领导制度,加强执政党建设。
        \item 必须坚持对外开放,借鉴和吸收人类文明成果建设社会主义,不能关起门来搞建设。
    \end{enumerate}

\end{spacing}
\end{document}