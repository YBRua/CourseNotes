\chapter{习近平新时代中国特色社会主义思想及其历史地位}


\section{中国特色社会主义进入新时代}
    \subsection{历史性成就和历史性变革}
        党的十八大以来,解决了许多长期想解决而没有解决的难题,办成了许多过去想办而没有办成的大事,推动党和国家事业取得了全方位的、开创性的历史性成就。
        \subsubsection{历史性成就}
        \begin{enumerate}
            \item 经济建设取得重大成就
            \item 全面深化改革取得重大突破
            \item 民主法治建设迈出重大步伐
            \item 思想文化建设取得重大进展
            \item 人民生活不断改善
            \item 生态文明建设成效显著
            \item 强军兴军开创新局面
            \item 港澳台工作取得新进展
            \item 全方位外交布局深入展开
            \item 全面从严治党成效显著
        \end{enumerate}

        \subsubsection{历史性变革}
        \begin{enumerate}
            \item 党的领导得到全面加强
            \item 贯彻新发展理念,发展观念不正确、发展方式粗放的状况得到明显改变
            \item 全面深化改革,各方面体制机制弊端阻碍发展活力和社会活力的状况得到明显改变
            \item 推进依法治国,有法不依、执法不严、司法不公问题严重的状况得到明显改变
            \item 党对意识形态工作的领导,社会思想舆论环境混乱的状况得到明显改变
            \item 推进生态文明建设,忽视生态环境保护、生态环境恶化的状况得到明显改变
            \item 推进国防和军队现代化,人民军队中一度存在的不良政治状况得到明显改变
            \item 推进中国特色大国外交,在国际力量对比中面临的不利状况得到明显改变
            \item 推进全面从严治党,管党治党宽松状况得到明显改变。
        \end{enumerate}

    \subsection{社会主要矛盾的变化}
        \subsubsection{旧版本}
        \begin{itemize}
            \item 1956年党的八大指出,我国的主要矛盾是人民对于建立先进的工业国的要求同落后的农业国的现实之间的矛盾;是人民对于经济文化迅速发展的需要同当前经济文化不能满足人民需要的状况之间的矛盾。
            \item 1981年十一届六中全会指出,我国所要解决的主要矛盾,是人民日益增长的物质文化需要同落后的社会生产之间的矛盾。
        \end{itemize}

        党的十九大明确指出,我国社会主要矛盾已经转化为人民日益增长的美好生活需要和不平衡不充分的发展之间的矛盾。

        主要依据有以下三个方面:
        \begin{enumerate}
            \item 经过改革开放40年的发展,我国社会生产力水平总体上显著提高,很多方面进入世界前列。这说明,我国进入社会主义初级阶段以来的“落后的社会生产”已经发生了新的阶段性变化。
            \item 人民生活水平显著提高,对美好生活的向往更加强烈,不仅对物质文化生活提出了更高要求,而且在民主、政治、公平、正义、安全、环境等方面的要求日益增长。这说明,人民群众对于日益增长的“物质文化需要”层次更高、内容范围更广,出现了阶段性的新特征。
            \item 影响满足人们们好生活需要的因素很多,但主要是发展的不平衡不充分问题。
            \begin{itemize}
                \item 不平衡,主要指各区域各领域各方面发展不平衡,制约了全国发展水平提升。
                \item 不充分,主要指一些地区、一些领域、一些方面还存在发展不足的问题,发展的任务仍然很重。
            \end{itemize}
        \end{enumerate}

        我国社会主要矛盾的变化,没有改变我们对我国社会主义所处历史阶段的判断,\emph{我国仍处于并将长期处于社会主义初级阶段的基本国情没有变,我国是世界最大发展中国家的国际地位没有变。}

    \subsection{新时代的内涵和意义}
        经过长期努力,中国特色社会主义进入了新时代,这是我国发展新的历史方位。
        \subsubsection{内涵}
        \begin{enumerate}
            \item 这个新时代是承前启后、继往开来,在新的历史条件下继续夺取中国特色社会主义伟大胜利的时代。
            \item 这个新时代是决胜全面建成小康社会、进而全面建设社会主义现代化强国的时代。
            \item 这个新时代是全国各族人民团结奋斗、不断创造美好生活、逐步实现全体人民共同富裕的时代。
            \item 这个新时代是全体中华儿女勠力同心、奋力实现中华民族伟大复兴中国梦的时代。
            \item 这个新时代是我国日益走近世界舞台中央、不断为人类作出更大贡献的时代。
        \end{enumerate}

        \subsubsection{意义}
        \begin{enumerate}
            \item 从中华民族复兴的历史进程来看,进入新时代意味着近代以来久经磨难的中华民族迎来了从站起来、富起来到强起来的伟大飞跃,迎来了实现中华民族伟大复兴的光明前景。新中国的成立使中国人民站起来,改革开放使中国人民逐步富起来,新时代中华民族要实现强起来的宏伟目标。
            \item 从科学社会主义发展进程看,进入新时代意味着科学社会主义在21世纪的中国焕发出强大生机活力,在世界上高高举起了中国特色社会主义伟大旗帜。
            \item 从人类文明进程看,进入新时代意味着中国特色社会主义道路、理论、制度、文化不断发展,为解决人类问题贡献了中国智慧和中国方案。
        \end{enumerate}


\section{新时代中国特色社会主义思想的主要内容}
    \subsection{核心要义和丰富内涵}
        \subsubsection{核心要义}
        \emph{坚持和发展中国特色社会主义,是改革开放以来我们党全部理论和实践的鲜明主题,也是习近平新时代中国特色社会主义思想的核心要义。}

        \subsubsection{丰富内涵}
        八个明确。似乎不是重点。
        \begin{enumerate}
            \item 明确坚持和发展中国特色社会主义:总任务是实现社会主义现代化和中华民族伟大复兴,在全面建成小康社会的基础上,分两步走在本世纪中叶建成富强民主文明和谐美丽的社会主义现代化强国
            \item 明确新时代我国社会主要矛盾,必须坚持以人民为中心的发展思想,不断促进人的全面发展、全体人民共同富裕
            \item 明确中国特色社会主义事业总体布局是“五位一体”,战略布局是“四个全面”,强调坚定道路自信、理论自信、制度自信、文化自信
            \item 明确全面深化改革总目标:完善和发展中国特色社会主义制度、推进国家治理体系和治理能力现代化
            \item 明确全面推进全面依法治国总目标:建设中国特色社会主义法治体系、建设社会主义法治国家
            \item 明确党在新时代的强军目标:建设一支听党指挥、能打胜仗、作风优良的人民军队,把人民军队建设成为世界一流军队
            \item 明确中国特色大国外交要推动构建新型国际关系,推动构建人类命运共同体
            \item 明确中国特色社会主义最本质的特征是中国共产党的领导,中国特色社会主义制度最大的优势是中国共产党的领导,提出新时代党的建设总要求,突出政治建设在党的建设中的重要地位
        \end{enumerate}

    \subsection{坚持和发展中国特色社会主义的基本方略}
        \subsubsection{十四个坚持}
        似乎不是重点。
        \begin{enumerate}
            \item 坚持党对一切工作的领导。
            \item 坚持以人民为中心。
            \item 坚持全面深化改革。
            \item 坚持新发展理念:创新协调绿色开放共享。
            \item 坚持人民当家做主。
            \item 坚持全面依法治国。
            \item 坚持社会主义核心价值体系。
            \item 坚持在发展中保障和改善民生。
            \item 坚持人与自然和谐共生。
            \item 坚持总体国家安全观。
            \item 坚持党对人民军队的绝对领导。
            \item 坚持一国两制和推进祖国统一。
            \item 坚持推动构建人类命运共同体。
            \item 坚持全面从严治党。
        \end{enumerate}

    八个明确是指导思想层面的表述,重点讲的是怎么看,回答的是新时代坚持和发展什么样的中国特色社会主义的问题;十四个坚持是行动纲领,重点讲的是怎么办,回答的是新时代怎样坚持和发展中国特色社会主义的问题。八个明确和十四个坚持体现了习近平新时代中国特色社会主义思想理论与实践的统一。


\section{新时代中国特色社会主义思想的历史地位}
    \emph{习近平新时代中国特色社会主义思想是马克思主义中国化的最新成果,是中国特色社会主义理论体系的重要组成部分,是当代中国马克思主义、21世纪马克思主义,是党和国家必须长期坚持并不断发展的指导思想,是全党全国人民为实现中华民族伟大复兴而奋斗的行动指南。}
    \subsection{马克思主义中国化的最新成果}
        与马克思列宁主义、毛泽东思想、邓小平理论、“三个代表”重要思想、科学发展观既一脉相承又与时俱进,是马克思主义中国化的飞跃,是当代中国马克思主义、21世纪马克思主义。

    \subsection{新时代的精神旗帜}
        着眼统揽伟大斗争、伟大工程、伟大事业、伟大梦想。

        党的十九大把习近平新时代中国特色社会主义思想确立为党的指导思想;十三届全国人大一次会议把这一思想载入宪法。

    \subsection{实现中华民族伟大复兴的行动指南}
