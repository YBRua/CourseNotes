\chapter{坚持和发展中国特色社会主义的总任务}

\section{是吸纳中华民族伟大复兴的中国梦}
    \subsection{中华民族近代以来最伟大的梦想}
        \firstOrder{坚持和发展中国特色社会主义的总任务,是实现社会主义现代化和中华民族伟大复兴},在全面建成小康社会的基础上,分两步走在本世纪中叶建成富强民主文明和谐美丽的社会主义现代化强国。中国梦是中华民族伟大复兴的形象表达。

    \subsection{中国梦的科学内涵}
        \firstOrder{中国梦的本质是国家富强、民族振兴、人民幸福。}
        \begin{itemize}
            \item 国家富强,是指我国综合国力进一步增强,中国特色社会主义事业进一步完善和发展。(经济、科技、政治、文化、社会、生态)
            \item 民族振兴,就是通过自身地不断发展与强大,继承并创造中华民族的优秀文化以及先进的文明成果,进而使中华民族再次处于世界领先的地位,再次以高昂的姿态屹立于世界民族之林。
            \item 人民幸福,就是人民权利保障更加充分、人人得享共同发展。
        \end{itemize}

        国家富强、民族振兴是人民幸福的基础和保障。\firstOrder{人民幸福是国家富强、民族振兴的根本出发点和落脚点。}

    \subsection{奋力实现中国梦}
        \emph{实现中国梦必须走中国道路、弘扬中国精神、凝聚中国力量}
        \begin{itemize}
            \item 实现中国梦必须走中国道路,这就是中国特色社会主义道路。
            \item 实现中国梦必须弘扬中国精神,这就是以爱国主义为核心的民族精神和以改革创新为核心的时代精神。
            \item 实现中国梦必须凝聚中国力量,这就是全国各族人民大团结的力量。
            \item 实现中华民族伟大复兴是海内外中华儿女的共同梦想。
            \item 实干才能梦想成真。
            \item 实现中国梦任重而道远,需要锲而不舍、驰而不息的艰苦努力。
            \item 实现中国梦需要和平,只有和平才能实现梦想。
        \end{itemize}


\section{建成社会主义现代化抢过的战略安排}
    \subsection{开启全面建设社会主义现代化强国的新征程}
    \subsection{实现社会主义现代化强国“两步走”战略}
        \firstOrder{两步走战略:}
        \begin{enumerate}
            \item 从2020年到2035年,基本实现社会主义现代化的目标要求
            \begin{itemize}
                \item 在经济建设方面,我国经济实力、科技实力将大幅跃升,跻身创新型国家前列。
                \item 在政治建设方面,人民平等参与、平等发展权利得到充分保障,法治国家、法治政府、法治社会基本建成,各方面制度更加完善,国家治理体系和治理能力现代化基本实现。
                \item 在文化建设方面,社会文明程度达到新高度,国家文化软实力显著增强,中华文化影响更加广泛深入。
                \item 在民生和社会建设方面,人民生活更为宽裕,中等收入群体比例明显提高,城乡区域发展差距和居民生活水平差距显著缩小,基本公共服务均等化基本实现,全体人民共同富裕迈出坚实步伐。
                \item 在生态文明建设方面,生态环境根本好转,美丽中国目标基本实现。
            \end{itemize}
            \item 从2035年到本世纪中叶,建成社会主义现代化强国的目标要求
        \end{enumerate}
