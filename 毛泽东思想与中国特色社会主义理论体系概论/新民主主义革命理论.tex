\chapter{新民主主义革命理论}\label{Chapter:新民主主义革命理论}


\section{新民主主义革命理论形成的依据}

    \subsection{近代中国国情和中国革命的时代特征}
    近代中国占支配地位的主要矛盾是帝国主义和中华民族的矛盾、封建主义和人民大众的矛盾,而\emph{帝国主义和中华民族的矛盾是各种矛盾中最主要的矛盾}。这决定了近代中国革命的\firstOrder{根本任务是推翻帝国主义、封建主义和官僚资本主义的统治(三座大山)}。

    \firstOrder{近代中国的社会性质和主要矛盾,决定了中国革命仍然是资产阶级民主革命。}

    \firstOrder{以五四运动的爆发为标志,中国资产阶级民主革命进入新民主主义革命的崭新阶段。}\footnote{新民主主义革命自1919年开始,旧民主主义到1922年结束,两者在时间上存在交叠。}

    无产阶级开始以独立的政治力量登上历史舞台,由自在阶级转变为自为的阶级。马克思主义在中国得到广泛传播,逐步成为中国革命的指导思想。

    它推翻帝国主义、封建主义和官僚资本主义的反动统治,但不破坏参加反帝反封建的资本主义成分。

    中国革命要分两步走,第一步是完成反帝反封建的新民主主义革命任务,第二步是完成社会主义革命任务,这是性质不同但又相互联系的两个革命过程。

    \subsection{新民主主义革命理论的实践基础}
    不考。


\section{新民主主义革命的总路线和基本纲领}

    \subsection{新民主主义革命的总路线}
    1939年,毛泽东在《中国革命和中国共产党》一文中首次提出“新民主主义革命”的科学概念。

    1948年,他在《在晋绥干部会议上的讲话》中完整地表述了总路线的内容:

    \begin{center}
        \emph{无产阶级领导的,人民大众的,反对帝国主义、封建主义和官僚资本主义的革命。}
    \end{center}

    \begin{enumerate}
        \item \firstOrder{新民主主义革命的对象.}
        分清敌友,这是革命的\textbf{首要问题}
        \begin{itemize}
            \item 帝国主义。是中国革命的\textbf{首要对象}。
            \item 封建地主阶级。
            \item 官僚资本主义。
        \end{itemize}
        \item \firstOrder{新民主主义革命的动力.}
        新民主主义革命的动力包括无产阶级、农民阶级、城市小资产阶级和民族资产阶级。
        \begin{itemize}
            \item \firstOrder{无产阶级是中国革命最基本的动力}。
            \item 农民是中国革命的主力军。
            \item 城市小资产阶级是无产阶级的可靠同盟者。包括广大的知识分子、小商人、手工业者和自由职业者。
            \item 民族资产阶级也是中国革命的动力之一。\footnote{注意中国革命的对象不是所有剥削阶级。城市小资产阶级和民族资产阶级不在革命对象范围内。}
        \end{itemize}
        \item \textbf{新民主主义革命的领导力量.}
        \firstOrder{无产阶级的领导权是中国革命的中心问题,也是新民主主义革命理论的核心问题。}

        区别新旧两种不同范畴的民主主义革命的根本标准是,\firstOrder{革命的领导权掌握在无产阶级手中还是掌握在资产阶级手中}。

        中国无产阶级具有一些特点和优点
        \begin{enumerate}
            \item 从诞生之日起,就身受外国资本主义、本国封建势力和资产阶级的三重压迫。
            \item 分布集中,有利于无产阶级队伍的组织和团结,有利于革命思想的传播和强大革命力量的形成。
            \item 和农民有着天然的联系,这是的无产阶级便于和农民结成亲密的联盟。
        \end{enumerate}
        无产阶级及其政党——中国共产党的领导,是中国革命取得胜利的根本保证。

        无产阶级及其政党实现对各革命阶级的领导,必须建立以工农联盟为基础的广泛地统一战线,这是实现领导权的关键。

        \item \textbf{新民主主义革命的性质和前途.}
        近代中国半殖民地半封建社会的性质和中国革命的历史人物,决定了中国革命的性质不是无产阶级社会主义革命,而是资产阶级民主主义革命。

        革命的前途是社会主义而不是资本主义。
    \end{enumerate}

    \subsection{新民主主义的基本纲领}
    \emph{只需要知道有三个纲领就行。}
    \begin{enumerate}
        \item \textbf{政治纲领}:推翻帝国主义和封建主义的统治,建立一个无产阶级领导的,以工农联盟为基础的、各革命阶级联合专政的新民主主义的共和国。
        \item \textbf{经济纲领}:没收封建地主阶级的土地归农民所有,没收官僚资产阶级的垄断资本归新民主主义的国家所有,保护民族工商业。
        \item \textbf{文化纲领}:无产阶级领导的人民大众的反帝反封建的文化,即民族的科学的大众的文化。
    \end{enumerate}


\section{新民主主义革命的道路和基本经验}

    \subsection{新民主主义革命的道路}
    即农村包围城市、武装夺取政权的革命道路。
    \paragraph{新民主主义革命道路的内容及意义.} 中国革命走农村包围城市、武装夺取政权的道路,根本在于处理好\textbf{土地革命}、\textbf{武装斗争}、\textbf{农村革命根据地建设}三者之间的关系。
    \begin{enumerate}
        \item 土地革命是民主革命的基本内容。
        \item 武装斗争是中国革命的主要形式,是农村根据地建设和土地革命的强有力保证。
        \item 农村革命根据地是中国革命的战略阵地,是进行武装斗争和开展土地革命的依托。
    \end{enumerate}

    \subsection{新民主主义革命的三大法宝}
        毛泽东在《<共产党人>发刊词》中,把\textbf{统一战线}、\textbf{武装斗争}、\textbf{党的建设}比作党在中国革命中战胜敌人的三个主要的法宝。

        \firstOrder{三大法宝}
        \begin{enumerate}
            \item 统一战线
            \item 武装斗争
            \item 党的建设
        \end{enumerate}

        \subsubsection{统一战线}
        总结党建立、巩固和发展统一战线的实践经验,主要在于这样几个方面:
        \begin{enumerate}
            \item 建立巩固的工农联盟。
            \item 正确对待资产阶级,尤其是民族资产阶级。
            \item 要采取区别对待的方针。
            \item \firstOrder{要坚持独立自主的原则。}
        \end{enumerate}

        \subsubsection{武装斗争}
        经验
        \begin{enumerate}
            \item \firstOrder{坚持党对军队的绝对领导}
            \item 建设全心全意为人民服务的军队
            \item 坚持正确的战略战术
        \end{enumerate}

        \subsubsection{党的建设}
        经验
        \begin{enumerate}
            \item 必须把思想建设\footnote{区别于新时代将党的政治建设放在首位}始终放在党的建设的首位
            \item 必须在任何时候都重视党的组织建设
            \item 必须重视党的作风建设:\firstOrder{形成了理论联系实际、密切联系群众、批评与自我批评相结合的三大优良作风}
        \end{enumerate}