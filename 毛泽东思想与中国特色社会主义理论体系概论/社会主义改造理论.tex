\chapter{社会主义改造理论}\label{Chapter:社会主义改造理论}


\section{从新民主主义到社会主义的转变}

    \subsection{新民主主义社会是一个过渡性的社会}
    新民主主义社会不是一个独立的社会形态,而是由新民主主义向社会主义转变的过渡性社会形态。\footnote{持续时间自新中国成立到社会主义改造基本完成。}

    \subsection{党在过渡时期的总路线及其理论依据}
        \subsubsection{1. 党在过渡时期的总路线的提出}
        党在过渡时期总路线的主要内容被概括为\textbf{“一化三改”}:一化即社会主义工业化;三改即对个体农业、手工业和资本主义工商业的社会主义改造。

        这是一条社会主义建设和社会主义改造同时并举的路线,体现了社会主义工业化和社会主义改造的紧密结合。


\section{社会主义改造道路和历史经验}

    \subsection{适合中国特点的社会主义改造道路}
        \subsubsection{农业、手工业的社会主义改造}
        农业:
        \begin{enumerate}
            \item 积极引导农民组织起来,走互助合作道路。
            \item 遵循自愿互利、典型示范和国家帮助的原则,以互助合作的优越性吸引农民走互助合作道路。
            \item 正确分析农村的阶级和阶层状况,制定正确的阶级政策。
            \item 坚持积极引导、稳步前进的方针,采取循序渐进的步骤。
        \end{enumerate}

        手工业:采取积极领导、稳步前进的方针。

        \subsubsection{资本主义工商业的社会主义改造}
        \begin{enumerate}
            \item 用和平赎买的方法改造资本主义工商业。
            \item 采取从低级到高级的国家资本主义的过渡形式。\footnote{这些企业的利润,按国家所得税、企业公积金、工人福利费、资方红利这四个方面进行分配,即当时所说的“四马分肥”。}
            \item 把资本主义工商业者改造为自食其力的社会主义劳动者。
        \end{enumerate}

    \subsection{社会主义改造的历史经验}
    \begin{enumerate}
        \item 坚持社会主义工业化建设与社会主义改造同时并举。
        \item 采取积极引导、逐步过渡的方式。
        \item 用和平方法进行改造。
    \end{enumerate}


\section{社会主义制度在中国的确立}

    \subsection{社会主义制度的基本确立及其理论依据}
    1956年底,我国对农业、手工业和资本主义工商业的社会主义改造基本完成,社会主义公有制已成为我国社会的经济基础,标志着中国历史上长达数千年的阶级剥削制度的结束和社会主义基本制度的确立。

    1954年9月,《中华人民共和国宪法》制定并颁布施行,明确规定了我国人民民主专政的国体和人民代表大会的政体。人民代表大会制度这一根本政治制度、中国共产党领导的多党合作和政治协商制度、民族区域自治制度这些基本政治制度的确立,表明我国有一个新民主主义的国家转变为社会主义国家。

    \subsection{确立社会主义基本制度的重大意义}
    社会主义基本制度的确立是中国历史上最深刻最伟大的社会变革,为当代中国一切发展进步奠定了制度基础,也为中国特色社会主义制度的创新和发展提供了重要前提。