\chapter{中国特色大国外交}

\section{坚持和平发展道路}
    \subsection{世界正处于大发展大变革大调整时期}
        冷战结束后,尤其是进入21世纪以来,国际形势发生了广泛而深刻的变化,但和平与发展仍然是时代主题,和平、发展、合作、共赢成为不可阻挡的时代潮流。

        \subsubsection{世界在多极化中曲折发展}
        \subsubsection{经济全球化深入发展}
        \subsubsection{文化多样化持续推进}
        \subsubsection{社会信息化快速发展}
        \subsubsection{科学技术孕育新突破}

    \subsection{坚持独立自主和平外交政策}
        中国独立自主的和平外交政策,
        \begin{itemize}
            \item 就是把国家主权和安全放在第一位,坚定地维护我国的国家利益
            \item 就是从我国人民和世界人民的根本利益出发
            \item 就是坚持各国的事务应由本国政府和人民决定
            \item 就是主张和平解决国际争端和热点问题
        \end{itemize}

    \subsection{推动建立新型国际关系}
        \begin{itemize}
            \item 维护世界和平、促进共同发展,是中国外交政策的宗旨
        \end{itemize}

        中国倡导建设相互尊重、公平正义、合作共赢的新型国际关系
        \subsubsection{核心}
        \begin{itemize}
            \item 维护联合国宪章的宗旨和原则
            \item 维护不干涉别国内政和尊重国家主权、独立、领土完整等国际关系基本准则
            \item 维护联合国及其安理会对世界和平承担的首要责任
        \end{itemize}

        \subsubsection{推动建设新型国际关系}
        \begin{itemize}
            \item 要坚决维护国家核心利益
            \item 要在和平共处五项原则基础上发展同世界各国的友好合作
            \item 要积极参与全球治理体系改革和建设
            \item 要加强涉外法律工作,完善涉外法律法规体系
            \item 要把互相尊重、公平正义、合作共赢理念体现到政治、经济、安全、文化等对外合作的方方面面,推动构建人类命运共同体
        \end{itemize}


\section{推动构建人类命运共同体}
    \subsection{构建人类命运共同体的思想内涵}
        \begin{itemize}
            \item 2012年党的十八大明确提出要倡导人类命运共同体意识
            \item 2015年博鳌亚洲论坛年会时提出通过迈向亚洲命运共同体,推动建设人类命运共同体
            \item 2015年9月在联合国总部发表讲话指出,构建以合作共赢为核心的新型国际关系,打造人类命运共同体
        \end{itemize}

        \subsubsection{时代背景}
        \begin{itemize}
            \item 国际
            \begin{itemize}
                \item 和平发展进步、经济全球化、社会信息化、前所未有的发展机遇。各国相互联系、相互依存、相互合作、相互促进
                \item 世界发展面临各种问题和挑战。经济低迷、发展鸿沟、地区冲突、恐怖主义和难民潮
            \end{itemize}
            \item 国内
            \begin{itemize}
                \item 中国特色社会主义进入新时代,国际影响力、感召力、塑造力进一步提高。中国有信心、有能力为世界作出更大贡献
            \end{itemize}
        \end{itemize}

        \subsubsection{构建人类命运共同体的核心}
        构建人类命运共同体思想,是一个科学完整、内涵丰富、意义深远的思想体系,其核心是:\emph{建设持久和平、普遍安全、共同繁荣、开放包容、清洁美丽的世界}。
        \begin{enumerate}
            \item 政治上,要互相尊重、平等协商,坚决摒弃冷战思维和强权政治,走对话而不对抗、结伴而不结盟的国与国交往新路。
            \item 安全上,要坚持以对话解决争端,以协商化解分歧,统筹应对传统和非传统安全威胁,反对一切形式的恐怖主义。
            \item 经济上,要同舟共济,促进贸易和投机自由化便利化,推动经济全球化朝着更加开放、包容、普惠、平衡、共赢的方向发展。
            \item 文化上,要尊重世界文明多样性,以文明交流超越文明隔阂、文明互鉴超越文明冲突、文明共存超越文明优越。
            \item 生态上,要坚持环境友好,合作应对气候变化,保护好人类赖以生存的地球家园。
        \end{enumerate}

        \subsubsection{意义}
        \begin{itemize}
            \item 继承和发展了新中国不同时期重大外交思想和主张
            \item 反映了中外优秀文化和全人类共同价值追求
            \item 适应了新时代中国与世界关系的历史性变化
            \item 成为中国引领时代潮流和人类文明进步方向的鲜明旗帜
            \item 多次写入联合国文件,对中国的和平发展、世界的繁荣进步都具有重大和深远意义
        \end{itemize}

    \subsection{促进“一带一路”国际合作}
        2013年9月和10月,习近平在出访中亚和东南亚国家期间,先后提出共建\emph{丝绸之路经济带}和\emph{21世纪海上丝绸之路}的重大倡议。

        党的十九大提出要以一带一路建设为重点
        \begin{enumerate}
            \item 坚持引进来和走出去兵种,深化双向投资合作。
            \item 坚持共商共建共享原则。
            \item 加强创新能力开放合作,主要加强技术创新合作、理论创新交流互鉴、创新人才资源交流合作。
            \item 把一带一路与构建人类命运共同体更加紧密结合起来,与落实2030年可持续发展议程紧密结合起来
        \end{enumerate}

    \subsection{共商共建人类命运共同体}
        \begin{enumerate}
            \item 坚持和平发展道路,推动建设新型国际关系。
            \item 不断完善外交布局,积极发展全球伙伴关系。
            \item 深度参与全球治理,积极引导国际秩序变革方向。
            \item 推动国际社会从伙伴关系、安全格局、经济发展、文明交流、生态建设等方面为建立人类命运共同体做出努力。
        \end{enumerate}
