\chapter{社会主义建设道路初步探索的理论成果}

\section{初步探索的重要理论成果}
\begin{enumerate}
    \item 调动一切积极因素为社会主义事业服务。
    \item 正确认识和处理社会主义矛盾的思想。
    \item 走中国工业化道路的思想。
\end{enumerate}

    \subsection{调动一切积极因素为社会主义事业服务}
    1956年4月和5月,毛泽东先后在中央政治局扩大会议和最高国务会议上,做了《论十大关系》的报告。

    《论十大关系》标志着党探索中国社会主义建设道路的良好开端。

    《论十大关系》确立了一个基本方针:\emph{努力把党内党外国内国外的一切积极的因素,直接的、间接的积极因素全部调动起来,为社会主义建设服务}。

    \subsection{正确认识和处理社会主义社会矛盾的思想}
    毛泽东在1957年2月所作的《关于正确处理人民内部矛盾的问题》的报告中,系统论述了社会主义社会矛盾的理论。
    
    有两类社会矛盾:\textbf{敌我矛盾}和\textbf{人民内部矛盾},这是两类性质完全不同的矛盾。敌我之间和人民内部这两类矛盾的性质不同,解决的方法也不同:采用\textbf{专政}和\textbf{民主}两种不同的方法。

    \subsection{走中国工业化道路的思想}
    毛泽东在《论十大关系》中论述的第一大关系,便是\emph{重工业、轻工业和农业的关系}。要走一条有别于苏联的中国工业化道路。

    提出了\emph{以农业为基础,以工业为主导,以农轻重魏旭发展国民经济的总方针}。


\section{初步探索的意义和经验教训}

    \subsection{初步探索的意义}
    \begin{enumerate}
        \item 巩固和发展了我国的社会主义制度。
        \item 为开创中国特色社会主义提供了宝贵经验、理论准备、物质基础。
        \item 丰富了科学社会主义的理论和实践。
    \end{enumerate}

    \subsection{初步探索的经验教训}
    \begin{enumerate}
        \item 必须把马克思主义与中国实际相结合,探索符合中国特点的社会主义建设道路。
        \item 必须正确认识社会主义社会的主要矛盾和根本任务,集中力量发展生产力。
        \item 必须从实际出发进行社会主义建设,建设规模和速度要和国力相适应,不能急于求成。
        \item 必须发展社会主义民主,健全社会主义法制。
        \item 必须坚持党的民主集中制和集体领导制度,加强执政党建设。
        \item 必须坚持对外开放,借鉴和吸收人类文明成果建设社会主义,不能关起门来搞建设。
    \end{enumerate}