\chapter{邓小平理论}
\emph{“什么是社会主义、怎样建设社会主义。”}

\section{邓小平理论的形成}
    \subsection{邓小平理论的形成条件}
        \begin{itemize}
            \item 和平与发展成为时代主题是邓小平理论形成的时代背景。
            \item 社会主义建设的经验教训是邓小平理论形成的历史根据
            \item 改革开放和现代化建设的实践是邓小平理论形成的现实依据。
        \end{itemize}

    \subsection{邓小平理论的形成过程}
        \begin{enumerate}
            \item 1978年12月召开的党的十一届三中全会
            \begin{itemize}
                \item \firstOrder{重新确立了解放思想、实事求是的思想路线}
                \item \firstOrder{停止使用以阶级斗争为纲的错误提法}
                \item \firstOrder{确定把全党工作重点转移到社会主义现代化建设上}
                \item 作出实行改革开放的重大决策
            \end{itemize}
            \item \firstOrder{1982年在党的十二大开幕词中明确指出建设有中国特色的社会主义。从此,中国特色社会主义成为党的全部理论和实践创新的主题。}
            \item \firstOrder{1987年党的十三大,第一次比较系统地论述了我国社会主义初级阶段理论。明确概括和全面阐发了党的“一个中心、两个基本点”的基本路线。}
            \item 1992年南方谈话重申了深化改革、加速发展的必要性和重要性。提出了一系列重要论断。
            \item 1992年党的十四大第一次比较系统地初步回答了什么是社会主义、怎样建设社会主义。
            \item \firstOrder{1997年党的十五大正式提出邓小平理论这一概念,同马克思列宁主义、毛泽东思想一起确立为党的指导思想并写入党章。}
            \item 1999年正式将邓小平理论载入宪法。
        \end{enumerate}


\section{邓小平理论的基本问题和主要内容}
    \subsection{邓小平理论回答的基本问题}
        \firstOrder{什么是社会主义、怎样建设社会主义},是邓小平在领导改革开放和现代化建设这一新的革命过程中,不断提出和反复思考的首要基本理论问题。

    \subsubsection{社会主义的本质} 1992年初,邓小平在南方谈话中对社会主义的本质作了总结性理论概括:\firstOrder{解放生产力,发展生产力,消灭剥削,消除两极分化,最终达到共同富裕。}

    \subsection{邓小平理论的主要内容}
        \subsubsection{解放思想、实事求是} \firstOrder{解放思想、实事求是的思想路线:是邓小平理论活的灵魂,是邓小平理论的精髓。}

        \subsubsection{社会主义初级阶段理论} 
        党的十三大系统的论述了社会主义初级阶段理论。

        \textbf{社会主义初级阶段},就是指我国在生产力落后、商品经济不发达条件下建设社会主义必然要经历的特定阶段,即从我国进入社会主义到基本实现社会主义现代化的整个历史阶段。
        \begin{enumerate}
            \item 我国已经进入社会主义社会,必须坚持而不能离开社会主义社会。
            \item 我国的社会主义社会还处在不发达的阶段,必须正视而不能超越初级阶段。
        \end{enumerate}

        党的十五大进一步阐述了社会主义初级阶段的基本特征(P99)。

        \subsubsection{党的基本路线}
        领导和团结全国各族人民,\emph{以经济建设为中心,坚持四项基本原则,坚持改革开放},自力更生,艰苦创业,为把我国建设成为富强、民主、文明(、和谐、美丽)的社会主义现代化国家(强国)而奋斗。
        \begin{enumerate}
            \item 建设富强民主文明和谐美丽的社会主义现代化国家。是党在社会主义初级阶段的奋斗目标,体现了社会主义社会全面发展的要求。
            \item 一个中心、两个基本点。是基本路线最主要的内容,是实现社会主义现代化奋斗目标的基本路径
            \item 领导和团结全国各族人民。是实现社会主义现代化奋斗目标的领导力量和依靠力量。
            \item 自力更生艰苦创业。是党的优良传统,也是实现社会主义初级阶段奋斗目标的根本立足点。
        \end{enumerate}

        \subsubsection{社会主义的根本任务}
        生产力是社会发展的最根本的决定性因素,社会主义的根本任务是\textbf{发展生产力}。社会主义革命是为了解放生产力、发展生产力。

        \emph{贫穷不是社会主义,社会主义要消灭贫穷。}

        \subsubsection{三步走战略}
        1987年4月,邓小平第一次提出了分三步走基本实现现代化的战略。同年10月,党的十三大把邓小平三步走的发展战略构想确定下来。
        \begin{enumerate}
            \item 从1981年到1990年实现国民生产总值比1980年翻一番,解决人民的温饱问题。
            \item 从1991年到20世纪末,使国民生产总值再翻一番,达到小康水平。
            \item \firstOrder{到21世纪中叶,国民生产总值再翻两番,达到中等发达国家水平,基本实现现代化。}\footnote{在习近平新时代中国特色社会主义思想中,这一任务目标提前至2035年。}
        \end{enumerate}

        为了顺利实现现代化发展战略,邓小平提出了台阶式发展的思想,要求抓住机遇,加快发展,争取隔几年是国民经济上一个新台阶。

        为了顺利实现现代化发展战略,邓小平还提出允许和鼓励一部分地区、一部分人先富起来逐步达到共同富裕的思想。

        \subsubsection{改革开放理论}
        (P105)。我国的渐进式增量改革,有别于苏联和俄罗斯的改革模式。

        新时期最鲜明的特点是改革开放。改革开放是中国的第二次革命。作为一次新的革命,不是也不允许否定和抛弃社会主义基本制度,它是社会主义制度的自我完善和发展。
        
        改革开放的实质和目标,是要从根本上改变舒服我国生产力发展的经济体制,建立充满生机和活力的社会主义新经济体制,同时相应地改革政治体制和其他方面的体制,以实现中国的社会主义现代化。

        判断改革和各方面工作的是非得失,归根到底,要以是否有利于发展社会主义的生产力,是否有利于增强社会主义国家的综合国力,是否有利于提高人民的生活水平为标准。

        \subsubsection{社会主义市场经济理论}
        \firstOrder{农村家庭联产承包责任制。}

        十二届三中全会通过的《中共中央关于经济体制改革的决定》提出了社会主义经济是公有制基础上有计划的商品经济的论断。

        在南方谈话中,邓小平指出,计划经济不等于社会主义,资本主义也有计划;市场经济不等于资本主义,社会主义也有市场。从根本上解除了把计划经济和市场经济看做社会基本制度范畴的思想束缚。

        社会主义市场经济理论的要点有
        \begin{enumerate}
            \item 计划经济和市场经济不是划分社会制度的标志
            \item 计划和市场都是经济手段,对经济活动的调节各有优势和长处,社会主义实行市场经济要把两者结合起来。
            \item 市场经济作为资源配置的一种方式本身不具有制度属性,可以和不同社会制度结合。
        \end{enumerate}

        \subsubsection{两手抓,两手都要硬}
        \begin{itemize}
            \item 一手抓物质文明,一手抓精神文明
            \item 一手抓建设,一手抓法制
            \item 一手抓改革开放,一手抓惩治腐败
        \end{itemize}

        \subsubsection{一国两制}
        和平统一、一国两制。(不考)

        \subsubsection{中国问题的关键在于党}
        建设中国特色社会主义,关键在于坚持、加强和改善党的领导。除了改善党的组织状况以外,还要改善党的领导工作状况、改善党的领导制度。


\section{邓小平理论的历史地位}
    \subsection{马克思列宁主义、毛泽东思想的继承和发展}
        是马克思列宁主义基本原理与当代中国实际和时代特征相结合的产物,是马克思列宁主义、毛泽东思想的继承和发展,是全党全国人民集体智慧的结晶。

    \subsection{中国特色社会主义理论体系的开山之作}
        响亮提出走自己的路、建设有中国特色的社会主义的伟大号召,从此中国特色社会主义称为我们党全部理论和实践一以贯之的主题。

    \subsection{改革开放和社会主义现代化建设的科学指南}
        邓小平理论是中国共产党和中国人民宝贵的精神财富,是改革开放和社会主义现代化建设的科学指南,是党和国家必须长期坚持的指导思想。