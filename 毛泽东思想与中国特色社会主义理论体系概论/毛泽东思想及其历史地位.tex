\chapter{毛泽东思想及其历史地位}
\par 毛泽东思想是在革命和建设的长期实践中,以毛泽东为主要代表的中国共产党人,根据马克思列宁主义基本原理,形成的适合中国情况的科学指导思想,是被实践证明了的关于中国革命和建设的正确的理论原则和经验总结,是中国共产党集体智慧的结晶。

\par 毛泽东思想是马克思主义中国化的第一个重大理论成果,是马克思列宁主义在中国的运用和发展,是被实践证明了的关于中国革命和建设的正确的理论原则和经验总结,是中国国共产党集体智慧的结晶,是党必须长期坚持的指导思想。


\section{毛泽东思想的形成和发展}

    \subsection{毛泽东思想形成发展的历史条件}
    Nothing here.

    \subsection{毛泽东思想形成发展的过程}
    书本中萌芽时期和形成时期统称为形成时期。
    \begin{enumerate}
        \item \textbf{萌芽.} 第一次国内革命战争时期。这一时期的著作
        \begin{itemize}
            \item 《中国社会各阶级的分析》
            \item 《湖南农民运动考察报告》
        \end{itemize}
        分析了中国社会各阶级在革命中的地位和作用,提出了\emph{新民主主义革命}的基本思想。
        \item \textbf{形成.} 土地革命战争时期。著作:
        \begin{itemize}
            \item 《中国的红色政权为什么能够存在?》
            \item 《井冈山的斗争》
            \item 《星星之火,可以燎原》
            \item 《反对本本主义》
        \end{itemize}
        \par 在同党内一度盛行的把马克思主义教条化、把共产国际决议和苏联经验神圣化的错误倾向的斗争中,提出并阐述了\emph{农村包围城市}、\emph{武装夺取政权}的思想,标志着毛泽东思想的初步形成。
        \item \textbf{成熟.} 著作:
        \begin{itemize}
            \item 《实践论》、《矛盾论》
            \item 《<共产党人>发刊词》
            \item 《中国革命和中国共产党》
            \item 《新民主主义论》
            \item 《论联合政府》
        \end{itemize}
        \par 运用马克思主义的认识论和辩证法,系统分析了党内“左”的和右的错误地思想根源。
        \par 新民主主义革命理论的系统阐述,实现了马克思主义与中国革命实践相结合的历史性飞跃,标志着毛泽东思想得到多方面展开而趋于成熟。
        \par \emph{1945年党的七大将毛泽东思想写入党章,确立为党必须长期坚持的指导思想。}
        \item \textbf{继续发展.} 著作:
        \begin{itemize}
            \item 《在中国共产党第七届中央委员会第二次全体会议上的报告》
            \item 《论人民民主专政》
            \item 《论十大关系》
            \item 《关于正确处理人民内部矛盾的问题》
        \end{itemize}
        先后提出\emph{人民民主专政理论}、\emph{社会主义改造理论}、\emph{关于严格区分和正确处理两类矛盾的学说}特别是正确处理人民内部矛盾的理论。
    \end{enumerate}


\section{毛泽东思想的主要内容和活的灵魂}

    \subsection{毛泽东思想的主要内容}
    \begin{enumerate}
        \item 新民主主义革命理论:第\ref{Chapter:新民主主义革命理论}章节
        \item 社会主义革命和社会主义建设理论:第\ref{Chapter:社会主义改造理论}章节
        \item 革命军队建设和军事战略的理论
        \item 政策和策略的理论
        \item 思想政治工作和文化工作的理论
        \item 党的建设理论
    \end{enumerate}

    \subsection{毛泽东思想活的灵魂}
    \par 贯穿于毛泽东思想各个组成部分的立场、观点和方法,是毛泽东思想的活的灵魂,它们有三个基本方面,即\textbf{实事求是},\textbf{群众路线},\textbf{独立自主}。

    \subsubsection{1. 实事求是} 实事求是,就是一切从实际出发,理论联系实际,坚持在实践中检验真理和发展真理。

    \subsubsection{2. 群众路线} 群众路线,就是一切为了群众,一切依靠群众,从群众中来,到群众中去,把党的正确主张变为群众的自觉行动。

    \subsubsection{3. 独立自主} 独立自主,就是坚持独立思考,走自己的路,就是坚定不移地维护民族独立、捍卫国家主权,把立足点放在依靠自己力量的基础上,同时积极争取外援,开展国际经济文化交流,学习外国一切对我们有益的先进事物。


\section{毛泽东思想的历史地位}

    \subsection{马克思主义中国化的第一个重大理论成果}
    毛泽东思想是马克思主义中国话第一次历史性飞跃的理论成果。在马克思主义中国话的历史进程中,毛泽东思想为中国特色社会主义理论体系的形成奠定了理论基础。

    \subsection{中国革命和建设的科学指南}

    \subsection{中国共产党和中国人民宝贵的精神财富}