\chapter{“三个代表”重要思想}
\emph{以江泽民为主要代表的中国共产党人,在世界社会主义陷入低谷时,坚决捍卫中国特色社会主义,并成功推向21世纪。}


\section{“三个代表”重要思想的形成}
    \subsection{“三个代表”重要思想的形成条件}
        \begin{enumerate}
            \item 在对冷战结束后国际局势科学判断的基础上形成的:世界多极化和经济全球化的趋势在曲折中发展,和平与发展仍是时代的主题。
            \item 在科学判断党的历史方位和总结历史经验的基础上提出的
            \item 在建设中国特色社会主义伟大实践的基础上形成的
        \end{enumerate}

    \subsection{“三个代表”重要思想的形成过程}
        \begin{itemize}
            \item 2000年2月25日,江泽民在广东考察工作时,从全面总结当的历史经验和如何适应新形势新任务的要求出发,首次对“三个代表”进行了比较全面的阐述。
            \item 2001年7月1日,江泽民在庆祝中国共产党成立80周年大会上的讲话中全面阐述了“三个代表”重要思想的科学内涵和基本内容。
            \item 党的十六大将“三个代表”重要思想与马克思列宁主义、毛泽东思想和邓小平理论一道确立为党必须长期坚持的指导思想,并写入党章。
        \end{itemize}


\section{“三个代表”重要思想的核心观点和主要内容}
    \emph{“中国共产党必须始终代表中国先进生产力的发展要求,代表中国先进文化的前进方向,代表中国最广大人民的根本利益。”}

    \subsection{“三个代表”重要思想的核心观点}
        \begin{itemize}
            \item 始终代表中国先进生产力的发展要求
            \item 始终代表中国先进文化的前进方向
            \item 始终代表中国最广大人民的根本利益
        \end{itemize}

    \subsection{“三个代表”重要思想的主要内容}
        \begin{itemize}
            \item 发展是党执政兴国的第一要务
            \item 建立社会主义市场经济体制
            \item 全面建设小康社会
            \item 建设社会主义政治文明
            \item 推进党的建设新的伟大工程
        \end{itemize}


\section{“三个代表”重要思想的历史地位}
    “三个代表”重要思想在邓小平理论的基础上,进一步回答了什么是社会主义、怎样建设社会主义的问题,创造性地回答了建设什么样的党、怎样建设党的问题,深化了对中国特色社会主义的认识。
    \subsection{中国特色社会主义理论体系的接续发展}
        这一思想把新时期党的建设目标、任务和要求,提到了一个新的高度,具有鲜明的时代特征,从根本上回答了在充满希望与挑战的新世纪,\emph{要把我们党建设成为一个什么样的党和怎样建设党}这样一个重大历史性问题。

    \subsection{加强和改进党的建设,推进中国特色社会主义视野的强大理论武器}
