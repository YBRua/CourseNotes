\chapter{毛泽东思想及其历史地位}
\noindent 毛泽东思想是马克思主义中国化的第一个重大理论成果,是马克思列宁主义在中国的运用和发展,是被实践证明了的关于中国革命和建设的正确的理论原则和经验总结,是中国国共产党集体智慧的结晶,是党必须长期坚持的指导思想。

\section{毛泽东思想的形成和发展}

\subsection{毛泽东思想形成发展的历史条件}
Nothing here.

\subsection{毛泽东思想形成发展的过程}
\begin{enumerate}
    \item \textbf{萌芽.} 这一时期的著作
    \begin{itemize}
        \item 《中国社会各阶级的分析》
        \item 《昏暗农民运动考察报告》
    \end{itemize}
    分析了中国社会各阶级在革命中的地位和作用,提出了\emph{新民主主义革命}的基本思想。
    \item \textbf{形成.} 著作:
    \begin{itemize}
        \item 《中国的红色政权为什么能够存在?》
        \item 《井冈山的斗争》
        \item 《星星之火,可以燎原》
        \item 《反对本本主义》
    \end{itemize}
    在同党内一度盛行的把马克思主义教条化、把共产国际决议和苏联经验神圣化的错误倾向的斗争中,提出并阐述了\emph{农村包围城市}、\emph{武装夺取政权}的思想,标志着毛泽东思想的初步形成。
    \item \textbf{成熟.} 著作:
    \begin{itemize}
        \item 《实践论》、《矛盾论》
        \item 《<共产党人>发刊词》
        \item 《中国革命和中国共产党》
        \item 《新民主主义论》
        \item 《论联合政府》
    \end{itemize}
    运用马克思主义的认识论和辩证法,系统分析了党内“左”的和右的错误地思想根源。

    新民主主义革命理论的系统阐述,实现了马克思主义与中国革命实践相结合的历史性飞跃,标志着毛泽东思想得到多方面展开而趋于成熟。

    1945年党的七大将毛泽东思想写入党章,确立为党必须长期坚持的指导思想。
    \item \textbf{继续发展.} 著作:
    \begin{itemize}
        \item 《在中国共产党第七届中央委员会第二次全体会议上的报告》
        \item 《论人民民主专政》
        \item 《论十大关系》
        \item 《关于正确处理人民内部矛盾的问题》
    \end{itemize}
    先后提出\emph{人民民主专政理论}、\emph{社会主义改造理论}、\emph{关于严格区分和正确处理两类矛盾的学说}特别是正确处理人民内部矛盾的理论。
\end{enumerate}

\section{毛泽东思想的主要内容和活的灵魂}
\section{毛泽东思想的历史地位}
\verb|raise NotImplementedError('还没讲')|