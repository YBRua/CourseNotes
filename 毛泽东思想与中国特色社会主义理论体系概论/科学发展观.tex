\chapter{科学发展观}

\section{科学发展观的形成}
    \subsection{科学发展观的形成条件}
        \begin{enumerate}
            \item 科学发展观是在深刻把握我国基本国情和新的阶段性特征的基础上形成和发展的。
            \item 科学发展观是在深入总结改革开放以来特别是党的十六大依赖实践经验的基础上形成和发展的
            \item 科学发展观是在深刻分析国际形势、顺应世界发展趋势、借鉴国外发展经验的基础上形成和发展的
        \end{enumerate}

    \subsection{科学发展观的形成过程}
        科学发展观是在\textbf{抗击非典疫情}和\textbf{探索完善社会主义市场经济体制}的过程中逐步形成。


\section{科学发展观的科学内涵和主要内容}
    \subsection{科学发展观的科学内涵}
        \firstOrder{科学发展观,第一要义是发展,核心立场是以人为本,基本要求是全面协调可持续,根本方法是统筹兼顾。}
        \begin{enumerate}
            \item 推动经济社会发展是科学发展观的第一要义
            \item 以人为本是科学发展观的核心立场
            \item 全面协调可持续是科学发展观的基本要求
            \item 统筹兼顾是科学发展观的根本方法
        \end{enumerate}

    \subsection{科学发展观的主要内容}
        \begin{enumerate}
            \item 加快转变经济发展方式
            \item 发展社会主义民主政治
            \item 推进社会主义文化强国建设
            \item 构建社会主义和谐社会
            \item 推进生态文明建设
            \item 全面提高党的建设科学化水平
        \end{enumerate}


\section{科学发展观的历史地位}
    \subsection{中国特色社会主义理论体系的接续发展}
    \subsection{发展中国特色社会主义必须长期坚持的指导思想}
