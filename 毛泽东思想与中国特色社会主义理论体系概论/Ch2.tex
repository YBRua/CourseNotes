\chapter{新民主主义革命理论}

\section{新民主主义革命理论形成的依据}

\subsection{近代中国国情和中国革命的时代特征}
以五四运动的爆发为标志,中国资产阶级民主革命进入新民主主义革命的崭新阶段\footnote{但是五四运动本身不算在新民主主义革命的阶段内。}。

无产阶级开始以独立的政治力量登上历史舞台,由自在阶级转变为自为的阶级。马克思主义在中国得到广泛传播,逐步成为中国革命的指导思想。

它推翻帝国主义、封建主义和官僚资本主义的反动统治,但不破坏参加反帝反封建的资本主义成分。

\subsection{新民主主义革命理论的实践基础}
不考。

\section{新民主主义革命的总路线和基本纲领}

\subsection{新民主主义革命的总路线}
1939年,毛泽东在《中国革命和中国共产党》一文中首次提出“新民主主义革命”的科学概念。

1948年,他在《在晋绥干部会议上的讲话》中完整地表述了总路线的内容:

\begin{center}
    \textit{无产阶级领导的,人民大众的,反对帝国主义、封建主义和官僚资本主义的革命。}
\end{center}

\begin{enumerate}
    \item \textbf{新民主主义革命的对象.}
    分清敌友,这是革命的\textbf{首要问题}
    \begin{itemize}
        \item 帝国主义。是中国革命的首要对象。
        \item 封建地主阶级。
        \item 官僚资本主义。
    \end{itemize}
    \item \textbf{新民主主义革命的动力.}
    新民主主义革命的动力包括无产阶级、农民阶级、城市小资产阶级和民族资产阶级。
    \begin{itemize}
        \item 无产阶级是中国革命最基本的动力。
        \item 农民是中国革命的主力军。
        \item 城市小资产阶级是无产阶级的可靠同盟者。包括广大的知识分子、小商人、手工业者和自由职业者。
        \item 民族资产阶级也是中国革命的动力之一。
    \end{itemize}
    \item \textbf{新民主主义革命的领导力量.}
    无产阶级的领导权是中国革命的\textbf{中心问题},也是新民主主义革命理论的\textbf{核心问题}。

    区别新旧两种不同范畴的民主主义革命的根本标准是,革命的领导权掌握在无产阶级手中还是掌握在资产阶级手中。

    中国无产阶级具有一些特点和优点
    \begin{enumerate}
        \item 从诞生之日起,就身受外国资本主义、本国封建势力和资产阶级的三重压迫。
        \item 分布集中,有利于无产阶级队伍的组织和团结,有利于革命思想的传播和强大革命力量的形成。
        \item 和农民有着天然的联系,这是的无产阶级便于和农民结成亲密的联盟。
    \end{enumerate}
    无产阶级及其政党——中国共产党的领导,是中国革命取得胜利的根本保证。

    无产阶级及其政党实现对各革命阶级的领导,必须建立以工农联盟为基础的广泛地统一战线,这是实现领导权的关键。

    \item \textbf{新民主主义革命的性质和前途.}
    近代中国半殖民地半封建社会的性质和中国革命的历史人物,决定了中国革命的性质不是无产阶级社会主义革命,而是资产阶级民主主义革命。

    革命的前途是社会主义而不是资本主义。
\end{enumerate}

\subsection{新民主主义的基本纲领}\footnote{只需要知道有三个纲领就行。}
\begin{enumerate}
    \item \textbf{政治纲领}:推翻帝国主义和封建主义的统治,建立一个无产阶级领导的,以工农联盟为基础的、各革命阶级联合专政的新民主主义的共和国。
    \item \textbf{经济纲领}:魔兽封建地主阶级的土地归农民所有,没收官僚资产阶级的垄断资本归新民主主义的国家所有,保护民族工商业。
    \item \textbf{文化纲领}:无产阶级领导的人民大众的反帝反封建的文化,即民族的科学的大众的文化。
\end{enumerate}

\section{新民主主义革命的道路和基本经验}

\subsection{新民主主义革命的道路}
即农村包围城市、武装夺取政权的革命道路。

\paragraph{新民主主义革命道路的内容及意义.} 中国革命走农村包围城市、武装夺取政权的道路,根本在于处理好\emph{土地革命、武装斗争、农村革命根据地建设}三者之间的关系。
\begin{enumerate}
    \item 土地革命是民主革命的基本内容
    \item 武装斗争是中国革命的主要形式,是农村根据地建设和土地革命的强有力保证
    \item 农村革命根据地是中国革命的战略阵地,是进行武装斗争和开展土地革命的依托
\end{enumerate}

\subsection{新民主主义革命的三大法宝}
\begin{enumerate}
    \item 统一战线
    \item 武装斗争
    \item 党的建设
\end{enumerate}