\chapter{“四个全面”战略布局}

\section{全面建成小康社会}
    \verb|全面建成小康社会 is deprecated. Use 全面建设社会主义现代化国家|

    \subsection{全面建成小康社会的内涵}
        \begin{itemize}
            \item 全面建成小康社会,更重要、更难做到的是“全面”。
            \begin{itemize}
                \item 小康讲的是发展水平
                \item 全面讲的是发展的平衡性、协调性、可持续性
            \end{itemize}
            \item 全面小康,覆盖的领域要全面,是五位一体全面进步的小康
            \item 全面建成小康社会,要实事求是,因地制宜
        \end{itemize}

    \subsection{全面建成小康社会的目标要求}
        \begin{itemize}
            \item 经济保持中高速增长
            \item 创新驱动成效显著
            \item 发展协调性明显增强
            \item 人民生活水平和质量普遍提高
            \item 国民素质和社会主义文明程度显著提高
            \item 生态环境质量总体改善
            \item 各方面制度更加成熟更加定型
        \end{itemize}

    \subsection{决胜全面建成小康社会}
        \begin{enumerate}
            \item 坚决打好防范化解重大风险攻坚战
            \item 坚决打好精准脱贫攻坚战
            \item 坚决打好污染防治攻坚战
            \item 确保经济社会持续健康发展
        \end{enumerate}


\section{全面深化改革}
    \subsection{坚定不移地全面深化改革}
        把全面深化改革作为四个全面战略布局中具有突破性和先导性的关键环节、具有新的历史特点的伟大斗争的重要方面。
        \begin{itemize}
            \item 全面深化改革,是顺应当今世界发展大势的必然选择。我们要顺应浩浩荡荡的历史潮流,承担起自己的历史责任,以更大的政治勇气和智慧、更有力的措施和办法推进改革
            \item \firstOrder{全面深化改革,是解决中国现实问题的根本途径。}破解发展中面临的难题,化解来自各方面的风险挑战,推动经济社会持续健康发展,必须依靠全面深化改革。
            \item 全面深化改革,关系党和人民事业前途命运,关系党的执政基础和执政地位。
        \end{itemize}

        \begin{itemize}
            \item 全面深化改革必须坚持党对改革的集中统一领导。党是改革的倡导者、推动者、领导者,改革能否顺利推进,关键取决于党,取决于党的领导。
            \item 全面深化改革必须坚持改革沿着中国特色社会主义方向前进。要始终坚持中国特色社会主义道路、中国特色社会主义理论体系、中国特色社会主义制度。
            \item 全面深化改革必须坚持改革往有利于维护社会公平正义、增进人民福祉方向前进。
            \item 全面深化改革必须坚持社会主义市场经济改个方向。
        \end{itemize}

    \subsection{全面深化改革的总目标和主要内容}
        \subsubsection{总目标}
            党的十八届三中全会通过《中共中央关于全面深化改革若干重大问题的决定》,\firstOrder{提出全面深化改革的总目标是完善和发展中国特色社会主义制度,推进国家治理体系和治理能力的现代化。}这两句话是一个整体,前一句规定了根本方向,后一句规定了实现路径。

    \subsection{正确处理全面深化改革中的重大关系}
        参考书P253-255。看书。理解,不用硬背。
        \begin{itemize}
            \item 处理好解放思想和实事求是的关系。全面深化改革,面临的挑战和困难前所未有,必须进一步解放思想、坚持实事求是。
            \item 处理好顶层设计和摸着石头过河的关系。
            \begin{itemize}
                \item 摸着石头过河就是便是渐变总结,从实践中获得真知,这是富有中国特色、符合中国国情的改革方法,也是符合马克思主义认识论和实践论的方法。
                \item 必须在深入调查研究的基础上提出全面深化改革的顶层设计,要对经济体制、政治体制、文化体制、社会体制、生态体质做出统筹设计
            \end{itemize}
            \item 处理好整体推进和重点突破的关系。注重系统性、整体性、协调性是全面深化改革的内在要求,也是推进改革的重要方法。也要注重抓主要矛盾和矛盾的主要方面,注重抓重要领域和关键环节。以重要领域和关键环节为突破口,可以对全面改革起到牵引和推动作用。
            \item 处理好胆子要大、步子要稳的关系。
            \begin{itemize}
                \item 改革再难也要向前推进,敢于担当,敢于啃硬骨头,敢于涉险滩。
                \item 方向要准,行驶要稳,尤其是不能犯颠覆性错误
            \end{itemize}
            \item 处理好改革、发展、稳定的关系。改革、发展、稳定是我国社会主义现代化建设的三个重要支点。
            \begin{itemize}
                \item 改革是社会经济发展的强大动力
                \item 发展是解决一切社会经济问题的关键
                \item 稳定是改革发展的前提
            \end{itemize}
        \end{itemize}


\section{全面依法治国}
    \subsection{全面依法治国方略的形成发展}
        \verb|NotImplementedError: 书P255-256页。|

    \subsection{中国特色社会主义法治道路}
        在坚持和拓展中国特色社会主义法治道路这个根本问题上,我们要树立自信、保持定力。
        \begin{itemize}
            \item \firstOrder{坚持中国共产党的领导。}党的领导和依法治国是高度统一的。\firstOrder{党的领导是社会主义法治最根本的保证。}
            \item 坚持人民在全面依法治国中的主体地位。坚持法治为了人民、依靠人民、造福人民、保护人民。
            \item 坚持法律面前人人平等。
            \item 坚持依法治国和以德治国相结合。
            \item 坚持从中国实际出发。
        \end{itemize}

    \subsection{深化依法治国实践的重点任务}
        \begin{enumerate}
            \item 推进中国特色社会主义法治体系建设
            \item 深化依法治国实践
        \end{enumerate}


\section{全面从严治党}
    \subsection{新时代党的建设总要求}
        \subsubsection{根本方针}
        \firstOrder{坚持党要管党、全面从严治党。}是新时代党的建设的根本方针。
        \begin{itemize}
            \item 全面是基础
            \item 严是关键
            \item 治是要害
        \end{itemize}

        \subsubsection{建设目标}
        \emph{把党建设成为始终走在时代前列、人民衷心拥护、勇于自我革命、经得起各种风浪考验、朝气蓬勃的马克思主义执政党。}

    \subsection{把党的政治建设摆在首位}
        \firstOrder{党的十九大首次把党的政治建设纳入党的建设总体布局},并强调以党的政治建设为统领,\firstOrder{把党的政治建设摆在首位。}
        \begin{enumerate}
            \item 旗帜鲜明讲政治是我们党作为马克思主义政党的根本要求
            \item 党的政治建设是党的根本性建设,决定党的建设方向和效果
            \item 注重抓党的政治建设是党的十八大以来全面从严治党的成功经验
        \end{enumerate}

        \firstOrder{党的政治建设的基本内容是:保证全党服从中央,坚持党中央权威和集中统一领导,是党的政治建设的首要任务。}

    \subsection{全面从严治党永远在路上}
        \begin{enumerate}
            \item 加强党的思想建设。\firstOrder{思想建设是党的基础性建设。}
            \item 加强党的组织建设
            \item 持之以恒正风肃纪
            \item 将制度建设贯穿党的各项建设中
            \item 深化标本兼治,夺取反腐败斗争压倒性胜利
        \end{enumerate}
